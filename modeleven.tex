\part{What is justice? What is equality?}
\label{ch.modeleven}
\addtocontents{toc}{\protect\mbox{}\protect\hrulefill\par}
\chapter{Part 25: Political Philosophy}


For this module, we are moving into an area of philosophy known as Political Philosophy. This area of philosophy deals with topics like justice, moral/immoral laws\footnote{It is possible, unless you are a moral relativist, to have laws which are immoral. For example, slavery was legal, but it was not moral. This area of Political Philosophy asks questions like "how would you enact moral rules into legal ones?"}, liberty\footnote{Questions here concern questions about the nature of liberty, freedom, how far does it extend, when should it be limited. This should not be confused with the free-will debate, as that's a question about whether our actions are deterministic and moral responsibility.}, punishment\footnote{Essentially "why do we punish people?", "how much should we punish?", and other such questions.}, and so on. For my content here, I am drawing much from John Rawls' A Theory of Justice (which best frames my personal beliefs regarding these things) and Robert Nozick's Anarchy, State, and Utopia (which is almost the exact opposite stance). Both of those works really frame the contemporary discussions in American Political Philosophy, with Nozick's ideas appearing often in conservative side of contemporary political discourse. To put a slogan to Rawls' notion of justice, it would be:
\begin{center}
Justice is equality
\end{center}
But, if that's the best way to encompass Rawls' views into three words (it is), then we will need to do a deep dive into the nature of equality in order to understand what it is and thereby understand what justice is. In doing this, also, we will see Nozick's view on justice come up from time to time, as it is in stark contrast to this notion.
\section{So what is equality exactly?}

Equality is often framed in terms of sameness. You and I are the same, so we are equal. We have the same job, so we should be paid equally. We are the same in social standing, so we are equal. Equality is often seen as a political goal, something worth shooting for. For example, if we notice a pay discrepancy, then we try to enact things which change that. In this case, you get the slogan "equal pay of equal work." Similarly, people want legal equality, and this is where we see Rawls' notion in its most obvious setting. If two people commit the same crime, we often think that it's unjust for one to get a lesser sentence than the other. People in protests about these things cry out that all people should be equal. Those who argue for equality, of any form, are called egalitarians.

Various intuitions about the nature of morality or other people have lead to the same, or subtly different, beliefs about the morality of equality.  In the case of Christian thought, it stems from the idea that all people are equal in the eyes of God. As a result, to have the kind of society which God would want for us, we should all be equal in it. Kantians, though often Christians, believe that, due to rationality, all people are due equal respect and from that should be equal within a society. Consequentialists, more particularly Utilitarians, tend to think that equality is the best way to ensure everyone’s well-being and get that all people should be equal to get the best society possible for people.

In general, we all have a pretty good intuition about what equality is, and this notion is applied in several different ways. For example, you can have numerical equality (2 = 2), equality in quality (being equally good), and many others. But, it should be pretty clear that all people are not equal in all things. Some people are smarter than others, prettier, stronger, taller, and so on. Pretty much any comparison you could make between two people will lead to some being more of that than others (in the case of natural, in born traits). It would be absurd for the egalitarians to argue that people should be equal in all things. Such a world would be one of clones, no one in any way better than any others. Egalitarians, therefore, can’t be making an argument with a conclusion like that (Rawls certainly does not).

But, despite this, many opponents of equality frame it in that way. They make the claim that their opponents are arguing something absolutely absurd in order to convince the people to vote/decide against them. This is called straw-manning, where you set up your opponent wrongly as to make them easily defeatable. Because of this easy confusion and the absurdity of the general, we must think of equality as being limited to various contexts. So, "all people should be equal" is not some statement about a world of indistinguishable people, but rather is a statement about equality in a particular context. With that being said, there are three places where people have argued for equality in sensible ways:
\begin{enumerate}
    \item Equality of the distribution of money
    \item Equality of employment opportunities
    \item Equality of political power 
\end{enumerate}
In order to make these sorts of equality sensible, various people have limited them even further, and in the case of justice as equality, have altered to scope of the equality, or given special exit-clauses.

\section{Equality of Money}

In Star Trek as well as The Orville, the future is portrayed without money. There are no paychecks for work done, there's no paying for food and water, and there's no paying for luxuries. The reason, explained in the shows, is that technology has removed the necessity of having money gauge rank, merit, or worth. Something, or several things, has replaced money (even digital coinage, like those cryptocurrencies) as the driving force to get things done. Basically, if we were to reach the point where we had something like the matter replicator from Star Trek, then either money would have already disappeared or it soon will.

But, as it sits now, we don't have that sort of technology, we need some kind intermediary to exchange goods.\footnote{For a great look at the history of money (and why simple barter systems just don't work), check out this series of videos from Extra Credits [link and citation needed]}. Equality of the Distribution of Money can be seen as a rather extreme view, way too utopian, but that has not stopped people from arguing for it and various lesser versions of it are found around the globe, as we will see. This is the stance that all money in a society should be equally distributed among all of the adults in the society. Everyone gets the same income, so to speak. Some have put this as a universal income regardless of the working status. But, as always, we need to look at the reasoning behind the idea to see whether it holds water.  There are, in general, two lines of thought to get to this idea:
\subsection{Utilitarian Considerations}

If you like a Utilitarian-style of thinking, then we have that having an equally distributed basic income will maximize the well-being of that society. This is because, as we have seen time and time again, and we will see later (after the discussion), massive income inequality can and does lead to an overall decrease in well-being. Sure, it's good for those who have a lot, but the state of those with little far outweighs that. People often think that others have the ability to rise up, given enough 'gumption', but this belief is caused by the exact same fallacy which we saw in the last module concerning harming animals and serial killers. 

Rawls used this kind of reasoning in A Theory of Justice. Though Rawls and the vast majority of Consequentialist Political Philosophers don't go this far, it is within the ballpark of possibility. Rawls and others claim that justice does entail equality of this sort, but with a massive caveat. Income inequality is just only if the population of the society, over all, is better off with that inequality than without it. Put into a more practical case, this means that if you are rich and you want a tax-break, then you will need to prove that you having more money than other people makes their lives better, not just yours. Or, in another example, reducing the taxes on the wealthiest of us would actually need to create jobs.\footnote{I used to use an example involving the Boeing Company for this, how their tax-breaks did not create jobs, but actually killed them, but I got too much recoil from it.} Putting this caveat on the equality of money does give you an even stronger Consequentialist position, but it still falls into the issues seen in the next page.
\subsection{Kantian-style Thinking}\footnote{This entire section on Kantian style thinking is up for revision, as an interesting paper has come to my attention concerning Kant on the redistribution of money, which is radically different than how Nozick portrays it.}

If you prefer a more Kantian way of thinking about these cases, then you might like this way of getting to equality of money. But, disclaimer, Kantians might not like this line of thinking. It is perfectly within their ability to deny the morality of distributing money, in that it would be taking a person's property without their consent. Kant, himself, thought that so long as the tax was applied fairly, as in a flat tax rate, then it would be fine. But the kind of taxation necessary to ensure equality of this kind would be a progressive tax rate, which would not be fair, per Kant. All people, morally, are equal and should be treated as such. From this moral equality you can get to a social equality. Meaning that the moral society is one where everyone is socially equal. What we see in the world is that money is expression of status in the world, so to make people socially equal, they would need to be monetarily equal.

\section{Equality of Money (Problems)}
\subsection{Problem 1: Impractical and Short-lived}

It should be clear that equally distributing money among all adults in a society is a logistical nightmare. There are many practical problems involved. Many people have claimed that whatever justice entails must be within practical limits, it can't be so utopian. So, justice must not entail this kind of equality because it's just so impractical. But, in reply, morality does seem to, sometimes, call on us to do the seeming impossible. Since justice is an aspect of morality, it's perfectly possible that justice requires such extreme measures.

At the same point, some have pointed to having, rather than a universal equal income, having a Universal Basic Income. In these cases, which are found in several places around the world, you can make more than a certain amount, but you can't make less. Ideally, this amount would pay for a roof over your head and food in your belly. But if you want more than that, you need to work for it.
\subsection{Problem 2: Different people deserve different amounts}

Another problem for this view is that, it is claimed, that different people deserve different amounts of money for the jobs that they do. For example, people often think that people doing really gross but necessary jobs, like garbage-people, make more money than other groups because few want to do it and it is necessary (this is not 100\% true, but it is a common conception).

The fundamental difference between egalitarians and those who believe in these inequalities is that the egalitarian holds that only small differences are acceptable and those correspond with need. From each according to their ability, to each according to their need. But the other side says that the inequality of any kind is acceptable, not just in need. To each according to their worth.
\subsection{Problem 3: Different people have different needs.}

Some people need more money to live than others. A person who needs expensive medical care daily to survive would have a very hard time living in a world with equality in this way. A method for distributing the money would need to include some kind of respect for basic individual humanity. Some kind of, in this case, universal health case, which is argued against as a step towards communism. 
\subsection{Problem 4: It is not right to redistribute.}

The best arguments for this can be found, not in Ayn Rand, but in Robert Nozick. This is that all of society should be structured in such a way as it ensures you keep what you make. “Taxation is theft” is a more contemporary way of saying the core point of Nozick’s work “taxation is slavery.”

\section{Equality of Opportunity}

This is NOT the stance that all people should have the same chance to get a job that they want, though it is sometimes given in that way. To see why saying all people should have the same shot at any job, take the following examples:

    \factoidbox{Suppose that I was applying to be the president of this college, the highest, top-tier position. I do not have any experience in that kind of work, aside from working with some committees here and there nor did I take any classes on college administration in school. Also suppose that at the same time another person was applying for the job (it can only go to one of us). This person took classes in college administration, they have experience in the field, and worked closely with the previous president, so they know the ins and outs of this particular college.  If equality of opportunity meant that all people had the same chance of getting any job if they apply, then this would mean that the hiring committee would need to flip a coin to decide who got the job.

    Similarly, suppose that two people had recently graduated from medical school, specializing in brain-surgery. One of them spent their summers working as a fugu-chef, getting some renown in that regard, the other spent their summers working as a mechanic on diesel trucks, also getting some renown. Assuming all else is equal, who would you rather have performing delicate surgery on you?}

But flipping a coin in this way is not how things should be done, and it should be clear why through looking at the second example. In the second case, it would be really strange for you to pick the mechanic. The delicate and precise hand-eye coordination required to prepare and serve blowfish (which is highly poisonous, if the smallest error is made in the process) is a relevant factor in making the choice. 

So, equality in opportunity can't be too extreme. With this in mind, we will say the following:
\begin{center}
Equality of opportunity is the stance that two people should have the same shot at a job given their abilities for the job.
\end{center}
No factors unrelated to the job should be included. Sometimes factors unrelated to their education or direct experience can be included (in the case of the mechanic vs fugu-chef), but those are related to the job. So, in the case of being president of the college, I should not be given the same chance as someone with experience, because I am unfit for the position. That being said, if two people are equally qualified for a position, then the just thing would be to flip a coin.

Failure to adhere to this sort of impartiality is most often seen in cases of nepotism. Nepotism, literally, is the favoring or giving preferential treatment to friends and relatives when making a hiring decision. Often, in such cases, the treatment is so preferential that people wholly unqualified for a position are given it solely because of their relationship with the person making the choice.  Doing such a thing is a very dangerous decision, because it's likely that they will do worse in the position than someone else. I believe it was in The Princess Diaries 2 where it was said "Nepotism belongs in the arts, not in plumbing". 

On the other-side, a common way to promote equality in opportunity given ability is found in the real world through Reverse Discrimination practices. Reverse discrimination means actively recruiting people from previously underprivileged groups. In other words, reverse discriminators deliberately treat job applicants unequally in that they are biased towards people from groups against which discrimination has usually been directed. The point of treating people unequally in this way is that it is intended to speed up the process of society becoming more equal, not only by getting rid of existing imbalances within certain professions, but also by providing role models for young people from the traditionally less privileged groups to imitate and look up to. Take the case of philosophy professors for example:

    \factoidbox{In 2003, 16.6\% of the full-time philosophy professor positions went to women, but women made up 27.1\% of the applicants, and there were no women of color at all. In order to encourage more women to pursue this field, especially since there is a history of them doing well, women applicants are starting to be treated better than their masculine counter-parts. This has lead to multiple departments around the country being mostly women, for example the department at Pierce College in Washington State (as of 2018).}

You will notice that the vast majority of the work read in this class was by men. This is in part because there aren't many readings appropriate for this level on the relevant subjects written by women. But, adding in more woman philosophers would be a change of pace for the standard classes (I happen to know that even PHIL\&101 classes taught by women have mostly masculine authors). In selecting my readings for this course, if I had given preferential treatment to works written by members of underrepresented groups, then I would have been engaged in reverse discrimination.

\section{Reverse Discrimination (Problems)}

Though reverse discrimination is seen today in the workplace as well as elsewhere (such as, this kind of thinking is applied in cases of political decisions, like who to vote for), there are some issues worth discussion. Most of the time, the issues which arise from applying this kind of procedure boil down to one, or both, of these two, but there are other issues which can, and do, appear from time to time.
\subsection{Problem 1: It is actually anti-egalitarian!}

Remember what I said previously about nepotism. This is giving preferential treatment to friends and family, when making a selection. The applicant's relationship with the reviewer/hiring person is irrelevant to the job, so nepotism just can't fit into equality of opportunity. At the same point, reverse discrimination just can't fit into equality of opportunity. Put into a slogan, this issue can de summed up as:
\begin{center}
Reverse Discrimination is still discrimination
\end{center}
The aims of reverse discrimination may be egalitarian, but some people feel that the way it achieves them is unfair. Remember what I said about this stance, equality of opportunity does not allow for any features not related to the job to be taken into account in the hiring process. For example, in the real world, I am not allowed to have my picture in my CV\footnote{A CV is a kind of resume used in academic and more brainy lines of work. It tends to be far more in depth, but far more dry.} because what I look like is a irrelevant factor. Similarly, if I had given favoritism to women philosophers in choosing my content, I would have been bringing in a factor which is not relevant to the quality of the course which I want to provide and create. For a staunch egalitarian, a principle of equality of opportunity in employment means that any form of discrimination on non-relevant grounds must be avoided. The only grounds for treating applicants differently is that they have relevantly different attributes. Yet the whole justification of reverse discrimination rests on the assumption that in most jobs such things as the sex, sexual orientation, gender, or racial origin of the applicant are not relevant.\footnote{Such identifying characteristics fall into what Rawls calls "non-moral factors" when discussing his test for how just a society is. In his case, reverse discrimination is bringing in "irrelevant factors".}
\subsection{Problem 2: Resentment}

This issue is one which is found quite often, though I am happy to say that I have never felt this way. Reverse discrimination is supposed to make access to various professions more evenly distributed across the population, but sometimes, this is not the case, in fact, it can have the opposite affect. Sometimes, in order to reach certain standards of equality, the groups will have no choice but to hire a person who is simply not qualified for the position, or intentionally not hire the best person for the job because they are not from a disadvantaged group. Those who fail to get a particular job because they happen not to come from a disadvantaged group may feel resentment against those who get jobs largely because of their sex or racial origin. Similarly, if a person gets a job because of their status and they fail at it, this can lead to other members of that group thinking that they, too, will fail.
\subsection{Problem 3: It attacks the symptom, but not the cause.}

Another issue with this is in how important it is. When we look at why certain groups are underrepresented in various fields, we see that it's actually the symptom of further, more deep seeded, issues. These issues include residues from racist economic and housing policies, backwards education spending measures, rising cost of an education, and many more. Some of these issues are more historical, but the results of them still ripple into today. Reverse discrimination practices do not treat these issues, rather they only treat the symptoms.

In order to truly resolve these issues in a fair and just way, and thereby get a more representative distribution of people across fields, we would need to radically change our mindsets about economic and educations policies. For example, we would need to make public colleges and universities tuition free (at the very least), better standardize and evenly distribute the quality of education across the country, and encourage and implement public works projects in the poorer communities to provide quality jobs in those regions. These things are all doable and do not involve anti-egalitarian practices.

\section{Equality of Political Power}

Another area where equality is argued for is in the sphere of government. It is said that all people should have the same impact in political decisions or have the same ‘volume’ when it comes to their voices being heard. In this case too, there is the potential for straw-manning, as we have seen before. The opponents to this kind of equality will claim that it means that a new born has the same political voice as a well-informed adult. This, frankly, is just wrong, and it should not be hard to think of why. Often, in cases like this, the right to make the choice falls to the guardian, meaning that if you have a ton of infant-children under your care, you get to vote on behalf of each of them. This is not what is meant. Rather we need to limit it:
\begin{center}
Equality of political power is the stance that all people, who are capable to make informed decisions for themselves, should have an equal impact in the decisions made by their government.
\end{center}
This is called a democracy. Often, democratic processes are done through voting, and in these cases, all people (having the relevant abilities) should be entitled to vote. This is different from a republic where the people have an equal say in electing people to make decisions for them.  Normally, the term democracy has two different senses. The first emphasizes the need for members of the population to have an opportunity to participate in the government of the state, usually through voting. The second emphasizes the need for a democratic state to reflect the true interests of the people, even though the people may themselves be ignorant of what their true interests may be. This is freedom of speech, as seen in the voting process.
\subsection{Direct Democracies}

Some of the earliest forms of democracy were direct democracies, these are where those eligible to vote have a say in each individual issue. This is like a Reddit-kind of government. For every issue or decision, all people can, and should, vote on it. The solution which gets the most votes is the winner and that is the course taken. Typically, these are only feasible with smaller societies or when there are not many choices to make. For larger societies or for those with many complex issues, there are practical limitations. For an interesting, humorous, Sci-Fi look at this, see the episode Majority Rule of The Orville. There are other issues with this, such as the people are not experts, they can and often do make the wrong choice.
\subsection{Representative Democracies}

In a representative democracy elections are held in which voters select their favored representatives. These representatives then take part in the day-to-day decision-making process, which may itself be organized on some sort of democratic principles. This streamlines the process a bit and helps remove some of the worries which appear in direct democracies. It also removes a lot of the pressure on you to make the decisions which affect others, the person elected gets that pressure. Essentially, in a representative democracy, the people vote and elect individuals from their midsts to make the day-to-day government decisions on their behalf.

There are several different ways in which such elections are conducted. Some demand a majority decision; this is where the representative must have gotten at least 51\% of the votes. For example, in these sorts of systems, we can have cases like these:

    \factoidbox{Michael, Annie, and Billie are all running for the same senate seat. Michael gets 10\% of the vote, Annie gets 30\%, and finally Billie gets 60\%. In this case, Billie wins the election and takes the office. But, running for president, we have John, Paul, George, and Ringo running. Paul gets 20\%, John gets 10\%, George gets 21\%, and Ringo, because of his acting in a hilarious slap-stick comedy, gets 49\% of the vote.  Because none of them got above 50\% of the vote, none of them take the seat.}

In such cases, there needs to be additional rules for what happens when none of the people running get the necessary majority.

Others, such as the one used in Britain, demand that the representative merely got the most votes. This removes the worry about needing to get some absolute majority and the need to have a procedure for when one is not reached. This leads to cases like these:

    \factoidbox{Murdoc, Tudee, Noodal, Hobbs, and Phoo are all running for a seat in Parliament. Phoo gets 10\%, Noodal gets 5\%, Murdoc gets 20\%, Hobbs gets 25\%, and Tudee gets 40\%. Since it's merely based on who got the most votes, Tudee will take the seat in Parliament.}

Representative democracies achieve government by the people in some ways but not in others. They achieve government by the people in so far as those elected have been chosen by the people. Once elected, however, the representatives are not usually bound on particular issues by the wishes of the people. Having frequent elections is a safeguard against abuse of office: those representatives who do not respect the wishes of the electorate are unlikely to be re-elected. Most democracies today are representative democracies.

\section{Democracies (Problems)}

Yes, we live in a representative democracy here in the United States. Luckily, we live in a society where we can and do question the nature of our government. Plato, for one, was opposed to democracies, not just because that sort of system killed Socrates, but also because he thought that the people can't make good decisions for themselves about what's good for them. Similarly, several thinkers throughout the world have found various issues with democracies. Here we will be covering three of them.
\subsection{Problem 1: It’s an illusion}

Some theorists have attacked these forms of democracy as providing a merely illusory sense of participation in political decision-making. They claim that voting procedures won’t guarantee rule by the people. Some voters may not understand where their best interests lie, or may be duped by skillful speech-makers. And besides, the range of candidates/choices offered in most elections doesn’t offer voters a genuine choice.

    \factoidbox{In the US (and this is true), various studies have been conducted concerning the political views of the average American. These are done by asking questions about various policies or giving them thought-experiments (the latter being the more accurate). They find that the average American tends to be center-left when it comes to policies. But if we apply the same metrics to the candidates being offered to the people, we find that the farthest left they go is center-right. This limited choice has slowly moved the policies and choices made to the right, when they should, according to the people, be at least, left of center.}

It is hard to see why this sort of democracy is so praised when it typically amounts to choosing between two or three candidates with virtually indistinguishable political policies. Similarly, when it comes to policy decisions, often the choices put forth are merely a yes-no choice on a single law which the people, elected or otherwise, would need to fully understand in order to make a decision on, and they, simply, are not given enough time. 
 
\subsection{Problem 2: Voters are not experts}

Plato and others have pointed out that sound political decision-making requires a great deal of expertise, expertise which many voters do not have. Thus direct democracy would very likely result in a very poor political system, since the state would be in the hands of people who don’t know what they are doing.
\begin{center}
The captain, not the passengers, should steer the ship.
\end{center}
A similar argument can be used to attack representative democracy. Many voters can’t judge how good a particular candidate is for the job, since they don’t know what they are doing in the case of political policy. People tend to choose their representatives on the basis of non-relevant features, such as how good-looking they are, they are the same gender as me, they speak in a way I can understand (5th grade reading level), that they aren't a politician (which should be a disqualifying factor in some cases), and so on. This could be seen as a form of nepotism, or at least something like it. Remember that nepotism involves taking in irrelevant features (in this case, relation to you) into the selection process. Here, they are bringing in irrelevant features as well. Similarly, another irrelevant feature is the political party which the person belongs. A person may run as a Democrat, for example, and be totally inappropriate for the position. As a result, many excellent potential representatives are deemed untouchable by the populous, and many unsuitable ones get chosen on the basis of irrelevant qualities they have, of which I am sure you can think of examples. 

However, this evidence could be turned around and used as an argument for educating citizens for participation in democracy, rather than abandoning democracy altogether. And even if this is not possible, it may still be true that representative democracy is, of all available alternatives, the most likely to promote the interests of the people.
\factoidbox{
    The major problem—one of the major problems, for there are several—one of the many major problems with governing people is that of whom you get to do it; or rather of who manages to get people to let them do it to them.
    To summarize: it is a well-known fact that those people who must want to rule people are, ipso facto, those least suited to do it.
    To summarize the summary: anyone who is capable of getting themselves made President should on no account be allowed to do the job.

    -Douglas Adams}

\subsection{Problem 3: The Paradox of Democracy}

For this, I will start with an example and then I will explain how this leads to a contradiction in my beliefs. I will need to, at the same time, say that we need to do something and that we shouldn't do it.

    \factoidbox{Suppose that I believe that eating meat is barbaric and should never occur in a civilized state (like Plato). If it came up for a vote, I would vote against meat-consumption. Yet, suppose that it did and I voted against it, and the majority decision is that it should be the case that people are allowed to eat meat.}

I could replace eating meat in this example with just about any moral belief which I have. In any such case, I am faced with a paradox. If I am committed to democratic principles, then I believe that the majority decision (or the decisions of the appropriately elected person) should be enacted. But, as a person, I will have certain strongly held moral beliefs about certain things. If I end up in the minority, the losing side, of a democratic vote about what I believe (for example, that refugees should be allowed in the country, that various income tax exemptions should be closed, etc), then I will both believe that it should not be and that it should be. In the case of eating meat, in the example, I would, at the same time, believe that eating meat should not be allowed and believe that it should be allowed. 