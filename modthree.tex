\part{What is The Mind? What is the Body?}
\addtocontents{toc}{\protect\mbox{}\protect\hrulefill\par}
\label{ch.modthree}
\chapter{The Mind-Body Problem by Tim Crane}\autocite{Crane1}
\label{mindbodyprob}
\newcounter{ff}
\setcounter{ff}{\thefootnote}
\setcounter{footnote}{0}
The mind-body problem is the problem of explaining how our mental states, events
and processes—like beliefs, actions and thinking—are related to the physical states,
events and processes in our bodies. A question of the form, ‘how is A related to B?’
does not by itself pose a philosophical problem. To pose such a problem, there has to
be something about A and B which makes the relation between them seem
problematic. Many features of mind and body have been cited as responsible for our
sense of the problem. Here I will concentrate on two: the fact that mind and body
seem to interact causally, and the distinctive features of consciousness.

A long tradition in philosophy has held, with René Descartes, that the mind
must be a non-bodily entity: a soul or mental substance. This thesis is called
‘substance dualism’ (or ‘Cartesian dualism’) because it says that there are two kinds
of substance in the world, mental and physical or material. One reason for believing
this is the belief that the soul, unlike the body, is immortal. Another reason for
believing it is that we have free will, and this seems to require that the mind is a
non-physical thing, since all physical things are subject to the laws of nature.

To say that the mind (or soul) is a mental substance is not to say that the mind is
made up of some non-physical kind of stuff or material. The use of the term
‘substance’ is rather the traditional philosophical use: a substance is an entity which
has properties and persists through change in its properties. A tiger, for instance, is a
substance, whereas a hurricane is not. To say that there are mental substances—
individual minds or souls—is to say that there are objects which are non-material or
non-physical, and these objects can exist independently of physical objects, like a
person’s body. These objects, if they exist, are not made of non-physical ‘stuff’: they
are not made of ‘stuff’ at all.

But if there are such objects, then how do they interact with physical objects?
Our thoughts and other mental states often seem to be caused by events in the world
external to our minds, and our thoughts and intentions seem to make our bodies
move. A perception of a glass of wine can be caused by the presence of a glass of
wine in front of me, and my desire for some wine plus the belief that there is a glass
of wine in front of me can cause me to reach towards the glass. But many think that
all physical effects are brought about by purely physical causes: physical states of
my brain are enough to cause the physical event of my reaching towards the glass.
So how can my mental states play any causal role in bringing about my actions?

Some dualists react to this by denying that such psychophysical causation really
exists (this view is called ‘epiphenomenalism’). Some philosophers have thought
that mental states are causally related only to other mental states, and physical states
are causally related only to other physical states: the mental and physical realms
operate independently. This ‘parallelist’ view has been unpopular in the 20th
century, as have most dualist views. For if we find dualism unsatisfactory, there is
another way to answer the question of psychophysical causation: we can say that
mental states have effects in the physical world precisely because they are, contrary
to appearances, physical states\autocite{Lewis1}. This is a *monist* view, since it
holds that there is *one* kind of substance, physical or material subtance. Therefore
it is known as ‘physicalism’ or ‘materialism’.

Physicalism comes in many forms. The strongest form is the form just
mentioned, which holds that mental *states* or *properties* are identical with
physical states or properties. This view, sometimes called the ‘type-identity theory’,
is considered an empirical hypothesis, awaiting confirmation by science. The model
for such an identity theory is the identification of properties such as the heat of a gas
with the mean kinetic energy of its constituent molecules. Since such an
identification is often described as part of the *reduction* of thermodynamics to
statistical mechanics, the parallel claim about the mental is often called a ‘reductive’
theory of mind, or ‘reductive physicalism’\autocite{Lewis2}.

Many philosophers find reductive physicalism an excessively bold empirical
speculation. For it seems committed to the implausible claim that all creatures who
believe that grass is green have one physical property in common—the property
which is identical to the belief that grass is green. For this reason (and others) some
physicalists adopt a weaker version of physicalism which does not have this
consequence. This version of physicalism holds that all particular objects and events
are physical, but allows that there are mental properties which are not identical to
physical properties. (Davidson\autocite{Davidson1} is one inspiration for such views.) This kind of
view, ‘non-reductive physicalism’, is a kind of dualism, since it holds there are two
kinds of property, mental and physical. But it is not *substance* dualism, since it
holds that all substances are physical substances.

Non-reductive physicalism is also sometimes called a ‘token-identity theory’
since it identifies mental and physical particulars or tokens, and it is invariably
supplemented by the claim that mental properties *supervene* on physical
properties. Though the notion can be refined in many ways, supervenience is
essentially a claim about the dependence of the mental on the physical: there can be
no difference in mental facts without a difference in some physical facts (see Kim\autocite{Kim1}; Horgan\autocite{Horgan1}).

If the problem of psychophysical causation was the whole of the mind-body
problem, then it might seem that physicalism is a straightforward solution to that
problem. If the only question is, ‘how do mental states have effects in the physical
world?’, then it seems that the physicalist can answer this by saying that mental
states are identical with physical states.

But there is a complication here. For it seems that physicalists can only propose
this solution to the problem of psychophysical causation if mental causes are
identical with physical causes. Yet if properties or states are causes, as many
reductive physicalists assume, then non-reductive physicalists are not entitled to this
solution, since they do not identify mental and physical properties. This is the problem of mental causation for non-reductive physicalists. (See Davidson\autocite{Davidson2},
Crane\autocite{Crane2}, Jackson\autocite{Jackson1}).

However, even if the physicalist can solve this problem of mental causation,
there is a deeper reason why there is more to the mind-body problem than the
problem of psychophysical interaction. The reason is that, according to many
philosophers, physicalism is not the *solution* to the mind-body problem, but
something which gives rise to a version of that problem. They reason as follows: we
know enough to know that the world is completely physical. So if the mind exists, it
too must be physical. However, it seems hard to understand how certain aspects of
mind—notably consciousness—could just be physical features of the brain. How can
the complex subjectivity of a conscious experience be produced by the grey matter
of the brain? As McGinn\autocite{McGinn1} puts it, neurones and synapses seem ‘the wrong
kind’ of material to produce consciousness. The problem here is one of intelligibility:
we know that the mental is physical, so consciousness must have its origins in the
brain; but how can we make sense of this mysterious fact?

Thomas Nagel dramatised this in a famous paper\autocite{Nagel1}. Nagel says that
when a creature is conscious, there is something it is *like* to be that creature: there
is something it is like to be a bat, but there is nothing it is like to be a stone. The heart
of the mind-body problem for Nagel is the apparent fact that we cannot understand
*how* consciousness can just be a physical property of the brain, even though we
know that in some sense physicalism is true (see also Chalmers\autocite{Chalmers1}).

Some physicalists respond by saying that this problem is illusory: if physicalism
*is* true, then consciousness is just a physical property, and it simply begs the
question against physicalism to wonder whether this *can* be true (see Lewis\autocite{Lewis3}).
But Nagel’s criticism can be sharpened, as it has been by what Frank Jackson calls
the ‘knowledge argument’ (Jackson\autocite{Jackson2}; see also Robinson\autocite{Robinson1}). Jackson argues
that even if we knew all the physical facts about, say, pain, we would not ipso facto
know what it is like to be in pain. Someone omniscient about the physical facts
about pain would learn something new when they learn what it is like to be in pain.

Therefore there is some knowledge—knowledge of what it is like—which is not
knowledge of any physical fact. So not all facts are physical facts. (For physicalist
responses to Jackson’s argument see Lewis\autocite{Lewis4}; Dennett\autocite{Dennett1}; Churchland\autocite{Churchland1}.)

In late twentieth century philosophy of mind, discussions of the mind-body
problem revolve around the twin poles of the problem of psychophysical causation
and the problem of consciousness. And while it is possible to see these as
independent problems, there is nonetheless a link between them, which can be
expressed as a dilemma: if the mental is not physical, then how we make sense of its
causal interaction with the physical? But if it is physical, how can we make sense of
the phenomena of consciousness? These two questions, in effect, define the
contemporary debate on the mind-body problem.
\setcounter{footnote}{\theff}
\chapter{Part 4: The Mind-Body Problem}
The mind-body problem is a very, very, old puzzle in philosophy. It is trying to explain how our 'what-it's-like-nesses' relate to our bodily states. It's trying to explain how our thoughts, emotions, and other mental qualities relate to physical (bodily) events. Basically, the Mind-Body Problem is trying to answer a question of the form:

\begin{center}How is X related to Y?\end{center}

There are many questions out there which are of this general form, for example, we have questions like:
\begin{earg}
    \item[]How is the cause related to the effect?
    \item[]How is exposition related to multiplication?
    \item[]How is matter related to gravity?
    \item[]How are the results of actions related to morality?
\end{earg}
For the Mind-Body Problem, the question is:

\begin{center}How is the mind related to the body?
How is the body related to the mind?\end{center}

Answering the question “how does one thing relate to another?” isn’t too hard to answer in most contexts. To make it a problem worth thinking about, the things need to have some obvious connection between them \emph{and} there needs to be something about the connection which seems problematic, or leads to problems. Conspiracy theories often try to make a connection between two things, but either there’s not an obvious connection or the proposed connection is problematic (crazy jumps in reasoning, odd consequences, etc.).\footnote{If you are interested in the Philosophy of Conspiracy Theories, I have a collection of readings, and an order to read them in, which can be at an intro-level.}

In this case, the question "how is the mind related to the body?" is not easy to answer and just about any claimed relationship between the two seems to have problems. But, it seems very obvious that the mind, my 'thinker' so to speak, does have some kind of relationship with my body. I think of things, my body does them, I feel sad, my body cries. So, the core question is:

\begin{center}How are our mental states, beliefs, feelings, or thinkings, related to our bodily states, the events which go on physically?\end{center}

There are two dominate theories, \gls{physicalism} and \gls{substance dualism}, about this and both have their problems, which we will go into detail about.

\section{Part 4.1: Substance Dualism}

This is a very classic stance that there are two sorts of things in the world, rather than just one. It was best put forth by a guy named Rene Descartes (who we will go in depth on in Module 6). Some people think that there’s only material, physical, things in the world. Descartes thought that there were two kinds of things:
\begin{earg}
    \item[]Physical Substances
    \item[]Mental Substances
\end{earg}
\subsection{Substance:}

In that tradition of philosophy, a \gls{substance} is a thing which has properties and can survive the change in those properties. For example, my car has the property ‘silver’ but I can paint it black and it would still be my car.

\newglossaryentry{substance}
{
name=substance,
description={The thing itself, which has properties and can survive changes in those properties.}
}


To say that there are mental substances is to say that there are non-material entities which exist independently of the physical stuff, like the body. The everyday term for a mental substance is 'soul'. We should be careful, this is a very confusing area; mental substances, or souls, are not composite according to the dualist. They are, to use some modern terminology, simple.Your typical physical substance is composed, built-up of, smaller physical things, like wood and metal, then the wood and metal are built of smaller parts, all the way down to strings (if String Theory is correct) or some other basic building-block. Mental substances (souls) aren't supposed to be like that. They are not composed of non-physical stuff and they are not composed of physical stuff, they are simple, basic, non-composite This is to say that mental substances do not have parts, they cannot be broken up, so to speak, or divided. 

\Gls{substance dualism} is sort of the default view which many people have when they enter into this kind of debate, they think that they have a soul or something like that, and there have been various reasons given to think that there is this sort of soul.\footnote{This is especially true if the person enters the debate with a prior belief in an afterlife or certain religious stances.} Some people think that there is some sort of afterlife. Though it is possible to get an afterlife (with consciousness) without a soul, the default view seems to be that one is required. Others claim that people have free will and this requires a non-material soul. Others still think that there are different properties had by \emph{you} and your \emph{body}. If two things are the same, then there wouldn't be that kind of difference. For example:
\newglossaryentry{substance dualism}
{
name=substance dualism,
description={The stance in that the world consists of two general kinds of substances, in the case of the Mind-Body Problem, these are mental and physical.}
}
\begin{earg}
    \item[1 ] I can imagine myself without a body.
    \item[2 ] If two things are the same, then if you imagine one, you imagine the other.
    \item[3 ] Therefore, I am not my body.
\end{earg}
From this we get that I am a soul, something non-physical.

Another argument which is given for this stance goes like this:
\begin{earg}
    \item[1 ] The mind is immortal, the body is not.
    \item[2 ] If the mind and the body were the same thing, this would not be the case.
    \item[3 ] Therefore, the soul is different from the body.
\end{earg}
And a third argument often given for this stance, again from Descartes, goes like this:
\begin{earg}
    \item[1 ] I have free will.
    \item[2 ] If I was just physical, then I would be subject to the laws of nature (no free will).
    \item[3 ] Therefore, I am not just physical.
\end{earg}
This particular argument is the center of a current project I am working on concerning free-will and this problem. We will explore more of this (free will) in Module 4.

\subsection{The Mary's Room Thought Experiment}

This is a relatively recent thought-experiment, a case to think about and try to understand with interesting implications. It was written in 1982 by Frank Jackson. The point of it is to give reason to think that everything is not just physical. This thought experiment leads to an argument for some flavor of dualism. I know that it is not very realistic, but I have amended it to make it more so. Here is the case:

\factoidbox{Two undercover spies fell in love and had a child. Due to the nature of their work, the government took the child and locked her a way in a room. The child in named “Mary”. Mary was forced to grow up in this room, but here’s the kicker, there’s only black and white. Mary never experiences colors at all. Mary, in this room, grows to be a brilliant scientist. She specializes in the neurophysiology of vision and acquires all the physical information there is to obtain about what goes on when we see a red rose, or the sky, and use terms like ‘red’, ‘blue’, and so on. She discovers, for example, just which wavelength combinations from the sky stimulate the retina, and exactly how this produces via the central nervous system the contraction of the vocal chords and expulsion of air from the lungs that results in the uttering of the sentence ‘The sky is blue’... Over the years, the political climate has changed and this sort of undercover work and the cloistering of the children becomes very taboo. The president finds out about Mary’s plight and orders her to be released. What will happen when Mary is released from her black and white room? On the day of her release, the president is present and hands her a red rose. Does she learn anything in that moment?}

\textbf{The Mary's Room Argument}
\begin{earg}
    \item[]Mary knows all of the physical facts about color vision.
    \item[]Mary has never experienced color.
    \item[]Upon seeing color for the first time, Mary learns something.
    \item[]If you learn something, then that thing is a fact you did not know before.
    \item[]So, Mary did not know a fact about color vision.
    \item[]If Mary did not know a fact about color vision, that fact must be non-physical.
    \item[]Therefore, there are some non-physical (mental) facts.
\end{earg}
This argument will \emph{either} get you substance dualism or a particular version of physicalism, but it will not get you certain kinds of physicalism. There are ways out for the physicalist, which we will see later.

\subsubsection{How Do These Substances Interact?}

Now we need to ask how mental substances cause physical events and vice versa. The stance that there is this sort of causation between them is called \gls{epiphenomenalism}. It seems clear that certain bodily states, like stubbing my toe, result in mental states (the feeling of pain) and that certain mental states, like feeling sad, result in bodily states (crying). But, how does this work?

The Causation Problem is essentially asking "how do body states cause mental states?" and, of even more importance, "how do mental states cause bodily states?" Many philosophers go with Epiphenomenalism because they think that we have some kind of free will. The various notions of free will and the arguments for and against it will come in Module 4. But, for now, let's just say that by 'free will' I mean that (at least some of) our choices are not deterministic, that some super-computer could not predict the choices we make before we make them.
\newglossaryentry{epiphenomenalism}
{
name=epiphenomenalism,
description={The stance in that certain physical events can cause mental events and vice versa.}
}

If our choices are not deterministic, then they must be from outside of the laws of nature (because the laws of nature are deterministic (if they aren't, that doesn't help either)).  If they come from outside of the laws of nature, then they must come from something non-physical (because the laws of nature govern physical things). So, something non-physical must be able to interact with something physical (EG the mental with the physical).

Boiling all of that last paragraph down, if our actions are non-deterministic, then something non-physical must be able to interact with something physical. This is far from a settled claim, as we will see when we cover free will. But, if you go with substance dualism to get free will, then you have a major problem. You need to have a way of getting the mental substances and the physical substances to interact, at least mental to physical. 

Some claim that this is an impossible task. The physical things are physical, the mental things are mental, and never the two shall meet. These folks claim that there can be no causation or interaction between the mental and physical, so by reasoning above, our actions are deterministic (uh-oh). Others claim that there can only be causation going from physical to mental, but not the other way around. This gives us the same problem as before.

\subsubsection{Pre-Established Harmony}
There is one theory which accepts, whole hog, the idea that there are two kinds of things in the world, mental and physical, and that the two cannot affect each other. This view, put forth by Leibniz,\footnote{references and citation needed} states that a person has two things, a soul and a physical body (a person is composed of a mental substance and a physical substance), but makes three further claims:
\begin{enumerate}
    \item[]No state of a mind can cause a state in another mind or body and no body can cause a state in another body or mind (basically, minds and bodies can't interact, minds and minds can't interact, and bodies and bodies can't interact).
    \item[]Every state of a substance which wasn't a miracle and wasn't its starting state, was caused by the previous state of that substance (basically, how some substance was determines how it will be).
    \item[]Minds and bodies are programmed (or pre-determined) to behave in mutual coordination with each-other. 
\end{enumerate}
This is the Pre-Established Harmony stance. The first claim gives us our answer to the Causation Problem, namely, there's not any causation between the mental and the physical. Rather, because of the third claim, it merely appears to be causation. There's correlation but not causation. The second claim smooths out any wrinkles which may appear in the stance because it gives us that the world is deterministic, lights and clock-work. 

For most versions of Pre-Established Harmony out there, it would seem, the 'programming' of the substances is arranged by God or some other divine architect.\footnote{This can be related to divine foreknowledge as a problem for free will.} Many people think that God is all-knowing, and this will appear again when we discuss arguments for and against the existence of God. If this is correct, then we can explain this by saying that God was the one who programmed the substances. But this leads to a further worry, and a potential problem. 

As I mentioned before, many people like Substance Dualism because it will get you some kind of Free Will, but Pre-Established Harmony denies the possibility of a substance doing other than how it was programmed, everything in the world is deterministic. This denies the possibility of Free-Will. People will have the illusion of control, but that control is much like a little kid holding a toy steering wheel. Sure they may mimic, without realizing, the movements of the driver perfectly, but they are not the one driving the car. 

\section{Part 4.2: Monism/Physicalism}

Another way to solve the causation problem and Mind-Body Problem is to say that there’s actually only one kind of substance in the world. This is a rejection of dualism and is called \gls{monism}. It can come in two forms.

\newglossaryentry{monism}
{
name=monism,
description={The stance that the world is comprised of only one kind of substance.}
}


The first of these forms is called \Gls{idealism}.This is the stance that there are only mental substances in the world. So, and I mean this jokingly, be nice to your table, it has feelings. The second is the real one which we will be concerned with for this class, but idealism does still have its adherents, is called \Gls{physicalism}. 
\newglossaryentry{idealism}
{
name=idealism,
description={The stance in that the world consists of only one general kind of substance, mental.}
}

\newglossaryentry{physicalism}
{
name=physicalism,
description={The stance in that the world consists of only one general kind of substance, physical.}
}


As a stance in the Mind-Body Problem as well as in other areas (though not as common), Physicalism comes in several different forms. But to really understand the distinction between these, we should cover what is meant by the terms "type" and "token." A type is a general class of things. For example, 'tree' is a type, same with 'car'. There are many individual things which are labeled as trees or as cars. Tokens, on the other hand, are individual instances of a type. When we talk about various things, it's useful to be careful about whether we are dealing with types or with tokens. For example, if someone were to claim that 'lying is morally wrong', we would need to know whether they are talking about all cases where a person knowingly misinforms another or just an individual instance of doing so. If they are talking about type of action and labeling all cases of lying as wrong, then all we need to do is point to a cases where it's OK to lie and that would disprove their claim. On the other hand, if they are talking about a token of the action, then we would need to look closely at the individual case to try and prove them wrong.

\subsection{Reductive Physicalism}

The different kinds of physicalism all share something in common, namely, that the mental states are the physical states, but they differ in whether they think in terms of types or tokens. The first, called "Reductive Physicalism" thinks in terms of types, they claim that the mental states you have are identical to your physical states, meaning that a type of mental experience maps to a type of physical state (likely in the brain). This kind of theory is far more empirical in nature than the others which philosophers typically deal with and would require a ton of brain scans to set up. The model for this kind of identity (type-identity) is often found in science, where they identify a general class of things with another. For example, how hot an object is and the mean kinetic energy of its molecules. 

Reductive Physicalism is the strongest one which you are going to find, it's making a very bold claim. Some claim that this is too bold of a claim. Reductive Physicalism entails that if two people have the same thought, then their brains had to be lit up (so to speak) in the same way. To some, this just doesn't seem plausible, as people come to the same conclusion about things all the time, but all brains are fundamentally different. One could bite the bullet and say that people have similar thoughts, but never the same thought, but that too requires a ton of experimental data. 

All that being said, the Reductive Physicalist does have a solution to the Mind Body Problem and the Causation Problem. For the Mind-Body Problem, the solution is that the mind is the body, there's no difference and for the Causation Problem, it's just physical to physical causation, so who cares? 

\subsection{Non-Reductive Physicalism}

Some people want to keep the physicalism, but don't want to say that all people have different thoughts, that two people can never have the same thought. This is where we get the other major kind of physicalism, Non-Reductive Physicalism. Rather than dealing with types, Non-Reductive Physicalism deals with tokens. Like Reductive Physicalism, Non-Reductive Physicalism says that there's only one kind of substance, namely physical, but Non-Reductive Physicalism claims that there are two different kinds of properties.

One way to think about this is in terms of colors. There are many different ways in which a certain shade of green can be produced. This can be from the particles on the surface of an object being arranged a certain way and having uncolored light bounce off of it or it can be from having colored light bouncing off of it and its particles being arranged in a different way. But, regardless, it's still the same shade of green. Similarly, the mental state, your thought, can be produced from a whole bunch of different arrangements of neurons in your brain. For the Non-Reductive Physicalist, the identification between the mind and the body is one of supervenience. 

Supervenience is a bit of a tricky topic, but mostly because it's a word that you hardly ever see, we encounter the concept all the time without realizing it. Supervenience is a kind of relation between two things. It is basically that one thing supervenes on another when there can’t be a change in the first without a change in the second. The first depends on the second. So, for example, whether or not something is beautiful supervenes on its arrangement. If you want to make something more or less beautiful, you fiddle with how it's arranged. Similarly, some claim that the morality of an action supervenes on the results, so if you want to make the right action, choose the one with the best results, you can't change the morality of an action without changing what the results of it were. Looking a little more politically, societies supervene on the people. So, if you want to change a society, you need to change the people in it (typically this means convince them of something). And, finally, if the color example worked for you, the color of an object supervenes on the arrangement of the particles on the surface and the light striking it. 

Going back to the point at hand, Non-Reductive Physicalism claims that the mental supervenes on the physical, meaning that there can be no change in the mental without a change in the physical. If you want to see this in action, look at videos of people getting fMRI scans. This is a sort of have your cake and eat it too kind of stance, they get that there is something mental 'up-stairs', but they also get all of the scientific power of Physicalism. This is likely the reason why most philosophers today are Non-Reductive Physicalists. 

Non-Reductive Physicalism gets all of the same answers to the Mind-Body Problem as well as the Causation Problem as the Reductive Physicalist, but it is less committed to such bold claims and it gets various other fun results. 

\section{Current Developments}
Non-Reductive Physicalism and Reductive Physicalism both have the massive support which they do for a reason, but they don't paint the whole picture. If the question of the Mind-Body Problem was just ‘how does the mind interact with the body (and vice versa)?' then the Physicalist, of either form, has an easy answer and would seem right. But assuming Physicalism is right, this leads to another problem. 

\factoidbox{We know enough to know that the world is completely physical. So if the mind exists, it too must be physical. However, it seems hard to understand how certain aspects of mind—notably consciousness—could just be physical features of the brain. How can the complex subjectivity of a conscious experience be produced by the grey matter of the brain?}

This is the biggest question in the Philosophy of Mind at the moment, I know it as the Hard Problem of Consciousness, basically, it's asking 'how do physical things make something as complicated as the mind?' Assuming that everything is physical, where does consciousness come from? This is the end of the road, paths here are still being paved. In the next section of this module, we will explore this problem, and others, in relation to contemporary computer science and artificial intelligence.

There is also a stance, which is an epistemological one, related to this which is called mysterianism which states that the problem of consciousness and the Mind-Body Problem in general is not possible for us to solve. It can also be phrased as saying that we know enough about the world to know that Physicalism is true, but how/why this is the case is beyond the ability of the human brain/mind to answer. Remember that an epistemological question is one which concerns whether or not one can or does know something where as a metaphysical question is about whether or not something is the case. Mysterians hold that Physicalism is true, which is a metaphysical stance, but at the same time say that it is impossible for us to know how that works.

\chapter{Minds, Brains and Programs by John Searle}\autocite{Searle1}
\label{mindsbrainsprograms}
\section{Abstract}
\newcounter{fn}
\setcounter{fn}{\thefootnote}
\setcounter{footnote}{0}

This article can be viewed as an attempt to explore the consequences of two propositions. (1) Intentionality in
human beings (and animals) is a product of causal features of the brain I assume this is an empirical fact about
the actual causal relations between mental processes and brains It says simply that certain brain processes are
sufficient for intentionality. (2) Instantiating a computer program is never by itself a sufficient condition of
intentionality The main argument of this paper is directed at establishing this claim The form of the argument is to
show how a human agent could instantiate the program and still not have the relevant intentionality. These two
propositions have the following consequences (3) The explanation of how the brain produces intentionality
cannot be that it does it by instantiating a computer program. This is a strict logical consequence of 1 and 2. (4)
Any mechanism capable of producing intentionality must have causal powers equal to those of the brain. This is
meant to be a trivial consequence of 1. (5) Any attempt literally to create intentionality artificially (strong AI)
could not succeed just by designing programs but would have to duplicate the causal powers of the human
brain. This follows from 2 and 4.

``Could a machine think?" On the argument advanced here only a machine could think, and only very special
kinds of machines, namely brains and machines with internal causal powers equivalent to those of brains And
that is why strong AI has little to tell us about thinking, since it is not about machines but about programs, and
no program by itself is sufficient for thinking.

\section{Minds, Brains, and Programs}

What psychological and philosophical significance should we attach to recent efforts at computer simulations of
human cognitive capacities? In answering this question, I find it useful to distinguish what I will call ``strong" AI
from ``weak" or ``cautious" AI (Artificial Intelligence). According to weak AI, the principal value of the computer
in the study of the mind is that it gives us a very powerful tool. For example, it enables us to formulate and test
hypotheses in a more rigorous and precise fashion. But according to strong AI, the computer is not merely a
tool in the study of the mind; rather, the appropriately programmed computer really is a mind, in the sense that
computers given the right programs can be literally said to understand and have other cognitive states. In strong
AI, because the programmed computer has cognitive states, the programs are not mere tools that enable us to
test psychological explanations; rather, the programs are themselves the explanations.

I have no objection to the claims of weak AI, at least as far as this article is concerned. My discussion here will
be directed at the claims I have defined as those of strong AI, specifically the claim that the appropriately
programmed computer literally has cognitive states and that the programs thereby explain human cognition.
When I hereafter refer to AI, I have in mind the strong version, as expressed by these two claims.

I will consider the work of Roger Schank and his colleagues at Yale\autocite{Schank1}, because I am
more familiar with it than I am with any other similar claims, and because it provides a very clear example of the
sort of work I wish to examine. But nothing that follows depends upon the details of Schank's programs. The
same arguments would apply to Winograd's SHRDLU\autocite{Winograd1}, Weizenbaum's ELIZA \autocite{Weizen1}, and indeed any Turing machine simulation of human mental phenomena.

Very briefly, and leaving out the various details, one can describe Schank's program as follows: the aim of the
program is to simulate the human ability to understand stories. It is characteristic of human beings' story-
understanding capacity that they can answer questions about the story even though the information that they
give was never explicitly stated in the story. Thus, for example, suppose you are given the following story:

-A man went into a restaurant and ordered a hamburger. When the hamburger arrived it was burned to a crisp,
and the man stormed out of the restaurant angrily, without paying for the hamburger or leaving a tip." Now, if
you are asked -Did the man eat the hamburger?" you will presumably answer, ' No, he did not.' Similarly, if
you are given the following story: '-A man went into a restaurant and ordered a hamburger; when the
hamburger came he was very pleased with it; and as he left the restaurant he gave the waitress a large tip
before paying his bill," and you are asked the question, -Did the man eat the hamburger?,-' you will presumably
answer, -Yes, he ate the hamburger." Now Schank's machines can similarly answer questions about restaurants
in this fashion. To do this, they have a -representation" of the sort of information that human beings have about
restaurants, which enables them to answer such questions as those above, given these sorts of stories. When the
machine is given the story and then asked the question, the machine will print out answers of the sort that we
would expect human beings to give if told similar stories. Partisans of strong AI claim that in this question and
answer sequence the machine is not only simulating a human ability but also
\begin{enumerate}
\item that the machine can literally be said to understand the story and provide the answers to questions, and
\item that what the machine and its program do explains the human ability to understand the story and answer
questions about it.
\end{enumerate}
Both claims seem to me to be totally unsupported by Schank's' work, as I will attempt to show in what follows.
One way to test any theory of the mind is to ask oneself what it would be like if my mind actually worked on
the principles that the theory says all minds work on. Let us apply this test to the Schank program with the
following Gedankenexperiment. Suppose that I'm locked in a room and given a large batch of Chinese writing.
Suppose furthermore (as is indeed the case) that I know no Chinese, either written or spoken, and that I'm not
even confident that I could recognize Chinese writing as Chinese writing distinct from, say, Japanese writing or
meaningless squiggles. To me, Chinese writing is just so many meaningless squiggles.

Now suppose further that after this first batch of Chinese writing I am given a second batch of Chinese script
together with a set of rules for correlating the second batch with the first batch. The rules are in English, and I
understand these rules as well as any other native speaker of English. They enable me to correlate one set of
formal symbols with another set of formal symbols, and all that 'formal' means here is that I can identify the
symbols entirely by their shapes. Now suppose also that I am given a third batch of Chinese symbols together
with some instructions, again in English, that enable me to correlate elements of this third batch with the first two
batches, and these rules instruct me how to give back certain Chinese symbols with certain sorts of shapes in
response to certain sorts of shapes given me in the third batch. Unknown to me, the people who are giving me
all of these symbols call the first batch ``a script," they call the second batch a ``story. ' and they call the third
batch ``questions." Furthermore, they call the symbols I give them back in response to the third batch ``answers
to the questions." and the set of rules in English that they gave me, they call ``the program."

Now just to complicate the story a little, imagine that these people also give me stories in English, which I
understand, and they then ask me questions in English about these stories, and I give them back answers in
English. Suppose also that after a while I get so good at following the instructions for manipulating the Chinese
symbols and the programmers get so good at writing the programs that from the external point of view that is,
from the point of view of somebody outside the room in which I am locked -- my answers to the questions are
absolutely indistinguishable from those of native Chinese speakers. Nobody just looking at my answers can tell
that I don't speak a word of Chinese.

Let us also suppose that my answers to the English questions are, as they no doubt would be, indistinguishable
from those of other native English speakers, for the simple reason that I am a native English speaker. From the
external point of view -- from the point of view of someone reading my ``answers" -- the answers to the
Chinese questions and the English questions are equally good. But in the Chinese case, unlike the English case,
I produce the answers by manipulating uninterpreted formal symbols. As far as the Chinese is concerned, I
simply behave like a computer; I perform computational operations on formally specified elements. For the
purposes of the Chinese, I am simply an instantiation of the computer program.

Now the claims made by strong AI are that the programmed computer understands the stories and that the
program in some sense explains human understanding. But we are now in a position to examine these claims in
light of our thought experiment.

1 As regards the first claim, it seems to me quite obvious in the example that I do not understand a word of the
Chinese stories. I have inputs and outputs that are indistinguishable from those of the native Chinese speaker,
and I can have any formal program you like, but I still understand nothing. For the same reasons, Schank's
computer understands nothing of any stories. whether in Chinese. English. or whatever. since in the Chinese
case the computer is me. and in cases where the computer is not me, the computer has nothing more than I have in the case where I understand nothing.

2. As regards the second claim, that the program explains human understanding, we can see that the computer
and its program do not provide sufficient conditions of understanding since the computer and the program are
functioning, and there is no understanding. But does it even provide a necessary condition or a significant
contribution to understanding? One of the claims made by the supporters of strong AI is that when I understand
a story in English, what I am doing is exactly the same -- or perhaps more of the same -- as what I was doing in
manipulating the Chinese symbols. It is simply more formal symbol manipulation that distinguishes the case in
English, where I do understand, from the case in Chinese, where I don't. I have not demonstrated that this
claim is false, but it would certainly appear an incredible claim in the example. Such plausibility as the claim has
derives from the supposition that we can construct a program that will have the same inputs and outputs as
native speakers, and in addition we assume that speakers have some level of description where they are also
instantiations of a program.

On the basis of these two assumptions we assume that even if Schank's program isn't the whole story about
understanding, it may be part of the story. Well, I suppose that is an empirical possibility, but not the slightest
reason has so far been given to believe that it is true, since what is suggested though certainly not demonstrated
-- by the example is that the computer program is simply irrelevant to my understanding of the story. In the
Chinese case I have everything that artificial intelligence can put into me by way of a program, and I understand
nothing; in the English case I understand everything, and there is so far no reason at all to suppose that my
understanding has anything to do with computer programs, that is, with computational operations on purely
formally specified elements. As long as the program is defined in terms of computational operations on purely
formally defined elements, what the example suggests is that these by themselves have no interesting connection
with understanding. They are certainly not sufficient conditions, and not the slightest reason has been given to
suppose that they are necessary conditions or even that they make a significant contribution to understanding.

Notice that the force of the argument is not simply that different machines can have the same input and output
while operating on different formal principles -- that is not the point at all. Rather, whatever purely formal
principles you put into the computer, they will not be sufficient for understanding, since a human will be able to
follow the formal principles without understanding anything. No reason whatever has been offered to suppose
that such principles are necessary or even contributory, since no reason has been given to suppose that when I
understand English I am operating with any formal program at all.

Well, then, what is it that I have in the case of the English sentences that I do not have in the case of the
Chinese sentences? The obvious answer is that I know what the former mean, while I haven't the faintest idea
what the latter mean. But in what does this consist and why couldn't we give it to a machine, whatever it is? I
will return to this question later, but first I want to continue with the example.

I have had the occasions to present this example to several workers in artificial intelligence, and, interestingly,
they do not seem to agree on what the proper reply to it is. I get a surprising variety of replies, and in what
follows I will consider the most common of these (specified along with their geographic origins).

But first I want to block some common misunderstandings about ``understanding": in many of these discussions
one finds a lot of fancy footwork about the word ``understanding." My critics point out that there are many
different degrees of understanding; that ``understanding" is not a simple two-place predicate; that there are even
different kinds and levels of understanding, and often the law of excluded middle doesn-t even apply in a
straightforward way to statements of the form ``x understands y; that in many cases it is a matter for decision and
not a simple matter of fact whether x understands y; and so on. To all of these points I want to say: of course, of
course. But they have nothing to do with the points at issue. There are clear cases in which ``understanding' literally applies and clear cases in which it does not apply; and these two sorts of cases are all I need for this argument 2 I understand stories in English; to a lesser degree I can understand stories in French; to a still lesser
degree, stories in German; and in Chinese, not at all. My car and my adding machine, on the other hand,
understand nothing: they are not in that line of business. We often attribute ``under standing" and other cognitive
predicates by metaphor and analogy to cars, adding machines, and other artifacts, but nothing is proved by such
attributions. We say, ``The door knows when to open because of its photoelectric cell," ``The adding machine
knows how) (understands how to, is able) to do addition and subtraction but not division," and ``The thermostat
perceives chances in the temperature."

The reason we make these attributions is quite interesting, and it has to do with the fact that in artifacts we
extend our own intentionality;3 our tools are extensions of our purposes, and so we find it natural to make
metaphorical attributions of intentionality to them; but I take it no philosophical ice is cut by such examples. The
sense in which an automatic door ``understands instructions" from its photoelectric cell is not at all the sense in
which I understand English. If the sense in which Schank's programmed computers understand stories is
supposed to be the metaphorical sense in which the door understands, and not the sense in which I understand
English, the issue would not be worth discussing. But Newell and Simon (1963) write that the kind of cognition
they claim for computers is exactly the same as for human beings. I like the straightforwardness of this claim,
and it is the sort of claim I will be considering. I will argue that in the literal sense the programmed computer
understands what the car and the adding machine understand, namely, exactly nothing. The computer
understanding is not just (like my understanding of German) partial or incomplete; it is zero.
Now to the replies:

I. The systems reply (Berkeley). ``While it is true that the individual person who is locked in the room does not
understand the story, the fact is that he is merely part of a whole system, and the system does understand the
story. The person has a large ledger in front of him in which are written the rules, he has a lot of scratch paper
and pencils for doing calculations, he has 'data banks' of sets of Chinese symbols. Now, understanding is not
being ascribed to the mere individual; rather it is being ascribed to this whole system of which he is a part."
My response to the systems theory is quite simple: let the individual internalize all of these elements of the
system. He memorizes the rules in the ledger and the data banks of Chinese symbols, and he does all the
calculations in his head. The individual then incorporates the entire system. There isn't anything at all to the
system that he does not encompass. We can even get rid of the room and suppose he works outdoors. All the
same, he understands nothing of the Chinese, and a fortiori neither does the system, because there isn't anything
in the system that isn't in him. If he doesn't understand, then there is no way the system could understand
because the system is just a part of him.

Actually I feel somewhat embarrassed to give even this answer to the systems theory because the theory seems
to me so implausible to start with. The idea is that while a person doesn't understand Chinese, somehow the
conjunction of that person and bits of paper might understand Chinese. It is not easy for me to imagine how
someone who was not in the grip of an ideology would find the idea at all plausible. Still, I think many people
who are committed to the ideology of strong AI will in the end be inclined to say something very much like this;
so let us pursue it a bit further. According to one version of this view, while the man in the internalized systems
example doesn't understand Chinese in the sense that a native Chinese speaker does (because, for example, he
doesn't know that the story refers to restaurants and hamburgers, etc.), still ``the man as a formal symbol
manipulation system" really does understand Chinese. The subsystem of the man that is the formal symbol
manipulation system for Chinese should not be confused with the subsystem for English.

So there are really two subsystems in the man; one understands English, the other Chinese, and ``it's just that the two systems have little to do with each other." But, I want to reply, not only do they have little to do with each other, they are not even remotely alike. The subsystem that understands English (assuming we allow ourselves
to talk in this jargon of ``subsystems" for a moment) knows that the stories are about restaurants and eating
hamburgers, he knows that he is being asked questions about restaurants and that he is answering questions as
best he can by making various inferences from the content of the story, and so on. But the Chinese system
knows none of this. Whereas the English subsystem knows that ``hamburgers" refers to hamburgers, the Chinese
subsystem knows only that ``squiggle squiggle" is followed by ``squoggle squoggle." All he knows is that various
formal symbols are being introduced at one end and manipulated according to rules written in English, and other
symbols are going out at the other end.

The whole point of the original example was to argue that such symbol manipulation by itself couldn't be
sufficient for understanding Chinese in any literal sense because the man could write ``squoggle squoggle" after
``squiggle squiggle" without understanding anything in Chinese. And it doesn't meet that argument to postulate
subsystems within the man, because the subsystems are no better off than the man was in the first place; they
still don't have anything even remotely like what the English-speaking man (or subsystem) has. Indeed, in the
case as described, the Chinese subsystem is simply a part of the English subsystem, a part that engages in
meaningless symbol manipulation according to rules in English.

Let us ask ourselves what is supposed to motivate the systems reply in the first place; that is, what independent
grounds are there supposed to be for saying that the agent must have a subsystem within him that literally
understands stories in Chinese? As far as I can tell the only grounds are that in the example I have the same
input and output as native Chinese speakers and a program that goes from one to the other. But the whole
point of the examples has been to try to show that that couldn't be sufficient for understanding, in the sense in
which I understand stories in English, because a person, and hence the set of systems that go to make up a
person, could have the right combination of input, output, and program and still not understand anything in the
relevant literal sense in which I understand English.

The only motivation for saying there must be a subsystem in me that understands Chinese is that I have a
program and I can pass the Turing test; I can fool native Chinese speakers. But precisely one of the points at
issue is the adequacy of the Turing test. The example shows that there could be two ``systems," both of which
pass the Turing test, but only one of which understands; and it is no argument against this point to say that since
they both pass the Turing test they must both understand, since this claim fails to meet the argument that the
system in me that understands English has a great deal more than the system that merely processes Chinese. In
short, the systems reply simply begs the question by insisting without argument that the system must understand
Chinese.

Furthermore, the systems reply would appear to lead to consequences that are independently absurd. If we are
to conclude that there must be cognition in me on the grounds that I have a certain sort of input and output and a
program in between, then it looks like all sorts of noncognitive subsystems are going to turn out to be cognitive.
For example, there is a level of description at which my stomach does information processing, and it instantiates
any number of computer programs, but I take it we do not want to say that it has any understanding\autocite{Pylyshyn1}. But if we accept the systems reply, then it is hard to
see how we avoid saying that stomach, heart, liver, and so on, are all understanding subsystems, since there is
no principled way to distinguish the motivation for saying the Chinese subsystem understands from saying that
the stomach understands. It is, by the way, not an answer to this point to say that the Chinese system has
information as input and output and the stomach has food and food products as input and output, since from the
point of view of the agent, from my point of view, there is no information in either the food or the Chinese -- the
Chinese is just so many meaningless squiggles. The information in the Chinese case is solely in the eyes of the
programmers and the interpreters, and there is nothing to prevent them from treating the input and output of my
digestive organs as information if they so desire.

This last point bears on some independent problems in strong AI, and it is worth digressing for a moment to
explain it. If strong AI is to be a branch of psychology, then it must be able to distinguish those systems that are
genuinely mental from those that are not. It must be able to distinguish the principles on which the mind works
from those on which nonmental systems work; otherwise it will offer us no explanations of what is specifically
mental about the mental. And the mental-nonmental distinction cannot be just in the eye of the beholder but it
must be intrinsic to the systems; otherwise it would be up to any beholder to treat people as nonmental and, for
example, hurricanes as mental if he likes. But quite often in the AI literature the distinction is blurred in ways
that would in the long run prove disastrous to the claim that AI is a cognitive inquiry. McCarthy, for example,
writes, '-Machines as simple as thermostats can be said to have beliefs, and having beliefs seems to be a
characteristic of most machines capable of problem solving performance" \autocite{McCarthy1}.

Anyone who thinks strong AI has a chance as a theory of the mind ought to ponder the implications of that
remark. We are asked to accept it as a discovery of strong AI that the hunk of metal on the wall that we use to
regulate the temperature has beliefs in exactly the same sense that we, our spouses, and our children have
beliefs, and furthermore that ``most" of the other machines in the room -- telephone, tape recorder, adding
machine, electric light switch, -- also have beliefs in this literal sense. It is not the aim of this article to argue
against McCarthy's point, so I will simply assert the following without argument. The study of the mind starts
with such facts as that humans have beliefs, while thermostats, telephones, and adding machines don't. If you get
a theory that denies this point you have produced a counterexample to the theory and the theory is false.

One gets the impression that people in AI who write this sort of thing think they can get away with it because
they don't really take it seriously, and they don't think anyone else will either. I propose for a moment at least, to
take it seriously. Think hard for one minute about what would be necessary to establish that that hunk of metal
on the wall over there had real beliefs beliefs with direction of fit, propositional content, and conditions of
satisfaction; beliefs that had the possibility of being strong beliefs or weak beliefs; nervous, anxious, or secure
beliefs; dogmatic, rational, or superstitious beliefs; blind faiths or hesitant cogitations; any kind of beliefs. The
thermostat is not a candidate. Neither is stomach, liver adding machine, or telephone. However, since we are
taking the idea seriously, notice that its truth would be fatal to strong AI's claim to be a science of the mind. For
now the mind is everywhere. What we wanted to know is what distinguishes the mind from thermostats and
livers. And if McCarthy were right, strong AI wouldn't have a hope of telling us that.

II. The Robot Reply (Yale). ``Suppose we wrote a different kind of program from Schank's program. Suppose
we put a computer inside a robot, and this computer would not just take in formal symbols as input and give out
formal symbols as output, but rather would actually operate the robot in such a way that the robot does
something very much like perceiving, walking, moving about, hammering nails, eating drinking -- anything you
like. The robot would, for example have a television camera attached to it that enabled it to 'see,' it would have
arms and legs that enabled it to 'act,' and all of this would be controlled by its computer 'brain.' Such a robot
would, unlike Schank's computer, have genuine understanding and other mental states."

The first thing to notice about the robot reply is that it tacitly concedes that cognition is not solely a matter of
formal symbol manipulation, since this reply adds a set of causal relation with the outside world \autocite{Fodor1}. 
But the answer to the robot reply is that the addition of such
``perceptual" and ``motor" capacities adds nothing by way of understanding, in particular, or intentionality, in
general, to Schank's original program. To see this, notice that the same thought experiment applies to the robot
case. Suppose that instead of the computer inside the robot, you put me inside the room and, as in the original
Chinese case, you give me more Chinese symbols with more instructions in English for matching Chinese
symbols to Chinese symbols and feeding back Chinese symbols to the outside. Suppose, unknown to me, some
of the Chinese symbols that come to me come from a television camera attached to the robot and other Chinese symbols that I am giving out serve to make the motors inside the robot move the robot's legs or arms.
It is important to emphasize that all I am doing is manipulating formal symbols: I know none of these other facts.
I am receiving ``information" from the robot's ``perceptual" apparatus, and I am giving out ``instructions" to its
motor apparatus without knowing either of these facts. I am the robot's homunculus, but unlike the traditional
homunculus, I don't know what's going on. I don't understand anything except the rules for symbol
manipulation. Now in this case I want to say that the robot has no intentional states at all; it is simply moving
about as a result of its electrical wiring and its program. And furthermore, by instantiating the program I have no
intentional states of the relevant type. All I do is follow formal instructions about manipulating formal symbols.

III. The brain simulator reply (Berkeley and M.I.T.). ``Suppose we design a program that doesn't represent
information that we have about the world, such as the information in Schank's scripts, but simulates the actual
sequence of neuron firings at the synapses of the brain of a native Chinese speaker when he understands stories
in Chinese and gives answers to them. The machine takes in Chinese stories and questions about them as input,
it simulates the formal l structure of actual Chinese brains in processing these stories, and it gives out Chinese
answers as outputs. We can even imagine that the machine operates, not with a single serial program, but with a
whole set of programs operating in parallel, in the manner that actual human brains presumably operate when
they process natural language. Now surely in such a case we would have to say that the machine understood
the stories; and if we refuse to say that, wouldn't we also have to deny that native Chinese speakers understood
the stories? At the level of the synapses, what would or could be different about the program of the computer
and the program of the Chinese brain?"

Before countering this reply I want to digress to note that it is an odd reply for any partisan of artificial
intelligence (or functionalism, etc.) to make: I thought the whole idea of strong AI is that we don't need to know
how the brain works to know how the mind works. The basic hypothesis, or so I had supposed, was that there
is a level of mental operations consisting of computational processes over formal elements that constitute the
essence of the mental and can be realized in all sorts of different brain processes, in the same way that any
computer program can be realized in different computer hardwares: on the assumptions of strong AI, the mind
is to the brain as the program is to the hardware, and thus we can understand the mind without doing
neurophysiology. If we had to know how the brain worked to do AI, we wouldn't bother with AI. However,
even getting this close to the operation of the brain is still not sufficient to produce understanding. To see this,
imagine that instead of a mono lingual man in a room shuffling symbols we have the man operate an elaborate
set of water pipes with valves connecting them. When the man receives the Chinese symbols, he looks up in the
program, written in English, which valves he has to turn on and off. Each water connection corresponds to a
synapse in the Chinese brain, and the whole system is rigged up so that after doing all the right firings, that is
after turning on all the right faucets, the Chinese answers pop out at the output end of the series of pipes.

Now where is the understanding in this system? It takes Chinese as input, it simulates the formal structure of
the synapses of the Chinese brain, and it gives Chinese as output. But the man certainly doesn-t understand
Chinese, and neither do the water pipes, and if we are tempted to adopt what I think is the absurd view that
somehow the conjunction of man and water pipes understands, remember that in principle the man can
internalize the formal structure of the water pipes and do all the ``neuron firings" in his imagination. The problem
with the brain simulator is that it is simulating the wrong things about the brain. As long as it simulates only the
formal structure of the sequence of neuron firings at the synapses, it won't have simulated what matters about
the brain, namely its causal properties, its ability to produce intentional states. And that the formal properties are
not sufficient for the causal properties is shown by the water pipe example: we can have all the formal
properties carved off from the relevant neurobiological causal properties.

IV. The combination reply (Berkeley and Stanford). 'While each of the previous three replies might not be
completely convincing by itself as a refutation of the Chinese room counterexample, if you take all three together they are collectively much more convincing and even decisive. Imagine a robot with a brain-shaped computer lodged in its cranial cavity, imagine the computer programmed with all the synapses of a human brain, imagine
the whole behavior of the robot is indistinguishable from human behavior, and now think of the whole thing as a
unified system and not just as a computer with inputs and outputs. Surely in such a case we would have to
ascribe intentionality to the system.

I entirely agree that in such a case we would find it rational and indeed irresistible to accept the hypothesis that
the robot had intentionality, as long as we knew nothing more about it. Indeed, besides appearance and
behavior, the other elements of the combination are really irrelevant. If we could build a robot whose behavior
was indistinguishable over a large range from human behavior, we would attribute intentionality to it, pending
some reason not to. We wouldn't need to know in advance that its computer brain was a formal analogue of the
human brain.

But I really don't see that this is any help to the claims of strong AI; and here-s why: According to strong AI,
instantiating a formal program with the right input and output is a sufficient condition of, indeed is constitutive of,
intentionality. As Newell\autocite{Newell1} puts it, the essence of the mental is the operation of a physical symbol system.
But the attributions of intentionality that we make to the robot in this example have nothing to do with formal
programs. They are simply based on the assumption that if the robot looks and behaves sufficiently like us, then
we would suppose, until proven otherwise, that it must have mental states like ours that cause and are
expressed by its behavior and it must have an inner mechanism capable of producing such mental states. If we
knew independently how to account for its behavior without such assumptions we would not attribute
intentionality to it especially if we knew it had a formal program. And this is precisely the point of my earlier
reply to objection 11.

Suppose we knew that the robot's behavior was entirely accounted for by the fact that a man inside it was
receiving uninterpreted formal symbols from the robot's sensory receptors and sending out uninterpreted formal
symbols to its motor mechanisms, and the man was doing this symbol manipulation in accordance with a bunch
of rules. Furthermore, suppose the man knows none of these facts about the robot, all he knows is which
operations to perform on which meaningless symbols. In such a case we would regard the robot as an ingenious
mechanical dummy. The hypothesis that the dummy has a mind would now be unwarranted and unnecessary,
for there is now no longer any reason to ascribe intentionality to the robot or to the system of which it is a part
(except of course for the man's intentionality in manipulating the symbols). The formal symbol manipulations go
on, the input and output are correctly matched, but the only real locus of intentionality is the man, and he doesn't
know any of the relevant intentional states; he doesn't, for example, see what comes into the robot's eyes, he
doesn't intend to move the robot's arm, and he doesn't understand any of the remarks made to or by the robot.
Nor, for the reasons stated earlier, does the system of which man and robot are a part.

To see this point, contrast this case with cases in which we find it completely natural to ascribe intentionality to
members of certain other primate species such as apes and monkeys and to domestic animals such as dogs. The
reasons we find it natural are, roughly, two: we can't make sense of the animal's behavior without the ascription
of intentionality and we can see that the beasts are made of similar stuff to ourselves -- that is an eye, that a
nose, this is its skin, and so on. Given the coherence of the animal's behavior and the assumption of the same
causal stuff underlying it, we assume both that the animal must have mental states underlying its behavior, and
that the mental states must be produced by mechanisms made out of the stuff that is like our stuff. We would
certainly make similar assumptions about the robot unless we had some reason not to, but as soon as we knew
that the behavior was the result of a formal program, and that the actual causal properties of the physical
substance were irrelevant we would abandon the assumption of intentionality.\footnote{\citetitle[See][]{Stich1}}

There are two other responses to my example that come up frequently (and so are worth discussing) but really
miss the point.

V. The other minds reply (Yale). ``How do you know that other people understand Chinese or anything else?
Only by their behavior. Now the computer can pass the behavioral tests as well as they can (in principle), so if
you are going to attribute cognition to other people you must in principle also attribute it to computers. '
This objection really is only worth a short reply. The problem in this discussion is not about how I know that
other people have cognitive states, but rather what it is that I am attributing to them when I attribute cognitive
states to them. The thrust of the argument is that it couldn't be just computational processes and their output
because the computational processes and their output can exist without the cognitive state. It is no answer to
this argument to feign anesthesia. In 'cognitive sciences" one presupposes the reality and knowability of the
mental in the same way that in physical sciences one has to presuppose the reality and knowability of physical
objects.

VI. The many mansions reply (Berkeley). ``Your whole argument presupposes that AI is only about analogue
and digital computers. But that just happens to be the present state of technology. Whatever these causal
processes are that you say are essential for intentionality (assuming you are right), eventually we will be able to
build devices that have these causal processes, and that will be artificial intelligence. So your arguments are in
no way directed at the ability of artificial intelligence to produce and explain cognition."
I really have no objection to this reply save to say that it in effect trivializes the project of strong AI by
redefining it as whatever artificially produces and explains cognition. The interest of the original claim made on
behalf of artificial intelligence is that it was a precise, well defined thesis: mental processes are computational
processes over formally defined elements. I have been concerned to challenge that thesis. If the claim is
redefined so that it is no longer that thesis, my objections no longer apply because there is no longer a testable
hypothesis for them to apply to.

Let us now return to the question I promised I would try to answer: granted that in my original example I
understand the English and I do not understand the Chinese, and granted therefore that the machine doesn't
understand either English or Chinese, still there must be something about me that makes it the case that I
understand English and a corresponding something lacking in me that makes it the case that I fail to understand
Chinese. Now why couldn't we give those somethings, whatever they are, to a machine?

I see no reason in principle why we couldn't give a machine the capacity to understand English or Chinese,
since in an important sense our bodies with our brains are precisely such machines. But I do see very strong
arguments for saying that we could not give such a thing to a machine where the operation of the machine is
defined solely in terms of computational processes over formally defined elements; that is, where the operation
of the machine is defined as an instantiation of a computer program. It is not because I am the instantiation of a
computer program that I am able to understand English and have other forms of intentionality (I am, I suppose,
the instantiation of any number of computer programs), but as far as we know it is because I am a certain sort
of organism with a certain biological (i.e. chemical and physical) structure, and this structure, under certain
conditions, is causally capable of producing perception, action, understanding, learning, and other intentional
phenomena. And part of the point of the present argument is that only something that had those causal powers
could have that intentionality. Perhaps other physical and chemical processes could produce exactly these
effects; perhaps, for example, Martians also have intentionality but their brains are made of different stuff. That
is an empirical question, rather like the question whether photosynthesis can be done by something with a
chemistry different from that of chlorophyll.

But the main point of the present argument is that no purely formal model will ever be sufficient by itself for
intentionality because the formal properties are not by themselves constitutive of intentionality, and they have by
themselves no causal powers except the power, when instantiated, to produce the next stage of the formalism
when the machine is running. And any other causal properties that particular realizations of the formal model
have, are irrelevant to the formal model because we can always put the same formal model in a different
realization where those causal properties are obviously absent. Even if, by some miracle Chinese speakers
exactly realize Schank's program, we can put the same program in English speakers, water pipes, or
computers, none of which understand Chinese, the program notwithstanding.

What matters about brain operations is not the formal shadow cast by the sequence of synapses but rather the
actual properties of the sequences. All the arguments for the strong version of artificial intelligence that I have
seen insist on drawing an outline around the shadows cast by cognition and then claiming that the shadows are
the real thing. By way of concluding I want to try to state some of the general philosophical points implicit in the
argument. For clarity I will try to do it in a question and answer fashion, and I begin with that old chestnut of a
question:

``Could a machine think?"

The answer is, obviously, yes. We are precisely such machines.

``Yes, but could an artifact, a man-made machine think?"

Assuming it is possible to produce artificially a machine with a nervous system, neurons with axons and
dendrites, and all the rest of it, sufficiently like ours, again the answer to the question seems to be obviously, yes.
If you can exactly duplicate the causes, you could duplicate the effects. And indeed it might be possible to
produce consciousness, intentionality, and all the rest of it using some other sorts of chemical principles than
those that human beings use. It is, as I said, an empirical question. ``OK, but could a digital computer think?"
If by ``digital computer" we mean anything at all that has a level of description where it can correctly be
described as the instantiation of a computer program, then again the answer is, of course, yes, since we are the
instantiations of any number of computer programs, and we can think.

``But could something think, understand, and so on solely in virtue of being a computer with the right sort of
program? Could instantiating a program, the right program of course, by itself be a sufficient condition of
understanding?"

This I think is the right question to ask, though it is usually confused with one or more of the earlier questions,
and the answer to it is no.

``Why not?"

Because the formal symbol manipulations by themselves don't have any intentionality; they are quite
meaningless; they aren't even symbol manipulations, since the symbols don't symbolize anything. In the linguistic
jargon, they have only a syntax but no semantics. Such intentionality as computers appear to have is solely in
the minds of those who program them and those who use them, those who send in the input and those who
interpret the output.

The aim of the Chinese room example was to try to show this by showing that as soon as we put something into the system that really does have intentionality (a man), and we program him with the formal program, you
can see that the formal program carries no additional intentionality. It adds nothing, for example, to a man's
ability to understand Chinese.

Precisely that feature of AI that seemed so appealing -- the distinction between the program and the realization
-- proves fatal to the claim that simulation could be duplication. The distinction between the program and its
realization in the hardware seems to be parallel to the distinction between the level of mental operations and the
level of brain operations. And if we could describe the level of mental operations as a formal program, then it
seems we could describe what was essential about the mind without doing either introspective psychology or
neurophysiology of the brain. But the equation, ``mind is to brain as program is to hardware" breaks down at
several points among them the following three:

First, the distinction between program and realization has the consequence that the same program could have
all sorts of crazy realizations that had no form of intentionality. Weizenbaum (1976, Ch. 2), for example, shows
in detail how to construct a computer using a roll of toilet paper and a pile of small stones. Similarly, the Chinese
story understanding program can be programmed into a sequence of water pipes, a set of wind machines, or a
monolingual English speaker, none of which thereby acquires an understanding of Chinese. Stones, toilet paper,
wind, and water pipes are the wrong kind of stuff to have intentionality in the first place -- only something that
has the same causal powers as brains can have intentionality -- and though the English speaker has the right
kind of stuff for intentionality you can easily see that he doesn't get any extra intentionality by memorizing the
program, since memorizing it won't teach him Chinese.

Second, the program is purely formal, but the intentional states are not in that way formal. They are defined in
terms of their content, not their form. The belief that it is raining, for example, is not defined as a certain formal
shape, but as a certain mental content with conditions of satisfaction, a direction of fit\footnote{\citeauthor[See][]{Searle2}}, and the
like. Indeed the belief as such hasn't even got a formal shape in this syntactic sense, since one and the same
belief can be given an indefinite number of different syntactic expressions in different linguistic systems.

Third, as I mentioned before, mental states and events are literally a product of the operation of the brain, but
the program is not in that way a product of the computer.

-Well if programs are in no way constitutive of mental processes, why have so many people believed the
converse? That at least needs some explanation."

I don't really know the answer to that one. The idea that computer simulations could be the real thing ought to
have seemed suspicious in the first place because the computer isn't confined to simulating mental operations, by
any means. No one supposes that computer simulations of a five-alarm fire will burn the neighborhood down or
that a computer simulation of a rainstorm will leave us all drenched. Why on earth would anyone suppose that a
computer simulation of understanding actually understood anything? It is sometimes said that it would be
frightfully hard to get computers to feel pain or fall in love, but love and pain are neither harder nor easier than
cognition or anything else. For simulation, all you need is the right input and output and a program in the middle
that transforms the former into the latter. That is all the computer has for anything it does. To confuse simulation
with duplication is the same mistake, whether it is pain, love, cognition, fires, or rainstorms.

Still, there are several reasons why AI must have seemed and to many people perhaps still does seem -- in
some way to reproduce and thereby explain mental phenomena, and I believe.we will not succeed in removing
these illusions until we have fully exposed the reasons that give rise to them.

First, and perhaps most important, is a confusion about the notion of information processing: many people in
cognitive science believe that the human brain, with its mind, does something called -information processing,"
and analogously the computer with its program does information processing; but fires and rainstorms, on the
other hand, don't do information processing at all. Thus, though the computer can simulate the formal features
of any process whatever, it stands in a special relation to the mind and brain because when the computer is
properly programmed, ideally with the same program as the brain, the information processing is identical in the
two cases, and this information processing is really the essence of the mental.

But the trouble with this argument is that it rests on an ambiguity in the notion of '- information." In the sense in
which people ``process information" when they reflect, say, on problems in arithmetic or when they read and
answer questions about stories, the programmed computer does not do -information processing." Rather, what
it does is manipulate formal symbols. The fact that the programmer and the interpreter of the computer output
use the symbols to stand for objects in the world is totally beyond the scope of the computer. The computer, to
repeat, has a syntax but no semantics. Thus, if you type into the computer '2 plus 2 equals?' it will type out '-4.'
But it has no idea that -4" means 4 or that it means anything at all. And the point is not that it lacks some
second-order information about the interpretation of its first- order symbols, but rather that its first-order
symbols don't have any interpretations as far as the computer is concerned. All the computer has is more
symbols.

The introduction of the notion of ``information processing" therefore produces a dilemma: either we construe the
notion of ``information processing" in such a way that it implies intentionality as part of the process or we don't. If
the former, then the programmed computer does not do information processing, it only manipulates formal
symbols. If the latter, then, though the computer does information processing, it is only doing so in the sense in
which adding machines, typewriters, stomachs, thermostats, rainstorms, and hurricanes do information
processing; namely, they have a level of description at which we can describe them as taking information in at
one end, transforming it, and producing information as output. But in this case it is up to outside observers to
interpret the input and output as information in the ordinary sense. And no similarity is established between the
computer and the brain in terms of any similarity of information processing.

Second, in much of AI there is a residual behaviorism or operationalism. Since appropriately programmed
computers can have input-output patterns similar to those of human beings, we are tempted to postulate mental
states in the computer similar to human mental states. But once we see that it is both conceptually and
empirically possible for a system to have human capacities in some realm without having any intentionality at all,
we should be able to overcome this impulse. My desk adding machine has calculating capacities, but no
intentionality, and in this paper I have tried to show that a system could have input and output capabilities that
duplicated those of a native Chinese speaker and still not understand Chinese, regardless of how it was
programmed. The Turing test is typical of the tradition in being unashamedly behavioristic and operationalistic,
and I believe that if AI workers totally repudiated behaviorism and operationalism much of the confusion
between simulation and duplication would be eliminated.

Third, this residual operationalism is joined to a residual form of dualism; indeed strong AI only makes sense
given the dualistic assumption that, where the mind is concerned, the brain doesn't matter. In strong AI (and in
functionalism, as well) what matters are programs, and programs are independent of their realization in
machines; indeed, as far as AI is concerned, the same program could be realized by an electronic machine, a
Cartesian mental substance, or a Hegelian world spirit. The single most surprising discovery that I have made in
discussing these issues is that many AI workers are quite shocked by my idea that actual human mental
phenomena might be dependent on actual physical/chemical properties of actual human brains.

But if you think about it a minute you can see that I should not have been surprised; for unless you accept some form of dualism, the strong AI project hasn't got a chance. The project is to reproduce and explain the mental
by designing programs, but unless the mind is not only conceptually but empirically independent of the brain you
couldn't carry out the project, for the program is completely independent of any realization. Unless you believe
that the mind is separable from the brain both conceptually and empirically -- dualism in a strong form -- you
cannot hope to reproduce the mental by writing and running programs since programs must be independent of
brains or any other particular forms of instantiation. If mental operations consist in computational operations on
formal symbols, then it follows that they have no interesting connection with the brain; the only connection
would be that the brain just happens to be one of the indefinitely many types of machines capable of instantiating
the program.

This form of dualism is not the traditional Cartesian variety that claims there are two sorts of substances, but it
is Cartesian in the sense that it insists that what is specifically mental about the mind has no intrinsic connection
with the actual properties of the brain. This underlying dualism is masked from us by the fact that AI literature
contains frequent fulminations against ``dualism'-; what the authors seem to be unaware of is that their position
presupposes a strong version of dualism.

``Could a machine think?" My own view is that only a machine could think, and indeed only very special kinds
of machines, namely brains and machines that had the same causal powers as brains. And that is the main
reason strong AI has had little to tell us about thinking, since it has nothing to tell us about machines. By its own
definition, it is about programs, and programs are not machines. Whatever else intentionality is, it is a biological
phenomenon, and it is as likely to be as causally dependent on the specific biochemistry of its origins as
lactation, photosynthesis, or any other biological phenomena. No one would suppose that we could produce
milk and sugar by running a computer simulation of the formal sequences in lactation and photosynthesis, but
where the mind is concerned many people are willing to believe in such a miracle because of a deep and abiding
dualism: the mind they suppose is a matter of formal processes and is independent of quite specific material
causes in the way that milk and sugar are not.

In defense of this dualism the hope is often expressed that the brain is a digital computer (early computers, by
the way, were often called ``electronic brains"). But that is no help. Of course the brain is a digital computer.
Since everything is a digital computer, brains are too. The point is that the brain's causal capacity to produce
intentionality cannot consist in its instantiating a computer program, since for any program you like it is possible
for something to instantiate that program and still not have any mental states. Whatever it is that the brain does
to produce intentionality, it cannot consist in instantiating a program since no program, by itself, is sufficient for
intentionality.

\setcounter{footnote}{\thefn}
\chapter{Part 5: Can Machines Think?}
\section{Artificial Intelligence and the Mind-Body Problem}
We encounter, in our modern lives, Artificial Intelligence (AI) more often than we would think. The spell-checker on our word-processors, Facebook's advertisements, driving directions on our phones/computers, some baby toys, speech-recognition software, and many other things all have AI built into them. They are made to make the machines intuitive to us and useful. These are examples of what is called 'weak AI' (a more precise definition coming later). When we think of AI, our minds are often plagued by thoughts of I, Robot, Star Trek, and other Sci-Fi stories. But, in the real world, can a machine, just lights and clockwork, have a mind? Can a computer be conscious? Can something like that really understand and really learn? Can there be a ghost in the machine?.

Those questions are all different versions of a seemingly simple question "can there ever be strong AI?" (again, more precise definition is later on). The philosopher John Searle in his paper Minds, Brains, and Programs was trying to answer just that. Is it possible for a machine to think?

In our exploration of the Philosophy of Artificial Intelligence\autocite{SEPAI}. as it relates to the Mind-Body Problem, we will mostly be looking into the argument made by Searle in that paper and the replies to it, but I will be including examples to make it more relevant to today (as it was written in the 80s) and references to related more recent works.

\section{Weak Vs Strong Artificial Intelligence}

Searle starts us off by going down the usual path for philosophy, making distinctions. We don't want to confuse simple forms of AI, like the kind found in a cell-phone, with complex forms, like the kind found in Data from Star Trek. Otherwise, if we were to think that all kinds of AI are the same, we would think that iPhones are iPeople. The kind of AI found in your phone and a lot of other computer-devices (including the one you are reading this on) is what we will call 'Weak AI'. This is:

\factoidbox{A form of machine intelligence, focused on a small task or a narrow range of (interconnected) tasks. This is also called 'narrow AI'. The principle value or purpose of weak AI is to solve problems in a methodological and precise way which humans either don't have the brain-power or the time to do ourselves.}

Weak AI simulates a person's thinking, typically in an ideal way, to get the best results. AI machines will not have the same irreliable aspects as a human's mind. For example, a suitably robust AI will not experience emotional fatigue which could result in a bias or missed factor, it will not (unless the programmer built it in) experience the cognitive biases which we have seen in the past, and it will not experience becoming tired of doing the same task over and over again. For example, imagine that you are driving in an area which you are not familiar with and are using your GPS (in a phone or some other device) and you take a wrong turn. The GPS receives the data about your current position and from the algorithms figures out that you are not on the route. From this, it runs other algorithms to generate the fastest route to the destination given your current position. In a very short period of time, it generates the new route and the instructions for you to get to the destination. Imagine having your friend in the passenger's seat (riding shotgun) and having them be your GPS. The speed and accuracy of their directions would be nothing compared to that of the weak AI in the GPS.

Your computer or phone or what have you likely has many different compartmentalized weak AI programs in it, each activated for a particular task. The GPS AI is most likely not the same as the predictive-text AI. Human persons (a concept we will encounter in Module 9) aren't like that. Our intelligence is more holistic. The 'mind' which figures out your math homework is the same one which imagined the above example. This is where we get to Strong AI. This is:

\factoidbox{A form of machine intelligence which is not focused on a small task or on a narrow range of tasks, but can handle just about any form of task which is thrown its way. Rather than merely simulating a person's thinking, the AI is, in fact, thinking. The principle value here is the same as that of any conscious human. Strong AI machines have minds.}

Science Fiction is full of what we can use as examples for Strong AI. I have already used Data as an example, but also we have R2D2 from Star Wars, Sonny from i-Robot, Wall-E (from the movie with the same name),  The Terminator (and many others if we include the plots of video games, like Cortana in Halo).  Searle has no problems with weak AI and, looking at the progress of technology, he was likely correct not to have any metaphysical qualms with it (ethical qualms are another story).  But he has problems with the notion of strong AI. He does not think that strong AI is possible.

\section{Roger Schank's AI}
Searle uses Roger Schank's AI as an example, because it's the one he's most familiar with. It should be noted that this AI is not special, but rather the basic way it works is found in AI even today, as I will explain in a moment. Schank's goal with this machine was to simulate how a person interprets a story when they're not given complete information. For example, take this story/question:

    \factoidbox{A man walks into a restaurant and orders a burger. It comes to him burnt to a crisp and  he storms out angrily without paying or leaving a tip. Did the man eat the burger?}

An answer to this question is not given in the story; it's not like you were asked "did he order a salad?" The answer is not spoon-fed to the machine or to us, rather we need to make logical conclusions given background information. In real life, many of the questions which we encounter do not contain all of the information which are necessary to answer them accurately. Most of the time, especially in the 'real world', we are going to need to make certain jumps in our reasoning based on background information. Such as our knowledge of normal human behavior.

More than likely, you answered the above question with something like "no, he didn't". The AI also generates the answer "no". This is because the AI in the machine was programmed or trained with cases involving reasonable human behavior and made predictions based on those assumptions. Next, take the following story (starting the same way):

    \factoidbox{A man walks into a restaurant and orders a burger. When it arrives, he is pleased with it and leaves a large tip before paying his bill. Did he eat the burger?}

Yet again, the answer is not spoon-fed to the machine or to you. Rather, we need to use some background information. While it is possible that the man did eat the burger in the first case and did not in the second, this is far from likely. You probably answered the second question with "yes, he did". Similarly, the AI also says "yes". In this case, too, we are relying on our training and experience in dealing with real world situations, our past. If, for example, we had trained the machine or a baby with only experiences which were the opposite of the normal, then the baby and the machine would likely generate the opposite responses to us with our normal upbringing. 

But, how do AI machines do this? Well, the bedrock level methodology has not changed much in the years, only getting faster and having larger data-bases for the relevant cases. The heart of it is a decision engine. This is some system of algorithms or if-then-else style statements which take in some input and produce some output. For example, we could have something like this going on:

    \factoidbox{If a person did not leave a tip, assume that they did not like the food. If a person did not like the food, assume that they didn't eat all of it.}

The current rage in computer science involves using what are called neural networks\footnote{They are called neural networks because the connections between stimulus and response resemble the connections between neurons in the brain and they are strengthened in much the same way (by repeated exposure).}  and machine learning (also called 'learning algorithms'). In this case, the decision engine is generated by the machine itself. The AI starts off with a large data-base of cases along with an 'answer-key' of sorts. The easiest example would be a large number of pictures of hand-written numbers. The machine scans the picture and then makes a guess using some previously given (likely by the programmer) algorithm. Often, this is based on the contrasting pixels of the image and arrangements of similarly colored pixels. If it generates the correct number, great and it moves on to the next. If, on the other hand, the machine gets the wrong number, it adjusts the amount of 'weight' it gives some factor in the image (in this case the pixels) until it generates the correct answer. Doing this millions of times with millions upon millions of examples creates a decision engine which can accurately predict the answer for cases like the ones which it has been programmed to handle. 

My case above using hand-written numbers is very oversimplified. For a more detailed account of what is really going on, the YouTuber 3Blue1Brown has several videos explaining this.\autocite{BlueBrown}  This methodology for machine learning and creating AI is not limited to cases like hand-written numbers and stories, rather it's found all over the place.

One shocking example of this in the real world is the case of Ashok Goel's Jill Watson\autocite{JillWatson1}. Jill is an AI built to be a Teaching Assistant for Georgia Tech's MASSIVE online courses on programming artificial intelligence (meta, I know). Coming from experience, for massive courses like these, the Teaching Assistant is often bogged down by answering the same question hundreds of times a day. Though the questions are phrased differently (like how the numbers in the hand-written cases look different), they all boil down to the same answer.  So, the professors at Georgia Tech collected a data-base of questions, sorted them by type and gave Jill an answer key. After successfully generating correct responses for the initial data-base, Jill was given real-time questions being submitted in a real class (but not actually able to reply, she replied in a mirror-forum) and then graded by a real person on her responses. Once she had a 97\% success rate (which is higher than some people I know), they let her loose into a real classroom forum. The remarkable thing is that very few people figured out that Jill was an AI. And even the ones who did, only did so because the class was on AI and they were already skeptical. In fact, the professor for the course needed to tell them that she was a machine. In the latest version of Jill which I know of, none were able to identify her as an AI.

\section{The Chinese Room Thought Experiment}

Given Jill Watson, hand-writing recognition, and Roger Schank's AI, we have some basic knowledge of how AI machines make decisions and generate their answers, which is enough to get the point of this. The core commonality across all AI programs is a decision engine. How that engine is generated is not all that important (though using the machine learning and neural network systems seems to be the most efficient and most accurate), but what is important is that there's this engine. All Searle needs to make his argument work is that the machine takes input, runs it through an engine like Jill's, and then spits out an output. 
\factoidbox{Imagine a native English speaker who knows no Chinese locked in a room full of boxes of Chinese symbols (a data base) together with a book of instructions for manipulating the symbols (the program). Imagine that people outside the room send in other Chinese symbols which, unknown to the person in the room, are questions in Chinese (the input). And imagine that by following the instructions in the program the man in the room is able to pass out Chinese symbols which are correct answers to the questions (the output). The program enables the person in the room to pass the Turing Test for understanding Chinese but he does not understand a word of Chinese.}

To help tie this together, in the case of Jill, the boxes of Chinese symbols is her data base of different answers. Similarly, in the case of Schank's AI, the possible responses to questions about stories are the Chinese symbols. The instruction manual is the core thing which we need to look into, this is the decision engine or sorting algorithm used by the computer to generate the answers. This can be written by a programmer with way too much time on their hands or through the kind of machine learning and neural networks I described before, that really doesn't matter. But what really matters is that all the machine is doing is, basically, crunching the numbers, running it through a bunch of 'if-then-else' style sorters to generate the answer.

If I were to lock you in the room with symbols/words not of your native tongue (one that you don't know), you would essentially be doing what the machine is doing, looking at the input, running through the instruction manual, and then giving the output. There would be no real understanding going on, no real learning in the process. If, on the other hand, I put you in a room full of symbols from your native tongue, there would be something more going on, something extra. There's actually interpretation happening. I don't know a lick of Chinese, but I do know English, Latin, and the basics of a few other languages. In the case of Chinese, I would be just spitting out uninterpreted symbols. In the case of English or Latin, I would be interpreting the symbols, there would be intention or thoughts behind the answers given.

\section{Strong AI Claims and Replies}

Strong AI Claims and Replies

Generalizing off of the Chinese Room Thought Experiment, the proponent of Strong AI would make two claims:
\begin{enumerate}
    \item The appropriately programmed AI (in this case the entirety of the Chinese Room) can be said, truthfully, to literally understand the applied case (in this case, the meaning of the Chinese symbols). 
    \item The machine and its programs explain the human ability to understand and reply the way we do in such cases (in this case, linguistic comprehension).
\end{enumerate}
Searle thinks that neither the evidence nor the way in which even the most broad AI would function support these claims. Here is is reasoning:

In regards to the first claim, that the machine literally understands the applied case, it seems clear that the person in the Chinese Room is the same as the computer (or what have you) running the program. Though the person might generate answers indistinguishable from those generated by a native Chinese speaker, thereby be able to pass the Turing Test, there still would not be any real understanding, like we would have if an English speaker was put in the room is English symbols. This gives us the first argument:
\begin{earg}
        \item[]If strong AI is possible, then the mere manipulation of symbols in a language would be enough to understand that language.
        \item[]The mere manipulation of symbols in a language is not enough to understand that language.
        \item[]Therefore, strong AI is not possible.
\end{earg}
For the second claim, that the machine explains the human ability to understand and reply in the way we do, Searle thinks that the programs described do not provide the sufficient conditions for understanding (if I am running the program, then I am understanding). This is because of core way in which they operate, formal symbol manipulation. It's possible to run the program without understanding (in the case of the Chinese Room).  On the other hand, do the programs provide a necessary condition for understanding (if I am understanding, then I am running the program)?

The person who thinks strong AI is possible might go down that route, claiming that when I understand a story or dialogue in English, I am doing symbol manipulation, just of a far more complex and intricate kind. Searle is smart to point out that he did not show that this is false ( claiming only that the machine does not give the sufficient condition for understanding). But, he does go a bit further, claiming that this is a truly incredible claim (an unbelievable claim). The plausibility of this claim rests on two claims. First, it is, in fact, possible to make a program indistinguishable from a native speaker of a language. Second, human persons are, at some level of description, programs. If you deny either of those, then you can't think that strong AI is possible.

In the case of the first claim, that it's possible for a machine to be indistinguishable from a native speaker, the jury is still out on that. All of the examples which I have encountered (and I will update this if ever I encounter a case otherwise) of a machine managing to trick a person into believing that it was a person, they were very sophisticated machines with a lot of exposure to various texts and responses but they were programmed to simulate a non-native speaker to play on our sympathies and make us hold them to a different standard than we would a person who was clearly fluent in English and not prone to make simple errors. In order to prove that it is possible, it would take an advancement in computer science which we are currently awaiting, if it is possible at all. 

In regards to the second claim, if one is a physicalist, then we could be willing to accept that human persons are on some level machines and functioning according to our programming (more on this in the Free Will Debate). However, there is a massive gap between the explanation in terms of the way our neurons are firing and the feeling, sense, of the world around us. Much like the Chinese Room, there is no room for the mental lives, understanding, feeling, in a purely physical explanation of the brain. The understanding must be over and above the matter in the brain (at least, that's what Searle is hinting at). 

\section{Other Potential AI Formats and Rebuttals}

Searle, during his time, presented this thought experiment around and got a few different replies to it, which he numbers and then gives the general regions where he got those replies. Many of the replies can be seen today in how some computer scientists are trying to make even more powerful AI.  So, here we go (and these are fast spark-notes, for most there are far more detailed and extra rebuttals):
\subsection{The Systems Reply}
\factoidbox{While it's true that the person in the Chinese Room does not understand Chinese, the system as a whole does. Understanding should not be ascribed to the individual, but to the system as a whole.}

    For this,  Searle's reply is quite simple, imagine that the person internalized all of the manual. The person has been locked in the room for so long that they have memorized the symbol manipulation rules. Would that person understand Chinese? Just as before, there seems to be something missing, some intentionality, which marks the real understanding. We see this often in second-language classrooms, or at least the old-school ones which I dealt with for a time. Memorize the rules, some vocabulary  (not knowing what the words mean), and then get trust into the language. Sure, the person might make the right replies in the right circumstances, but, essentially, the lights would be on and no one would be home.
    
\subsection{The Robot Reply}
\factoidbox{Rather than the program being in a immobile computer, suppose that we put it in a robot. The robot would not be 'bolted to the floor', but rather would be free to move around, eat, drink, make coffee, it would get sensory input from cameras and sensors, all of this would be controlled by the computer 'brain'.}

    Something akin to this has been made, and is one of the more successful examples of AI, not quite 100\%, still has the problems. The difference between this machine and the first kind and Jill or Schank's AI is that they concede that the machine needs more than just formal symbol manipulation, more than just inputs and outputs, but needs to have the ability to interact with the world and develop a data-base from the real world experience. But, adding in the 'perceptual' and 'motor' qualities doesn't add anything to the base-level way that machines 'think'. For example, suppose that we replay the Chinese Room case, but this time the input is coming from a camera and sensors in a robot. The outputs would need to be more complex, obviously, but at the end of the day, it's still just a decision engine, there's no understanding in the machine.
    \subsection{The Brain Simulator Reply}

\factoidbox{Suppose that we make a program which doesn't represent information which we have about the world, but rather simulates the actual sequence of neurons firing in a human brain. It takes in stories and simulates how the brain fires upon seeing the scripts and acts like the brain would command a body.}

    Now, where is the understanding in this system? Calling back to my A\&P courses when I went to Community College, the brain is, simply put, an arrangement of neurons. The neurons fire and cause others to fire across the brain according to their arrangement. This is just a really, really, complicated decision engine, a really, super, complex set of if-then-else statements.  It would only be simulating the structure of the brain, not the mind. If Non-Reductive Physicalism is true, then this, we could say is conscious… Maybe…
\subsection{The Combination Reply}

\factoidbox{What if we combined all three of them together. While each had a problem, namely versions of the Chinese Room, if we combined them we could get a way out. For example, what if we made a robot with a body indistinguishable from a human's. In the 'skull' cavity, there is a brain shaped computer. This computer would run a simulation of a human brain. We raise it as a human child, making no special circumstances for it (beyond those a parent would give their child), and so on. Imagine that the behavior of this robot is indistinguishable from a human person's. Surely, in this case we would say that it has intentional states (feelings and understanding).}

    In the real world, there have been attempts at this, with the robot (though we can't simulate the brain yet) given a blank slate and raise as a child. The results of these cases were quite promising, but due to the limitations on technology, the brain power never really got far. Searle wants to point out a difference between appearance and reality. Certainly, if we come across a human on the street and they behave in the ways which we have come to expect people to behave, we would attribute to this human intentional states (feelings). But these are all appearances, NPCs in video-games have gotten quite good at appearing to have emotions and reactions, it's what makes contemporary video-games so awesome. For some interesting content on this, check out the The Uncanny Valley\autocite{extrahistory2012} (in this case, it's the reactions of the NPC, not the appearance, but the case still applies). But does an NPC really have those emotions? No, they don't, they are running a script. At the end of the day, the machine is still lights and clockwork. But a rebuttal to this reply leads to:
\subsection{The Other Minds Reply}
\factoidbox{How do you know that other people have intentional states (feelings)? I have first person access to my mental states, but other people's states are totally blocked to me. All I have to go on is their behavior. So, if you're going to attribute mental states to other humans (who can pass our intuitive tests), then by the same principle you must attribute them to a computer which can pass the tests.}

    This reply is, at first, very intuitive and seems to cut to the core of Searle's objections. Basically, the only reason that you think that I have a mind is because I look like you and I act a certain way. Isn't that enough for me to actually have a mind?  However, as is often the case in philosophy, the beauty of a reply is only skin-deep. This reply is making a jump between an epistemic question (a question about what and how we know) and a metaphysical question (a question about what is actually the case). The core question for the strong AI proponent is not whether we know or believe that a machine as feelings, but rather whether the machine actually does have feelings. The study of knowledge is a separate question which we will cover in Module 6. In psychology and in our daily lives it is presupposed the reality and knowledge of other minds, just like how in the physical sciences and in our daily lives it's presupposed the reality and knowledge of the external world. These presumptions are not, necessarily, actually the case.
\subsection{The Many Mansions Reply}
 \factoidbox{The core of Searle's argument makes an assumption, that AI is only about analog or digital computers. This just so happens to be the present state of technology, but what about the future? Whatever processes necessary for these intentional states, eventually, some system will be able to replicate it and that's what we will call strong AI. While we have weak AI today, nothing presupposes that some other means of making AI can't generate a strong one.}

    The only reply to this is that it moves the goalposts. The reason we are making AI in the way we are, aside from making life easier on us, is to hopefully explain some aspect of human intentionality. If we define strong AI as whatever gets us a really understanding and feeling thing, then making a baby is making strong AI. The core thesis, found in AI researchers even today, is that the mind is the brain, physical (remember that this is the core premise of physicalism!).  If this thesis or claim is reframed or redefined so that it's no longer physicalism, then the objections no longer apply and they no longer have a testable hypothesis. 
