\part{Do We Have Free Will? What is it?}
\label{ch.modFour}
\addtocontents{toc}{\protect\mbox{}\protect\hrulefill\par}
\chapter{Part 6: The Free Will Debate}
The topic for this week is the notion of free will, what it is, whether people have it, etc. Stances regarding free will will influence your stances concerning problems in Neurology, Psychology, Criminal Justice, reward and punishment (praise and blame), morality (in general), physics, and many other areas of study. As a starting point, we will say that Free Will is whatever is necessary for moral responsibility. Other, more common notions of free will start with this as the beginning and then add things which they believe are necessary for such responsibility. 

This means, basically, that if you aren't acting in accordance with your free will (or you don't have free will) then you aren't morally responsible for it. Others might think that you are and praise/blame you for the actions, but you aren't actually responsible for it, it just wasn't up to you. As a first example, suppose that some mad scientist puts a microchip in my brain and uses me as a remote control robot. She could make me do awful things and others, prior to knowing about my plight may hold me accountable for it, but in reality, I am not responsible for the actions made by my body. 

That example brings us to an interesting point about moral responsibility/free will. It seems that there needs to be some kind of control which the doer has over the action, something more than automated things. So, for example, take this case:

    \factoidbox{Sally is drinking a cup of coffee with a person she finds attractive (call it a date). They make a lewd comment and then, due to a strange muscle spasm, her hand flies forward as she's about to take a drink and splashes the coffee all over their white clothes. They will be furious of course and they will likely blame her for the strange event, but is she ACTUALLY responsible for it?} 

More often than not, regardless of whether you think that she should have splashed the coffee on them, you will think that she wasn't responsible in this case. So, what's missing? Now, let's look at this example, be sure to notice the difference:

    \factoidbox{Sally is drinking a cup of coffee with a person she finds attractive (call it a date). They make a lewd comment and then, in shock and anger, her hand flies forward as she's about to take a drink and splashes the coffee all over their white clothes. They will be furious of course and they will likely blame her for the strange event, but is she ACTUALLY responsible for it?}

In this case, more often than not, you will say that she is responsible, for good or ill. So, what's the difference?  
\section{Part 6.1:  Free Will and (Moral) Responsibility}

If there is no such thing as free will, there is no such thing as moral responsibility. This is because ‘free will’ is defined as whatever is necessary to have moral responsibility. So, we now need to ask various questions which can help us narrow the field concerning this responsibility:
\noindent
\begin{tabular}{p{2.75in}|p{2.75in}|}
Question&Likely Answer\\\hline
Am I responsible for failing to do something I physically couldn’t do?&If I can't do it, I can't be responsible for not doing it.\\
\hline
Am I responsible for doing something which I physically couldn’t not do (as in, it was impossible for me not to do it)?&If I had to do it (no choice), I am not responsible for doing it.\\
\hline
Am I responsible for something outside of my control?&If the thing is outside of my control, I can’t be responsible for it.
\end{tabular}

To help drive these points home, take these two examples, they are much like the examples which we have seen in this module already, so again, notice the difference:

    \factoidbox{You are you, as you are now, you can’t fly, you can’t run faster than a speeding bullet, you aren’t bullet proof. A friend of yours is in New York and is outside of an orphanage on fire, with kids trapped inside. You are in Washington.  Are you responsible for not saving the children?} 

As before, if I predict the ordinary intuition right, you will say that you aren't responsible for saving the kiddies. This is because it's just not within your power to save them. However, if we add in and change some things to the case, then we could think otherwise.

   \factoidbox{ Last night, the chemistry labs on campus were busy, a meteor fell into a pool of some unstable compounds and you were exposed to the splash. You are not as you are now, you can fly, you can run/fly faster than a speeding bullet, you are bullet proof. A friend of yours is in New York and is outside of an orphanage on fire, with kids trapped inside. You are in Washington. Are you responsible for not saving the children?}

If I predict your answer right, you will say “yes”.  But, what is the difference, maybe Uncle Ben had it right, "With great power comes great responsibility." Many people claim that the difference between the cases, as it regards moral responsibility is that in the first case, there’s nothing you can do, it’s not possible for you to help the kiddies but in the second, you can.  This means that whatever is necessary for moral responsibility must have the requirement, built in, that you can do it. Some of you might have heard the phrase "ought implies can" which basically means that if something is your moral responsibility, if you are actually responsible for something, then you need to be able to do it. This also applies in reverse, if you can't do something, then you aren't responsible for it. 

\section{Part 6.2: The Free Will Debate}

Trying to answer questions like those is the basis for the free will debate. What makes those two cases different? Why do we say ‘no’ to the first example and ‘yes’ to the second example? And some of the possible was of handling this were mentioned lightly above. Of the 8 possible stances which one can hold (taking the possible answers to various questions and using those to form stances), there are three which are actually palatable, able to hold their own. These stances are:
\begin{earg}
   \item[]  Hard Determinism
    \item[] Libertarianism (not the political stance)
    \item[] Compatibilism 
\end{earg}
Both the Hard Determinist and the Compatibilist hold that Determinism is true. The Libertarian rejects Determinism. The debate concerns which of these stances is correct. 
\chapter{Part 7: Determinism, Incompatibilism, and Hard Determinism}
\section{Part 7.1 Determinism}

This is the stance that every event is the result of preceding causes in accordance with the unchanging laws of nature. Basically, if some super amazing God-like computer knew all of the laws of nature and the state of the universe down to the smallest particles, it could predict with 100\% accuracy the future. For example, take the muscle spasm coffee launch, the computer would have predicted it (prior to it happening). If the world was set-up differently (like in the second coffee case), the computer would have predicted the coffee splash there too. Also, there is nothing about the computer which is special, it is not determining your fate, it is just saying what will happen. The weather person saying that it will be sunny tomorrow does not make it the case that it will be sunny tomorrow. It is worth noting that your actions, choices, and thoughts are all events. This is pretty uncontroversial. So, if a stance says that determinism is true, then it follows that all of the choices you make are just as determined by the past and the laws of nature as any other event, like an apple falling from a tree. This is a common area of confusion for students. Many think that Compatibilism includes provisions that everything else in the universe is determined except your choices. This is not true. Libertarianism is willing to make that claim. Compatibilism, as we will see, holds that determinism is true (your actions, thoughts, and choices are determined) but you are still responsible for your actions, under the right circumstances (which are more common than you would think).  Actually, thinking about questions concerning our justice system, like 'why punish?' and 'how much?', from a deterministic mindset is why certain countries have a very low recidivism rate, and prison conditions are far better than in other countries. In those cases, they treat punishment as rehabilitation, removing the reason the person behaved as they did. 

There is a lot of evidence in favor of determinism. In fact, the vast majority of the empirical sciences (biology, macro-level physics (not quantum mechanics (the jury is out there)), and chemistry to name a few) take determinism as a base level assumption. If two events have the same preceding causes (or relevantly similar preceding causes), then those events will be the same (or relevantly similar). We can see this in our daily lives as well. When we make a choice, there are many factors which enter into it, such as our past, our mental acuity in that moment, and our understanding of the case at hand. How we weigh or consider those factors have various causes as well, such as our past, dispositions (either nurtured into us or in our nature), and so on. All of those things, according to determinism, cause our choice with 100\% certainty and, like the examples above, a super-computer, if it knew all of those factors, could predict, with certainty, the choices we make before we make them. The positive of this, then, is that if you want to alter your behavior, or the behavior of another, you need to think about why you or they made or will make those choices and remove or change those factors.  

\subsection{Incompatibilism}

Incompatiblism is a family of stances regarding the relationship between free will and determinism. All stances in this family answer the question "Were determinism true, would we still have free will?" (or, in other words, "Were determinism true, would we still have moral responsibility?") the same way. If the stance says that there couldn't be responsibility in a universe where determinism is true, then the stance is Incompatibilist (free will and determinism are incompatible).\footnote{Not all determinists are incompatibilists. Also, for time-travel to be possible (as in traveling back in time), determinism must be true.  I will not defend that claim here. As a side note, fatalism is determinism and if a model for the topology of time implies that time-travel is possible, then it implies fatalism; but it should be noted that there is some debate on this.} There are two, general, theories which fall into this family. These two are radically different in that they only agree about the answer to that question. These are Hard Determinism and Libertarianism. 

\section{Part 7.2: Hard Determinism}

Hard Determinism belongs to the incompatiblist family. This means that it claims that responsibility and determinism can't exist together. It also makes the positive claim that determinism is true, there is no randomness in the universe. It is called 'Hard' because it draws a hard line about the cases, takes the hard, difficult to swallow, parts of determinism to the extreme. As a result, from these two claims, it claims that there’s no such thing as moral responsibility. We may hold each other responsible, but it is not a feature of the world (one might claim that morality is a useful fiction). There are several examples of this sort of thinking in the real world. For example, in a court case, suppose that the accused had a hard upbringing and up until that point lived a difficult life with few opportunities. They are in court accused of theft. The more you know about the accused and their upbringing, the less harshly you will want to hold them responsible or punish them because you quickly realize that in their mind, given their background and experiences, there wasn't another option.\footnote{This thought, taken in a slightly different direction, can be found in Martha Nussbaum's Equity and Mercy} Hard Determinism can use this fact to their advantage. We don't have the ability to have all of the factors of other people in our minds, but if we could, we would see that the responsibility we attribute to others is actually just a necessary result of the prior events, all stemming back to circumstances which the person had no control over. The control doesn't magically appear in some choices (because of determinism), so responsibility must just be a fiction our small minds concocted.

\subsection{An Argument for Hard Determinism}
This is the basic argument for Hard Determinism, it relies on various assumptions which are quite commonly made in science, for example, that the laws of nature are constant (they don't change), that there aren't random events (even if there were, you would not get free will, in the sense of moral responsibility, but that's an argument for later), and, to a lesser degree, that the world is physicalist (not overly necessary, but makes it easier to argue).  The bold text are the lines to the argument.
\subsubsection{1. The past controls the present and the future.}

This is the core tenant of determinism. It is basically that what happened in the past will tell you (assuming that the physical world is not random) what will happen in the present and the future. One way to think about it is how ordinary physical events, an apple falling, a rock rolling, etc. happen. Some change in the past determined a change in the present, and that determined a change in the future. Also, for a brain-bender (and people who speak languages which a different style of tense system than English will find this easier to grasp), what was the present is now the past and what was the future will be the present and thereby the past. This is the core tenant of determinism. It is basically that what happened in the past will tell you (assuming that the physical world is not random) what will happen in the present and the future. One way to think about it is how ordinary physical events, an apple falling, a rock rolling, etc. happen. Some change in the past determined a change in the present, and that determined a change in the future. Also, for a brain-bender (and people who speak languages which a different style of tense system than English will find this easier to grasp), what was the present is now the past and what was the future will be the present and thereby the past.
\subsubsection{2. You can't control the past.}

This is likely the most intuitive of the lines of the argument. What happened happened, you can’t change what happened, and make it different (and no, time travel stories where the characters change the past are not time travel stories or they are not possible, see the previous articles on this to see why). This might seem a little obvious; What has happened is set in stone, some might think that the future isn't written yet (though the determinist will disagree), but that notion relies on some aspect of the present being variable, random, or undetermined. The past can't be changed. 
\subsubsection{3. You can't control the way the past controls the present and the future.}

This is a special emphasis on one aspect of the previous line. Basically, it's here to remind us that we don't have control over how nature does its thing. The laws of nature don't change because I want really hard and wish upon a star, the universe is horrifyingly indifferent. The core of it is that gravity will do its thing regardless of whether a human wills it one way or the other, and all other natural phenomena will work out the same regardless of us. With humans, we are just another cog in the machine. Remember that determinism extends to all events, which includes your thoughts, actions, and choices. You can't change who raised you, the past choices you made, the laws of nature and the past which lead up to the choice which you currently face. Those factors, according to determinism, determine the choice you will make in that moment. The 'up to you' aspects of responsibility doesn't appear.
\subsubsection{4. So, you can't control the present and the future.}

The word ‘so’ in this context is a marker for something derived from the other lines. And this line is derived from them. The only way out of this is to say that one of the previous lines is wrong. In this case, say that you can control the present/future, you will need to say that either you can change the past or change the laws of nature.
\subsubsection{5. If the way the present is and the future will be are outside of my control, you aren't not morally responsible for it.}

This is from a common reply to the question “am I responsible for things outside of my control?’. When you think about it,  being-in-control is required for responsibility. All I did was say that if the present and future are outside of my control, I am not morally responsible for it. The examples which I have given up to this point will drive home this point. For those examples of not being responsible for things outside of your control, think about this case:

   \factoidbox{A young person is driving to work, using the same back, forested, roads which they drive everyday. We could go into the causal factors which lead them to take these roads everyday, but that would be going in a different direction. The road bends and as they finish the corner, a deer leaps from the side of the road striking the front of their vehicle.}

 The various factors outside of this person's control, even if they were doing everything right up until that point, function as more than enough excuse to say that they weren't responsible for this. Relatedly, the nature of the laws of nature and the past also remove the responsibility which people could have for any action they take, any event which befalls them, because the factors leading up to them were outside of their control.
\subsubsection{6. Therefore, you aren't morally responsible for anything (AKA, No Free Will)}

This is the conclusion, it is what all of the lines up until this point have been leading up to. If you accept all of the lines leading up to this point, you have to accept this one (really, no choice).  People who accept this line are called Hard Determinists. The two main assumptions made are that Determinism is true and that responsibility requires some ability to do otherwise. But, there is another flavor of it called Libertarianism (not the political stance) which is incompatiblist, but thinks that determinism is false.

\chapter{Part 8: What If We Denied Determinism?}
Basically, what would happen if we said that determinism was false? There are a few ways to deny determinism. Some posit that the laws of nature do have some randomness to them, pointing to some findings in Quantum Mechanics. They say that certain events on a quantum level ‘just happen’, there are no hidden variables. Two systems, they say, can be exactly the same, but behave differently. But how does this fair with responsibility? Well, it is not what is wanted. If there’s this randomness in the world, it makes our ideas of control even more elusive. Take for example, the muscle spasm coffee splash case. In that case, we say that she wasn't responsible because it was outside of her control. Random events are always outside of our control. So, we don't have free will for determined acts because we couldn't do otherwise and we don't have free-will for random acts because we don't have control.  But, there are some who deny this sort of reasoning, these are the Libertarians.

\section{Part 8.1: Libertarianism}

Like the Hard Determinists, Libertarians are Incompatibilists. They hold, to say this again, that responsibility and determinism can't play together. They disagree with the Hard Determinist, however, about whether determinism is true. They both hold that either determinism is true or there is moral responsibility (but not both). Hard Determinists say that determinism is true, so that removes responsibility (because it can't be both, this is not a formal fallacy, in this case). Libertarians say that determinism is false, so they get that there must be some indeterminacy and therefore responsibility. In particular, they deny the third line of the Hard Determinism Argument (you can't control the way the past controls the present and future). Although they are not determinists, Libertarians could be fine with physical events being deterministic, they want something extra beyond the laws of nature to intervene and give you control over the way the past controls the present and the future (there have been attempts to have a physicalist Libertarianism, but I argue against those attempts in some extra reading I wrote and provide). It posits some kind of substance dualism (typically, though there have been attempts otherwise, to have physicalism and libertarianism). The mental has control over the physical, within certain restraints from outside of nature: a "contra-causal" freedom, in which the mental is distinct from the causal order of nature, yet mysteriously able to alter it. In this way, we can deny the line in the hard determinist’s arguments which says that past controls the present and future. According to these guys, we have some influence in there. The world is not deterministic, and we have moral responsibility. We could call that conception interventionist control. 

\subsection{An argument for Libertarianism}

This is an argument which I made up for libertarianism, sometimes when I make up arguments for a stance, for this class, I will intentionally give holes for you to find and exploit. This is not one of those cases. It starts from the basic assumption that there is moral responsibility, that morality is more than just a social fiction (it's an actual part of the world). We will see arguments for and against this in Module 7. This particular argument is a shorter, simplified version of one I am giving in a paper.
\begin{earg}
    \item[1] There is moral responsibility (that is, a person is morally responsible for their actions, some or all).
    \item[2] If a person is morally responsible for their actions, then they must be able to do otherwise.
    \item[3] If determinism is true, then a person is not able to do otherwise.
    \item[4] So (from 1 and 2), a person is able to do otherwise.
    \item[5] So (from 3 and 4), determinism is false.
    \item[6] Therefore (from 1 and 5), there is moral responsibility and determinism is false.
\end{earg}
The big area of contention is the second line, and that is worthy of a bit of a defense. Imagine the following case:

    \factoidbox{Suppose that a meteor crashed in a field near your home and an alien spore escaped filling the air, quickly infecting people’s nervous systems. These spores grow and take over the mind of the host. Sometimes, the actions which the host would do are the same as those which the spores would cause them to do, however, when they are not, the spores switch up the nerves to make the host do as they would want. Since the host could not do otherwise, it would seem that they are not morally responsible for their actions (as even if they were their choice, the spores would stop them from making a different one).}

To apply this more to the case at hand, the spores are the past and the laws of nature. We want to say that they are different than those, but the only reason for that is that the past and the laws of nature, according to determinism, have been with us forever, while the spores are a recent addition.

\chapter{Part 9: Compatibilism and Soft Determinism}
The core similarity between the hard determinist and the libertarian is that they both think that determinism and free will can’t play together, you get one or you get the other. That is, they are both incompatibilist. The opposite stance, compatibilism, says that they are compatible. One could hold that determinism is false but were it true, there would still be responsibility. This is all to say that one doesn't negate the other. Now, from my experience, versions of this stance are the most popular in philosophy at the moment. This distinction is useful because incompatibilism vs compatibilism is a choice or assumption we make prior to formulating or taking a stance in this debate. 

\section{Part 9.1: Soft Determinism}

Soft Determinism is a nicer, softer, interpretation of the implications of Determinism. This is the stance that determinism is true and that we have free will. Some people, causing some confusion, like to call this stance 'Compatibilism'. This stance is that the world is 100\% determined but we still have some kind of control. I like to think of it as having your cake and eating it too. Take this argument (shorter version of what was before):

\begin{earg}
    \item[]The past controls the present and future.
    \item[]You can't control the past.
    \item[]Also, you can't control the way the past controls the present and future.
    \item[]So, you can't control the present and future.
\end{earg}

The Soft Determinist denies that we have no control over the future. Rather they say that we have control in virtue of being a cog in the machine. A common error which people have when they encounter this stance is that they think that the Libertarians, Hard Determinists, and the Compatibilists all mean the same thing when they use the term 'free will'. While it is true that all three mean 'whatever is necessary for moral responsibility', the Libertarians and the Hard Determinists hold that this responsibility requires that you physically be able to do otherwise. Determinism holds that it is physically impossible for you to do otherwise but it does not say that it was absolutely impossible for you to do otherwise. The Soft Determinists latch onto this sense. You are free because it was possible for you to do otherwise, even though you were physically guaranteed not to. For example, suppose that I am driving down the road and I come to a fork. I turn right. It was certainly be within my ability to turn left and drive that way just as much as it was within my ability to turn right. Determinism says that it was determined that I would turn right and I physically could not have taken a left, the fact still remains that I \emph{could} have done it, making me responsible. I had control over the future, even though it was determined how I would use that control.

There are three ways that this intuition can be accounted for and each is stronger than the last. 

\subsection{The Flash Definition}

    \factoidbox{A subject acted freely if she could have done otherwise in the right sense. The subject could have done otherwise in this sense provided she would have done otherwise if she had chosen differently.}

So, your action was free if you chose to do it; you would have done something different if you chose something different. Another explanation: For the libertarian, you get free will because you have control from outside of nature,  (more than likely, through substance dualism, but that is a contemporary on going debate I am personally involved in). For the compatibilist of this stripe, you have control from within the laws of nature. Your freedom is in how you process the information and make your choices, though what choice you make is determined, you have free will when the choice is made in the right way. But, there is an issue for this, and I like to call it the mini-Martians problem, but the issue which I gave with the alien spores will work as well. 

    \factoidbox{Imagine the invasion of the mini-Martians. These are incredibly small, organized, and mischievous beings: small enough to invade our brains and walk around in them. If they do so, they can set our modules pretty well at will. We become puppets in their hands. Of course, the mini-Martians might set us to do what we would have done anyhow. But they might throw the chemical switches so that we do quite terrible things. Then let us suppose that, fortunately, science invents a scan to detect whether the Martians have invaded us. Won't we be sympathetic to anyone who suffered this misfortune? Wouldn't we immediately recognize that he was not responsible for his wrongdoings? But, says the incompatibilist, why does it make a difference if it was mini-Martians, or causal agencies of a more natural kind?}

Basically, there's no difference between the mini-Martians and the laws of nature, the more we learn about a person the more likely we are to think that they weren't responsible for their actions (which is why a Deterministic mind-set in cases of Crime and Punishment leads to such different results, you tend to see more rehabilitation rather than retribution).  In the face of this, some philosophers have thought to rephrase the stance and add some more content, which gives us:

\subsection{Revised Definition}

    \factoidbox{The subject acted freely if she could have done otherwise in the right sense. This means that she would have done otherwise if she had chosen differently and, under the impact of other thoughts or considerations, she would have chosen differently.}

But this one, too, does have its issues. In this case, I can point to more trigger-warning-worthy real world examples, but I will leave those aside. The core issue involves Bad Luck. 

    \factoidbox{Although there is no randomness in this sort of world, we can still talk about ‘luck’ in the since that something ‘just wasn’t in the cards’ or it was just the case that something would not happen. Sometimes, it is just not going to happen that the ‘right’ thoughts don’t arise.}

Some philosophers like to associate freedom with understanding. We are free in so far as we understand. This is attractive to those who like political freedoms, like of speech and information. Including this in there will make the stance stronger. We need to figure out what “other thoughts and considerations” are. This is done by adding in that these are (1) accurate to the given situation and (2) available to the choicer.
\subsection{The Revised Revised Definition}

This is the strongest of the bunch and best gets the contemporary views from this stance. It is a little more complicated, but when you apply it to cases in the real world where we don't hold people responsible, it seems to fit well:

    \factoidbox{The subject acted freely if she could have done otherwise in the right sense. This means that she would have done otherwise if she had chosen differently and, under the impact of other true and available thoughts or considerations, she would have chosen differently. True and available thoughts and considerations are those that represent her situation accurately, and are ones that she could reasonably be expected to have taken into account.}
    
\chapter{Are Libertarianism and Physicalism Compatible?}
\section{By Davis Smith}
\section{Abstract}

This paper concerns two seemingly unrelated topics, the Mind-Body Problem and the Free Will Debate. More particularly, I am worried about whether Libertarianism and Physicalism are compatible. First, I show that Libertarianism implies that there are actions such that the doer could have physically done otherwise and which the doer had control over. Next, I show that Physicalism implies that all actions are either deterministic or non-deterministic. And finally, I show that a Physicalist deterministic action eliminates the ability to do otherwise and that a Physicalist non-deterministic action eliminates the doer’s control. These prove that Libertarianism and Physicalism are contradictory.
\section{Introduction}

One would think that the more desperate two topics in philosophy are, the less impact a stance in one would have on the other. While this is true for some things, it does not seem to hold in the case of the Free Will Debate and the Mind-Body Problem. In this paper, I am going to argue that Libertarianism and Physicalism are not compatible.\footnote{There have been a few papers in cognitive science and experimental philosophy which have shown that people who believe in Physicalism, more particularly, Reductive Physicalism, are less likely to believe in Libertarian free will. In other words, that there is a correlation between belief in Substance Dualism and Libertarian free will. \cite{Wisniewski1} for one such study.} This is to say that a universe with Libertarian free will cannot be a Physicalist universe. I do not claim that Determinism and Substance Dualism are incompatible\footnote{I, personally, believe that they are compatible.} nor do I make the more robust claim that Libertarianism is self-contradictory,\footnote{This is a longer project which starts with a paper much like this one.} rather, I am claiming only that Libertarian free will is contrary to Physicalism. This argument has 3 central premises, which the others prove/support. First, if Libertarianism is true, then there are some actions such that the doer could have (physically) done otherwise and the action was within the doer's control. Second, if Physicalism is true, then for any action, that action is either deterministic or non-deterministic. And finally, if an action is deterministic, then the doer could not have (physically) done otherwise and if an action is non-deterministic, the action is not within the doer’s control. These three premises, if they are accurate, lead to the necessary conclusion that if Libertarianism is true, then Physicalism is false.
\section{Terms}

For this phase of the argument, there are four (4) terms which I need to precisely define and clarify. For some, this will seem like simple review, but it is important that we are all on the same page. These terms are ‘Determinism', ‘Physicalism', ‘Free Will', and ‘Libertarianism'.
\subsection{Determinism}

\begin{center}An event A is deterministic if and only if, given the state of the world and the laws of nature prior to A, A will occur necessarily.\footnote{Put in a more technical and precise way, this means that an event A (at t at w) is deterministic if and only if there is no possible world, w’, such that the conditions of the world prior to t at w’ and the laws of nature at t at w’ are the same as those prior to and at t at w and A does not occur at t at w’.}
\end{center}
Basically, for all possible worlds with the exact same laws of nature and they have the exact same past events leading up to the event, then the event will occur in all the possible worlds, in the same way and at the same time. The opposite of this is to say that an event is non-deterministic, and we can define this like so:
\begin{center}
An event A is non-deterministic if and only if, given the state of the world and the laws of nature prior to A, A will not occur necessarily.\footnote{Put in a more technical way, we have that event A (at t at w) is non-deterministic if and only if there’s possible world, w’, such that the conditions of the world prior to t at w’ and the laws of nature at t at w’ are the same as those prior to and at t at w and A does not occur at t at w’.}
\end{center}
This is not to say that the event will not occur, but rather it is saying that it is possible that the event will not occur. If an event is non-deterministic, then it is possible for the event not to occur in a world the same prior to it as a world where it does. There are two ways in which an event could be non-deterministic. First, the event could be the probabilistic result of a preceding cause or it could be completely uncaused. In the case of the former, the laws of nature would need to be in such a way that for at least some sets of circumstances, the probability of the event occurring is less than 1. The probability of the event occurring could be very close to 1, but if it is less than that, there is still a chance that the event does not occur. A totally uncaused event, on the other hand, would be one in which the preceding events in no way determine or make more likely that it occurs, such an event would be, for all intents and purposes, random.

With these two definitions, we can define ‘Determinism' in the following way: Determinism is the stance that all events in the actual world are deterministic. If Determinism is accurate, then it follows that the laws of nature do not contain randomness and that the preceding events predict with 100\% accuracy future events.\footnote{This way of defining ‘Determinism’ is used in several places, but \cite[most notably in][ ]{Popper1}.} Some point to the seemingly random events at a quantum level as evidence against determinism, but some recent works\autocite{Barrett1} show that it is impossible to determine whether any notion of randomness can characterize the data in Quantum Mechanics. This means that appeals to such notions are going to, fundamentally, be based on intuitions and not the data.\footnote{I will often use the phrase ‘deterministic universe’ which means that Determinism is true at the possible world in question. Similarly, ‘non-deterministic universe’ means that Determinism is not true at the possible world in question.}
\subsection{Physicalism}

For our purposes, though there may be other Physicalisms out there, this is a stance within the possible responses to the Mind-Body Problem. This stance is that all the objects/substances in the world are physical. There are no mental substances which do not reduce to the physical or which do not supervene on the physical. For our purposes, also, there is no reason to distinguish between Reductive and Non-Reductive Physicalism, as they both lead to the same relevant place.\footnote{I will often use the phrase ‘Physicalist universe’, which means that the possible world in question is one in which Physicalism is true.} Even in the case of Non-Reductive Physicalism, if a Physicalist universe is non-deterministic, then the indeterminacy is in the natural laws. Since mental events, at the very least, supervene on physical events, the choices we make (mental) are caused by changes in the physical nature of the brain. So, non-deterministic choices must be the result of a non-deterministic physical event.
\subsection{Free Will}

For our purposes, I am defining ‘free will’ as whatever is necessary/sufficient for moral responsibility.\footnote{Others may fail to hold the doer morally responsible for the action or it may be the case that the degree of rectitude is inconsequential. Either way, the doer is still morally responsible.} There are two ways which this can be interpreted:
\begin{enumerate}
    \item If a person does an act freely, then they are morally responsible for that action.
    \item If a person is morally responsible for an action, then they did that action freely.
\end{enumerate}
The second interpretation seems to be more common. For example, many claim that if there’s no free will, then there’s no moral responsibility. The first notion, however, is the one which we will be using as this is seemingly what the Libertarians are after in the case of free will. Merely asserting that we have free will because Determinism is false does not guarantee that we have moral responsibility. This would be a fallacy. Libertarians, therefore, must hold the first conditional. They want free will because it grants moral responsibility.\footnote{Even those who wish to distance free will and moral responsibility will claim that free action is at least a sufficient condition for moral responsibility.}
\subsection{Libertarianism}

Libertarianism is an incompatibilist stance, meaning that it holds that Determinism and free will are not compatible. You could have one or the other, but not both. More particularly, it holds that Determinism is false, so there is free will and therefore moral responsibility. Since it implies that determinism is false, it states that there are some non-deterministic events. Some implied features of Libertarianism include:
\begin{enumerate}
    \item To have moral responsibility, the person must have physically been able to do otherwise.
    \item The events which cause the actions which we are morally responsible for, must be non-deterministic (because the doer must be able to do otherwise).
\end{enumerate}
When it comes to the first feature of Libertarianism, this is the Principle of Alternative Possibilities\footnote{\cite{Frankfurt1}.}, but we need to be careful to include the term ‘physically'. Without it, the statement could be one which a Compatibilist would approve of. They would say that, sure, you could do otherwise, but you physically are not able to. Though the Frankfurt-style cases found in the literature are convincing to some that this is not an essential feature of moral responsibility,\footnote{We will see a Frankfurt-style case in Part 1 of the argument.} I believe that they would be unconvincing to a sensible Libertarian.\footnote{This is because the passive coercion in the Frankfurt-style cases is quite like the indirect causal pressures which the past and the laws of nature have on our choices in a deterministic universe. If they reject the possibility of moral responsibility in a deterministic universe, then they would equally need to reject the possibility of moral responsibility in a Frankfort-style case. For an alternative argument, \cite[see][ ]{Widerker1}.} The Libertarians would want the ability to physically do otherwise. With the second feature, if the events which cause our actions are wholly deterministic, then it would not be possible for us to do otherwise, which means that, from the first feature, we would not be morally responsible for them.
\section{The Argument}

This argument is broken into three parts, which come together at the end to show that Physicalism and Libertarianism are not compatible. The premises for the argument have four (4) conditional statements, each of which are proven in the relevant section. These are the same premises which I mentioned in the introduction. They are also the section titles.
\subsection{Part 1: If Libertarianism is true, then there are some actions such that the doer could have done otherwise and the action was within the doer's control.}

Proving this line of the argument requires us to show two different things; both of which are derived from a basic understanding of ‘moral responsibility'. From the very definition of Libertarianism, we have that we must have free will and therefore must be morally responsible for at least some of our actions. But what exactly does it mean to have moral responsibility for some of our actions?

It seems that there are at least two necessary features for a doer to be morally responsible for their action, and these are the two features of the above conditional. First, the doer must have been able to do otherwise. The doer must have had the ability to physically refrain from doing it or do something other than what they did. For example, take this thought experiment to drive home the idea:\footnote{This is a case similar to the ones \cite[found in][ ]{Frankfurt1} \cite[and][ ]{Fischer1}.}
\factoidbox{
    Suppose that a meteor crashed in a field near your home and an alien spore escaped filling the air, quickly infecting people’s nervous systems. These spores grow and take over the mind of the host. Sometimes, the actions which the host would do are the same as those which the spores would cause them to do, however, when they are not, the spores switch up the nerves to make the host do as they would want.}

Since the host could not do otherwise, they are not morally responsible for their actions (as even if it were their choice, the spores would stop them from making a different one). When we think of cases where a person is forced or coerced to do something, then we are less inclined to hold them responsible for their actions. In the above case, the alien spores are passively forcing the hosts to do actions, so it seems clear that they are at least less responsible for the actions.

The other necessary feature for moral responsibility is a sense of control or up-to-us-ness. The doer must have chosen\footnote{Making a choice, it would seem, needs to be both voluntary and with deliberation.} to perform the action and they must have directed it to the outcome. For example, look at this case:
\factoidbox{
    Suppose that you are at a dinner party having a discussion with various people and sipping wine. During a deep discussion on whether Libertarianism is compatible with Physicalism, you have a muscle spasm which launches the wine all over another’s white cloths.}

Whether the universe in this thought experiment is deterministic is not relevant. You have no control over what happens when you have a muscle spasm like this. That lack of control serves as a more than adequate excuse to relinquish moral responsibility for the destruction of another person's clothing.\footnote{For more on this, \cite[see][ ]{Austin1}.} One could also characterize this as an event which was not ultimately up to you. Your thoughts and attitudes, the reasons which you may have, did not enter the event. This means that, at least for the Libertarian, the ability-to-do-otherwise is not enough for moral responsibility. It is certainly the case that it’s possible for you not to have that spasm, but the action must be up to you.\footnote{Here is a break-down of the argument: First, if Libertarianism is true, then people have free will. Second, if people have free will, then people are morally responsible for some of their actions. If people are morally responsible for some of their actions, then there are actions such that the person could have done otherwise and which the person was in control over. Therefore, if Libertarianism is true, then there are some actions such that the doer could have done otherwise and which was in the doer’s control.}
\subsection{Part 2: If Physicalism is true, then for any action, that action is either deterministic or non-deterministic.}

This is straight forward, but I am the sort of person who does not like to leave anything without a proof. To start off, from the definition of ‘Physicalism', we have that there are only physical substances. This is that there is only one kind of substance in the world and those substances are physical. This certainly removes the possibility of Substance Dualism, at least for the purposes of this argument.

The next line of the proof is that if there are only physical substances, then there are only physical events. Events require that there be objects or substances (one or more) which are involved in that event. An event A involves an object/substance B if and only if for all minimalist and accurate accounts the event A, the accounts contain reference to B. For a proof, try to imagine an event which does not have anything involved in it. Even if there’s quantum randomness where a particle just appears out of nothing, that particle is still involved in the event. Thus, if there are only physical substances, then there can only be physical events. Some might claim that there are mental events, but physicalism has it that those, at the very least, reduce to physical events or they are simply a different perspective on the physical event.\footnote{Also, a minimalist account of those mental events, if Physicalism is true, would only involve physical objects.}

Third, if there can only be physical events, then for any event, that event is either deterministic or non-deterministic. This can be derived using the law of excluded middle. Since there is no third alternative between being deterministic or non-deterministic, any event must be one or the other and the law of non-contradiction would have it that they cannot be both. From this, all we need is that all actions are events, which seems too obvious to deny.\footnote{As before, here is a breakdown of the argument: If Physicalism is true, then there are only physical substances. If there are only physical substances, then there are only physical events. If there are only physical events, then those events are ether deterministic or non-deterministic. All actions are events. Therefore, if Physicalism is true, for any action, that action is either deterministic or non-deterministic.} As a tie between this part and the previous, Physicalist Libertarianism would say that there are at least some non-deterministic actions which we are, therefore, morally responsible for.  
\subsection{Part 3: If an action is deterministic, then the doer could not have done otherwise and if an action is non-deterministic, the action is not within the doer’s control.}

This line of the argument has two halves. For both examples used to illustrate these conditionals, we can assume that they are in a Physicalist universe. For the first half of this part of the argument, this requires some quibbles about the exact meanings of the terms. The relevant sense of being able to do otherwise is the one which the Libertarian uses. So, it follows naturally that a deterministic action lacks alternative possibilities.

The second half of this part of the argument is more involved. First, if an action is non-deterministic, then it is either a probabilistic effect of a preceding cause or it is totally uncaused. In the case of probabilistic events, though one event may be more likely than another, it is still, on a base level, random which of those events happens. For example, take the following thought experiment:\footnote{The percent likelihood of the two different possibilities is, I would think, a worst-case scenario for the moral responsibility of the man in question. Other ratios of possibility are useable.}

\factoidbox{
    A man is standing in front of a switch, there is a run-away trolley. If the man does nothing, then the trolley will kill 5 people, if they flip the switch, then the trolley will only kill one person. Since this is a non-deterministic universe, there is a 50\% chance that they will flip the switch and a 50\% chance that they will do nothing.}

In this case, the core difference between it and other trolley problems is that there is this element of chance. Since this is a physicalist universe, the indeterminacy of the choice must be because of some manner of non-deterministic physical event. A minimalist account of the choice would not contain reference to the doer’s reasons. The choice was not up to them.\footnote{It could be generalized that, since this is a Physicalist universe, the choice reduces to physical events and those must be probabilistic. On that level, the reasoning, the up-to-him-ness, is not present. A similar line of thought, though about artificial intelligence, can be \cite[found in][ ]{Searle1}.} The random nature of the choice takes the action outside of the man’s control. In the case of totally uncaused events, the amount of control had by the doer becomes even more elusive. Take, for example, the following thought experiment:\footnote{This case is similar in form to the ones \cite[seen in][ ]{Elzein1}, but the thought experiments there are used to show that the principle of alternative possibilities is important to moral responsibility and that the alternative possibilities must be relevant to the case.}
\factoidbox{
    One evening, Jones is sitting back to watch a little television when a quantum particle appears in his brain and then vanishes. This event causes a chain reaction which results in him choosing to begin growing edible mushrooms. Acting on this choice, he becomes very knowledgeable about the subject and eventually he discovers a new form of hypoallergenic penicillin, saving countless lives.}

For ease of use, we can assume that the appearance of the particle is the only non-deterministic event relevant to the case.\footnote{Also, because other possible worlds are not relevant, we can assume that this is a physicalist universe.} Since the initial cause of the choice was completely random and the resulting events which lead to the discovery were the deterministic results of the initial event, we could hardly say that Jones was in control of the action and, thereby, morally responsible for saving the countless lives.\footnote{Here is a break-down of the argument: First, if an action is deterministic, then the person could not have done otherwise. If an action is non-deterministic, then it is either a probabilistic effect of a preceding cause or it’s totally uncaused. If the action is probabilistic in this way, then the action was not within the person’s control and if the action is totally uncaused, then the action was not within the person’s control. Therefore, if an action is deterministic, then the person could not have done otherwise and if the action is non-deterministic, then the action was not within the person’s control.} Along the same vein, going outside of the scope of this paper, if this were a Substance Dualist universe and the non-deterministic event occurred mentally, then Jones’ choice still, ultimately, would not have been up to him.
\section{Conclusion: If Libertarianism is true, then Physicalism is false.}

So far, I have shown three (3) things. First, If Libertarianism is true, then there are some actions such that the doer could have done otherwise and that the doer was in control of that action. This is from the seemingly necessary features of moral responsibility. Next, I have shown that if Physicalism is true, then all our actions are either deterministic or non-deterministic. There is no third alternative. And finally, I have shown that if an action is deterministic, then it's not possible for the doer to have done otherwise and if the action is non-deterministic, then the action was not in the doer's control. From the second and the third, we have that if Physicalism is true, then for any action, the doer either could not have done otherwise or the doer was not in control. But the first part claims that if Libertarianism is true, then there are some actions where the doer has both of those features. This directly contradicts the result from Physicalism and therefore shows that if Libertarianism is true, then Physicalism is false. \footnote{Here the complete argument: If Physicalism is true, then there are only physical substances. If there are only physical substances, then there can only be physical events. If there can only be physical events, then for any event, that event is either deterministic or random. All actions are events. If an action is deterministic, then the person could not have done otherwise. If an action is random, then it is either a probabilistic effect of a preceding cause or it’s totally uncaused. If the action is probabilistic in this way, then the action was not within the person’s control and if the action is totally uncaused, then the action was not within the person’s control. If Libertarianism is true, then a person has free will. If a person has free will, then they are morally responsible for some actions. If a person is morally responsible for an action, then a) that person could have done otherwise and b) that person is in control of the action. Therefore, if Libertarianism is true, then Physicalism is false.}


\chapter{Are Dualism and Libertarianism Compatible? by Davis Smith}
\label{libertarianismdualism}
\section{By Davis Smith}
\section{Abstract}
\newcounter{fb}
\setcounter{fb}{\thefootnote}
\setcounter{footnote}{0}

This paper concerns two seemingly unrelated areas of Metaphysics, the Free Will Debate and the Mind-Body Problem. I will be arguing that Libertarianism and Substance Dualism are not compatible. The argument goes like so:  First, if Libertarianism is true then there are some actions such that the actor could have physically done otherwise and which the actor had control over. Second, if Dualism is true, then there are three places where there could be the indeterminacy necessary for Libertarianism. Third, all three of these decrease the actor’s control over their actions.  Therefore, if Dualism is true, then Libertarianism is false.

\section{Introduction}
Metaphysics is a very wide-reaching field and as such one would think that the conclusions one reaches in a far distant area would have little impact on the conclusions one could reasonably hold in a closer one. While this might be true for some areas of Metaphysics, it does not seem to hold for the Free Will Debate and the Mind-Body Problem. In this paper, I will be arguing that Libertarianism and Dualism are not compatible.\footnote{I admit that this is counter-intuitive to some. There have been a few papers in Cognitive Science and Experimental Philosophy which point to a correlation between believing in Dualism and believing in Libertarianism. \cite{Wisniewski1} for one such study. While I have no issue with Dualism, for the purposes of this paper, the issues I have are with Libertarianism.}  In a previous paper\footnote{Davis Smith, “Are Libertarianism and Physicalism Compatible?”, unpublished, presented at the 72nd Northwest Philosophy Conference.  My main intent for this paper is to ultimately merge it with the Physicalism one, showing that Libertarianism is incompatible with the mind, whatever it may be.}, I showed that Libertarianism and Physicalism are contradictory, so here, using a similar methodology, I will show that Libertarianism and Dualism are contradictory. To give a brief overview of the argument:  First, if Libertarianism is true then there are some actions such that the actor could have physically done otherwise and which the actor had control over. Second, if Dualism is true, then there are three places where there could be the indeterminacy necessary for Libertarianism. Third, all three of these decrease the actor’s control over their actions.  Therefore, if Libertarianism is true, then Dualism is false. We will start by being particular about the meanings of the terms central to this proof: Determinism, Dualism, and Libertarianism.
\section{Terms}
\subsection{Determinism}
\begin{center}An event A is deterministic if and only if, given the state of the world and the laws of nature\footnote{“Given the state of the world and the laws of nature” is meant to be neutral in regards to physicalism vs dualism.}  prior to A, A will occur necessarily.\footnote{Put in a more technical and precise way, this means that an event A (at t at w) is deterministic if and only if there is no possible world, w’, such that the conditions of the world prior to t at w’ and the laws of nature at t at w’ are the same as those prior to and at t at w and A does not occur at t at w’.}\end{center} 
Basically, for all possible worlds with the exact same laws of nature and they have the exact same past events leading up to the event, then the event will occur in all the possible worlds, in the same way and at the same time. The opposite of this is to say that an event is non-deterministic, and we can define this like so:
\begin{center}An event A is non-deterministic if and only if, given the state of the world and the laws of nature prior to A, it is possible for A not to occur.\footnote{Put in a more technical way, we have that event A (at t at w) is non-deterministic if and only if there’s possible world, w’, such that the conditions of the world prior to t at w’ and the laws of nature at t at w’ are the same as those prior to and at t at w and A does not occur at t at w’.}\end{center} 
This is not to say that the event will not occur, but rather it is saying that it is possible that the event will not occur. There are two ways in which an event could be non-deterministic. First, the event could be the probabilistic result of a preceding cause or it could be completely uncaused. In the case of the former, the laws of nature would need to be in such a way that for at least some sets of circumstances, the probability of the event occurring is less than 1. The probability of the event occurring could be very close to 1, but if it is less than that, there is still a chance that the event does not occur. A totally uncaused event, on the other hand, would be one in which the preceding events in no way determine or make more likely that it occurs, such an event would be, for all intents and purposes, random.

With these two definitions, we can define ‘Determinism' in the following way: Determinism is the stance that all events in the actual world are deterministic. If Determinism is accurate, then it follows that the laws of nature do not contain randomness and that the preceding events predict with 100\% accuracy future events.\footnote{This way of defining ‘Determinism’ is used in several places, but \cite[most notably in][ ]{Popper1}.}  Some point to the seemingly random events at a quantum level as evidence against determinism, but some recent works\footnote{\cite{Barrett1}}  show that it is impossible to determine whether any notion of randomness can characterize the data in Quantum Mechanics.\footnote{I will often use the phrase ‘deterministic universe’ which means that Determinism is true at the possible world in question. Similarly, ‘non-deterministic universe’ means that Determinism is not true at the possible world in question.}
\subsection{Dualism}
Dualism a stance in the Mind-Body Problem which says that rather than there being one kind of substance in the world, there are two, namely mental and physical. Under many normal interpretations of this stance, the physical substance is the physical body while the mental substance is the immaterial ‘soul’, this houses our feelings, sensations, thoughts, and the ‘what-it’s-like-ness’. Some Dualist theories hold that the mental and physical do not interact, there is no causal relationship between them. It assumes Determinism because it would be very contrary to our experience otherwise. We do not need to spend much time thinking about it for this project. Other Dualist theories, however, do posit that the substances interact. While there is an open question about how this could be, we can assume that they do interact for our purposes. 
\subsection{Libertarianism}
Libertarianism holds that Determinism and responsibility for our actions are not compatible and holds that Determinism is false, so there is responsibility for our actions. From this, it states that there are some non-deterministic events. In general, being responsible for an action implies at least two aspects (according to Libertarianism):
\begin{enumerate}
\item The doer must have physically been able to do otherwise.
\item The events which cause the action which the doer is responsible for must be non-deterministic.\footnote{Mark Balaguer, who we will encounter later, claims that the indeterminacy must increase the control, a concept which we will explore in Part 1 of the argument. \cite{Balaguer2}} 
\end{enumerate}
When it comes to the first feature of Libertarianism, this is the Principle of Alternative Possibilities\footnote{\cite[As presented in][ ]{Frankfurt1}}, but we need to be careful to include the term ‘physically'. Without it, the statement could be one which a Compatibilist would approve of. They would say that, sure, you could do otherwise, but you physically are not able to. Though the Frankfort-style cases found in the literature are convincing to some that this is not an essential feature of responsibility,\footnote{We will see a Frankfurt-style case in Part 1 of the argument.}  I believe that they would be unconvincing to a sensible Libertarian.\footnote{This is because the passive coercion in the Frankfort-style cases is quite like the indirect causal pressures which the past and the laws of nature have on our choices in a deterministic universe. If they reject the possibility of moral responsibility in a deterministic universe, then they would equally need to reject the possibility of moral responsibility in a Frankfort-style case. For an alternative argument \cite[see][ ]{Widerker1}.}  The Libertarians want the ability to physically do otherwise. Similarly, if the events which cause our actions are deterministic, then it would not be possible for us to do otherwise, which means that we would not be responsible for them. Both of these, however, have reference to responsibility. Responsibility, as I will argue later, has at least two features which I believe contradicts these two features. 

Most generally, Libertarians can be divided into two general teams. On one side we have the Agent-Causal Libertarians and on the other we have the Event-Causal Libertarians. Both teams agree that there are at least some actions which we are responsible for, they differ in how they think this functions. Agent-Causal Libertarians hold that the doer has the ability to be the first cause of a chain of events. These Libertarians hold that when we act freely, we are causing something to be without ourselves being caused.\footnote{\cite[687]{Franklin1} Some could see a similarity between this sort of idea and theological considerations. Some hold that God is the first cause of our universe and that God was not caused. Some also hold that God created man in His image. As a result, some could jump to the conclusion that God created us with the ability to be a first cause.}  I hold that Agent-Causal Libertarianism requires Substance Dualism, in other words, it is not possible to have this ability to be the ‘first cause’ without mental substances.\footnote{The primary purpose of this paper is to show that all forms of Libertarianism, as I have constructed it, are incompatible with dualism. Agent-Causal Libertarianism seems a little more brazen about claiming that it requires dualism and the rejection of physicalism. If Agent-Causal Libertarianism does require dualism, then this paper serves as a proof that the stance is contradictory.}  Event-Causal Libertarians take on a more metaphysically modest stance. These theorists generally have the best aspects of the compatibilists’ theories but add the claim that there must be indeterminacy in the action. All Libertarians need to be able to show that the indeterminacy adds something to the control which is lacked in a Compatibilist model.\footnote{I will argue in Part 2 that the indeterminacy detracts from control and thereby responsibility. The more indeterminacy we add to a given model, the less responsibility the agent has.}
  
\section{The Argument}
This argument is broken into three parts, which come together at the end to show that Dualism and Libertarianism are not compatible. 
\subsection{Part 1: If Libertarianism is true, then there are some actions such that the actor could have physically done otherwise and had control over.} 
Proving this line of the argument requires us to show two different things; both of which are derived from a basic understanding of ‘responsibility'. From the very definition of Libertarianism, there must be some responsibility for at least some of our actions. But what exactly does it mean to have moral responsibility for some of our actions?

It seems that there are at least two necessary features for a doer to be responsible for their action, and these are the two features of the above conditional. First, the doer must have been able to do otherwise.\footnote{And this must be a physical ability, not a metaphysical one, according to Libertarianism. So, the doer must have had the ability to physically refrain from doing it or do something other than what they did.}  As an example, take this thought experiment:\footnote{This is a case similar to the ones \cite[found in][ ]{Frankfurt1} \cite[and][ ]{Fischer1}.}
\factoidbox{Suppose that a meteor crashed in a field near your home and psychic alien spores escaped filling the air. Once some spores choose a host, they begin psychically implanting thoughts and take over their mind. Sometimes, the actions which the host would do are the same as those which the spores would cause them to do, however, when they are not, the spores implant thoughts which make the host do as they would want.} 
Since the host could not do otherwise, they are not responsible for their actions (as even if it were their choice, the spores would stop them from making a different one). When we think of cases where a person is forced or coerced to do something, then we are less inclined to hold them responsible for their actions. In the above case, the alien spores are passively forcing the hosts to do actions, so it seems clear that they are at least less responsible for the actions.

The other necessary feature for responsibility is a sense of control or ‘up-to-us-ness.’\footnote{This control requires that the doer had chosen to make the action and directed it to the outcome. Making a choice, it would seem, needs to be both voluntary and with deliberation.} For example, look at this case:
\factoidbox{Suppose that you are at a dinner party having a discussion with various people and sipping wine. During a deep discussion on whether Libertarianism is compatible with Physicalism, you have a muscle spasm which launches the wine all over another’s white cloths.}
Whether the universe in this thought experiment is deterministic is not relevant. You have no control over what happens when you have a muscle spasm like this and the lack of control serves as an excuse to relinquish responsibility for the destruction of another person's clothing.\footnote{For more on this, \cite[see][ ]{Austin1}}  One could also characterize this as an event which was not ultimately up to you. This means that the ability-to-do-otherwise is not enough for responsibility. While it’s possible for you not to have that spasm, but the action must be up to you.
\subsection{Part 2: If Dualism is true, then there are three places where the indeterminacy could appear in an actor’s action; either A) in the physical substance, B) in the mental substance, or C) in the interaction between the mental and the physical substances.}
In the previous part, we looked closely at what it takes to be responsible for our actions; the physical ability to do otherwise and control or ‘up-to-us-ness’. For us to be able to do otherwise, in the Libertarian sense, the action or the events leading to the action\footnote{There could have been an event in the distant past which was not deterministic which lead to the train of event leading to the doer’s action, but such an event seems hardly relevant to the question of whether the doer could have done otherwise. For another example, suppose that one person had the ability to do otherwise while another did not. In their interactions, the free person caused the determined person’s actions and the determined person could have, in a sense, done otherwise, but the determined person’s actions were determined by the free person’s.}  cannot be deterministic, meaning that there must be some indeterminacy in the actual process of making an action. It follows, then, that there are three possible places where the indeterminacy could occur: in the physical substance, in the mental, or in their interactions. Call these, in order, \emph{physical-indeterminacy}, \emph{mental-indeterminacy}, and \emph{interaction-indeterminacy}. Of these, the indeterminacy therein could be totally uncaused or a probabilistic result of previous events. In the following sub-parts, I will show that each of these diminish or eliminate the doer’s control. 
\subsubsection{Part 2A: If there is physical-indeterminacy, then it decreases the control the actor has over their actions.}
In a previous paper, I argued that Physicalism and Libertarianism are incompatible on the grounds that physical-indeterminacy diminishes the doer’s control or the ‘up-to-us-ness’. This holds true here as well, for the same reasons. For example, take the following thought experiment:
\factoidbox{A man is standing in front of a switch, there is a run-away trolley. If the man does nothing, then the trolley will kill 5 people, if they flip the switch, then the trolley will only kill one person. Since this is a non-deterministic universe, there is a 50\% chance that they will flip the switch and a 50\% chance that they will do nothing.}
In this case, the core difference between it and other trolley problems is that there is this element of chance.\footnote{The percent likelihood of the two different possibilities is, I would think, a worst-case scenario for the responsibility of the man in question. Other ratios of possibility are useable.}  In a Dualistic universe, we can suppose that the mental substance had deliberated and, because of the past experiences and events in their life, choose to flip the switch. Since there is physical-indeterminacy, whether to flip the switch was not up to the man.\footnote{It could be generalized that in an extreme case like this, the doer’s reasons and feelings do not play a part in the choice and thereby make it up to him. A similar line of thought, though about artificial intelligence, can be \cite[found in][ ]{Searle1}}.  Generalizing this, if there is any physical-indeterminacy relevant to our actions, then our control is diminished by it. 

In the case of a totally uncaused physical event leading to the action, take this thought experiment:\footnote{This case is similar in form to the ones \cite[seen in][ ]{Elzein1}, but the thought experiments there are used to show that the principle of alternative possibilities is important to moral responsibility and that the alternative possibilities must be relevant to the case.}
\factoidbox{One evening, Jones is sitting back to watch a little television when a quantum particle appears in his brain and then vanishes. This event causes a chain reaction which results in him choosing to begin growing edible mushrooms. Acting on this choice, he becomes very knowledgeable about the subject and eventually he discovers a new form of hypoallergenic penicillin, saving countless lives.}  
For ease of use, we can assume that the appearance of the particle is the only non-deterministic event relevant to the case. Since the initial cause of the choice was completely random and the resulting events which lead to the discovery were the deterministic results of the initial event, we could hardly say that Jones was in control of the action and, thereby, responsible for saving the countless lives.
\subsubsection{Part 2B: If there is mental-indeterminacy, then it decreases the control the actor has over their actions.}
To illustrate is point, we can reuse the thought experiments from the previous part, with some minor alterations. The events in the mental substance could be either totally uncaused or probabilistic. For the probabilistic version, take the trolley problem case from before and think about it as indeterminism in the mental substance. In such a case, the person would not have control over their own thoughts and thereby their actions. They would be, in a sense, undisciplined. One could be faced with a choice where they are unsure what to do and deliberate about it. If the choice contains a hint of randomness, then that diminishes the control they had. Similarly, if the causal mental event is totally uncaused, then it would be very similar to the case involving hypoallergenic penicillin. The thought which spurs the action would be completely random and potentially radically out of character for Jones. This randomness further depreciates the responsibility Jones has. 

\subsubsection{Part 2C: If there is interaction-indeterminacy, then it decreases the control the actor has over their actions.} 
For the third and final place where this indeterminacy could be, take a look at the thought experiments once more. For probabilistic events in the interaction, it would be like a case where the mind chose to pull the leveler and there is a 50-50 shot about whether the body would get the correct message. If the body, magically, got the message, then they would have gotten lucky in regards to doing the action Often, when someone is trying to clam responsibility for an action or event, others will diminish that responsibility by claiming that they got lucky. If the indeterminacy required for the ability to do otherwise was in the interaction between the mind and the body, then it would be lucky that the body did what it was told. For totally uncaused events, this would be like the mental substance not giving an order to the body and the body magically getting an order. This too would be very troubling to how we could say that they are responsible for their actions. 
\subsection{Part 3: So, if Dualism is true, then indeterminacy decreases the control an actor has over their actions.}
This is simple enough to prove. I have shown that, if Dualism is true, then there are three places where the indeterminacy required for responsibility, according to Libertarianism, could appear. Each of these actually diminish or eliminate the control the doer has over their actions. This means that if Dualism is true, then any form of indeterminacy (relevant to how our actions are taken) does not increase the responsibility one may have but rather decreases it.
\section{Conclusion: If Dualism is true, then Libertarianism is false.}
To conclude this paper, I will outline what I have done. First, according to Libertarianism, Determinism is false and there are some actions which we are responsible for. These actions must be ones where we could have physically done otherwise (namely, there is a core aspect of indeterminacy) and those actions were within our control. Dualism allows for three places where the indeterminacy could appear: In the physical, mental, or in their interactions. I then went on to show that indeterminacy in the physical diminishes the control the doer has and it also does so when it is in the mental and also in their interaction. Libertarianism makes a central claim that this indeterminacy increases our responsibility for our actions, not decreases it. The implication from this is that if Dualism is true, then Libertarianism is false because in every place where the indeterminacy could be, it harms our responsibility, not help it.

\setcounter{footnote}{\thefb}

