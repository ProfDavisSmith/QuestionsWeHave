\chapter{Minds, Brains and Programs by John Searle}\autocite{Searle1}
\label{mindsbrainsprograms}
\section{Abstract}
This article can be viewed as an attempt to explore the consequences of two propositions. (1) Intentionality in
human beings (and animals) is a product of causal features of the brain I assume this is an empirical fact about
the actual causal relations between mental processes and brains It says simply that certain brain processes are
sufficient for intentionality. (2) Instantiating a computer program is never by itself a sufficient condition of
intentionality The main argument of this paper is directed at establishing this claim The form of the argument is to
show how a human agent could instantiate the program and still not have the relevant intentionality. These two
propositions have the following consequences (3) The explanation of how the brain produces intentionality
cannot be that it does it by instantiating a computer program. This is a strict logical consequence of 1 and 2. (4)
Any mechanism capable of producing intentionality must have causal powers equal to those of the brain. This is
meant to be a trivial consequence of 1. (5) Any attempt literally to create intentionality artificially (strong AI)
could not succeed just by designing programs but would have to duplicate the causal powers of the human
brain. This follows from 2 and 4.

"Could a machine think?" On the argument advanced here only a machine could think, and only very special
kinds of machines, namely brains and machines with internal causal powers equivalent to those of brains And
that is why strong AI has little to tell us about thinking, since it is not about machines but about programs, and
no program by itself is sufficient for thinking.

\section{Minds, Brains, and Programs}

What psychological and philosophical significance should we attach to recent efforts at computer simulations of
human cognitive capacities? In answering this question, I find it useful to distinguish what I will call "strong" AI
from "weak" or "cautious" AI (Artificial Intelligence). According to weak AI, the principal value of the computer
in the study of the mind is that it gives us a very powerful tool. For example, it enables us to formulate and test
hypotheses in a more rigorous and precise fashion. But according to strong AI, the computer is not merely a
tool in the study of the mind; rather, the appropriately programmed computer really is a mind, in the sense that
computers given the right programs can be literally said to understand and have other cognitive states. In strong
AI, because the programmed computer has cognitive states, the programs are not mere tools that enable us to
test psychological explanations; rather, the programs are themselves the explanations.

I have no objection to the claims of weak AI, at least as far as this article is concerned. My discussion here will
be directed at the claims I have defined as those of strong AI, specifically the claim that the appropriately
programmed computer literally has cognitive states and that the programs thereby explain human cognition.
When I hereafter refer to AI, I have in mind the strong version, as expressed by these two claims.

I will consider the work of Roger Schank and his colleagues at Yale\autocite{Schank1}, because I am
more familiar with it than I am with any other similar claims, and because it provides a very clear example of the
sort of work I wish to examine. But nothing that follows depends upon the details of Schank's programs. The
same arguments would apply to Winograd's SHRDLU\autocite{Winograd1}, Weizenbaum's ELIZA \autocite{Weizen1}, and indeed any Turing machine simulation of human mental phenomena.

Very briefly, and leaving out the various details, one can describe Schank's program as follows: the aim of the
program is to simulate the human ability to understand stories. It is characteristic of human beings' story-
understanding capacity that they can answer questions about the story even though the information that they
give was never explicitly stated in the story. Thus, for example, suppose you are given the following story:

-A man went into a restaurant and ordered a hamburger. When the hamburger arrived it was burned to a crisp,
and the man stormed out of the restaurant angrily, without paying for the hamburger or leaving a tip." Now, if
you are asked -Did the man eat the hamburger?" you will presumably answer, ' No, he did not.' Similarly, if
you are given the following story: '-A man went into a restaurant and ordered a hamburger; when the
hamburger came he was very pleased with it; and as he left the restaurant he gave the waitress a large tip
before paying his bill," and you are asked the question, -Did the man eat the hamburger?,-' you will presumably
answer, -Yes, he ate the hamburger." Now Schank's machines can similarly answer questions about restaurants
in this fashion. To do this, they have a -representation" of the sort of information that human beings have about
restaurants, which enables them to answer such questions as those above, given these sorts of stories. When the
machine is given the story and then asked the question, the machine will print out answers of the sort that we
would expect human beings to give if told similar stories. Partisans of strong AI claim that in this question and
answer sequence the machine is not only simulating a human ability but also
\begin{enumerate}
\item that the machine can literally be said to understand the story and provide the answers to questions, and
\item that what the machine and its program do explains the human ability to understand the story and answer
questions about it.
\end{enumerate}
Both claims seem to me to be totally unsupported by Schank's' work, as I will attempt to show in what follows.
One way to test any theory of the mind is to ask oneself what it would be like if my mind actually worked on
the principles that the theory says all minds work on. Let us apply this test to the Schank program with the
following Gedankenexperiment. Suppose that I'm locked in a room and given a large batch of Chinese writing.
Suppose furthermore (as is indeed the case) that I know no Chinese, either written or spoken, and that I'm not
even confident that I could recognize Chinese writing as Chinese writing distinct from, say, Japanese writing or
meaningless squiggles. To me, Chinese writing is just so many meaningless squiggles.

Now suppose further that after this first batch of Chinese writing I am given a second batch of Chinese script
together with a set of rules for correlating the second batch with the first batch. The rules are in English, and I
understand these rules as well as any other native speaker of English. They enable me to correlate one set of
formal symbols with another set of formal symbols, and all that 'formal' means here is that I can identify the
symbols entirely by their shapes. Now suppose also that I am given a third batch of Chinese symbols together
with some instructions, again in English, that enable me to correlate elements of this third batch with the first two
batches, and these rules instruct me how to give back certain Chinese symbols with certain sorts of shapes in
response to certain sorts of shapes given me in the third batch. Unknown to me, the people who are giving me
all of these symbols call the first batch "a script," they call the second batch a "story. ' and they call the third
batch "questions." Furthermore, they call the symbols I give them back in response to the third batch "answers
to the questions." and the set of rules in English that they gave me, they call "the program."

Now just to complicate the story a little, imagine that these people also give me stories in English, which I
understand, and they then ask me questions in English about these stories, and I give them back answers in
English. Suppose also that after a while I get so good at following the instructions for manipulating the Chinese
symbols and the programmers get so good at writing the programs that from the external point of view that is,
from the point of view of somebody outside the room in which I am locked -- my answers to the questions are
absolutely indistinguishable from those of native Chinese speakers. Nobody just looking at my answers can tell
that I don't speak a word of Chinese.

Let us also suppose that my answers to the English questions are, as they no doubt would be, indistinguishable
from those of other native English speakers, for the simple reason that I am a native English speaker. From the
external point of view -- from the point of view of someone reading my "answers" -- the answers to the
Chinese questions and the English questions are equally good. But in the Chinese case, unlike the English case,
I produce the answers by manipulating uninterpreted formal symbols. As far as the Chinese is concerned, I
simply behave like a computer; I perform computational operations on formally specified elements. For the
purposes of the Chinese, I am simply an instantiation of the computer program.

Now the claims made by strong AI are that the programmed computer understands the stories and that the
program in some sense explains human understanding. But we are now in a position to examine these claims in
light of our thought experiment.

1 As regards the first claim, it seems to me quite obvious in the example that I do not understand a word of the
Chinese stories. I have inputs and outputs that are indistinguishable from those of the native Chinese speaker,
and I can have any formal program you like, but I still understand nothing. For the same reasons, Schank's
computer understands nothing of any stories. whether in Chinese. English. or whatever. since in the Chinese
case the computer is me. and in cases where the computer is not me, the computer has nothing more than I have in the case where I understand nothing.

2. As regards the second claim, that the program explains human understanding, we can see that the computer
and its program do not provide sufficient conditions of understanding since the computer and the program are
functioning, and there is no understanding. But does it even provide a necessary condition or a significant
contribution to understanding? One of the claims made by the supporters of strong AI is that when I understand
a story in English, what I am doing is exactly the same -- or perhaps more of the same -- as what I was doing in
manipulating the Chinese symbols. It is simply more formal symbol manipulation that distinguishes the case in
English, where I do understand, from the case in Chinese, where I don't. I have not demonstrated that this
claim is false, but it would certainly appear an incredible claim in the example. Such plausibility as the claim has
derives from the supposition that we can construct a program that will have the same inputs and outputs as
native speakers, and in addition we assume that speakers have some level of description where they are also
instantiations of a program.

On the basis of these two assumptions we assume that even if Schank's program isn't the whole story about
understanding, it may be part of the story. Well, I suppose that is an empirical possibility, but not the slightest
reason has so far been given to believe that it is true, since what is suggested though certainly not demonstrated
-- by the example is that the computer program is simply irrelevant to my understanding of the story. In the
Chinese case I have everything that artificial intelligence can put into me by way of a program, and I understand
nothing; in the English case I understand everything, and there is so far no reason at all to suppose that my
understanding has anything to do with computer programs, that is, with computational operations on purely
formally specified elements. As long as the program is defined in terms of computational operations on purely
formally defined elements, what the example suggests is that these by themselves have no interesting connection
with understanding. They are certainly not sufficient conditions, and not the slightest reason has been given to
suppose that they are necessary conditions or even that they make a significant contribution to understanding.

Notice that the force of the argument is not simply that different machines can have the same input and output
while operating on different formal principles -- that is not the point at all. Rather, whatever purely formal
principles you put into the computer, they will not be sufficient for understanding, since a human will be able to
follow the formal principles without understanding anything. No reason whatever has been offered to suppose
that such principles are necessary or even contributory, since no reason has been given to suppose that when I
understand English I am operating with any formal program at all.

Well, then, what is it that I have in the case of the English sentences that I do not have in the case of the
Chinese sentences? The obvious answer is that I know what the former mean, while I haven't the faintest idea
what the latter mean. But in what does this consist and why couldn't we give it to a machine, whatever it is? I
will return to this question later, but first I want to continue with the example.

I have had the occasions to present this example to several workers in artificial intelligence, and, interestingly,
they do not seem to agree on what the proper reply to it is. I get a surprising variety of replies, and in what
follows I will consider the most common of these (specified along with their geographic origins).

But first I want to block some common misunderstandings about "understanding": in many of these discussions
one finds a lot of fancy footwork about the word "understanding." My critics point out that there are many
different degrees of understanding; that "understanding" is not a simple two-place predicate; that there are even
different kinds and levels of understanding, and often the law of excluded middle doesn-t even apply in a
straightforward way to statements of the form "x understands y; that in many cases it is a matter for decision and
not a simple matter of fact whether x understands y; and so on. To all of these points I want to say: of course, of
course. But they have nothing to do with the points at issue. There are clear cases in which "understanding' literally applies and clear cases in which it does not apply; and these two sorts of cases are all I need for this argument 2 I understand stories in English; to a lesser degree I can understand stories in French; to a still lesser
degree, stories in German; and in Chinese, not at all. My car and my adding machine, on the other hand,
understand nothing: they are not in that line of business. We often attribute "under standing" and other cognitive
predicates by metaphor and analogy to cars, adding machines, and other artifacts, but nothing is proved by such
attributions. We say, "The door knows when to open because of its photoelectric cell," "The adding machine
knows how) (understands how to, is able) to do addition and subtraction but not division," and "The thermostat
perceives chances in the temperature."

The reason we make these attributions is quite interesting, and it has to do with the fact that in artifacts we
extend our own intentionality;3 our tools are extensions of our purposes, and so we find it natural to make
metaphorical attributions of intentionality to them; but I take it no philosophical ice is cut by such examples. The
sense in which an automatic door "understands instructions" from its photoelectric cell is not at all the sense in
which I understand English. If the sense in which Schank's programmed computers understand stories is
supposed to be the metaphorical sense in which the door understands, and not the sense in which I understand
English, the issue would not be worth discussing. But Newell and Simon (1963) write that the kind of cognition
they claim for computers is exactly the same as for human beings. I like the straightforwardness of this claim,
and it is the sort of claim I will be considering. I will argue that in the literal sense the programmed computer
understands what the car and the adding machine understand, namely, exactly nothing. The computer
understanding is not just (like my understanding of German) partial or incomplete; it is zero.
Now to the replies:

I. The systems reply (Berkeley). "While it is true that the individual person who is locked in the room does not
understand the story, the fact is that he is merely part of a whole system, and the system does understand the
story. The person has a large ledger in front of him in which are written the rules, he has a lot of scratch paper
and pencils for doing calculations, he has 'data banks' of sets of Chinese symbols. Now, understanding is not
being ascribed to the mere individual; rather it is being ascribed to this whole system of which he is a part."
My response to the systems theory is quite simple: let the individual internalize all of these elements of the
system. He memorizes the rules in the ledger and the data banks of Chinese symbols, and he does all the
calculations in his head. The individual then incorporates the entire system. There isn't anything at all to the
system that he does not encompass. We can even get rid of the room and suppose he works outdoors. All the
same, he understands nothing of the Chinese, and a fortiori neither does the system, because there isn't anything
in the system that isn't in him. If he doesn't understand, then there is no way the system could understand
because the system is just a part of him.

Actually I feel somewhat embarrassed to give even this answer to the systems theory because the theory seems
to me so implausible to start with. The idea is that while a person doesn't understand Chinese, somehow the
conjunction of that person and bits of paper might understand Chinese. It is not easy for me to imagine how
someone who was not in the grip of an ideology would find the idea at all plausible. Still, I think many people
who are committed to the ideology of strong AI will in the end be inclined to say something very much like this;
so let us pursue it a bit further. According to one version of this view, while the man in the internalized systems
example doesn't understand Chinese in the sense that a native Chinese speaker does (because, for example, he
doesn't know that the story refers to restaurants and hamburgers, etc.), still "the man as a formal symbol
manipulation system" really does understand Chinese. The subsystem of the man that is the formal symbol
manipulation system for Chinese should not be confused with the subsystem for English.

So there are really two subsystems in the man; one understands English, the other Chinese, and "it's just that the two systems have little to do with each other." But, I want to reply, not only do they have little to do with each other, they are not even remotely alike. The subsystem that understands English (assuming we allow ourselves
to talk in this jargon of "subsystems" for a moment) knows that the stories are about restaurants and eating
hamburgers, he knows that he is being asked questions about restaurants and that he is answering questions as
best he can by making various inferences from the content of the story, and so on. But the Chinese system
knows none of this. Whereas the English subsystem knows that "hamburgers" refers to hamburgers, the Chinese
subsystem knows only that "squiggle squiggle" is followed by "squoggle squoggle." All he knows is that various
formal symbols are being introduced at one end and manipulated according to rules written in English, and other
symbols are going out at the other end.

The whole point of the original example was to argue that such symbol manipulation by itself couldn't be
sufficient for understanding Chinese in any literal sense because the man could write "squoggle squoggle" after
"squiggle squiggle" without understanding anything in Chinese. And it doesn't meet that argument to postulate
subsystems within the man, because the subsystems are no better off than the man was in the first place; they
still don't have anything even remotely like what the English-speaking man (or subsystem) has. Indeed, in the
case as described, the Chinese subsystem is simply a part of the English subsystem, a part that engages in
meaningless symbol manipulation according to rules in English.

Let us ask ourselves what is supposed to motivate the systems reply in the first place; that is, what independent
grounds are there supposed to be for saying that the agent must have a subsystem within him that literally
understands stories in Chinese? As far as I can tell the only grounds are that in the example I have the same
input and output as native Chinese speakers and a program that goes from one to the other. But the whole
point of the examples has been to try to show that that couldn't be sufficient for understanding, in the sense in
which I understand stories in English, because a person, and hence the set of systems that go to make up a
person, could have the right combination of input, output, and program and still not understand anything in the
relevant literal sense in which I understand English.

The only motivation for saying there must be a subsystem in me that understands Chinese is that I have a
program and I can pass the Turing test; I can fool native Chinese speakers. But precisely one of the points at
issue is the adequacy of the Turing test. The example shows that there could be two "systems," both of which
pass the Turing test, but only one of which understands; and it is no argument against this point to say that since
they both pass the Turing test they must both understand, since this claim fails to meet the argument that the
system in me that understands English has a great deal more than the system that merely processes Chinese. In
short, the systems reply simply begs the question by insisting without argument that the system must understand
Chinese.

Furthermore, the systems reply would appear to lead to consequences that are independently absurd. If we are
to conclude that there must be cognition in me on the grounds that I have a certain sort of input and output and a
program in between, then it looks like all sorts of noncognitive subsystems are going to turn out to be cognitive.
For example, there is a level of description at which my stomach does information processing, and it instantiates
any number of computer programs, but I take it we do not want to say that it has any understanding\autocite{Pylyshyn1}. But if we accept the systems reply, then it is hard to
see how we avoid saying that stomach, heart, liver, and so on, are all understanding subsystems, since there is
no principled way to distinguish the motivation for saying the Chinese subsystem understands from saying that
the stomach understands. It is, by the way, not an answer to this point to say that the Chinese system has
information as input and output and the stomach has food and food products as input and output, since from the
point of view of the agent, from my point of view, there is no information in either the food or the Chinese -- the
Chinese is just so many meaningless squiggles. The information in the Chinese case is solely in the eyes of the
programmers and the interpreters, and there is nothing to prevent them from treating the input and output of my
digestive organs as information if they so desire.

This last point bears on some independent problems in strong AI, and it is worth digressing for a moment to
explain it. If strong AI is to be a branch of psychology, then it must be able to distinguish those systems that are
genuinely mental from those that are not. It must be able to distinguish the principles on which the mind works
from those on which nonmental systems work; otherwise it will offer us no explanations of what is specifically
mental about the mental. And the mental-nonmental distinction cannot be just in the eye of the beholder but it
must be intrinsic to the systems; otherwise it would be up to any beholder to treat people as nonmental and, for
example, hurricanes as mental if he likes. But quite often in the AI literature the distinction is blurred in ways
that would in the long run prove disastrous to the claim that AI is a cognitive inquiry. McCarthy, for example,
writes, '-Machines as simple as thermostats can be said to have beliefs, and having beliefs seems to be a
characteristic of most machines capable of problem solving performance" \autocite{McCarthy1}.

Anyone who thinks strong AI has a chance as a theory of the mind ought to ponder the implications of that
remark. We are asked to accept it as a discovery of strong AI that the hunk of metal on the wall that we use to
regulate the temperature has beliefs in exactly the same sense that we, our spouses, and our children have
beliefs, and furthermore that "most" of the other machines in the room -- telephone, tape recorder, adding
machine, electric light switch, -- also have beliefs in this literal sense. It is not the aim of this article to argue
against McCarthy's point, so I will simply assert the following without argument. The study of the mind starts
with such facts as that humans have beliefs, while thermostats, telephones, and adding machines don't. If you get
a theory that denies this point you have produced a counterexample to the theory and the theory is false.

One gets the impression that people in AI who write this sort of thing think they can get away with it because
they don't really take it seriously, and they don't think anyone else will either. I propose for a moment at least, to
take it seriously. Think hard for one minute about what would be necessary to establish that that hunk of metal
on the wall over there had real beliefs beliefs with direction of fit, propositional content, and conditions of
satisfaction; beliefs that had the possibility of being strong beliefs or weak beliefs; nervous, anxious, or secure
beliefs; dogmatic, rational, or superstitious beliefs; blind faiths or hesitant cogitations; any kind of beliefs. The
thermostat is not a candidate. Neither is stomach, liver adding machine, or telephone. However, since we are
taking the idea seriously, notice that its truth would be fatal to strong AI's claim to be a science of the mind. For
now the mind is everywhere. What we wanted to know is what distinguishes the mind from thermostats and
livers. And if McCarthy were right, strong AI wouldn't have a hope of telling us that.

II. The Robot Reply (Yale). "Suppose we wrote a different kind of program from Schank's program. Suppose
we put a computer inside a robot, and this computer would not just take in formal symbols as input and give out
formal symbols as output, but rather would actually operate the robot in such a way that the robot does
something very much like perceiving, walking, moving about, hammering nails, eating drinking -- anything you
like. The robot would, for example have a television camera attached to it that enabled it to 'see,' it would have
arms and legs that enabled it to 'act,' and all of this would be controlled by its computer 'brain.' Such a robot
would, unlike Schank's computer, have genuine understanding and other mental states."

The first thing to notice about the robot reply is that it tacitly concedes that cognition is not solely a matter of
formal symbol manipulation, since this reply adds a set of causal relation with the outside world \autocite{Fodor1}. 
But the answer to the robot reply is that the addition of such
"perceptual" and "motor" capacities adds nothing by way of understanding, in particular, or intentionality, in
general, to Schank's original program. To see this, notice that the same thought experiment applies to the robot
case. Suppose that instead of the computer inside the robot, you put me inside the room and, as in the original
Chinese case, you give me more Chinese symbols with more instructions in English for matching Chinese
symbols to Chinese symbols and feeding back Chinese symbols to the outside. Suppose, unknown to me, some
of the Chinese symbols that come to me come from a television camera attached to the robot and other Chinese symbols that I am giving out serve to make the motors inside the robot move the robot's legs or arms.
It is important to emphasize that all I am doing is manipulating formal symbols: I know none of these other facts.
I am receiving "information" from the robot's "perceptual" apparatus, and I am giving out "instructions" to its
motor apparatus without knowing either of these facts. I am the robot's homunculus, but unlike the traditional
homunculus, I don't know what's going on. I don't understand anything except the rules for symbol
manipulation. Now in this case I want to say that the robot has no intentional states at all; it is simply moving
about as a result of its electrical wiring and its program. And furthermore, by instantiating the program I have no
intentional states of the relevant type. All I do is follow formal instructions about manipulating formal symbols.

III. The brain simulator reply (Berkeley and M.I.T.). "Suppose we design a program that doesn't represent
information that we have about the world, such as the information in Schank's scripts, but simulates the actual
sequence of neuron firings at the synapses of the brain of a native Chinese speaker when he understands stories
in Chinese and gives answers to them. The machine takes in Chinese stories and questions about them as input,
it simulates the formal l structure of actual Chinese brains in processing these stories, and it gives out Chinese
answers as outputs. We can even imagine that the machine operates, not with a single serial program, but with a
whole set of programs operating in parallel, in the manner that actual human brains presumably operate when
they process natural language. Now surely in such a case we would have to say that the machine understood
the stories; and if we refuse to say that, wouldn't we also have to deny that native Chinese speakers understood
the stories? At the level of the synapses, what would or could be different about the program of the computer
and the program of the Chinese brain?"

Before countering this reply I want to digress to note that it is an odd reply for any partisan of artificial
intelligence (or functionalism, etc.) to make: I thought the whole idea of strong AI is that we don't need to know
how the brain works to know how the mind works. The basic hypothesis, or so I had supposed, was that there
is a level of mental operations consisting of computational processes over formal elements that constitute the
essence of the mental and can be realized in all sorts of different brain processes, in the same way that any
computer program can be realized in different computer hardwares: on the assumptions of strong AI, the mind
is to the brain as the program is to the hardware, and thus we can understand the mind without doing
neurophysiology. If we had to know how the brain worked to do AI, we wouldn't bother with AI. However,
even getting this close to the operation of the brain is still not sufficient to produce understanding. To see this,
imagine that instead of a mono lingual man in a room shuffling symbols we have the man operate an elaborate
set of water pipes with valves connecting them. When the man receives the Chinese symbols, he looks up in the
program, written in English, which valves he has to turn on and off. Each water connection corresponds to a
synapse in the Chinese brain, and the whole system is rigged up so that after doing all the right firings, that is
after turning on all the right faucets, the Chinese answers pop out at the output end of the series of pipes.

Now where is the understanding in this system? It takes Chinese as input, it simulates the formal structure of
the synapses of the Chinese brain, and it gives Chinese as output. But the man certainly doesn-t understand
Chinese, and neither do the water pipes, and if we are tempted to adopt what I think is the absurd view that
somehow the conjunction of man and water pipes understands, remember that in principle the man can
internalize the formal structure of the water pipes and do all the "neuron firings" in his imagination. The problem
with the brain simulator is that it is simulating the wrong things about the brain. As long as it simulates only the
formal structure of the sequence of neuron firings at the synapses, it won't have simulated what matters about
the brain, namely its causal properties, its ability to produce intentional states. And that the formal properties are
not sufficient for the causal properties is shown by the water pipe example: we can have all the formal
properties carved off from the relevant neurobiological causal properties.

IV. The combination reply (Berkeley and Stanford). 'While each of the previous three replies might not be
completely convincing by itself as a refutation of the Chinese room counterexample, if you take all three together they are collectively much more convincing and even decisive. Imagine a robot with a brain-shaped computer lodged in its cranial cavity, imagine the computer programmed with all the synapses of a human brain, imagine
the whole behavior of the robot is indistinguishable from human behavior, and now think of the whole thing as a
unified system and not just as a computer with inputs and outputs. Surely in such a case we would have to
ascribe intentionality to the system.

I entirely agree that in such a case we would find it rational and indeed irresistible to accept the hypothesis that
the robot had intentionality, as long as we knew nothing more about it. Indeed, besides appearance and
behavior, the other elements of the combination are really irrelevant. If we could build a robot whose behavior
was indistinguishable over a large range from human behavior, we would attribute intentionality to it, pending
some reason not to. We wouldn't need to know in advance that its computer brain was a formal analogue of the
human brain.

But I really don't see that this is any help to the claims of strong AI; and here-s why: According to strong AI,
instantiating a formal program with the right input and output is a sufficient condition of, indeed is constitutive of,
intentionality. As Newell\autocite{Newell1} puts it, the essence of the mental is the operation of a physical symbol system.
But the attributions of intentionality that we make to the robot in this example have nothing to do with formal
programs. They are simply based on the assumption that if the robot looks and behaves sufficiently like us, then
we would suppose, until proven otherwise, that it must have mental states like ours that cause and are
expressed by its behavior and it must have an inner mechanism capable of producing such mental states. If we
knew independently how to account for its behavior without such assumptions we would not attribute
intentionality to it especially if we knew it had a formal program. And this is precisely the point of my earlier
reply to objection 11.

Suppose we knew that the robot's behavior was entirely accounted for by the fact that a man inside it was
receiving uninterpreted formal symbols from the robot's sensory receptors and sending out uninterpreted formal
symbols to its motor mechanisms, and the man was doing this symbol manipulation in accordance with a bunch
of rules. Furthermore, suppose the man knows none of these facts about the robot, all he knows is which
operations to perform on which meaningless symbols. In such a case we would regard the robot as an ingenious
mechanical dummy. The hypothesis that the dummy has a mind would now be unwarranted and unnecessary,
for there is now no longer any reason to ascribe intentionality to the robot or to the system of which it is a part
(except of course for the man's intentionality in manipulating the symbols). The formal symbol manipulations go
on, the input and output are correctly matched, but the only real locus of intentionality is the man, and he doesn't
know any of the relevant intentional states; he doesn't, for example, see what comes into the robot's eyes, he
doesn't intend to move the robot's arm, and he doesn't understand any of the remarks made to or by the robot.
Nor, for the reasons stated earlier, does the system of which man and robot are a part.

To see this point, contrast this case with cases in which we find it completely natural to ascribe intentionality to
members of certain other primate species such as apes and monkeys and to domestic animals such as dogs. The
reasons we find it natural are, roughly, two: we can't make sense of the animal's behavior without the ascription
of intentionality and we can see that the beasts are made of similar stuff to ourselves -- that is an eye, that a
nose, this is its skin, and so on. Given the coherence of the animal's behavior and the assumption of the same
causal stuff underlying it, we assume both that the animal must have mental states underlying its behavior, and
that the mental states must be produced by mechanisms made out of the stuff that is like our stuff. We would
certainly make similar assumptions about the robot unless we had some reason not to, but as soon as we knew
that the behavior was the result of a formal program, and that the actual causal properties of the physical
substance were irrelevant we would abandon the assumption of intentionality. [See "Cognition and
Consciousness in Nonhuman Species BBS 1(4) 1978.]

There are two other responses to my example that come up frequently (and so are worth discussing) but really
miss the point.

V. The other minds reply (Yale). "How do you know that other people understand Chinese or anything else?
Only by their behavior. Now the computer can pass the behavioral tests as well as they can (in principle), so if
you are going to attribute cognition to other people you must in principle also attribute it to computers. '
This objection really is only worth a short reply. The problem in this discussion is not about how I know that
other people have cognitive states, but rather what it is that I am attributing to them when I attribute cognitive
states to them. The thrust of the argument is that it couldn't be just computational processes and their output
because the computational processes and their output can exist without the cognitive state. It is no answer to
this argument to feign anesthesia. In 'cognitive sciences" one presupposes the reality and knowability of the
mental in the same way that in physical sciences one has to presuppose the reality and knowability of physical
objects.

VI. The many mansions reply (Berkeley). "Your whole argument presupposes that AI is only about analogue
and digital computers. But that just happens to be the present state of technology. Whatever these causal
processes are that you say are essential for intentionality (assuming you are right), eventually we will be able to
build devices that have these causal processes, and that will be artificial intelligence. So your arguments are in
no way directed at the ability of artificial intelligence to produce and explain cognition."
I really have no objection to this reply save to say that it in effect trivializes the project of strong AI by
redefining it as whatever artificially produces and explains cognition. The interest of the original claim made on
behalf of artificial intelligence is that it was a precise, well defined thesis: mental processes are computational
processes over formally defined elements. I have been concerned to challenge that thesis. If the claim is
redefined so that it is no longer that thesis, my objections no longer apply because there is no longer a testable
hypothesis for them to apply to.

Let us now return to the question I promised I would try to answer: granted that in my original example I
understand the English and I do not understand the Chinese, and granted therefore that the machine doesn't
understand either English or Chinese, still there must be something about me that makes it the case that I
understand English and a corresponding something lacking in me that makes it the case that I fail to understand
Chinese. Now why couldn't we give those somethings, whatever they are, to a machine?

I see no reason in principle why we couldn't give a machine the capacity to understand English or Chinese,
since in an important sense our bodies with our brains are precisely such machines. But I do see very strong
arguments for saying that we could not give such a thing to a machine where the operation of the machine is
defined solely in terms of computational processes over formally defined elements; that is, where the operation
of the machine is defined as an instantiation of a computer program. It is not because I am the instantiation of a
computer program that I am able to understand English and have other forms of intentionality (I am, I suppose,
the instantiation of any number of computer programs), but as far as we know it is because I am a certain sort
of organism with a certain biological (i.e. chemical and physical) structure, and this structure, under certain
conditions, is causally capable of producing perception, action, understanding, learning, and other intentional
phenomena. And part of the point of the present argument is that only something that had those causal powers
could have that intentionality. Perhaps other physical and chemical processes could produce exactly these
effects; perhaps, for example, Martians also have intentionality but their brains are made of different stuff. That
is an empirical question, rather like the question whether photosynthesis can be done by something with a
chemistry different from that of chlorophyll.

But the main point of the present argument is that no purely formal model will ever be sufficient by itself for
intentionality because the formal properties are not by themselves constitutive of intentionality, and they have by
themselves no causal powers except the power, when instantiated, to produce the next stage of the formalism
when the machine is running. And any other causal properties that particular realizations of the formal model
have, are irrelevant to the formal model because we can always put the same formal model in a different
realization where those causal properties are obviously absent. Even if, by some miracle Chinese speakers
exactly realize Schank's program, we can put the same program in English speakers, water pipes, or
computers, none of which understand Chinese, the program notwithstanding.

What matters about brain operations is not the formal shadow cast by the sequence of synapses but rather the
actual properties of the sequences. All the arguments for the strong version of artificial intelligence that I have
seen insist on drawing an outline around the shadows cast by cognition and then claiming that the shadows are
the real thing. By way of concluding I want to try to state some of the general philosophical points implicit in the
argument. For clarity I will try to do it in a question and answer fashion, and I begin with that old chestnut of a
question:

"Could a machine think?"

The answer is, obviously, yes. We are precisely such machines.

"Yes, but could an artifact, a man-made machine think?"

Assuming it is possible to produce artificially a machine with a nervous system, neurons with axons and
dendrites, and all the rest of it, sufficiently like ours, again the answer to the question seems to be obviously, yes.
If you can exactly duplicate the causes, you could duplicate the effects. And indeed it might be possible to
produce consciousness, intentionality, and all the rest of it using some other sorts of chemical principles than
those that human beings use. It is, as I said, an empirical question. "OK, but could a digital computer think?"
If by "digital computer" we mean anything at all that has a level of description where it can correctly be
described as the instantiation of a computer program, then again the answer is, of course, yes, since we are the
instantiations of any number of computer programs, and we can think.

"But could something think, understand, and so on solely in virtue of being a computer with the right sort of
program? Could instantiating a program, the right program of course, by itself be a sufficient condition of
understanding?"

This I think is the right question to ask, though it is usually confused with one or more of the earlier questions,
and the answer to it is no.

"Why not?"

Because the formal symbol manipulations by themselves don't have any intentionality; they are quite
meaningless; they aren't even symbol manipulations, since the symbols don't symbolize anything. In the linguistic
jargon, they have only a syntax but no semantics. Such intentionality as computers appear to have is solely in
the minds of those who program them and those who use them, those who send in the input and those who
interpret the output.

The aim of the Chinese room example was to try to show this by showing that as soon as we put something into the system that really does have intentionality (a man), and we program him with the formal program, you
can see that the formal program carries no additional intentionality. It adds nothing, for example, to a man's
ability to understand Chinese.

Precisely that feature of AI that seemed so appealing -- the distinction between the program and the realization
-- proves fatal to the claim that simulation could be duplication. The distinction between the program and its
realization in the hardware seems to be parallel to the distinction between the level of mental operations and the
level of brain operations. And if we could describe the level of mental operations as a formal program, then it
seems we could describe what was essential about the mind without doing either introspective psychology or
neurophysiology of the brain. But the equation, "mind is to brain as program is to hardware" breaks down at
several points among them the following three:

First, the distinction between program and realization has the consequence that the same program could have
all sorts of crazy realizations that had no form of intentionality. Weizenbaum (1976, Ch. 2), for example, shows
in detail how to construct a computer using a roll of toilet paper and a pile of small stones. Similarly, the Chinese
story understanding program can be programmed into a sequence of water pipes, a set of wind machines, or a
monolingual English speaker, none of which thereby acquires an understanding of Chinese. Stones, toilet paper,
wind, and water pipes are the wrong kind of stuff to have intentionality in the first place -- only something that
has the same causal powers as brains can have intentionality -- and though the English speaker has the right
kind of stuff for intentionality you can easily see that he doesn't get any extra intentionality by memorizing the
program, since memorizing it won't teach him Chinese.

Second, the program is purely formal, but the intentional states are not in that way formal. They are defined in
terms of their content, not their form. The belief that it is raining, for example, is not defined as a certain formal
shape, but as a certain mental content with conditions of satisfaction, a direction of fit (see Searle 1979), and the
like. Indeed the belief as such hasn't even got a formal shape in this syntactic sense, since one and the same
belief can be given an indefinite number of different syntactic expressions in different linguistic systems.

Third, as I mentioned before, mental states and events are literally a product of the operation of the brain, but
the program is not in that way a product of the computer.

-Well if programs are in no way constitutive of mental processes, why have so many people believed the
converse? That at least needs some explanation."

I don't really know the answer to that one. The idea that computer simulations could be the real thing ought to
have seemed suspicious in the first place because the computer isn't confined to simulating mental operations, by
any means. No one supposes that computer simulations of a five-alarm fire will burn the neighborhood down or
that a computer simulation of a rainstorm will leave us all drenched. Why on earth would anyone suppose that a
computer simulation of understanding actually understood anything? It is sometimes said that it would be
frightfully hard to get computers to feel pain or fall in love, but love and pain are neither harder nor easier than
cognition or anything else. For simulation, all you need is the right input and output and a program in the middle
that transforms the former into the latter. That is all the computer has for anything it does. To confuse simulation
with duplication is the same mistake, whether it is pain, love, cognition, fires, or rainstorms.

Still, there are several reasons why AI must have seemed and to many people perhaps still does seem -- in
some way to reproduce and thereby explain mental phenomena, and I believe.we will not succeed in removing
these illusions until we have fully exposed the reasons that give rise to them.

First, and perhaps most important, is a confusion about the notion of information processing: many people in
cognitive science believe that the human brain, with its mind, does something called -information processing,"
and analogously the computer with its program does information processing; but fires and rainstorms, on the
other hand, don't do information processing at all. Thus, though the computer can simulate the formal features
of any process whatever, it stands in a special relation to the mind and brain because when the computer is
properly programmed, ideally with the same program as the brain, the information processing is identical in the
two cases, and this information processing is really the essence of the mental.

But the trouble with this argument is that it rests on an ambiguity in the notion of '- information." In the sense in
which people "process information" when they reflect, say, on problems in arithmetic or when they read and
answer questions about stories, the programmed computer does not do -information processing." Rather, what
it does is manipulate formal symbols. The fact that the programmer and the interpreter of the computer output
use the symbols to stand for objects in the world is totally beyond the scope of the computer. The computer, to
repeat, has a syntax but no semantics. Thus, if you type into the computer '2 plus 2 equals?" it will type out '-4."
But it has no idea that -4" means 4 or that it means anything at all. And the point is not that it lacks some
second-order information about the interpretation of its first- order symbols, but rather that its first-order
symbols don't have any interpretations as far as the computer is concerned. All the computer has is more
symbols.

The introduction of the notion of "information processing" therefore produces a dilemma: either we construe the
notion of "information processing" in such a way that it implies intentionality as part of the process or we don't. If
the former, then the programmed computer does not do information processing, it only manipulates formal
symbols. If the latter, then, though the computer does information processing, it is only doing so in the sense in
which adding machines, typewriters, stomachs, thermostats, rainstorms, and hurricanes do information
processing; namely, they have a level of description at which we can describe them as taking information in at
one end, transforming it, and producing information as output. But in this case it is up to outside observers to
interpret the input and output as information in the ordinary sense. And no similarity is established between the
computer and the brain in terms of any similarity of information processing.

Second, in much of AI there is a residual behaviorism or operationalism. Since appropriately programmed
computers can have input-output patterns similar to those of human beings, we are tempted to postulate mental
states in the computer similar to human mental states. But once we see that it is both conceptually and
empirically possible for a system to have human capacities in some realm without having any intentionality at all,
we should be able to overcome this impulse. My desk adding machine has calculating capacities, but no
intentionality, and in this paper I have tried to show that a system could have input and output capabilities that
duplicated those of a native Chinese speaker and still not understand Chinese, regardless of how it was
programmed. The Turing test is typical of the tradition in being unashamedly behavioristic and operationalistic,
and I believe that if AI workers totally repudiated behaviorism and operationalism much of the confusion
between simulation and duplication would be eliminated.

Third, this residual operationalism is joined to a residual form of dualism; indeed strong AI only makes sense
given the dualistic assumption that, where the mind is concerned, the brain doesn't matter. In strong AI (and in
functionalism, as well) what matters are programs, and programs are independent of their realization in
machines; indeed, as far as AI is concerned, the same program could be realized by an electronic machine, a
Cartesian mental substance, or a Hegelian world spirit. The single most surprising discovery that I have made in
discussing these issues is that many AI workers are quite shocked by my idea that actual human mental
phenomena might be dependent on actual physical/chemical properties of actual human brains.

But if you think about it a minute you can see that I should not have been surprised; for unless you accept some form of dualism, the strong AI project hasn't got a chance. The project is to reproduce and explain the mental
by designing programs, but unless the mind is not only conceptually but empirically independent of the brain you
couldn't carry out the project, for the program is completely independent of any realization. Unless you believe
that the mind is separable from the brain both conceptually and empirically -- dualism in a strong form -- you
cannot hope to reproduce the mental by writing and running programs since programs must be independent of
brains or any other particular forms of instantiation. If mental operations consist in computational operations on
formal symbols, then it follows that they have no interesting connection with the brain; the only connection
would be that the brain just happens to be one of the indefinitely many types of machines capable of instantiating
the program.

This form of dualism is not the traditional Cartesian variety that claims there are two sorts of substances, but it
is Cartesian in the sense that it insists that what is specifically mental about the mind has no intrinsic connection
with the actual properties of the brain. This underlying dualism is masked from us by the fact that AI literature
contains frequent fulminations against "dualism'-; what the authors seem to be unaware of is that their position
presupposes a strong version of dualism.

"Could a machine think?" My own view is that only a machine could think, and indeed only very special kinds
of machines, namely brains and machines that had the same causal powers as brains. And that is the main
reason strong AI has had little to tell us about thinking, since it has nothing to tell us about machines. By its own
definition, it is about programs, and programs are not machines. Whatever else intentionality is, it is a biological
phenomenon, and it is as likely to be as causally dependent on the specific biochemistry of its origins as
lactation, photosynthesis, or any other biological phenomena. No one would suppose that we could produce
milk and sugar by running a computer simulation of the formal sequences in lactation and photosynthesis, but
where the mind is concerned many people are willing to believe in such a miracle because of a deep and abiding
dualism: the mind they suppose is a matter of formal processes and is independent of quite specific material
causes in the way that milk and sugar are not.

In defense of this dualism the hope is often expressed that the brain is a digital computer (early computers, by
the way, were often called "electronic brains"). But that is no help. Of course the brain is a digital computer.
Since everything is a digital computer, brains are too. The point is that the brain's causal capacity to produce
intentionality cannot consist in its instantiating a computer program, since for any program you like it is possible
for something to instantiate that program and still not have any mental states. Whatever it is that the brain does
to produce intentionality, it cannot consist in instantiating a program since no program, by itself, is sufficient for
intentionality.