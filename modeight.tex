\part{How Should I Act?}
\label{ch.modeight}
\addtocontents{toc}{\protect\mbox{}\protect\hrulefill\par}
\chapter{Part 17: Normative Ethics}
Our focus in this module will be normative ethics. It is worth noting that many people confuse normative ethics and descriptive ethics. Descriptive Ethics is the study of how people do, in fact, think about ethics. Different cultures and people think about ethics differently, this is fine and totally normal. Normative Ethics is concerned how we should be motivated and how we should act. Confusing these two has lead many people to moral relativism, but it's rooted in a confusion, not a fact. Confusing these two things also is the main support for the Cultural Differences Argument which we saw before. The project in Normative Ethics is to find a theory which best explains our moral intuitions and best describes universally held moral standards. One way to think of the difference is that Descriptive Ethics is the realm of sociology, while Normative Ethics is the realm of philosophy.  Different theories have been made to best describe our moral intuitions and are still being made, each improving on the last (so there is progress). 
\section{Ethical Theories}
As we saw in the last module, Moral Relativism just can't be correct. The reasons for supporting it contradict each other or it leads to truly awful beliefs. If we still want to be Moral Realists, then we are going to need to go with something objective, a theory of morality which has a set of rules or guiding principles to tell us what to do.  Trying to answer this question has lead down two different paths, the first we will cover belongs to the \Glspl{consequentialism}, who hold that we can tell whether an action was morally right or wrong based on the consequences, hence the name. In general, this branch thinks in terms of well-being and happiness/sadness. The second path people have taken in trying to get to an Ethical Theory goes belongs to the \Glspl{non-consequentialism}, who hold that the consequences of an action have no baring on the morality. In general, this branch points to respect, free-will, the human spirit, and rationality to get the morally right actions. Sometimes you will find middle-ground theories, but these tend to come from the Consequentialists, not the Non-consequentialists. Both of these branches share one thing in common, though, they are trying to answer the question ``what should I do?" There are other ethical questions out there concerning different questions, for example, the field of Virtue Ethics, found in Aristotle and most of Eastern Philosophy, is trying to answer the question ``who should I be?", which is very different. It seems clear that a good person can make a moral error and a bad person can do the right thing.  For this class, we are mostly only going to be concerned with the `what should I do?' question, but other philosophy classes are out there which cover the other question.

\newglossaryentry{consequentialism}
{
  name=consequentialism,
  description={The general stance that the morality or permissibility of an action is determined by the consequences of that action. This is a family of theories which all share the basic command to promote the good, though they differ on what the good is and how one ought to promote it},
  plural=consequentialists
}


\newglossaryentry{non-consequentialism}
{
  name=non-consequentialism,
  description={The general stance that the morality or permissibility of an action is not determined by the consequences of that action. This is a family of theories which all share the basic idea that some actions are just wrong regardless of the good they might (will) produce; typically, these theories are duty-based, meaning that an action is moral if and only if you have a duty to do it and immoral if and only if you have a duty to refrain from doing it},
  plural=non-consequentialists
}

For both the consequentialist and non-consequentialist branches of the tree, we can add further stances by asking various questions. For example, if we ask the question ``Do I have moral duties to other people?" and get either ``no" or ``sometimes", then, likely, we are dealing with a flavor of ethical egoism, which is the stance that something is moral if it benefits you selfishly. You might have heard of this stance before, but by a different name. The most famous supporter of this was Ayn Rand.

\section{The Scientific Method for Ethics}

When you are given a general Ethical Theory, there is a scientific style method for testing it. When philosophers propose a hypothesis about what makes an action right or wrong (good or bad) (in this context, we call them theories), they come from looking at the facts about people’s general moral intuitions and why we think a person did right or wrong in their actions/why a person is a good or bad one.  The ‘why’ aspect is key. Most of the time, our moral intuitions are a gut instinct, from a sort of mental-shortcut. The vast majority of the time, moral disagreements can be settled by exposing the shortcuts. This moves the moral debate into a metaphysical one, one which we can debate about. 

The philosopher then looks at the predictions made by the theory, what it tells us to do in some case, and then checks to see whether it lines up with the world around us, and the generally accepted moral duties which we have. This is not just looking at the philosopher's own culture, rather it also comes from looking at the perspectives of other cultures, for example, there have been very interesting developments from comparing the moral intuitions of the Ashanti People of Ghana and how they line up with general moral principles, and if the prediction and the intuitions don't line up, the philosopher amends the theory and tries again. This is how we do things in the real world; we are told some principle, we follow it until it makes a wrong prediction, then we adjust and amend, and repeat. For this class, we will be using this argument structure:
\begin{earg}
    \item[]If theory A is correct, then action B is what I should do.
    \item[]Action B is not what I should do.
    \item[]Therefore, theory A is not correct.
\end{earg}
This also comes in a slightly different form. We chose which according to the prediction of the theory.
\begin{earg}
    \item[]If theory A is correct, then I shouldn't do action B.
    \item[]I should do action B.
    \item[]Therefore, theory A is not correct.
\end{earg}
We start with what the theory claims would be the case and then see whether it lines up with the data (for this class, the data is your gut instinct when it comes to the ethical case. There are areas in psychology and philosophy which actually test people's moral intuitions by, for example, putting them in a VR situation and measuring the response). This system works for most of the theories covered in this class. The trick is to figure out what the theory says about a particular case and then see whether you think that this is correct.

Sometimes, however, it is harder to pinpoint something that the theory gets wrong in terms of actions. In these cases, it pays to look into how the theory chooses whether an action is right or wrong. Those cases don’t tend to involve me making a fun story, which I do for almost all of them (there are exceptions, Divine Command Theory, Virtue Ethics, and Feminist Ethics). The more applicable the theory is to individual choices, the easier making a case where it makes a prediction is.

\chapter{Part 18: Utilitarianism}

Utilitarianism is based on the idea that happiness is good. It is Consequentialist in nature. Utilitarian thinkers have traditionally understood happiness in terms of pleasure and the absence of pain. The general ideas behind this theory are found the world over, for example, you can find this theory being discussed and spread in Ancient China, in the Mohist School, between 479 BCE and 221 BCE. It was first really formalized by Jeremy Bentham (for a fun fact about him, look up the Auto-Icon) and then was further developed by John Stuart Mill, who is its best known advocate. Mill, characterizes \gls{utilitarianism} as the view that “an action is right in-so-far as it tends to produce pleasure and the absence of pain.” Mill, also, has a fascinating life story as well, if you care to look into it.  If happiness, conceived of as pleasure and the absence of pain, is the one thing that has positive value and pain/sadness being pleasure's opposite (the one thing with negative value), then this criterion of right action should follow fairly quickly. This leads us to the Principle of Utility.

\newglossaryentry{utilitarianism}
{
  name=utilitarianism,
  description={A form of Consequentialism which says that the morality of our actions is determined by the amount of happiness caused and suffering prevented. This is a form of Direct Consequentialism which takes happiness as the intrinsic good and suffering as the intrinsic bad},
  plural=Utilitarian
}


In any given scenario, the actions we make in that scenario will have consequences. We assign those consequences value based on the amount of happiness (pleasure) caused and the amount of sadness (pain) caused, for all beings affected, that action's utility. The utility of an action is the net total of pleasure caused by the action minus any pain caused by that action. In calculating the utility of an action, we need to consider all of the effects of the action, both long run and short run. Given the utilities of all available courses of action, utilitarianism says that the correct course of action is the one that has the greatest utility. So an action is right if it produces the greatest net total of pleasure over pain of any available alternative action. Note that sometimes no possible course of action will produce more pleasure than pain. This is not a problem for utilitarianism as we’ve formulated it. Utilitarianism will simply require us to pursue the lesser evil. The action with the highest utility can still have negative utility.

There are a few things which are worth noting about Utilitarianism. First, this has no room for self-interest bias. You are not special in the moral figuring. To use this theory, you need to count your own wants and desires equally to all other beings who could be affected. For example, suppose that I really want a million dollars, so I rob a bank. If I only counted myself in the equation, then it would tell me that this is the right course of action (assuming I don't get caught). However, there are other people and I am not, as it were, an island. The grief and suffering caused to others by my action would greatly outweigh the benefit to myself, and that makes the action wrong.

Second, this theory only requires that you cause the most good (once the bad is subtracted). This good does not need to be evenly distributed. For example, suppose that a government has a surplus and the president wants to use that money to help the citizens. The Pres. has a couple of choices. She could distribute the money in the form of gas-cards to the top 90\% of the population. This would cause some happiness for the majority. The other option is to use the money to help fight starvation, provide education/work training, and otherwise help the bottom 10\%. While the first option causes good things for the most people, in the grand-scheme of things, dedicating the money to the bottom 10\% is the better option because the suffering removed is greater than the pleasure which would have been added. 

Third, this theory does not make a distinction between long term and short term consequences. This makes sense most of the time, but it could be an issue for the theory when it comes to practical use. For example, allowing your child to eat all of the candy and junk food they want will, likely, maximize their happiness in the short term. In the long-term, however, there are a bunch of different health problems and behavioral issues which arise from that sort of diet, making it wrong to feed a kid a ton of junk food.

Here is an example of a course of action I could take, supposing that I came into some money:

\noindent	
\begin{tabular}{p{1in}|p{1.5in}|p{1.5in}|p{1in}|p{1in}}
Options&Hedons (happy-points)&Dolors (unhappy-points)&Total\\\hline

Buying New Shoes&10&500&-490\\

Donating to Oxfam&500&10&490
\end{tabular}


For this case, suppose that I have a choice between buying myself a new pair of shoes or donate to the charity Oxfam, which is there to alleviate world hunger and poverty. Think of `hedons' and `dolors' as made up units of pleasure and pain. I like to think of them relatively, so we can roughly say that donating to Oxfam will cause 50x as much good as buying the new shoes, and it will also cause a 50th of the suffering. According to the Utilitarian, this is the route you should go, make the most hedons and the least dolors.  Remember, these are not actual units, they are just conceptual tools to help understand the theory. You can use different unit sizes for different cases, just keep them the same in the same case.

Utilitarianism has no room for the individual making the choice seeing themselves as special, weighing their happiness as greater. It’s the total happiness that matters, not just your happiness. So utilitarianism can call for great personal sacrifice. The happiness of a child during their lifetime might require great personal sacrifice from the parent/caretaker during the first few decades. Utilitarianism says that all beings are equal, morally speaking, so long as they are equal in their ability to feel pleasure and experience pain.

Likewise, Utilitarianism has no room for favoring the immediate consequences over the long term. In this primitive form, an actions utility covers all time after the action. So, while it might maximize a small child’s pleasure in the short run to be given ice cream whenever they want it, the long run utility of this might not be so good given the habits formed and the health consequences of an over-indulged sweet tooth.

Here is another example of this at work. In this case, each of these choices on how to spend my money have different ripples through time and different amounts of people that they affect. For an interesting exercise, think about how long the results of the actions will help others and how many they help.
	
\begin{tabular}{p{1in}|p{1.5in}|p{1.5in}|p{1in}|p{1in}}
Options&Hedons (happy-points)&Dolors (unhappy-points)&Total\\\hline
Buying a new laptop&40&20&20\\
Donating to a foodbank&80&20&60\\
Give money to broke relative&60&20&40
\end{tabular}

Like I said, you can use different sizes, just keep them the same in the same case. As we can see, the amount of suffering caused by my action is around the same across the board. But the amount of pleasure is different, the one with the most is to donate the money to a food bank. According to the utilitarian, this is what I should do.

\section{Why People Like This Theory}

There are several reasons why people tend to like this theory. In other words, there are several positive aspects of it which seem to get morality right. The first reason is that the theory is impartial. It does not have in it any sort of bias for or against a person or even animal. As Jeremy Bentham put it, it doesn't matter whether they can speak or whether they can problem-solve, rather it only matters whether they can suffer. As a result, it doesn't matter whether you are rich, poor, black, white, male, female, man woman, part of some religious sect, or none at all. Morally speaking, you are equally worthy of consideration. Sometimes we take this sort of equality for granted, but when this theory was being formulated, it was not the norm, and even today, equality of this sort is fought for. You don’t need to argue for it in utilitarianism, it comes built in.

The second reason people like this theory so much is it gives ample justification for moral claims. Utilitarians are no strangers to being controversial. Classical Utilitarians have been on the front lines of many periods of history: the abolishing of slavery (Bentham, 1748-1832), equal rights for women (John Stuart Mill, 1806-1873), and animal rights (Peter Singer 1946-present) to name a few. That said, utilitarians can give reasons why many of our deeply held moral beliefs are correct. They see this as a major plus for the view. For example, many of the things which we find repugnant morally, slavery, killing innocent people for no reason, and others, all tend to do more harm than good. At the same time, things we have strong moral feelings in favor of, helping others, telling the truth, bravery, and others, all tend to make the best outcomes.

Third, a really important feature in a theory is that it can provide practical advice about what to do in a given situation and how to resolve conflicts. When you are in a situation where you are not sure what to do, the utilitarian can apply their theory and tell you what the best option is, there’s something in the world which they can point to, the outcomes, and say which one is the best. Similarly, when two groups are in disagreement about what the morally right thing is, the Utilitarian can point to certain base-level ground-work facts about the case and resolve the issue.

The fourth plus which this theory has going for it is its flexibility. As I mentioned in the part about justification, certain rules tend to produce the best outcomes, so the utilitarian supports them. But, those rules are not absolutes, they are there because they tend to get the best result. Sometimes, we need to break those rules in extreme situations. Utilitarians are just fine with this and can easily give the reasons why those rules should be broken. Its flexibility lets it work in most any situation. Take this historical case as an example: In the winter between 1846-1847, members of the Donner Party, traveling west, found themselves in heavy mountain snows. About half of the 87 members of the party died after food and supplies ran out. Those left had a terrible choice: Starve to Death or Eat the Remains of their fellows. Some may think that the rules against cannibalism are absolutes. Not to be violated under any conditions. The Utilitarian disagrees. This disagreement is not based on cannibalism, but rather that no act-type is absolutely wrong. While it’s wonderful that many of us are against these sorts of actions, utilitarians understand that desperate times call for desperate measures.

\section{Concerns for Utilitarianism}

Now, this theory is really good at a lot of things, it is especially useful in high-stakes situations, but there are some concerns which need to be addressed and areas of improvement. The first is an epistemological worry which should have caught your eye in the last page. Often, We don’t know (and sometimes we can't know) what the consequences of our actions will be in the long term. Sometimes, even in the short term, we may be uncertain about what will happen. For example, what if one of the people affected is particularly strange and gets happiness from pain? These worries point to a concern about the practicality of the theory.  It's difficult to use it in our daily lives. This worry/concern does NOT take Utilitarianism out of running for being a good Normative Ethical Theory, though. As a normative ethical theory, utilitarianism is aimed at identifying the standard for right action, not telling when a particular action meets that standard. Setting the standard for right action and figuring out how to meet that standard are two different projects. But they aren't the only worries for this primitive form of Utilitarianism (Act-Utilitarianism). There are other worries which are troubling for the very core of the theory, the Principle of Utility.

\subsection{The Porky the Pig Problem Problem}

When we speak of utility as pleasure and the absence of pain, we need to take “pleasure” and “pain” in the broadest sense possible. There are social, intellectual and aesthetic pleasures to consider as well as sensual pleasures. Recognizing this is important to answering what Mill calls the “doctrine of swine” objection to Utilitarianism. This objection takes the Utilitarianism to be unfit for humans because it recognizes no higher purpose to life than the mere pursuit of pleasure. This objection only applies to treating pleasure and pain in their most basic, animalistic, senses.  This was not originally mine, but a version of it was given as a two day lecture when I was in community college. The very original version is not as fun to give and it was published by G.E. Moore in 1903. If you want to see one other objection, one which you could write about in the assignment as well as the original objection to hedonism (which is a more primitive form of utilitarianism, without special amendments, utilitarianism falls into both of these problems). But, here we go:

    \thoughtex{Porky the Pig Farmer}{Way deep in the back woods, there lives a pig farmer named Porky. Porky raises prize winning swine and sells them to cover his basic needs. He has no wife/husband and doesn’t really want one. One day, when he was particularly bored, he noticed his swine having fun wallowing in the mud. Thinking that this looked like a good time, he stripped down and jumped in, having a whale of a time. As time goes on, Porky needs more entertainment than merely wallowing with his sows. He starts thinking “man, that’s a pretty little piggy”. Over time, late in the evening, his neighbors start hearing strange sounds coming from the direction of Porky’s shack. They think nothing of it, maybe Porky is just doing some late night breeding to get better piggies for market. Porky is doing some late night breeding of a sort. He is engaged in bestiality! And O! Is he enjoying himself!}{Porky.jpg}

With all this information about how awesome and pleasurable Porky's life is with his piggy time, we now have to ask ``is what he is doing moral?" Well, according to Utilitarianism, if we take pleasure to be in the animalistic sense, it is.
\begin{earg}
    \item[]The morally right action is the one which produces the greatest amount of happiness/pleasure for the greatest number of people.
    \item[]If Porky doesn’t engage with his pigs in this way, he will have very few pleasures.
    \item[]If Porky does engage with his pigs in this way, he has a lot of pleasures.
    \item[]If he has a lot of pleasures, then the greatest amount of happiness over the greatest number will be served.
    \item[]Therefore, the morally right action is to engage with his pigs in this way.
\end{earg}
But this can't be right. The core of it is that there is no consent (among other things), which means that it can't be right.
\begin{earg}
    \item[]If Utilitarianism is correct, Porky’s actions would be morally permissible.
    \item[]Porky’s actions are not morally permissible.
    \item[]Therefore, Utilitarianism is not correct.
\end{earg}

\subsubsection{The Reply to Porky}

Because of this objection, Mill and others don't take pleasure in the animalistic sense, but rather in a far more broad sense, where the other intellectual and emotional pleasures are taken into account. Mill responds that it is the person who raises this objection that portrays human nature in a degrading light, not the utilitarian theory of right action. People are capable of pleasures beyond mere sensual indulgences and the utilitarian theory concerns these as well. Mill then argues that social and intellectual pleasures are of an intrinsically higher quality than sensual pleasure. This response seems OK to some, but others argue that a sufficient amount of physical pleasures can, in principle, outweigh the intellectual.

\subsection{The Utility Monster Objection}

One objection to Utilitarianism isn't concerned with what it measures to determine morality or even how it is measured, rather this objection concerned how it determines the morally right action given the outcomes. As Act-Utilitarianism sits right now, it claims that the morally right action is the one with the highest outcome given the available options. This, however, leads to the possibility of a utility monster. A utility monster is a being which receives a massive quantity of pleasure (happiness, the good) from consuming resources, higher than any other being, by a significant margin, often at the expense of others. Just a cursory glance over the numbers would have Act-Utilitarianism claim that such a utility monster is doing the right thing in exploiting or harming others for their own benefit, because of the massive amount of good which they receive. To put this idea as an example, take the following case:

\thoughtex{The Utility Monster}{Peter Singer\footnote{Peter Singer is an Australian philosopher and is best known for his, seemingly, extreme views regarding Utilitarianism. He holds that the theory is correct and applies it to many contemporary issues. For example, one of his foundations, The Life You Can Save, tracks the spending habits of various charities and connects donors with the one which will get the most `bang for their buck' in the issue which they are concerned with. Singer, holding true to this belief, lives well below his means and donates substantial amounts to charity. He is mostly concerned with world hunger and poverty, but he has been known to be outspoken about animal `rights' and welfare.}, late in the evening, while he is resting at home, hears loud banging and commotion coming from his basement. So, he grabs his flashlight and goes down to investigate. He sees a green, round, being with arms and legs tearing apart Singer's plumbing causing massive flooding. ``What are you?" Singer exclaims, ``and what are you doing to my pipes?" The creature pauses their destruction for a moment and turns to Singer. ``I am destroying your plumbing" they explain, ``you see, I love wrecking pipes, far more than any suffering caused to you. I am a utility monster."}{UtilityMonster.jpg}

In this case, most of us would say that it's wrong for the utility monster to destroy Singer's pipes. There are concerns about personal property and there are concerns about the harm done to Singer. However, Act-Utilitarianism, without any modifications, measures pleasure and pain with the same metric. One unit of happiness cancels out one unit of sadness. As a result, so long as the Utility Monster didn't have any other options which would cause a higher total, Act-Utilitarianism would say that they did the right thing. This does not square with general intuitions about morality. 

Act-Utilitarianism gets the wrong result in this sort of case. And this objection is so basic, to the core of Act-Utilitarianism, that it might lead us to a change in theory, amend the theory to better fit our intuitions. One possible alteration is to change what is measured to determine morality. In this case, rather than measuring happiness and sadness equally, you only measure the sadness caused by the action or inaction. This is called negative utilitarianism. Negative Utilitarianism was proposed, most notably, by Karl Popper in his work On the Open Society and Its Enemies. Popper states that the morally right action is the one which minimizes suffering, rather than maximizing pleasure. Going this route avoids both the Utility Monster Objection as well as the Porky the Pig Problem but it also leads to certain other problems, if you take the letter, rather than the spirit, of the moral theory. 

\subsection{The Organ Harvest Problem}

This problem is often seen as a more gruesome version of the Trolley Problem for Ethics. This objection to Act-Utilitarianism stems from the idea that only the results matter, the ends justify the means. I personally have other versions of problems like this which involve framing an innocent person, forging evidence, and rigging elections, all of which have (due to the situation that they are in) the best consequences. Consider the following case:

    \thoughtex{A Check-Up}{Bob goes to the doctor for a check up. His doctor finds that Bob is in perfect health. And his doctor also finds that Bob is biologically compatible with six other patients she has who are all dying of various sorts of organ failure. Let’s assume that if Bob lives out his days he will live a typically good life, one that is pleasant to Bob and also brings happiness to his friends and family. But we will assume that Bob will not discover a cure for AIDs or bring about world peace. And let us make similar assumptions about the six people suffering from organ failure.}{OrganHarvest.jpg}

The question for the Act-Utilitarian is ``what should the doctor do?" According to the theory, it seems that the good doctor should quickly and as painlessly as possible kill Bob and harvest his organs, getting them to the 6 other patients as quickly as possible. This is because, to quote Spock, ``the needs of the many outweigh the needs of the few." The overall outcome of letting Bob go is the value of one average good life minus the values of six average good lives and the overall outcome of killing Bob is the value of six average good lives minus the value of one. But this can't be right. Our intuitions clearly speak to the immorality of this.  Act-Utilitarianism, again, gets the wrong result in this sort of case. 

\section{Rule Utilitarianism}

After having the concerns with Act-Utilitarianism, more particularly, cases similar to the Organ Harvest Problem, many people have thought to amend Utilitarianism to better fit those cases. These amendments are perfectly fine, but they have reached the point where they have developed into two different theories in this family. The first, which we will cover in this class, is called Rule Utilitarianism and the second, which we will not be covering, is called Negative Utilitarianism.

Rule Utilitarians move away from the idea that we need to max-out the utility of a given action. These Consequentialists look at the rule which was followed in making the action (hence the name). They say to act according to the rule which, if followed by everyone, would lead to the greatest utility generally. So, for example, if we look at the Organ Harvest Problem, a rule which had doctors kill their patients to save others (kill one to save six) would not have a very high utility if all doctors followed it. People would stop going to the doctor, they wouldn't trust medicine, etc. On the other hand, a rule which had doctors do no harm (forcing a difference between doing and allowing harm) would lead to a greater utility overall. We trust our doctors, in part, because of this. So, a move to this sort of Rule Utilitarianism might be a step in the right direction. But, there is a third rule with an even bigger utility.

Consider the following rule: Doctors should never harm their patients except when doing so would maximize utility. Now suppose that doctors ordinarily refrain from harming their patients and as a result people trust their doctors. But in Bob’s case, his doctor realizes that she can maximize utility by killing Bob and distributing his organs. She can do this in a way that no one will ever discover, so her harming Bob in this special case will not undermine people’s faith in the medical system. The possibility of rules with “except when utility is maximized” clauses renders Rule Utilitarianism vulnerable to the same kinds of counterexamples we found for Act Utilitarianism. In effect, Rule Utilitarianism collapses back into Act Utilitarianism.

In order to deal with the original problem of Bob and his vital organs, the Rule Utilitarian must find a principled way to exclude certain sorts of utility maximizing rules. Rather than pursuing this further for the Utilitarian, I want to consider further just how simple Act Utilitarianism goes wrong in Bob’s case (and I say this as a person who honestly thinks Consequentialism is correct). Utilitarianism evaluates the goodness of actions in terms of their consequences. Utilitarian considerations of good consequences seem to leave out something that is ethically important. Specifically, in this case, it leaves out a proper regard for Bob as person with a will of his own. What makes Bob’s case a problem case is something other than consequences, namely, his status as a person and the sort of regard this merits. This problem case for utilitarian moral theory seems to point towards the need for a theory based on the value of things other than an action’s consequences.

\chapter{Part 19: Kantianism}

Like Utilitarianism, Imannual Kant’s moral theory states that there is an absolute moral rule. But, unlike Utilitarianism, it states that the consequences of the action don't matter morally. It is, however, grounded in a theory of intrinsic value. Rather than going with the Principle of Utility, Kant's theory has it that the only thing with moral worth is the Good Will, which we find in persons. Persons, for this theory are  autonomous rational moral agents. This theory, from the very start, makes a certain metaphysical assumption: People have free will. This can't be the kind of free will proposed by the Compatibilist in the sense we have seen in this class, as Kant was fundamentally opposed to it, so the sense of free will must be the Libertarian sense. There have been attempts to make a Kantian Compatibilism, but those seem to have been unsuccessful.

\glspl{kantianism} rests on this notion of the \glspl{good will}, so we should be clear about what that is. The opening passage of Immanuel Kant’s Groundwork for a Metaphysic of Morals proclaims that “it is impossible to conceive of anything in the world, or indeed beyond it, that can be understood as good without qualification except for a good will.” This eloquent sentence places this `Good Will' at the center of Kant's value theory. The one thing that has intrinsic value, for Kant, is the autonomous good will of a person. But we can't understand Kant's Good Will in the ordinary sense. In everyday discourse we might speak of someone being a person of good will if they want to do good things. But this would be a Consequentialist take. On Kant’s view, the person with Good Will wants  good things out of a sense of moral duty, not just some habit or tendency.
\factoidbox{The naturally generous philanthropist doesn’t demonstrate their good will through their giving according to Kant, but the selfish greedy person does show their good will when they give to the poor out of a recognition of their moral duty to do so even though they’d really rather not.}

\newglossaryentry{kantianism}
{
  name=kantianism,
  description={The philosophical theories proposed and defended by Immanual Kant},
  plural=Kant's Moral Theory
}

\newglossaryentry{good will}
{
  name=good will,
  description={The one intrinsic good according to Kant's non-consequentialist theory of morality. All rational autonomous people have this `good will'. It comes from our ability to rationally and freely choose to act on our duties. According to Kantianism, anything which thwarts or hinders either another's free will or ratonality is wrong},
  plural=Good Will
}


So, to be worthy of dignity and moral regard, for Kant, we need to have the ability to see what our moral duty is and act according to it. For Kant, then, in order to be worthy of moral consideration, we need to be able to act in a way which is opposed to our desires/conditioning. Having an autonomous good will with the capacity to act from moral duty is central to being a person in the moral sense and it is the basis, the metaphysical grounding, for an ethics of respect for persons. Now what it is to respect a person merits some further analysis.

Kant's version of the Principle of Utility, his fundamental principle to guide our actions and thereby give people the dignity and respect worthy of their Good Will is called The Categorical Imperative. An imperative is a command or order given. Kant, and others, explain this kind of imperative by contrasting it with another kind, a hypothetical imperative. A hypothetical imperative is a command, but it only applies conditionally according to your desires, what your goal is. For example, ``if you want to avoid traffic, leave 15min early." This imperative tells you what to do if you want to avoid traffic, but it fails to tell you what to do in a case where you don't want to avoid traffic. A categorical imperative (though Kant thinks there's only one) is different in that it tells you what to do regardless of your goal or desires. It applies according to the kind of action you are taking, not why you are taking it. Kant also holds that if there is a moral law, it will be the Categorical Imperative. Treating it that way, moral reasons must always overshadow any other sort of reasons which can be given. You might, for instance, think you have a self-interested reason to cheat on exam. But if morality is grounded in a categorical imperative, then your moral reason against cheating overrides your self-interested reason for cheating. If we think considerations of moral obligation trump self-interested considerations, Kant’s idea that the fundamental law of morality is a categorical imperative accounts for this nicely.

Although Kant gives 4 different statements of the Categorical Imperative and claims that they all boil down to the same basic idea, there is some debate about whether they do. Two of the formulations just seem to be restatements of the other two, and the debate is over whether these two can be formed together. Once you see them, you will see that they certainly don't look like they are expressing the same idea. The first formulation which we will cover is called The Principle from Humanity, for this class, though other names for it are out there. The second formulation is called The Principle from Universalizability for this class, though, again, other names are out there.

\section{The Principle From Humanity}
\begin{center}
Always treat persons (including yourself) as ends in and of themselves and never as a mere-means
\end{center}
This formulation tells us to treat individuals as ends in themselves, but what does that mean? How do I use people as mere-means? How could I use myself that way? This formulation or principle is noted for really highlighting the notion that people are intrinsically valuable. To say that persons have intrinsic value is to say that they have value independent of their usefulness for this or that purpose.  They are not a tool or resource which you can use without consideration of their worth. the Principle from Humanity does not say that you can never use a person for your own purposes (using them as a means). If this were the case, you taking a class from me would be morally wrong. It tells us never to use others as a mere-means.

\subsection{Means Vs Mere-Means}

We treat people as a means to our own ends in ways that are not morally problematic all the time. When I go to a grocery store to pick up some food, I treat the clerk as a means to my end of buying food. But I do not treat that person as a mere-means to my own end. I accomplish my end of buying food through my interaction with the clerk only with the understanding that the clerk is acting autonomously in serving me. My interaction with the clerk is morally acceptable so long as the clerk is serving me voluntarily, or acting autonomously for his own reasons.

By contrast, we use someone as a mere-means to our own ends if we force them to do our will, or if we deceive them into doing our will. We are, in a sense, using them without their consent, where it would not be possible for them to consent. Sometimes we use people as mere-means when we don't take their goals into account in our interactions with them.

    \thoughtex{Smith Tower}{Suppose that I have the goal of building a tower with my name emblazoned on the top.  I have the money to do this and I hire workers to build it (they have the understanding that they will get paid when the job is done). Just before they finish the work and would expect to get paid, I take all of my money, put it in an off-shore account under my son's name and declare bankruptcy. Using the laws in this regard, I get the contracts waved so that I don't pay the workers.  Later, when the dust settles, I transfer the money back into my name.}{SmithTower.jpg}

In doing this, I have used those workers as mere-means. I have the goal of getting the building, they have the goal of getting paid. In my interactions with them, according to Kantianism, I should have, in a sense, made their goals my own and had the intermediate end of paying them for the work. In the case above, the workers and I do not share the same end, so I am using them merely as a means.

Coercion and deception are paradigm violations of the categorical imperative. In coercing or deceiving another person, we disrupt their autonomy and violate their will. This is what the categorical imperative forbids. Respecting persons requires refraining from violating their autonomy.

Here is an example of this sort of theory at work, where it seems to get the right answer:
\begin{earg}
    \item[] To take a person’s life, liberty, or legitimately acquired property without that person’s consent is to use that person as a means to an end (if they give consent, then they are being treated as an end). It is to treat a person as a tool or resource placed here for your convenience.
    \item[] It is always wrong to treat a person in that way.
    \item[] Therefore, each person has a moral right to life, liberty, and property, regardless of the consequences.
\end{earg}
So, killing is always wrong, so is stealing. Another example of this formulation getting things right comes up when we talk about slavery (though the Utilitarians were on the front-lines against slavery before the Kantians):
\begin{earg}
    \item[] Each person is intrinsically valuable regardless of race, religion, ethnicity, or national origin and deserves to be treated as such.
    \item[] Slavery does not treat people as an intrinsically valuable being.
    \item[] Therefore, slavery is morally wrong.
\end{earg}
\section{Problems for Humanity}
\subsection{Taxation}
\begin{earg}
    \item[] Taxation is the taking of a person’s property and using it to benefit others, without their consent.
    \item[] Taking a person’s property without their consent is treating them as a mere means to an end.
    \item[] Treating a person as a mere means to an end is always morally wrong.
    \item[] Therefore, taxation is always morally wrong.
\end{earg}
Some of my more conservative students may like the sound of that, ``Taxation is Theft" they tend to shout. But many of you will have a problem with the consequences of taxation always being morally wrong. Public schools, as they are paid by taxes, gone. Most public roads, if they do not lead to some person's business and they did not pay for it, gone. The cost of your college education will sky-rocket, as they are subsidized by taxes. And many others.
\subsection{Lying}

This is an interesting case, and one which Kant himself thought was right, as in he thought that the theory got the correct answer here, however, this does not line up with most people's moral intuitions about the case. Take this case as an example (which will appear in the discussion for this module):

    \thoughtex{Kant's Axe}{Your roommate is in the shower after having a late night with a recently gained boy/girlfriend, who had gone through a recent bad break-up (their former significant other is ``the ex", this is not the ex of your roommate, for clarity) . You hear a knock on the door and go to answer it. You find the ex standing at the door with an axe, murder in his/her eyes. The ex asks you “Where is [insert roommate name]?” Neither of you can hear the shower running. You know that if you tell the ex, then they will rush right through you/knock you out/whatever and get to the roommate killing them.}{KantsAxe.jpg} 

According to The Humanity Principle, we should never use others as a mere-means. But, does lying count as a mere-means? According to Kant, it does. Kant says that in lying, you are misinforming one person for the benefit of another (or yourself), without appreciating their intrinsic worth. This is always wrong, according to the principle, so lying, no matter what, is always wrong. Sure, you could refuse to answer, slam the door, and so on, but you can never lie. This means that lying to the axe-murderer is wrong. So, if you need to speak, then you must tell them that your roommate is in the shower, and more than likely, deal with the Psycho scene later. 

However, this is an issue, because it certainly seem right that there are cases where lying is permissible, so this seems wrong, too hard lined. 
\subsection{The Rendering Aid Problem}

This particular problem addresses something interesting in this formulation of the Categorical Imperative (and this feature should be found in any rephrasing of it, as it's core). In it, it states that we should not use ourselves as mere-means. Since Kant explicitly stated that we shouldn't, it must thereby be possible to use ourselves as mere-means. This was quite the puzzle for me, personally, how could I possibly do something without taking my own goals into account? How could I force myself to do something against my will? The examples which Kant himself gives have not aged well, they involve sexual acts and suicide, which I don't want to use for this course. 

I have hence asked around and some examples do seem problematic and I will look at the issue of rendering aid. Often we glorify people who help others at the cost of their own wants and goals. In such a case, they are not taking their own goals into account and using themselves to benefit others. This, by its very nature, would be the person using themselves as a mere-means. Here is an example, in an ordinary case:

    \thoughtex{Changing a Tire}{Suppose that I have the goal of buying a new video game, The Elder Scrolls 20 or some such, and I know that given the shear demand for the game, if I am not there just as the shop opens, it will be sold out. As I am driving to the shop early,  I notice a young man having a hard time changing a tire. Feeling as though I should help, I pull over to render aid. This prevents me from making it to the shop on time and violates my end.}{changingtire.jpg}

In this case, it would seem that I have used myself as a mere-means and cases of rendering aid like this would be morally wrong.    
\section{The Principle from Universalizability}
\begin{center}
Act only on the maxim that you can consistently will to be a universal law
\end{center}
This version is also known as the formula of the universal law. The maxim of our action is the base-level reason or principle that determines what we are doing. We act for our own reasons and different goals might lead to similar actions. For example, a person might wash their clothes regularly because they don't want to smell bad while another person might do the same because they don't want their significant other to complain about the stack of clothes near the closet. Though they have the same action attached to them, the maxim behind the action will be different.  For Kant, intentions matter and this formulation really gets at this point. He evaluates the moral status of actions not according to the action itself or according to its consequences, but according to the maxim of the action. Whether an action is right or wrong  is determined by the actor’s intentions or reasons for acting.

It should be noted that a \gls{maxim} should not include your personal wants and desires, those would make it a hypothetical imperative. A maxim should be written/phrases as something along the lines of ``I will A in order to realize C." Where A is the action you want to perform and C is the reasoning behind the action. So, suppose that I am deep in debt to a loan shark because of my gambling habit. I go to the bank to get a loan to pay the loan shark and save my knee-caps. The maxim for my action here would be something along the lines of ``I will lie in order to get money."

\newglossaryentry{maxim}
{
  name=maxim,
  description={The reason and goal for which you are acting. This is typically of the form ``whenever I am A then I will E''}
}


According to this formulation, what makes an action morally acceptable is that its maxim is universalizable. That is, morally permissible action is action that is motivated by an intention that we can rationally will that others act on similarly. A morally prohibited action is just one where we can’t rationally will that our maxim is universally followed. Basically, ask yourself ``am I making a special exception for myself?" ``could anyone in my situation do this?" If anyone with similar desires could do what you are doing and accomplish them, then you are morally in the clear, otherwise you are morally up a creek.

Here is an example of this thought at work:

    \factoidbox{Suppose that I really want to win at a game, so I think about cheating. The principle I go on is “whenever I want to win, I will cheat in order to do so”. But, if everyone did this, the concept of a game would be mute, no one would play the game by the rules, so cheating would not be cheating. This is a contradiction, so cheating is always wrong.}

There is no higher moral authority than the rational autonomous person according to Kant. Morality is not a matter of following rules laid down by some higher authority. It is rather a matter of writing rules for ourselves that are compatible with the rational autonomous nature we share with other persons. We show respect for others through restraining our own will in ways that demonstrate our recognition of them as moral equals.
Problems for this one
\subsection{The Traffic Jam Problem}

Suppose that I regularly get caught in traffic at 6:45AM. I know that if I leave 15 minutes earlier, I will arrive where there’s traffic at 6:45AM at 6:30AM, missing the traffic. So, I ponder the following:
\begin{center}
    Whenever I want to avoid traffic, I will leave 15min earlier.
\end{center}
But, everyone wants to avoid traffic, so what would happen if everyone did this? If everyone wanted to avoid traffic and left 15min earlier, it is reasonable to say that the traffic would not be at 6:45AM, but now at 6:30AM. This means that if everyone left early to avoid traffic, they would not avoid traffic. This is a contradiction. If I leave early, I avoid traffic and if everyone leaves early, no one avoids traffic. Therefore, according to Kant, me leaving early is morally wrong.
\subsection{The Breaking Promises Problem}

Suppose that I promise my mother on her deathbed to sing and play a certain song on my banjo at her funeral. She asks me to play/sing In Hell I'll Be In Good Company, by The Dead South. But, on the day of the funeral, I can't bring myself to play such an inappropriate song, the lyrics in this context would make everyone's grieving worse. So, I leave the banjo to the side. The principle you are going with is “whenever I make a promise that I don’t want to keep, I will break that promise.” Now, what would happen if everyone broke promises they didn’t want to keep? If everyone broke promises they did not want to keep, then the very notion of promising would go out the window. If there are no promises, you can’t break them. That is a contradiction. If you promise, you break it. If everyone broke promises, there would be no promises. If there are no promises, you can’t break them. Therefore, if everyone broke promises, they can’t break promises. So, breaking promises is always morally wrong.

But in not playing that song, did I really do something wrong? Most will say no.

\chapter{The Moral Status of Bloodbending by Davis Smith}
\label{bloodbending}
\section{Introduction}
\newcounter{fd}
\setcounter{fd}{\thefootnote}
\setcounter{footnote}{0}

During their travels around the world of Avatar, our heroes have encountered many people with strange or unique bending abilities. For example, they encountered Combustion Man, who could make things explode with his mind.\footnote{First seen in “The Headband,” (23:24-23:48) but his abilities are displayed for the first time in “The Beach” (15:21-16:16).} None, however, are more troubling and ethically interesting than Hama, a waterbender from the Southern Tribe who can, on a full moon night, bend the very blood in a person’s veins and manipulate their body to do her bidding.\autocite{puppetmaster} Such a violation of a person’s control over themselves would, rightfully, be the stuff of nightmares. However, are all cases of this bloodbending wrong? To answer this question, we turn to two well-established theories in Ethics, an area of Philosophy which, among other things, tries to answer the question “what is the moral thing to do?” Almost as if by design, bloodbending works as a fantastic example of how these theories determine the morality of an action and how they can come to radically different conclusions on a single case.

The episode gives us four different situations in which bloodbending is used and each can serve as examples for the different ways one can evaluate an action to determine its morality. In all four cases, the bender is forcing a person to do something against their will. So, if our moral intuition or the theory gives different responses to the question “was that the right thing to do?”, then it is not the manipulation of a person which makes that difference. Because the cases are so different, if an ethical theory gives the same response to all four cases, then the manipulation is the cause.

\section{The Ethical Theories}

To start us off, I would like to give a basic overview of the different ways in which one could classify ethical theories, theories about whether some action is right or wrong. First, we can classify them according to what they use to make their determination. For this, there are two natural categories. On one side, we have theories which make their moral evaluation according to the consequences, the results of the action. These are called Consequentialist theories. The ultimate command for a Consequentialist theory is to leave the world a better place. Theories in this family differ in how they determine ‘better’ and in how they measure it, but they all share that primary command. On the other side, we have Non-Consequentialist theories. As the name implies, these theories are the opposite of the Consequentialist ones. They hold that the consequences are irrelevant to the morality of an action. Some actions, according to these theories, are just wrong, regardless of the outcome.

We could also classify them according to whether they are concerned with individual instances of an action or kinds of actions. Act-Type theories start by asking about the kind of action it is. These theories tend to move from general principles or commands about categories of actions and use those to determine moral status of an individual case. For example, one could have the general principle “lying is always wrong” or the command “don’t lie” and from that determine whether some case of lying is wrong (it is, according to that principle). On the other hand, some theorists think that this approach is too impersonal and not down-to-Earth. This is where we get the Act-Token theories. Rather than moving from some general principle to make a judgement, Act-Token theories make their evaluations by just looking at the case in isolation. These theories have no problem claiming that most of the time lying is wrong, but they are more than willing to make exceptions and give reasons for those exceptions.

These two different ways of classifying ethical theories gives us four different, generic, types of theories. In this paper, we will be focusing in on the two major ones, Utilitarianism (a Consequentialist Act-Token theory) and Kantianism (a Non-Consequentialist Act-Type theory). Because of their radically different stances regarding the measurement of morality and what is to be measured, these theories serve as a wonderful foundation to explore the moral status of bloodbending.

\section{The Utilitarian Account} 

Jeremy Bentham (1748-1832)\autocite{Bentham1}, John Stuart Mill (1806-1873)\autocite{Mill1}, and Harriot Taylor Mill (1807-1858)\footnote{Harriot Taylor Mill’s husband was J.S. Mill and by his own words, she was heavily influential in his formulations and refinements of Utilitarianism.} built off each other and formalized a theory of morality called Utilitarianism. This account of the permissibility of actions is Consequentialist in nature and, though it is possible for this account to use act-types, Mill and Bentham both certainly preferred focusing on act-tokens. These two features, almost immediately, give us a possible answer to the question “is bloodbending always wrong?”, namely, “no.” In addition to this, when we think about the more general question about whether it is permissible to violate a person’s autonomy, this theory would say “sometimes.” In fact, because it is an Act-Token theory, it will always allow for some exceptions to this kind of general rule.

There is one principle, one overarching metric, according to Utilitarianism, which cannot have exceptions, and which determines the morality of an action; The Principle of Utility.\autocite[pg.12]{Bentham1} This says that the action you should do is the one which results in the greatest amount of happiness (or well-being) once the suffering (sadness, pain) has been subtracted from it. It should be noted, however, that the doer’s interests and the effects on them are not treated as especially important. Your well-being is treated equally to all other human (and potentially non-human) souls. It is part of the core command of this theory that you think of others and ensure that, all else being equal, their lives are made better by your actions. Morally speaking, this theory has it that all beings are equal in so far as their ability to suffer or experience happiness is equal.\autocite[pg.245]{Bentham1}

The world of Avatar has no shortage of examples of Utilitarian style thinking, seemingly, getting morality correct, outside of cases of bloodbending. To start off, in Book 1, Episode 5, we first encounter one of my favorite side characters, the Cabbage Merchant. As he is attempting to enter the city of Omashu, we see the guards, without regard for the merchant’s desires or the effects of having a cabbage merchant in the city, use earthbending to launch his cart into a canyon.\autocite[02:53-03:07]{Omashu1} Given what we know about the Cabbage Merchant and the results of this, we can say that the guards were wrong in launching the cart. In that same episode, we find Team Avatar preparing to use the delivery system of the city as a slide, for their amusement. In doing so, they cause a lot of damage to various people’s property and they cause even further distress to the Cabbage Merchant.\autocite[04:54-07:29]{Omashu1} This destruction and the suffering, for lack of a better word, caused certainly outweighs the enjoyment had by the team. It follows from this that Utilitarianism would say that what they did was wrong. 

In Book 1, Episode 11, we encounter rival tribes (with radically different personalities) each seeking passage through a dangerous canyon.\autocite[00:00-23:37]{Divide1} The guide warns them that bringing food into the canyon will attract the carnivorous and dangerous animals. Both tribes end up bringing food into the canyon, believing that the other will. Given the risk that they are putting the people, collectively, in by doing so, we can see that this, like before, outweighs the benefit of not going a day (or so) without food. Utilitarianism, as a result, says that the tribes did something wrong here.  To finish off our set of examples, we turn to Book 2, Episode 2, where Iroh is deciding about whether to make tea out of a strange plant.\autocite[03:14-03:53, 06:11-07:06]{Lovers1} As he put it, this plant is either “delectable tea or deadly poison.” If we suppose that it is a 50-50 shot, we need to ask whether the temporary enjoyment from such a “heartbreaking” tea is equal to the suffering caused by his death. Since we know the negatives which would come from Iroh’s death, we can say rather confidently, using Utilitarian reasoning, that Iroh making the tea was not worth the risk.

\section{The Kantian Account}

Some people do not like this Utilitarian account. They think that it is missing something fundamental to the morality of actions. For example, Utilitarianism does not have a respect for personhood built in. Showing a person respect and dignity, for a Utilitarian, is seen as secondary. If you can get the best results while being respectful, do so, but the main driving force is to make the world a better place. This is where we get Kantianism, named after its inventor Immanuel Kant (1724-1804).\autocite{Kant1} Kant was a Non-consequentialist, meaning that he held that the consequences have no bearing on the morality of actions. Kant’s ethical theory, also, is based around act-types rather than tokens. This puts Kantianism in direct opposition to Utilitarianism.

For Kant, and the Kantians, there are many moral imperatives, commands, which we need to follow to in order to act morally. These imperatives act categorically according to the type of action it is. For example, you have simple commands like “don’t make false statements,” “don’t cheat,” and “don’t steal.” For this theory, we don’t need to wait until the dust settles to know whether we did the right thing, rather we just need to know what kind of action it was and what the imperative associated with it is.

Determining the imperative associated with a given act-type can be quite difficult. Luckily, however, according to Kant, they all reduce to one simple command. This is the Categorical Imperative.  Kant, himself, gives four different formulations of the Categorical Imperative and claims that they all mean essentially the same thing. There is no small debate about whether they do, but it is certainly clear that two of the four mean the same as the other two, so the debate is really about whether the two formulations get at the same thing. That being said, the most relevant formulation to our question about the moral status of bloodbending is the Formula from Humanity.\autocite[pg. 38]{Kant1} This imperative tells us to treat all of humanity, ourselves included, as an end in and of themselves and never merely as a means to an end. Within that formulation, there is a central concept which we need to be clear on before we can use the theory to evaluate bloodbending. This is the idea of using someone as a mere-means.

We all have goals and aspirations which we seek to accomplish in our lives. For Aang, it is to master all four elements and bring balance to the world. To accomplish these goals, we often need to use each other by getting help or through normal exchanges, such as Aang using Katara, Toph, and Zuko to learn the different styles of bending. In these cases, we are using them as a means to our end but we are not necessarily using them merely as a means. We take the other person’s goals and aspirations into account when we use them and, thereby, help them achieve their goals. Using someone merely as a means is a different matter entirely. In these cases, we do not take their goals into consideration, we trick, coerce, or force them to do something which they otherwise would not consent to. Kant believed that our autonomy (our free will) and our rationality were the most valuable things in the world. Any action which hindered or depreciated a person’s autonomy or rationality, regardless of the consequences, is wrong. In lying, for example, we are intentionally deceiving a rational autonomous agent for the betterment of ourselves or another. This is using them merely as a means because we are not appreciating their great worth.

As with the Utilitarian account, Avatar has several examples which we could apply Kantian thinking to in order to make a moral judgement. In the very first episode of the series, Aang lies about being the Avatar. In doing so, Aang is misinforming Katara and Sokka for his benefit, as he later admits.\autocite[11:31-11:52]{Iceberg1} In such a simple case, we can see that Kant would say that Aang did something wrong. In a similar situation, in Book 1, Episode 4, Team Avatar arrives at the island of Kyoshi, named after one of Aang’s past lives.\autocite[06:05-07:35]{Warriors1} There, Aang chooses to be honest about being the Avatar. Despite the consequences of this, namely Zuko finding out and going to the island, Kantianism says that Aang did the right thing, because him lying would have been wrong.

For other examples, in Book 1, Episode 9, Team Avatar encounters a band of pirates who have, through “high risk trading,” acquired a waterbending scroll, with forms and moves which Katara and Aang can learn to forward their quest to master the element.\autocite[07:04-10:00]{Scroll1} As the story progresses, we learn that Katara stole the scroll from the pirates. Even though the scroll was stolen by them, stealing from a thief is still stealing. In doing so, Katara used them as a mere means and violated the Categorical Imperative. This means that Kant would say that her theft was morally wrong. The guilt for the action is on Katara, just as the initial theft of the scroll by the pirates is on the pirates. Similarly, in Book 3, Episode 5, a few Fire Nation soldiers spot Team Avatar and the proceed to send a message to the Fire Lord.\autocite[03:46-04:20, 08:09-08:56]{Beach1} As the messenger hawk flies away, Combustion Man steals the scroll. Kant and those who think like him would say that Combustion Man did something wrong because he is stealing and the Categorical Imperative strictly forbids theft.

\section{The Theories Applied to Bloodbending}

As you may have guessed from my examples, both Kantianism and Utilitarianism truly shine when put to the test, used in cases, and the World of Avatar has no shortage of examples. For many of the moral cases in Avatar, Kantianism and Utilitarianism give remarkably different responses. In Book 2, Episode 3, in order to help the citizens of Omashu flee the Fire Nation invaders, Team Avatar and the populace use a small octopus-like creature to fake an epidemic, called ‘penta-pox’.\autocite[08:48-10:57]{Return1} Doing so was an affective and non-violent way to get out of the city and Utilitarianism would approve. Kantianism, on the other hand, holds that any form of deception, which is a kind of lying, is wrong. In Book 3, Episode 1, Team Avatar and their cohorts are on a stolen Fire Nation ship.\autocite[06:58-08:00]{Awakening1} The act of stealing this ship would be seen as fine according to Utilitarianism but be seen as morally wrong according to Kantianism. Similarly, in the same episode, they are impersonating Fire Nation soldiers.\autocite[08:46-10:00]{Awakening1} This is deceitful and is lying. Kantianism gives a clear and absolute prohibition to lying, meaning that they say this is wrong. But Utilitarians disagree. These fundamental disagreements become even more pronounced when we apply the theories to the four cases of bloodbending seen in The Puppetmaster.

The, chronologically, first case of bloodbending is Hama using it to escape from prison. Hama is in prison after being captured by the Fire Nation during one of their raids. Hama, feeling the power of the full moon, realizes that living creatures, like elephant-rats and humans, have blood in their veins and that blood has water. Using her waterbending, Hama learns to manipulate that blood in the elephant rats and use them as puppets. This is not, however, enough to make her escape. On another full moon, Hama uses bloodbending to force a guard to open her cell and thereby escape from the prison.\autocite[18:04-19:33]{puppetmaster} The moral question which this scene raises and which these theories need to answer is whether it was permissible for Hama to use her skill to escape.

Looking at this prison escape, we see that Utilitarianism says that Hama did the right thing. In fact, she should have continued to use bloodbending to free the other Southern Water Tribe benders as well (which she may have). Doing so, in this case, results in the best outcome given her options, freedom to pursue her happiness and (in freeing the others) their happiness and, according to Utilitarianism, the right action is the one with the best consequences given your options. Many of us will likely agree with this assessment. Morally speaking, it was wrong for the Fire Nation to imprison those benders in the first place, so it seems only right to use her skills to escape.

Kantianism, on the other hand, holds that Hama did something wrong, regardless of the consequences. Hama bloodbends the guard of the prison and thereby forces him to release her. In doing this, she is using the guard merely as a means to an end (namely, escape from prison). She does not care about the goals or aspirations of the guard, rather she is forcing him to do something which he would not do otherwise. Kant and those who think like him would likely say that there is nothing wrong with her escaping from prison, rather it was wrong for her to use bloodbending to get do it. This is in stark contrast to Utilitarianism, which says that the results are all that matter, the steps used to get there are not relevant (unless there is a better way). Kant does not think that the ends justify the means, how you achieve a goal is just as important, if not more important, than what you achieved.

For the next case, we find Hama, significantly older, settled down in a Fire Nation Village. She is still a strong waterbender and still knows how to bloodbend. On full moon nights, Hama uses this ability to force people to leave their homes and walk deep into the forest, where she has a cave and she chains them to the walls, making them her prisoners.\autocite[15:12-16:49, 17:10-18:03]{puppetmaster} Her motivation is vengeance, not just against the soldiers who imprisoned her or against the soldiers which raided her tribe, but rather against the entirety of the Fire Nation. In her mind, all Fire Nation citizens are guilty and worthy of her wrath. For this case, the questions which needs answering is whether it was permissible for Hama to use her skill to have vengeance in this way.

Utilitarianism would say that Hama kidnapping these citizens, regardless of the method used, is wrong. We likely will agree with this assessment too. Hama’s actions are certainly causing more harm than good. The people of the village are terrified and those who she as imprisoned are experiencing similar pain and suffering to her own imprisonment years before. There are very little, if any, positive results from her kidnapping of the villagers. Since the suffering outweighs the happiness caused, Hama’s actions are wrong, according to Utilitarianism. The Kantians agree with this assessment, but for different reasons. According to Kantianism, Hama is clearly compelling people to do something which they would not normally do, namely be chained up in a cave. She is not taking their goals and aspirations into account and is using them as a mere-means. We need to note that the pain, suffering, and emotional turmoil which Hama is causing does not enter into the Kantian framework. That would be consequentialist and Utilitarian thinking, which is not Kantian. Hama’s actions here are wrong solely because of the violation of their autonomy.  

In the third case, we have Hama and Katara using waterbending to fight each other. Katara, a skilled waterbender in her own right, uses some of the skills which she learned from Hama to pull water from the surrounding plant life. Hama switches tactics and uses bloodbending to take control of Aang and Sokka’s bodies, using them to fight for her against Katara. Hama then forces Sokka to raise his Space-sword and fly towards Aang.\autocite[22:08-22:53]{puppetmaster} If nothing is done, Aang will die. The question here is whether it was permissible for Hama to use bloodbending to force Sokka to attempt to kill Aang.

Utilitarianism gives a similar response to The Kidnapping, claiming that Hama was doing something wrong. This is because, had Hama succeeded, she would have killed the Avatar, which would have stopped him from defeating the Fire Lord and ending the war. Not only that, but it would have allowed the destruction of the Earth Kingdom at the Fire Lord’s hands, because Aang would not have been there to stop it. All of the pain and suffering which Aang will have prevented would equally fall on Hama’s shoulders. All of those extreme factors make for a very clear Utilitarian response; Hama was doing something wrong. This example, in particular, leads to a possible response. Hama was ignorant of these facts, does that enter into it? For Utilitarianism, a person’s knowledge does not change the morality of an action. The Principle of Utility does not have room for the state of mind the doer is in. If a person is ignorant of the potential consequences of their actions, we may hold them less responsible, we may place less blame on them for their wrongdoing, but that does not make what they did any less wrong.

For this case, Kant would agree with the Utilitarians and say that Hama was in the wrong. Again, different reasoning is used. Killing a person (either yourself or another), especially for your own benefit, is the ultimate case of using them as a mere-means. In doing so, you are removing their rationality and autonomy from the world and, at the same time, violating that autonomy because you are removing all of their abilities. This reasoning, again, is very different than the kind which the Utilitarian would use. The Utilitarian does not think that killing a person is always wrong, rather they work on a case-by-case basis. In choosing whether to kill the Fire Lord, the Utilitarian would likely say that killing him would be a better choice than merely removing his bending, as there would be less net suffering in the world because of it. The Kantian would disagree, killing a person is always wrong.

For the final case, we pick up right where the previous left off. Sokka, sword raised, is flying towards Aang. Katara is the only one who can save them, and she only has one option, use bloodbending on Hama to stop her. This would save our heroes and subdue Hama. Katara, realizing this, uses bloodbending for the first time and defeats Hama, who is captured by her former prisoners and taken away. As she leaves, Hama says “my work is done. Congratulations, Katara. You’re a bloodbender.”\autocite[22:53-23:43]{puppetmaster} The ethical question which this case gives is whether it is permissible to use bloodbending to save a life.

The Kantians and the Utilitarians, yet again, disagree. Utilitarianism has it that Katara did the right thing. If she had not bloodbended Hama, The Avatar would have died. All of the good which Aang would cause in the world would not happen. In fact, in a sense, all of those positives act in favor of saving him, by any means. In bloodbending Hama, Katara is partially responsible for all of the good which Aang would cause thereafter. All of those factors make the response pretty clear. Katara may feel bad and cry because of what she had to do to save Aang’s life, but those feelings are a drop in the ocean compared to the alternatives. For this case, unlike The Attempt to Kill Aang, Katara knew all of those factors. Though this does not change the actual moral rectitude of the action, it should change how much praise or blame we place on her. Katara knowing this makes it so that we should not blame her for bloodbending, in fact, we should thank and congratulate her for it. If Katara were a good Utilitarian, she would feel good for saving Aang’s life and not sadness for how she had to do it. Since Utilitarianism is a Consequentialist stance, it has that the ends justify the means. Bloodbending, in this case, is a means to saving the Avatar’s life, which more than justifies its use.

The Kantians, on the other hand, will say that Katara using bloodbending to save Aang is wrong. This case mirrors very closely a thought experiment which Kant himself encountered. Suppose that your roommate is in the shower and you hear a knock at the door. You open it to find an axe-murderer, weapon out, asking about the whereabouts of your roommate. You have a choice. You could lie to save your friend or you could tell them the truth. In telling them the truth, they will die at the hands of the axe-murderer; but, Kant argued, you cannot lie to them. If you lie to the axe-murderer, you are using them merely as a means to an end and the results of that lie are on you, morally speaking. Whereas, if you tell the truth, you are not using them merely as a means and the results of their actions are not on you.\autocite[346-350]{Beck1} \footnote{There are more than 23 cases of lying or deception in the three seasons of the show and all would be counted as immoral according to Kantianism. There are 23 cases in the first two seasons (books) alone.} This distinction could be generalized to a difference between killing and letting die. Killing a person is always wrong but letting a person to die at the hands of another (keeping your own hands clean, so to speak) is fine.\footnote{A disagreement about killing vs letting die is even in one of the final episodes of Avatar. In Sozin’s Comet Part 2, while Aang is contacting his past lives, he speaks with Avatar Kyoshi. Kyoshi describes her encounter with Chin the Conqueror, to which Aang replies “You didn't really kill Chin. Technically, he fell to his own doom because he was too stubborn to get out of the way.” And then Kyoshi says “Personally, I don't really see the difference, but I assure you, I would have done whatever it took to stop Chin.” Since Kyoshi does not see a difference between killing and letting die, we could, arguably, claim that she is more Utilitarian than Kantian. \cite[32:59-33:13]{Sozen1}} Katara’s use of bloodbending is very similar to the axe-murderer case. In violating Hama’s autonomy, the results of the action are on Katara and she did something morally wrong. If she had done nothing, on the other hand, the results of Aang’s death would not be on her, morally speaking and she would not have done something wrong. This, again, points to a radically different way of thinking about morality. The Utilitarian does not see a distinction between killing and letting a person die, because both have the same results. The Utilitarian, also, would say that the results of Aang’s death would, at least partially, be on Katara because she could have prevented it.

But, what do these theories say about violating a person’s autonomy? In all four cases, we have situations where a bender is forcing another to do something which they would not otherwise do. In all four cases, we see a person’s autonomy is being violated. Utilitarianism gives mixed results for these cases. As a result, this means that Utilitarianism gives us situations where a person’s autonomy should be violated. Of course, most of the time, we should not force people or coerce them into doing things that they would not otherwise want, but, according to this theory, there are always exceptions. According to Utilitarianism, you should violate another’s autonomy when your failure to do so would result in less than optimal results. Most of the time, violating a person’s autonomy is not the best course of action, but there are going to be cases where it is what you need to do.   

For all of these examples, Kantianism gives the same response, bloodbending is morally wrong. This means that the theory says that a person’s autonomy should never be violated, through manipulation or through bloodbending. Could there be a case where Kantianism says that it is permissible? Such a case would require a person to voluntarily, in good faith and without deception, wish to be bloodbended. Even then, there could be a sense in which that person, in acting on that wish, is using themselves merely as a means to an end.
\section{Conclusion}

In this paper, we have used the two major theories in Ethics to figure out the moral status of bloodbending. According to Utilitarianism, sometimes using the ability is wrong and other times it is the right thing to do. The evaluation has nothing to do with bloodbending itself, rather the judgement is made on the basis of the consequences of the action. The Utilitarian will claim that Hama escaping from prison was permissible, Hama kidnapping the villagers was wrong, Hama attempting to kill Aang was wrong, and Katara saving Aang’s life was permissible. The Kantian, on the other hand, will make their evaluation based on the nature of bloodbending itself, regardless of the consequences. They claim that the prison escape was wrong, the kidnapping was wrong, attempting to kill Aang was wrong, and Katara saving the Avatar was wrong. In each case, the bender is violating a person’s autonomy which is wrong.  So, what is the moral status of bloodbending? The answer to this question depends on the answer to a bigger question: Which theory is more accurate, Utilitarianism or Kantianism?

\setcounter{footnote}{\thefd}
