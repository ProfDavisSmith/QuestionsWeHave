\chapter{Death By Thomas Nagel}\autocite{Nagel2}
\label{death}
\newcounter{fa}
\setcounter{fa}{\thefootnote}
\setcounter{footnote}{0}
If death is the unequivocal and permanent end of our existence, the question
arises whether it is a bad thing to die.

There is conspicuous disagreement about the matter: some people think death
is dreadful; others have no objection to death per se, though they hope their
own will be neither premature nor painful. Those in the former category tend to
think those in the latter are blind to the obvious, while the latter suppose the
former to be prey to some sort of confusion. On the one hand it can be said
that life is all we have and the loss of it is the greatest loss we can sustain. On
the other hand it may be objected that death deprives this supposed loss of its
subject, and that if we realize that death is not an unimaginable condition of the
persisting person, but a mere blank, we will see that it can have no value
whatever, positive or negative.

Since I want to leave aside the question whether we are, or might be, immortal
in some form, I shall simply use the word 'death' and its cognates in this
discussion to mean permanent death, unsupplemented by any form of
conscious survival. I want to ask whether death is in itself an evil; and how great
an evil, and of what kind, it might be. The question should be of interest even
to those who believe in some form of immortality, for one's attitude towards
immortality must depend in part on one's attitude toward death.

If death is an evil at all, it cannot be because of its positive features, but only
because of what it deprives us of. I shall try to deal with the difficulties
surrounding the natural view that death is an evil because it brings to an end
all the goods that life contains. We need not give an account of these goods
here, except to observe that some of them, like perception, desire, activity, and
thought, are so general as to be constitutive of human life. They are widely
regarded as formidable benefits in themselves, despite the fact that they are
conditions of misery as well as of happiness, and that a sufficient quantity of
more particular evils can perhaps outweigh them. That is what is meant, I think
by the allegation that it is good simply to be alive, even if one is undergoing
terrible experiences. The situation is roughly this: There are elements which, if
added to one's experience, make life better; there are other elements which if
added to one's experience, make life worse. But what remains when these are
set aside is not merely neutral: it is emphatically positive. Therefore life is worth
living even when the bad elements of experience are plentiful, and the good
ones too meager to outweigh the bad ones on their own. The additional positive
weight is supplied by experience itself, rather than by any of its consequences.
I shall not discuss the value that one person's life or death may have for others,
or its objective value, but only the value that it has for the person who is its
subject. That seems to me the primary case, and the case which presents the 
greatest difficulties. Let me add only two observations. First, the value of life
and its contents does not attach to mere organic survival; almost everyone
would be indifferent (other things equal) between immediate death and
immediate coma followed by death twenty years later without reawakening.
And second, like most goods, this can be multiplied by time: more is better than
less. The added quantities need not be temporally continuous (though
continuity has its social advantages). People are attracted to the possibility of
long-term suspended animation or freezing, followed by the resumption of
conscious life, because they can regard it from within simply as a continuation
of their present life. If these techniques are ever perfected, what from outside
appeared as a dormant interval of three hundred years could be experienced
by the subject as nothing more than a sharp discontinuity in the character of
his experiences. I do not deny, or course, that this has its own disadvantages.
Family and friends may have died in the meantime; the language may have
changed; the comforts of social, geographical, and cultural familiarity would be
lacking. Nevertheless those inconveniences would not obliterate the basic
advantage of continued, though discontinuous, existence.

If we turn from what is good about life to what is bad about death, the case is
completely different. Essentially, though there may be problems about their
specification, what we find desirable in life are certain states, conditions, or
types of activity. It is being alive, doing certain things, having certain
experiences, that we consider good. But if death is an evil, it is the loss of life,
rather than the state of being dead, or nonexistent, or unconscious, that is
objectionable.\footnote{It is often said that those who object to death have made the mistake of trying to
imagine what it is like to be dead.} This asymmetry is important. If it is good to be alive, that
advantage can be attributed to a person at each point of his life. It is a good of
which Bach had more than Schubert, simply because he lived longer. Death,
however, is not an evil of which Shakespeare has so far received a larger
portion than Proust. If death is a disadvantage, it is not easy to say when a man
suffers it.

There are two other indications that we do not object to death merely because
it involves long periods on nonexistence. First, as has been mentioned, most
of us would not regard the temporary suspension of life, even for substantial
intervals, as in itself a misfortune. If it ever happens that people can be frozen
without reduction of the conscious lifespan, it will be inappropriate to pity those
who are temporarily out of circulation. Second, none of us existed before we
were born (or conceived), but few regard that as a misfortune. I shall have more
to say about this later.

The point that death is not regarded as an unfortunate state enables us to refute
a curious but very common suggestion about the origin of the fear of death. It is alleged that the failure to realize that this task is logically impossible (for the banal reason that there is nothing to
imagine) leads to the conviction that death is mysterious and therefore a
terrifying prospective state. But this diagnosis is evidently false, for it is just as
impossible to imagine being totally unconscious as to imagine being dead
(though it is easy enough to imagine oneself, from the outside, in either of those
conditions). Yet people who are averse to death are not usually averse to
unconsciousness (so long as it does not entail a substantial cut in the total
duration of waking life).

If we are to make sense of the view that to die is bad, it must be on the ground
that life is a good and death is the corresponding deprivation or loss, bad not
because of any positive features but because of the desirability of what it
removes. We must now turn to the serious difficulties which this hypothesis
raises, difficulties about loss and privation in general, and about death in
particular.

Essentially, there are three types of problem. First, doubt may be raised
whether anything can be bad for a man without being positively unpleasant to
him: specifically, it may be doubted that there are any evils which consist
merely in the deprivation or absence of possible goods, and which do not
depend on someone's minding that deprivation. Second, there are special
difficulties, in the case of death, about how the supposed misfortune is to be
assigned to a subject at all. There is doubt both to who its subject is, and as to
when he undergoes it. So long as a person exists, he has not yet died, and
once he has died, he no longer exists; so there seems to be no time when
death, if it is a misfortune, can be ascribed to its unfortunate subject. The third
type or difficulty concerns the asymmetry, mentioned above, between our
attitudes to posthumous and prenatal nonexistence. How can the former be
bad if the latter is not?

It should be recognized that if these are valid objections to counting death as
an evil, they will apply to many other supposed evils as well. The first type of
objection is expressed in general form by the common remark that what you
don't know can't hurt you. It means that even if a man is betrayed by his friends,
ridiculed behind his back, and despised by people who treat him politely to his
face, none of it can be counted as a misfortune for him so long as he does not
suffer as a result. It means that a man is not injured if his wishes are ignored
by the executor of his will, or if, after his death, the belief becomes current that
all the literary works on which his fame rest were really written by his brother,
who died in Mexico at the age of 28. It seems to me worth asking what
assumptions about good and evil lead to these drastic restrictions.

All the questions have something to do with time. There certainly are goods
and evils of a simple kind (including some pleasures and pains) which a person
possesses at a given time simply in virtue of his condition at that time. But this
is not true of all the things we regard as good or bad for a man. Often we need to know his history to tell whether something is a misfortune or not; this applies
to ills like deterioration, deprivation, and damage. Sometimes his experiential
state is relatively unimportant – as in the case of a man who wastes his life in
the cheerful pursuit of a method of communicating with asparagus plants.
Someone who holds that all goods and evils must be temporally assignable
states of the person may of course try to bring difficult cases into line by pointing
to the pleasure or pain that more complicated goods and evils cause. Loss,
betrayal, deception, and ridicule are on this view bad because people suffer
when they learn of them. But it should be asked how our ideas of human value
would have to be constituted to accommodate these cases directly instead.
One advantage of such an account might be that it would enable us to explain
why the discovery of these misfortunes causes suffering – in a way that makes
it reasonable. For the natural view is that the discovery of betrayal makes us
unhappy because it is bad to be betrayed – not that betrayal is bad because its
discovery makes us unhappy.

It therefore seems to me worth exploring the position that most good and ill
fortune has as its subject a person identified by his history and his possibilities,
rather than merely by his categorical state of the moment – and that while this
subject can be exactly located in a sequence of places and times, the same is
not necessarily true of the goods and ills that befall him.\footnote{t is certainly not true in general of the things that can be said of him. For example, Abraham
Lincoln was taller than Louis XIV. But when?}

These ideas can be illustrated by an example of deprivation whose severity
approaches that of death. Suppose an intelligent person receives a brain injury
that reduces him to the mental condition of a contented infant, and that such
desires as remain to him can be satisfied by a custodian, so that he is free from
care. Such a development would be widely regarded as a severe misfortune,
not only for his friends and relations, or for society, but also and primarily, for
the person himself. This does not mean that a contented infant is unfortunate.
The intelligent adult who has been reduced to this condition is the subject of
the misfortune. He is the one we pity, though of course he does not mind his
condition. It is in fact the same condition he was in at the age of three months,
except that he is bigger. If we did not pity him then, why pity him now; in any
case, who is there to pity? The intelligent adult has disappeared, and for a
creature like the one before us, happiness consists in a full stomach and a dry
diaper.

If these objections are invalid, it must be because they rest on a mistaken
assumption about the temporal relation between the subject of a misfortune
and the circumstances which constitute it. If, instead of concentrating
exclusively on the oversized baby before us, we consider the person he was,
and the person he could be now, then his reduction to this state and the
cancellation of his natural adult development constitute a perfectly intelligible
catastrophe.

This case should convince us that it is arbitrary to restrict the goods and evils
that can befall a man to nonrelational properties ascribable to him at particular
times. As it stands, that restriction excludes not only such cases of gross
degeneration, but also a good deal of what is important about success and
failure, and other features of a life that have the character of processes. I
believe we can go further, however. There are goods and evils which are
irreducibly relational; they are features of the relations between a person, with
spatial and temporal boundaries of the usual sort, and circumstances which
may not coincide with him either in space or in time. A man's life includes much
that does not take place within the boundaries of his life. These boundaries are
commonly crossed by the misfortunes of being deceived, or despised, or
betrayed. (If this is correct, there is a simple account of what is wrong with
breaking a deathbed promise. It is an injury to the dead man. For certain
purposes it is possible to regard time as just another type of distance.). The
case of mental degeneration shows us an evil that depends on a contrast
between the reality and the possible alternatives. A man is the subject of good
and evil as much because he has hopes which may or may not be fulfilled, or
possibilities which may or may not be realized, as because of his capacity to
suffer and enjoy. If death is an evil, it must be accounted for in these terms,
and the impossibility of locating it within life should not trouble us.

When a man dies we are left with his corpse, and while a corpse can suffer the
kind of mishap that may occur to an article of furniture, it is not a suitable object
for pity. The man, however, is. He has lost his life, and if he had not died, he
would have continued to live it, and to possess whatever good there is in living.
If we apply to death the account suggested for the case of dementia, we shall
say that although the spatial and temporal locations of the individual who
suffered the loss are clear enough, the misfortune itself cannot be so easily
located. One must be content just to state that his life is over and there will
never be any more of it. That fact, rather than his past or present condition,
constitutes his misfortune, if it is one. Nevertheless if there is a loss, someone
must suffer it, and he must have existence and specific spatial and temporal
location even if the loss itself does not. The fact that Beethoven had no children
may have been a cause of regret to him, or a sad thing for the world, but it
cannot be described as a misfortune for the children that he never had. All of
us, I believe, are fortunate to have been born. But unless good and ill can be
assigned to an embryo, or even to an unconnected pair of gametes, it cannot
be said that not to be born is a misfortune. (That is a factor to be considered in
deciding whether abortion and contraception are akin to murder.)

This approach also provides a solution to the problem of temporal asymmetry,
pointed out by Lucretius. He observed that no one finds it disturbing to
contemplate the eternity preceding his own birth, and he took this to show that
it must be irrational to fear death, since death is simply the mirror image of the
prior abyss. That is not true, however, and the difference between the two explains why it is reasonable to regard them differently. It is true that both the
time before a man's birth and the time after his death is time of which his death
deprives him. It is time in which, had he not died then, he would be alive.
Therefore any death entails the loss of some life that its victim would have led
had he not died at that tor any earlier point. We know perfectly well what it
would be for him to have had it instead of losing it, and there is no difficulty in
identifying the loser.

But we cannot say that the time prior to a man's birth is time in which he would
have lived had he been born not then but earlier. For aside from the brief margin
permitted by premature labor, he could not have been born earlier: anyone born
substantially earlier than he would have been someone else. Therefore the time
prior to his birth prevents him from living. His birth, when it occurs, does not
entail the loss to him of any life whatever.

The direction of time is crucial in assigning possibilities to people or other
individuals. Distinct possible lives of a single person can diverge from a
common beginning, but they cannot converge to a common conclusion from
diverse beginnings. (The latter would represent not a set of different possible
lives of one individual, but a set of distinct possible individuals, whose lives
have identical conclusions.) Given an identifiable individual, countless
possibilities for his continued existence are imaginable, and we can clearly
conceive of what it would be for him to go on existing indefinitely. However
inevitable it is that this will not come about, its possibility is still that of the
continuation of a good for him, if life is the good we take it to be.\footnote{I confess to being troubled by the above argument, on the ground that it is too sophisticated
to explain the simple differences between our attitudes to prenatal and posthumous
nonexistence. For this reason I suspect that something essential is omitted from the account
of the badness of death by an analysis which treats it as a deprivation of possibilities. My
suspicion is supported by the following suggestion of Robert Nozick. We could imagine
discovering that people developed from individual spores that had existed indefinitely far in
advance of their birth. In this fantasy, birth never occurs naturally more than a hundred years
before the permanent end of the spore's existence. But then we discover a way to trigger the
premature hatching of these spores, and people are born who have thousands of years of
active life before them. Given such a situation, it would be possible to imagine oneself having
come into existence thousands of years previously. If we put aside the question whether this
would really be the same person, even given the identity of the spore, then the consequence
appears to be that a person's birth at a given time could deprive him of many earlier years of
possible life. Now while it would be cause for regret that one had been deprived of all those
possible years of life by being born too late, the feeling would differ from that which many
people have about death. I conclude that something about the future prospect of permanent
nothingness is not captured by the analysis in terms of denied possibilities. If so, then Lucretius'
argument still awaits as answer. I suspect that it requires a general treatment of the difference
between past and future in our attitudes toward our own lives. Our attitudes toward past and
future pain are very different, for example. Derek Parfit's unpublished writings on this topic
have revealed its difficulty to me.}


We are left, therefore, with the question whether the nonrealization of this
possibility is in every case a misfortune, or whether it depends on what can
naturally be hoped for. This seems to me the most serious difficulty with the
view that death is always an evil. Even if we can dispose of the objections
against admitting misfortune that is not experienced, or cannot be assigned to
a definite time in the person's life, we still have to set some limits on how
possible a possibility must be for its nonrealization to be a misfortune (or good
fortune, should the possibility be a bad one). The death of Keats at 24 is
generally regarded as tragic; that of Tolstoy at 82 is not. Although they will both
be dead forever, Keats' death deprived him of many years of life which were
allowed to Tolstoy; so in a clear sense Keats' loss was greater (though not in
the sense standardly employed in mathematical comparison between infinite
quantities). However, this does not prove that Tolstoy's loss was insignificant.
Perhaps we record an objection only to evils which are gratuitously added to
the inevitable; the fact that it is worse to die at 24 than at 82 does not imply that
it is not a terrible thing to die at 82, or even at 806. the question is whether we
can regard as a misfortune any limitations, like mortality, that is normal to the
species. Blindness or near-blindness is not a misfortune for a mole, nor would
it be for a man, if that were the natural condition of the human race.

The trouble is that life familiarizes us with the goods of which death deprives
us. We are already able to appreciate them, as a mole is not able to appreciate
vision. If we put aside doubts about their status as goods and grant that their
quantity is in part a function of their duration, the question remains whether
death, no matter when it occurs, can be said to deprive its victim of what is in
the relevant sense a possible continuation of life.

The situation is an ambiguous one. Observed from without, human beings
obviously have a natural lifespan and cannot live much longer than a hundred
years. A man's sense of his own experience, on the other hand, does not
embody this idea of a natural limit. His existence defines for him an essentially
open-ended possible future, containing the usual mixture of goods and evils
that he has found so tolerable in the past. Having been gratuitously introduced
to the world by a collection of natural, historical, and social accidents, he finds
himself the subject of a life, with an indeterminate and not essentially limited
future. Viewed in this way, death, no matter how inevitable, is an abrupt
cancellation of indefinitely extensive possible goods. Normality seems to have
nothing to do with it, for the fact that we will all inevitably die in a few score
years cannot by itself imply that it would be good to live longer. Suppose that
we were all inevitably going to die in agony – physical agony lasting six months.
Would inevitability make that prospect any less unpleasant? And why should it
be different for a deprivation? If the normal lifespan were a thousand years,
death at 80 would be a tragedy. As things are, it may just be a more widespread
tragedy. If there is no limit to the amount of life that it would be good to have,
then it may be that a bad end is in store for us all.
\setcounter{footnote}{\thefa}