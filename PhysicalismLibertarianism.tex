\chapter{Are Libertarianism and Physicalism Compatible?}
\section{By Davis Smith}
\section{Abstract}

This paper concerns two seemingly unrelated topics, the Mind-Body Problem and the Free Will Debate. More particularly, I am worried about whether Libertarianism and Physicalism are compatible. First, I show that Libertarianism implies that there are actions such that the doer could have physically done otherwise and which the doer had control over. Next, I show that Physicalism implies that all actions are either deterministic or non-deterministic. And finally, I show that a Physicalist deterministic action eliminates the ability to do otherwise and that a Physicalist non-deterministic action eliminates the doer’s control. These prove that Libertarianism and Physicalism are contradictory.
\section{Introduction}

One would think that the more desperate two topics in philosophy are, the less impact a stance in one would have on the other. While this is true for some things, it does not seem to hold in the case of the Free Will Debate and the Mind-Body Problem. In this paper, I am going to argue that Libertarianism and Physicalism are not compatible.\footnote{There have been a few papers in cognitive science and experimental philosophy which have shown that people who believe in Physicalism, more particularly, Reductive Physicalism, are less likely to believe in Libertarian free will. In other words, that there is a correlation between belief in Substance Dualism and Libertarian free will. \cite{Wisniewski1} for one such study.} This is to say that a universe with Libertarian free will cannot be a Physicalist universe. I do not claim that Determinism and Substance Dualism are incompatible\footnote{I, personally, believe that they are compatible.} nor do I make the more robust claim that Libertarianism is self-contradictory,\footnote{This is a longer project which starts with a paper much like this one.} rather, I am claiming only that Libertarian free will is contrary to Physicalism. This argument has 3 central premises, which the others prove/support. First, if Libertarianism is true, then there are some actions such that the doer could have (physically) done otherwise and the action was within the doer's control. Second, if Physicalism is true, then for any action, that action is either deterministic or non-deterministic. And finally, if an action is deterministic, then the doer could not have (physically) done otherwise and if an action is non-deterministic, the action is not within the doer’s control. These three premises, if they are accurate, lead to the necessary conclusion that if Libertarianism is true, then Physicalism is false.
\section{Terms}

For this phase of the argument, there are four (4) terms which I need to precisely define and clarify. For some, this will seem like simple review, but it is important that we are all on the same page. These terms are ‘Determinism', ‘Physicalism', ‘Free Will', and ‘Libertarianism'.
\subsection{Determinism}

\begin{center}An event A is deterministic if and only if, given the state of the world and the laws of nature prior to A, A will occur necessarily.\footnote{Put in a more technical and precise way, this means that an event A (at t at w) is deterministic if and only if there is no possible world, w’, such that the conditions of the world prior to t at w’ and the laws of nature at t at w’ are the same as those prior to and at t at w and A does not occur at t at w’.}
\end{center}
Basically, for all possible worlds with the exact same laws of nature and they have the exact same past events leading up to the event, then the event will occur in all the possible worlds, in the same way and at the same time. The opposite of this is to say that an event is non-deterministic, and we can define this like so:
\begin{center}
An event A is non-deterministic if and only if, given the state of the world and the laws of nature prior to A, A will not occur necessarily.\footnote{Put in a more technical way, we have that event A (at t at w) is non-deterministic if and only if there’s possible world, w’, such that the conditions of the world prior to t at w’ and the laws of nature at t at w’ are the same as those prior to and at t at w and A does not occur at t at w’.}
\end{center}
This is not to say that the event will not occur, but rather it is saying that it is possible that the event will not occur. If an event is non-deterministic, then it is possible for the event not to occur in a world the same prior to it as a world where it does. There are two ways in which an event could be non-deterministic. First, the event could be the probabilistic result of a preceding cause or it could be completely uncaused. In the case of the former, the laws of nature would need to be in such a way that for at least some sets of circumstances, the probability of the event occurring is less than 1. The probability of the event occurring could be very close to 1, but if it is less than that, there is still a chance that the event does not occur. A totally uncaused event, on the other hand, would be one in which the preceding events in no way determine or make more likely that it occurs, such an event would be, for all intents and purposes, random.

With these two definitions, we can define ‘Determinism' in the following way: Determinism is the stance that all events in the actual world are deterministic. If Determinism is accurate, then it follows that the laws of nature do not contain randomness and that the preceding events predict with 100\% accuracy future events.\footnote{This way of defining ‘Determinism’ is used in several places, but \cite[most notably in][ ]{Popper1}.} Some point to the seemingly random events at a quantum level as evidence against determinism, but some recent works\autocite{Barrett1} show that it is impossible to determine whether any notion of randomness can characterize the data in Quantum Mechanics. This means that appeals to such notions are going to, fundamentally, be based on intuitions and not the data.\footnote{I will often use the phrase ‘deterministic universe’ which means that Determinism is true at the possible world in question. Similarly, ‘non-deterministic universe’ means that Determinism is not true at the possible world in question.}
\subsection{Physicalism}

For our purposes, though there may be other Physicalisms out there, this is a stance within the possible responses to the Mind-Body Problem. This stance is that all the objects/substances in the world are physical. There are no mental substances which do not reduce to the physical or which do not supervene on the physical. For our purposes, also, there is no reason to distinguish between Reductive and Non-Reductive Physicalism, as they both lead to the same relevant place.\footnote{I will often use the phrase ‘Physicalist universe’, which means that the possible world in question is one in which Physicalism is true.} Even in the case of Non-Reductive Physicalism, if a Physicalist universe is non-deterministic, then the indeterminacy is in the natural laws. Since mental events, at the very least, supervene on physical events, the choices we make (mental) are caused by changes in the physical nature of the brain. So, non-deterministic choices must be the result of a non-deterministic physical event.
\subsection{Free Will}

For our purposes, I am defining ‘free will’ as whatever is necessary/sufficient for moral responsibility.\footnote{Others may fail to hold the doer morally responsible for the action or it may be the case that the degree of rectitude is inconsequential. Either way, the doer is still morally responsible.} There are two ways which this can be interpreted:
\begin{enumerate}
    \item If a person does an act freely, then they are morally responsible for that action.
    \item If a person is morally responsible for an action, then they did that action freely.
\end{enumerate}
The second interpretation seems to be more common. For example, many claim that if there’s no free will, then there’s no moral responsibility. The first notion, however, is the one which we will be using as this is seemingly what the Libertarians are after in the case of free will. Merely asserting that we have free will because Determinism is false does not guarantee that we have moral responsibility. This would be a fallacy. Libertarians, therefore, must hold the first conditional. They want free will because it grants moral responsibility.\footnote{Even those who wish to distance free will and moral responsibility will claim that free action is at least a sufficient condition for moral responsibility.}
\subsection{Libertarianism}

Libertarianism is an incompatibilist stance, meaning that it holds that Determinism and free will are not compatible. You could have one or the other, but not both. More particularly, it holds that Determinism is false, so there is free will and therefore moral responsibility. Since it implies that determinism is false, it states that there are some non-deterministic events. Some implied features of Libertarianism include:
\begin{enumerate}
    \item To have moral responsibility, the person must have physically been able to do otherwise.
    \item The events which cause the actions which we are morally responsible for, must be non-deterministic (because the doer must be able to do otherwise).
\end{enumerate}
When it comes to the first feature of Libertarianism, this is the Principle of Alternative Possibilities\footnote{\cite{Frankfurt1}.}, but we need to be careful to include the term ‘physically'. Without it, the statement could be one which a Compatibilist would approve of. They would say that, sure, you could do otherwise, but you physically are not able to. Though the Frankfurt-style cases found in the literature are convincing to some that this is not an essential feature of moral responsibility,\footnote{We will see a Frankfurt-style case in Part 1 of the argument.} I believe that they would be unconvincing to a sensible Libertarian.\footnote{This is because the passive coercion in the Frankfurt-style cases is quite like the indirect causal pressures which the past and the laws of nature have on our choices in a deterministic universe. If they reject the possibility of moral responsibility in a deterministic universe, then they would equally need to reject the possibility of moral responsibility in a Frankfort-style case. For an alternative argument, \cite[see][ ]{Widerker1}.} The Libertarians would want the ability to physically do otherwise. With the second feature, if the events which cause our actions are wholly deterministic, then it would not be possible for us to do otherwise, which means that, from the first feature, we would not be morally responsible for them.
\section{The Argument}

This argument is broken into three parts, which come together at the end to show that Physicalism and Libertarianism are not compatible. The premises for the argument have four (4) conditional statements, each of which are proven in the relevant section. These are the same premises which I mentioned in the introduction. They are also the section titles.
\subsection{Part 1: If Libertarianism is true, then there are some actions such that the doer could have done otherwise and the action was within the doer's control.}

Proving this line of the argument requires us to show two different things; both of which are derived from a basic understanding of ‘moral responsibility'. From the very definition of Libertarianism, we have that we must have free will and therefore must be morally responsible for at least some of our actions. But what exactly does it mean to have moral responsibility for some of our actions?

It seems that there are at least two necessary features for a doer to be morally responsible for their action, and these are the two features of the above conditional. First, the doer must have been able to do otherwise. The doer must have had the ability to physically refrain from doing it or do something other than what they did. For example, take this thought experiment to drive home the idea:\footnote{This is a case similar to the ones \cite[found in][ ]{Frankfurt1} \cite[and][ ]{Fischer1}.}
\factoidbox{
    Suppose that a meteor crashed in a field near your home and an alien spore escaped filling the air, quickly infecting people’s nervous systems. These spores grow and take over the mind of the host. Sometimes, the actions which the host would do are the same as those which the spores would cause them to do, however, when they are not, the spores switch up the nerves to make the host do as they would want.}

Since the host could not do otherwise, they are not morally responsible for their actions (as even if it were their choice, the spores would stop them from making a different one). When we think of cases where a person is forced or coerced to do something, then we are less inclined to hold them responsible for their actions. In the above case, the alien spores are passively forcing the hosts to do actions, so it seems clear that they are at least less responsible for the actions.

The other necessary feature for moral responsibility is a sense of control or up-to-us-ness. The doer must have chosen\footnote{Making a choice, it would seem, needs to be both voluntary and with deliberation.} to perform the action and they must have directed it to the outcome. For example, look at this case:
\factoidbox{
    Suppose that you are at a dinner party having a discussion with various people and sipping wine. During a deep discussion on whether Libertarianism is compatible with Physicalism, you have a muscle spasm which launches the wine all over another’s white cloths.}

Whether the universe in this thought experiment is deterministic is not relevant. You have no control over what happens when you have a muscle spasm like this. That lack of control serves as a more than adequate excuse to relinquish moral responsibility for the destruction of another person's clothing.\footnote{For more on this, \cite[see][ ]{Austin1}.} One could also characterize this as an event which was not ultimately up to you. Your thoughts and attitudes, the reasons which you may have, did not enter the event. This means that, at least for the Libertarian, the ability-to-do-otherwise is not enough for moral responsibility. It is certainly the case that it’s possible for you not to have that spasm, but the action must be up to you.\footnote{Here is a break-down of the argument: First, if Libertarianism is true, then people have free will. Second, if people have free will, then people are morally responsible for some of their actions. If people are morally responsible for some of their actions, then there are actions such that the person could have done otherwise and which the person was in control over. Therefore, if Libertarianism is true, then there are some actions such that the doer could have done otherwise and which was in the doer’s control.}
\subsection{Part 2: If Physicalism is true, then for any action, that action is either deterministic or non-deterministic.}

This is straight forward, but I am the sort of person who does not like to leave anything without a proof. To start off, from the definition of ‘Physicalism', we have that there are only physical substances. This is that there is only one kind of substance in the world and those substances are physical. This certainly removes the possibility of Substance Dualism, at least for the purposes of this argument.

The next line of the proof is that if there are only physical substances, then there are only physical events. Events require that there be objects or substances (one or more) which are involved in that event. An event A involves an object/substance B if and only if for all minimalist and accurate accounts the event A, the accounts contain reference to B. For a proof, try to imagine an event which does not have anything involved in it. Even if there’s quantum randomness where a particle just appears out of nothing, that particle is still involved in the event. Thus, if there are only physical substances, then there can only be physical events. Some might claim that there are mental events, but physicalism has it that those, at the very least, reduce to physical events or they are simply a different perspective on the physical event.\footnote{Also, a minimalist account of those mental events, if Physicalism is true, would only involve physical objects.}

Third, if there can only be physical events, then for any event, that event is either deterministic or non-deterministic. This can be derived using the law of excluded middle. Since there is no third alternative between being deterministic or non-deterministic, any event must be one or the other and the law of non-contradiction would have it that they cannot be both. From this, all we need is that all actions are events, which seems too obvious to deny.\footnote{As before, here is a breakdown of the argument: If Physicalism is true, then there are only physical substances. If there are only physical substances, then there are only physical events. If there are only physical events, then those events are ether deterministic or non-deterministic. All actions are events. Therefore, if Physicalism is true, for any action, that action is either deterministic or non-deterministic.} As a tie between this part and the previous, Physicalist Libertarianism would say that there are at least some non-deterministic actions which we are, therefore, morally responsible for.  
\subsection{Part 3: If an action is deterministic, then the doer could not have done otherwise and if an action is non-deterministic, the action is not within the doer’s control.}

This line of the argument has two halves. For both examples used to illustrate these conditionals, we can assume that they are in a Physicalist universe. For the first half of this part of the argument, this requires some quibbles about the exact meanings of the terms. The relevant sense of being able to do otherwise is the one which the Libertarian uses. So, it follows naturally that a deterministic action lacks alternative possibilities.

The second half of this part of the argument is more involved. First, if an action is non-deterministic, then it is either a probabilistic effect of a preceding cause or it is totally uncaused. In the case of probabilistic events, though one event may be more likely than another, it is still, on a base level, random which of those events happens. For example, take the following thought experiment:\footnote{The percent likelihood of the two different possibilities is, I would think, a worst-case scenario for the moral responsibility of the man in question. Other ratios of possibility are useable.}

\factoidbox{
    A man is standing in front of a switch, there is a run-away trolley. If the man does nothing, then the trolley will kill 5 people, if they flip the switch, then the trolley will only kill one person. Since this is a non-deterministic universe, there is a 50\% chance that they will flip the switch and a 50\% chance that they will do nothing.}

In this case, the core difference between it and other trolley problems is that there is this element of chance. Since this is a physicalist universe, the indeterminacy of the choice must be because of some manner of non-deterministic physical event. A minimalist account of the choice would not contain reference to the doer’s reasons. The choice was not up to them.\footnote{It could be generalized that, since this is a Physicalist universe, the choice reduces to physical events and those must be probabilistic. On that level, the reasoning, the up-to-him-ness, is not present. A similar line of thought, though about artificial intelligence, can be \cite[found in][ ]{Searle1}.} The random nature of the choice takes the action outside of the man’s control. In the case of totally uncaused events, the amount of control had by the doer becomes even more elusive. Take, for example, the following thought experiment:\footnote{This case is similar in form to the ones \cite[seen in][ ]{Elzein1}, but the thought experiments there are used to show that the principle of alternative possibilities is important to moral responsibility and that the alternative possibilities must be relevant to the case.}
\factoidbox{
    One evening, Jones is sitting back to watch a little television when a quantum particle appears in his brain and then vanishes. This event causes a chain reaction which results in him choosing to begin growing edible mushrooms. Acting on this choice, he becomes very knowledgeable about the subject and eventually he discovers a new form of hypoallergenic penicillin, saving countless lives.}

For ease of use, we can assume that the appearance of the particle is the only non-deterministic event relevant to the case.\footnote{Also, because other possible worlds are not relevant, we can assume that this is a physicalist universe.} Since the initial cause of the choice was completely random and the resulting events which lead to the discovery were the deterministic results of the initial event, we could hardly say that Jones was in control of the action and, thereby, morally responsible for saving the countless lives.\footnote{Here is a break-down of the argument: First, if an action is deterministic, then the person could not have done otherwise. If an action is non-deterministic, then it is either a probabilistic effect of a preceding cause or it’s totally uncaused. If the action is probabilistic in this way, then the action was not within the person’s control and if the action is totally uncaused, then the action was not within the person’s control. Therefore, if an action is deterministic, then the person could not have done otherwise and if the action is non-deterministic, then the action was not within the person’s control.} Along the same vein, going outside of the scope of this paper, if this were a Substance Dualist universe and the non-deterministic event occurred mentally, then Jones’ choice still, ultimately, would not have been up to him.
\section{Conclusion: If Libertarianism is true, then Physicalism is false.}

So far, I have shown three (3) things. First, If Libertarianism is true, then there are some actions such that the doer could have done otherwise and that the doer was in control of that action. This is from the seemingly necessary features of moral responsibility. Next, I have shown that if Physicalism is true, then all our actions are either deterministic or non-deterministic. There is no third alternative. And finally, I have shown that if an action is deterministic, then it's not possible for the doer to have done otherwise and if the action is non-deterministic, then the action was not in the doer's control. From the second and the third, we have that if Physicalism is true, then for any action, the doer either could not have done otherwise or the doer was not in control. But the first part claims that if Libertarianism is true, then there are some actions where the doer has both of those features. This directly contradicts the result from Physicalism and therefore shows that if Libertarianism is true, then Physicalism is false. \footnote{Here the complete argument: If Physicalism is true, then there are only physical substances. If there are only physical substances, then there can only be physical events. If there can only be physical events, then for any event, that event is either deterministic or random. All actions are events. If an action is deterministic, then the person could not have done otherwise. If an action is random, then it is either a probabilistic effect of a preceding cause or it’s totally uncaused. If the action is probabilistic in this way, then the action was not within the person’s control and if the action is totally uncaused, then the action was not within the person’s control. If Libertarianism is true, then a person has free will. If a person has free will, then they are morally responsible for some actions. If a person is morally responsible for an action, then a) that person could have done otherwise and b) that person is in control of the action. Therefore, if Libertarianism is true, then Physicalism is false.}

