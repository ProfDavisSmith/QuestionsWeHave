\thispagestyle{empty}

\pagestyle{empty}

\vspace*{80pt}

\begin{raggedleft}
\fontsize{30pt}{24pt}\sffamily
\selectfont
  \textbf{Questions We Have }

\medskip\fontsize{18pt}{20pt}\selectfont

\textbf{An Introduction to Philosophy}

\vfill
\fontsize{12pt}{16pt}\selectfont \textit{By }  \textbf{Davis A. Smith}\\
\end{raggedleft}

\newpage

\noindent%
Questions We Have by Davis Smith is, for the most part, an original introduction to Philosophy textbook. It consists of eleven (11) Modules (sections or parts) which each handle a different philosophical question a person may have wondered about. Aside from the in depth discussions of the various topics which Davis Smith provides, there are 10 primary source works which are reprinted into this text. The questions discussed span from the abstract, such as the purpose of Philosophy, to the concrete, like our moral duties to other people and animals. 
The primary sources used are, in order:
\begin{enumerate}
\item[1] Plato's Apology
\item[2] Death by Thomas Nagel
\item[3] The Mind-Body Problem by Tim Crane
\item[4] Minds, Brains and Programs by John Searle
\item[5] Are Libertarianism and Physicalism Compatible? by Davis Smith
\item[6] Are Dualism and Physicalism Compatible? by Davis Smith
\item[7] Meditations on First Philosophy by Ren\'e Descartes
\item[8] The Challenge of Cultural Relativism by James Rachels
\item[9] The Moral Status of Bloodbending by Davis Smith
\item[10] The Moral and Legal Status of Abortion by Mary Warren
\item[11] I Was Once a Fetus by Alexander Pruss
\end{enumerate}

The two papers by Davis Smith concerning Libertarianism will be merged into one with future editions of this textbook.

This work is licensed under a \href{https://creativecommons.org/licenses/by/4.0/}{Creative Commons Attribution 4.0} license.
You are free to copy and redistribute the material in any medium or format, and  remix, transform, and build upon the material for any purpose, even commercially, under the following terms:
\begin{itemize}
\item You must give appropriate credit, provide a link to the license, and indicate if changes were made. You may do so in any reasonable manner, but not in any way that suggests the licensor endorses you or your use.
\item You may not apply legal terms or technological measures that legally restrict others from doing anything the license permits.
\end{itemize}


\bigskip
