\thispagestyle{empty}

\pagestyle{empty}

\vspace*{80pt}

\begin{raggedleft}
\fontsize{30pt}{24pt}\sffamily
\selectfont
  \textbf{Questions We Have }

\medskip\fontsize{18pt}{20pt}\selectfont

\textbf{An Introduction to Philosophy}

\vfill
\fontsize{12pt}{16pt}\selectfont \textit{By }  \textbf{Davis A. Smith}\\
\end{raggedleft}

\newpage

\noindent%
\section*{Information Regarding This Textbook}
Questions We Have by Davis Smith is, for the most part, an original introduction to Philosophy textbook. It consists of eleven (11) modules (sections or parts) which each handle a different philosophical question a person may have wondered about. Aside from the in depth discussions of the various topics which Davis Smith provides, there are 11 (soon to be 10) primary source works which are reprinted into this text. The questions discussed span from the abstract, such as the purpose of Philosophy, to the concrete, like our moral duties to other people and animals. 
The primary sources used are, in order:
\begin{earg}
\item[1] \nameref{platoapology}
\item[2] \nameref{death}
\item[3] \nameref{mindbodyprob}
\item[4] \nameref{mindsbrainsprograms}
\item[5] \nameref{libertarianismphysicalism}
\item[6] \nameref{libertarianismdualism}
\item[7] \nameref{descartesmeditations}
\item[8] \nameref{challengerelativism}
\item[9] \nameref{bloodbending}
\item[10] \nameref{statusabortion} 
\item[11] \nameref{iwasafetus} 
\end{earg}

These works have been lightly edited to fit the typestting and citation style of this text. When applicable, the originals of these works are cited at the start of the section which they are in, for ease of locating them. The two papers by Davis Smith concerning Libertarianism will be merged into one with future editions of this textbook.

\subsection{Credits for Images and the Typesetting}
Aside from the content which is cited in the textbook itself, the images and pictures used were all AI generated. This includes the cover of this textbook. They were generated by Bing's Designer, which is powered by DALL-E 3. This textbook was typeset using \LaTeX. The \LaTeX  code used was based on the code used for \href{https://github.com/ProfDavisSmith/forallxR3#readme}{\emph{forallx:} \emph{$R^3$}} by Davis A. Smith, used under a 
\href{https://creativecommons.org/licenses/by/4.0/}{CC BY 4.0} license, which, in turn, was based on the typesetting code used for \href{https://forallx.openlogicproject.org/}{\emph{forallx:} \emph{Calgary}}, 
by Aaron Thomas-Bolduc and Richard Zach, used under a 
\href{https://creativecommons.org/licenses/by/4.0/}{CC BY 4.0} license. 

\subsection{Copyright Status}

This work is licensed under a \href{https://creativecommons.org/licenses/by/4.0/}{Creative Commons Attribution 4.0} license.
You are free to copy and redistribute the material in any medium or format, and  remix, transform, and build upon the material for any purpose, even commercially, under the following terms:
\begin{itemize}
\item You must give appropriate credit, provide a link to the license, and indicate if changes were made. You may do so in any reasonable manner, but not in any way that suggests the licensor endorses you or your use.
\item You may not apply legal terms or technological measures that legally restrict others from doing anything the license permits.
\end{itemize}

\subsection{Notes For Instructors}

This book was designed for a quarter-long (10-11 weeks) Introduction to Philosophy. In my classes, I (Davis A. Smith) cover all modules, spending 1 week per module. The CC BY license gives you the right to download and distribute the book yourself. In order to ensure that all your students have the same version of the book throughout the term you’re using it, you should upload the PDF you decide to use to your LMS rather than merely give your students a link to the source. You are also free to have the PDFs printed by your bookstore.

\subsection{Funding}

The production of the 2022 edition of this work was made possible by Professional Development funds provided by \href{https://www.pierce.ctc.edu/elad}{Pierce College's Employee Learning and Development (ELAD)}. 


\includegraphics{marcom-PierceCollege-Logo.png}


\bigskip
