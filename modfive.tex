\part{Does God Exist? If so, why is there evil?}
\label{ch.modfive}
\addtocontents{toc}{\protect\mbox{}\protect\hrulefill\par}
\chapter{Part 10: Arguments for the existence of God}
In philosophy, ‘God’ is given a standard three part definition, though some traditions include more features, others include less (I know of very few which include less). These features are:
\begin{earg}
    \item[]Omniscient: All-knowing
    \item[]Omnipotent: All-powerful
    \item[]Omnibenevolent: All-good
\end{earg}
So, these ultimately boil down to the idea that God is the ultimate being, there is nothing which She can't do, nothing She doesn't know, and She would never do anything morally wrong or unjust. The core question in philosophy of religion (I wish it had a different name, as this one can be misleading) is whether such an entity exists. A related question is whether we can prove it? Whether such an entity exists is a metaphysical question while whether we can prove it or know whether such a being exists before our deaths is a different kind of question, an epistemic one. There are many different proofs for the existence of God and all have at least one glaring problem.

For this section, we will be covering 4 different arguments for the existence of God and 1 argument against the existence of such a being. The arguments in favor of God's existence are The Design Argument, The Fine Tuning Argument, The First Cause Argument, and the Ontological Argument. I am willing to wager that if you believe in such a deity, the justification for the belief (likely) is some variation on the arguments seen in this class. For example, I love watching town-hall style school-board debates (when science textbooks are discussed) because I find arguments like the ones covered here 'in the wild'. Though, I will admit, it's very unlikely that I will ever see the last argument covered here in such an environment, which is a shame, as it's by far the strongest you are going to find, the rest are pretty bluntly seen.  

It is very important that you know this is what philosophers mean by 'God', it may be different than the one you think of and many of the arguments for this being are from a monotheistic tradition. Talk of gods (plural) and arguments for them are not really found in the western tradition. But, I will also say, that all but the last argument can work for a pantheon of deities. 

\section{The Design Argument}

The design argument is actually a generic class of arguments which are also called \glspl{teleological argument}. This style of reasoning goes as far back as Plato (I am serious), and in its most recent form, it can be found in the arguments for intelligent design (knowing this style of arguments and their flaws are helpful in the real world today because of this; regardless of the side you are on, you should know either your own flaws to defend yourself or the flaws in your opponent's stance to take them out). Plato argued that there was a designer of the universe because of the organization and structure of it. The odds of the universe coming about because of chance are very unlikely, so Plato and others believe that there must have been some designer. This is just an example to get the right thought-process in your head:

\newglossaryentry{teleological argument}
{
name=teleological argument,
description={An argument which moves from a seeming purpose, end, or design exhibited by the universe to the claim that there must be a designer of the universe},
plural=teleological arguments
}


    \factoidbox{Suppose that you are walking through the forest, on a hiking trail, and you can across a computer, hooked to a running generator, mouse, keyboard, all on a desk. The computer is on and playing the classic Doom. Would you say that this appeared randomly, or would you say that someone built it?}

More than likely, you will say that someone built it. However, the human body, with its cells and structures, is far more complicated than some computer. So, why do we say that it lacked a builder/designer, while the computer had one? 
\subsection{The Argument}
There are various different versions of this argument out there, and this is just my more formal, and improved version based on the spirit of it:

\subsubsection{1. There are complex structures in the natural world.}

For evidence of this, just take a look outside. The trees, the mountains, the cells of the living animals running around, the living animals themselves all seem to be quite complicated, far more robust than a computer in a forest even.

    \subsubsection{2. If there are complex structures in the natural world, then they must have had a designer.}

This is the leap in the reasoning which you should look for in this particular argument, but there's another jump in the reasoning which is found across all of the arguments for the existence of God. This references back to the original intuition about the computer in the forest. If we are willing to say that there was a designer for this, then why not say that there was a designer for the even more complex living organisms and the world itself? There's (seemingly) no reason not to. 

    \subsubsection{3. If there is such a designer, then that designer is God.}

This is the second jump in the reasoning and is the one which is found across the arguments for the existence of God. God is defined as the all-knowing, all-powerful, and all-good. If any being is capable of being the designer of the universe, then it most certainly would be God. However, as we will see later, this is not the only being capable of creating the universe... 

    \subsubsection{4. If that designer is God, then God exists.}
    \subsubsection{Therefore, God exists.} 

\section{Problems with the Design Argument}
\subsection{Problem 1: The Weakness of the Analogy}

This is an attack on the first line of the argument. It only attacks why we think that it’s true. The argument itself is valid. The first line seems intuitive, seems right, because of the examples used. The examples pointed to are things like watches and computers, very complex things which obviously needed to have some kind of designer. But the analogy between natural complexities and a watch seemingly has strength to it, but in actuality, the similarity is quite weak. There are many commonalities between two kinds of watches, but the commonality between the eye and a watch is that both are complex, there is nothing more to go on. When the analogy is weak, the conclusion from the analogy is equally weak.

Here are two arguments, so that you can see the weakness of the analogy. If you accept the first, based solely on the reasoning used, then you will also need to accept the second:
\begin{tabular}{p{1.75in}|p{1.75in}}
    The brain is like a computer.&Guns are like hammer.\\
    They both have complex, inter-working parts and it’s very clear when one part fails to function properly.&They both have metal parts and could be used to kill someone.\\
    Yet, it would be ridiculous to claim that a computer didn’t have a designer.&Yet, it would be ridiculous to restrict the purchase of hammers.\\
    So, claiming that the brain didn’t have a designer is equally ridiculous.&So, restrictions on purchasing guns are equally ridiculous.
\end{tabular}

\subsection{Problem 2: Evolution}

The Theory of Evolution gives us another reason to doubt the analogy given in the design argument. Although it does not disprove the existence of God, it just moves Him back a step, evolution gives us a way of saying that the eye and other natural complexities did not need a designer. The natural process of survival of the fittest (no, that does not mean survival of the best, but survival of the ones which best fit into their environment) shows how these complexities will come about without a designer.

\subsection{Problem 3: Limitation of the conclusion}

This is shown by looking at the jump found in line 3 of the argument. This was:

    \begin{center}The only thing which could have designed the complex structures in the natural world is God.\end{center}

The point here is that even if we assume that there was a designer, we do not get that this designer was all-knowing, all-powerful, and all-good. More over, we do not get that there was only one designer. For example, in the designing of a new watch, rarely is it one person, but a team of people working together. There is no reason to think that the divine designer of the human eye was the same who designed the brain, and there is no reason to think that either or both of those designers was any of the tripartite features ascribed to God. Following the same sort of structure as before, if you like the first argument, then you will also need to like the second:

\begin{tabular}{p{1.75in}|p{1.75in}}
    Human bodies are remarkably complex and beautiful structures.&The pyramids are remarkably complex and beautiful structures.\\
    Such structures do not appear naturally without a designer/builder.&Such structures do not appear naturally without a designer/builder.\\
    So, the Flying Spaghetti Monster, with His great noodly appendages, made them.&So, someone, named Steve, all by himself, made them.
\end{tabular}

\section{The Fine-Tuning Argument}
Even if we accept that there was no need for a designer in the minute machinations of human creation, there is another reason to think that there was some kind of divine player involved. At its core, the fine tuning argument points at the odds of some event happening. The odds of the world being as it is an suitable for human evolution and survival without the work of a divine architect are so astronomically small that it is more likely that it would have never happened at all. But, if you bring a divine architect into the mix, the odds are 100\%. Since it did happen this way, the odds are far more likely that there was a divine architect. That Architect is God.

The Fine Tuning Argument could be seen as a variation or a more fine-tuned version of the Design Argument which we just covered. Although the Design Argument is quite ancient, the Fine-Tuning variation is relatively recent (as far as I am aware). It mostly comes out of the advancements in the empirical sciences. 

The argument goes like this, with the bold being the lines of the argument and the other being the explanation, note that when I extract an argument, I always make it as strong as possible, giving as much benefit of the doubt as possible. You don't want to kick a man when they're down:

\subsection{1. The universe is suitable for human evolution.}

This should be pretty obvious. Given our knowledge of the world, we can say that we did evolve from some proto-human. Which means that the universe must be suitable for that event. 

\subsection{2. There are three ways that this could have happened: by chance, by necessity, or by design.}

This is supposed to cover all of the possible ways we have to account for the fact that the world has us on it. We could have been insanely lucky (1 out of $10^50$ is far larger than the real chance by luck), it could be that the world (universe) just needed to be that way (could not have been otherwise), or something could have designed it for life. 

\subsection{3. The odds of this happening by chance are almost 0.}

This is basically saying that the odds are not in our favor. As I mentioned before, the odds are so indescribably small that the thought that it happened by chance should not enter our minds. (This is a particularly weak line, ripe for attack).

\subsection{4. There is no reason to think that the laws of nature could not have been otherwise.}

In other words, there's no reason to think that the laws of nature are necessary. It seems that it's within the realm of possibility that the laws of nature could be different. (Again, this is a weak line).

\subsection{5. So, the world must be so suitable by design.}

This is the consequence of the previous lines, the second line gives us 3 options, the first 2 got taken out, so the third must be the right one. (This line is not open to attack, you need to take out a previous one).

\subsection{6. If it happened by design, there must have been a designer.}

This is the same sort of move which we have seen before, relatively uncontroversial. If something was made, then there was a maker.

\subsection{7. If there was a designer, then that designer was God.}

This is the God-jump which we have seen before. The reasoning here is essentially the same as the design argument. 

\subsection{8. So, that designer was God.}

\subsection{9. Therefore, God exists.}

\section{Problems with The Fine Tuning Argument}
With The Fine Tuning Argument, we can think of it as a more robust version of the Design Argument, so it does avoid some of the issues, for example it does not fall to Weakness of Analogy. In the text, rather than having the Theory of Evolution as an issue, we have The Lottery Objection and as with all the arguments for the existence of God, we will have Limitations on the Conclusion. There are three other objections which were not covered in the textbook, which I have added.
\subsection{The Lottery Objection}

The core different objection to this is concerning its very reasoning. Very unlikely events can still happen. Suppose that you are one of millions who bought a lottery ticket to win several million dollars. Now, one of the tickets will win, the odds of it being yours are small, but that is true for all of the other tickets. Suppose that you pick a planet in the universe at random (your lotto numbers), the odds of that planet being one which can support life are small, but that is true for any planet you could have picked. Given the sheer number of planets, one or more will support life. Whatever creatures evolved on those planets may think that their position was put there by a designer too.
\subsection{Limitation of the Conclusion}

As with the Design Argument, there is a jump in the reasoning which is unsupported. In this case, the jump is from there being something which tuned the world/universe to be suitable for evolution/intelligent life to that thing being God. To illustrate, just as before, take the following arguments, if you accept the reasoning in the first, then you must also, by the same logic, accept the second, but they are not compatible:
\begin{tabular}{p{2in}|p{2in}}
    The world is suitable for human evolution.&The world is suitable for human evolution.\\
    The odds of this happening without a divine architect are next to 0.& The odds of this happening without a divine architect are next to 0.\\
    The odds of this happening with a divine architect are 100\%.&The odds of this happening with a divine architect are 100\%.\\
    So, there is a divine architect.&So, there is a divine architect.\\
    That divine architect is God.&That divine architect is a teenager doing a science fair project.\\
    Therefore, God exists.&Therefore, a teenager doing a science fair project exists.\\
\end{tabular} 
\subsection{Scientific Progress}

The basis for the Fine Tuning Argument is that there is currently no explanation for the constants of nature being so precise. However, this is an issue only currently. In the history of humanity in general, there have been many phenomena for which we lacked explanations and which were eventually explained away (such as the weather, the sun, stars, and natural disasters). The empirical sciences may, one day, find an explanation for these constants which is completely satisfactory. 
\subsection{Exotic Life}

Another area of contention for the Fine Tuning Argument is that it says that it's suitable for human life. While humans are quite remarkable beings, why should we think that the universe was specially made for us? It's at least conceivable that in a universe where the constants are not suitable for human life, there could be exotic life which is remarkably different than our own. 

There are limits to this response however, such a universe would need to be set up to overcome the second law of thermodynamics, which is easier than some would think (there's more possibilities for that than normally presented).

\section{The First Cause Argument}
First Cause Arguments are also called \glspl{cosmological argument} as they, rather than looking at how the world is, they look at the cause and effect relations which led to the world as it is. Everything in the natural world has a cause. All of these causes in turn have a cause. There are things in the natural world. So, there must have been a first cause, something not in the natural world, which created this chain (the cause of the big bang is the new standard place to put it). This first cause was God. 
\newglossaryentry{cosmological argument}
{
name=cosmological argument,
description={An argument which moves from observations of the world around us, more particularly those concerning causation, to the claim that there must have been a first cause},
plural=cosmological arguments
}



\subsection{1. Everything has a cause other than itself.}

This seems sort of self-evident, but might be worth explanation. Think of where you came from. Personally, my existence was caused by a drop in copper prices coupled with the series of events which lead my parents to meet at a desert party while my dad was a student at ASU and them moving to California. I can go farther, my father (whose story I am more familiar with) was caused by my grandfather and grandmother meeting, along with the events leading up to that, and so on. No where in the chain of cause and effect (unless backwards time-travel is possible/actual) will you find a case where something, directly or through transitivity, causes itself. More over, all things are a part of a causal chain like this. 

\subsection{2. Everything which is a cause has itself a cause other than itself (directly or indirectly).}

This is mostly an expansion/explanation of the first line, meaning that there are no existing things which weren't caused to exist by something previous to them in a chain. As I noted in parentheses above, there is an issue if you allow for backwards time-travel. In such a case, it would be possible for a thing/person to cause themselves to exist. For example, take this case from the Futurama episode "Roswell that Ends Well":

    \factoidbox{Philip Fry travels back in time to Roswell, New Mexico in the year 1947, where his 'grandfather' was stationed. As events play out, Fry ends up causing the death of his 'grandfather'. But, Fry does not fade away to nothing, and though some faulty reasoning, comes to believe that the woman he is consoling is not in fact his grandmother (she is, though), and Fry ends up sleeping with her. As a result, she becomes pregnant and Fry becomes his own grandfather.}

Here, we have a case where, thanks to time-travel, a person was their own cause. This can be an issue for this argument, but it's better as the 'Not a Proof' objection in the next page. 

\subsection{3. There are things which exist.}

This is another one of those 'no duh' sort of lines. Basically, it points out that these non-looped causal chains are in the world and need some kind of explanation. So, using an interesting jump in the reasoning (which will be pointed out later), we get the next line:

\subsection{4. So, there must have been a first cause to kick start the chain.}

As always, the word 'so' indicates that we have an intermediate conclusion, this means that it's supposed to follow from the previous lines, whether or not it does is a different question. Basically, the reasoning here is that since there are things in the world, and everything needs to have a cause other than itself, there needs to have been something to start it, something which got the ball rolling. 

\subsection{5. If there was a first cause, that cause was God.}

This sort of jump should look familiar, as it is found in both the Design and the Fine-Tuning Arguments. This jump is often found when people look at the big bang and ask "OK, what caused that?" When they find they get something like "hey, we don't know" they jump to God. This jump is just as exploitable as it was in the previous arguments. And this is where, from this line and the previous, we get the conclusion:

\subsection{6. Therefore, God exists.}

\section{Problems with the First Cause Argument}
\subsection{Problem 1: Self-contradictory}

In some versions of the argument, they are not careful with their phrasing and claim that \emph{everything} is said to have a cause other than itself, so God must have had a cause and there could not be a first cause. For example, take this argument:
\begin{enumerate}
   \item If it exists, it has a cause other than itself.
   \item If something is a cause, it has a further cause.
   \item God exists.
   \item So, God much have had a cause other what Herself.
   \item If there was such a cause, that cause was Meta-God.
    \item If that cause was Meta-God, then Meta-God exists.
    \item Therefore, Meta-God exists
\end{enumerate}
All I did here was apply the exact same reasoning which the First Cause argument uses to get that God exists, but applied it to God Himself (with the same restrictions). Since I am not a fan of taking on the letter of an argument and I prefer to attack the spirit of it, I have made this argument stronger by limiting the 'things' to objects in the natural world.  Many of the versions found in popular culture have this error and this saddens me because it's such an easy fix:
\begin{enumerate}
    \item Everything in the natural world has a cause other than itself.
    \item Everything which is a cause in the natural world has itself a cause other than itself.
    \item There are things in the natural world.
    \item So, there must have been a first cause from outside of the natural world to kick start the chain.
    \item If there was a first cause from outside the natural world, that cause was God.
    \item Therefore, God exists.
\end{enumerate}
This version, also, has its own problems which are in the very spirit of the argument, some of which we saw in the explanation of the argument.
\subsection{Problem 2: Not a proof}

This argument makes two background assumptions each of which can be used to show that it does not take all relevant possible cases into account, and thereby not really work as a proof. First, the argument assumes that the series of cause and effect can’t go back infinitely. But, this assumption requires its own argument to prove and may be intuitive because of our limited understanding of the earliest moments in the universe. This under-supported assumption does imply that there must have been a first cause. But, if you look at that assumption, there’s no reason to think that.

More over, the second line of the argument claims that nothing (at least in the natural world) can, directly or indirectly, cause itself to exist. But, in the previous page, I gave a seemingly possible case, assuming that you can have time-travel, where circular causation occurs. The chain doesn't, in a sense, continue back indefinitely and it does not have a first cause. Applying this to the universe itself and without the need for time-travel, it could be the case that all of time is circular, where the cause of the 'first' moment of the universe was the 'last' moment of the universe. This can be confusing, as in a circular model of time, ordering moments in terms of 'first' and 'last' is misleading, there are no such moments. 
\subsection{Problem 3: Limitations of the conclusion}

Just like with the other arguments for the existence of God, even if we accept all of their reasoning, there is still the jump. There is no reason to think that the first cause was God or any being with one or more of the tripartite features of God. It could have been a really evil kid playing with the equivalent of an ant farm. The universe could have been his ant farm. To illustrate this using other examples, take these. As before, if you accept the reasoning in the First Cause Argument, then you must also accept these, which are equally strong arguments:
\begin{tabular}{p{2in}|p{2in}}
    If it exists in the natural world, it has a cause other than itself.&If it exists in the natural world, it has a cause other than itself.\\
    If something is a cause in the natural world, it has a further cause.&    If something is a cause in the natural world, it has a further cause.\\
    Some things exist in the natural world.&Some things exist in the natural world.\\
    So, there must have been a cause from outside of the natural world to start it.&So, there must have been a cause from outside of the natural world to start it.\\
    If there was such a cause, that cause was random fluctuations of quantum strings.&    If there was such a cause, that cause was The Flying Spaghetti Monster.\\
    If that cause was random fluctuations of quantum, then random fluctuations of quantum strings exists.& If The Flying Spaghetti Monster was the cause, then it exists.\\
    Therefore, random fluctuations of quantum strings exists.&Therefore, The Flying Spaghetti Monster exists.\\
\end{tabular}
The left most argument, if I remember correctly, is roughly one given by Steven Hawking when there was a debate about this topic between physicists and theologians, but this was many years ago.

\section{The Ontological Argument}
To start us off, I will define a special term, which you might have encountered before. \Gls{ontology}, properly speaking, is an area of philosophy dealing with existence. It is the study of being, becoming, and some features of reality relating to existence.  I spent the majority of my undergraduate work dealing with these sorts of questions as it relates to time and also to composition. Sometimes, Ontology is called "Metametaphysics", but Metametaphysics tends to be more of Meta-ontology.  

\newglossaryentry{ontology}
{
name=ontology,
description={An area of Philosophy which concerns the study of what exists},
plural=ontologies
}


This term is also used in Computer Science, but there it's the formal representation of categories within a data structure. They use this term, I think, because much of the work in Ontology in Philosophy does relate to the representation and naming of objects in the world. 

As the name implies, \glspl{ontological argument} are arguments which deal with the existence of something by looking at the very nature of that thing. These sorts of arguments are very different from those we have seen, and they are typically not employed in pop-culture. They do not use evidence about how the world is, was, or will be, to show that the thing must exist. This was the major flaw in the previous arguments, because we can just counter those assertions with other possible explanations. Not relying on evidence, in this context, makes the arguments significantly stronger. Also, these kinds of arguments are a bit harder to wrap your head around, which is probably why they aren't being used in school-board debates.
\newglossaryentry{ontological argument}
{
name=ontological argument,
description={An argument which moves from the properties which an object/thing is purported to have to the claim that the object/thing must exist},
plural=ontological arguments
}

\section{Anselm’s Ontological Argument}

The oldest, and one of the strongest, ontological arguments was written by St. Anselm. Anselm was born in 1033 CE and died in 1109 CE. He was proclaimed a saint in 1163 CE and his feast day is April 21st. As far as I could find, he is not the patron saint of anything. 

Anselm starts by defining God in a different way than the one which we have seen. Thus far, we have had God be whatever has these three features: All-Knowing, All-Powerful, and All-Good. But, Anselm defines God as:
\begin{center}
"That being than which nothing greater can be conceived."
\end{center}
This definition does rely on a few things, which weaken the letter, but not the spirit of the argument (such as the powers of human imagination), so we will make it a little stronger by saying that God is:
\begin{center}
"The being such that there is nothing can be greater than it."
\end{center}
Essentially, we are saying that God is the greatest possible being, in all respects. But, first off, are we talking about the same thing? Some could easily argue yes. When it comes to power, nothing can be more powerful than God (as Anselm defined the term), so God is all-powerful. Nothing can know more than God, so God is all-knowing. With goodness, we run into an interesting point though. Anselm thinks, as many do, that evil is the absence of good, so 'the greatest evil' is a misnomer and it should be 'the least good'. This means that God can't be the greatest evil possible, but is the greatest good.  

Remember that Ontological Arguments look at the very nature of a thing to show that it must exist. Does being the greatest in everything imply that God exists? According to those who like this argument, YES! And here is their argument:

\subsection{1. God is the greatest possible being.}

This is the definition we are working from, remember that this gets us the being which we have discussed before, but also gets us some other features which may have been assumed by some of you before, such as omnipresence. 

\subsection{2. Any existing being will be greater in some respects than a non-existing being.}

This one is a little different, so I will use certain examples to help. I exist and because of this, I have certain powers and abilities which non-existent things lack. For example, I can interact directly with physical objects, while non-existing things, like fictional characters can't do that, their powers are limited to the will of the author and the minds of the reader. So, in at least some respects, I am greater than a non-existing thing, in virtue of my existence. This does not just apply to me, but also to any existing thing, I would just need to fiddle with the examples. 

\subsection{3. The greatest possible being cannot have anything greater than it in any respects.}

As we have defined it, the greatest possible being is the greatest in all respects. As such, no being, existing or not, can be greater than it in any respects. But, as a tie in, remember that any existing thing will be greater than a non-existing one in some respect.

\subsection{4. So, if the greatest possible being does not exist, then there are things which are greater than it in some respects.}

If existing things are always greater than non-existing ones in some respects, then if the greatest possible being doesn't exist, there would be something, namely an existing thing, which has powers, abilities, positive features which it lacks. But, the greatest possible being having something greater than it should seem contradictory. 

\subsection{5. So, the greatest possible being exists.}

This comes from the fourth line. The seeming contradiction is actual (or defensible). Assuming that the greatest possible being doesn't exist gives us a contradiction, so long as there are existing things (which should be obvious). If something leads to a contradiction, then we know that it can't be true. So, the greatest possible being must exist. 

\subsection{6. Therefore, God exists.}

Remember how we defined God, God is the greatest possible being. What is true for one is true for the other, they are the same thing.

\section{Problems for the Ontological Argument}
\subsection{Problem 1: The Perfect Island}

This was actually the inspiration behind my versions of the arguments seen in the Limitation of the Conclusion sections. It is easy to imagine the absolutely perfect island. Perfect weather, wildlife, beach, and so on (think Hawai’i without the tourists). But, that island would be more perfect if it existed (same reasoning as before). This island is the most perfect island, therefore, the perfect island must exist. We can do that for the perfect taco, the perfect wife/husband, or anything, the perfect version of it must exist. Now I have to ask, where is my perfect island?

This is the Ontological Argument's equivalent of Limitation of Conclusion, but you need to be a little more careful about the counter examples used.  Sometimes, the example won't work because of the nature of the thing, for example, the perfect fictional character can't be used for this because the perfect fictional character can't exist, because then it would not be fictional. Take for example, these two arguments, which follow the same reasoning as the Ontological Argument:
\begin{tabular}{p{2in}|p{2in}}
    Alex is the greatest possible significant other.&God is the greatest possible being.\\
    Any existent being is greater in some respects than non-existent beings.&Any existent being is greater in some respects than non-existent beings.\\
    The greatest possible significant other can not have anything greater than it in any relevant respects.&The greatest possible being can not have anything greater than it in any respects.\\
    So, if the greatest possible significant other does not exist, then there are things which are greater than it in some relevant respects.&So, if the greatest possible being does not exist, then there are things which are greater than it in some respects.\\
    So, the greatest possible significant other exists.&So, the greatest possible being exists.\\
    Therefore, my significant other, Alex, who lives in Canada, exists.&Therefore, God exists.\\
\end{tabular}
 
\subsection{Problem 2: Existence is not a property}

Take for example, ‘unicorn’. The definition of that word is ‘a horse with one horn on its head’. Now, if I said ‘unicorns exist’, I am not adding another property to unicorns. For anything to be a unicorn, it must first exist. But the concept of unicorns does not change whether or not there are unicorns.

Applying this to the argument, we notice that it treats existence as just another property. Something which a thing is greater for having. But this is a mistake, existence is often seen as a precondition for having properties. We can have the concept of the greatest possible being, but that does not entail that such a being exists.

In grad-school, I did several presentations arguing against the idea that existence is not a property and also against that existence is a precondition for having properties. But that was really advanced stuff and I will not go into depth about it here.

\subsection{Problem 3: Evil}
This is our segue into the next topic of this module, which is the problem of evil. If we look at the world, there is a HUGE amount of pain and suffering in it (AKA evils). If God does actually exist, then there would not be these evils (because He is all good). So, God must not exist. 

\chapter{Part 11: An Argument Against the Existence of God}

The \gls{Problem of Evil}, which I abbreviate as POE, is the family of arguments against the existence of God which we will be discussing. This is one of those arguments where what doesn't kill it makes it stronger. As a result, there are many different versions of it, each with their own special objection (explanations for how God can exist despite the argument). The objections to the arguments for the existence of God were not arguments against the existence of God, rather they are ways that the argument for the existence of God is flawed. You do not want flawed arguments. For this class, we will deal with, in general, two kinds of POE, which are separated by the kind of evil they use. 

\newglossaryentry{Problem of Evil}
{
name=Problem of Evil,
description={An argument against the existence of God which points to the pain and suffering on the Earth and asks how an all-knowing, all-powerful, and all-good god could allow for such a thing},
plural=Problems of Evil
}


First, we have what I'll call \gls{man-made evil}. These are also sometimes called 'moral evils'. Man-made evils are the sufferings which people cause, the 'bad' things we do. Do not get hung up on the use of the word 'evil'. We will see later why good and evil aren't relative or cultural, but evil, in either case, here concerns suffering, which is not a man-made concept. Man-made evils are things like torture, murder, famine (yes, famines are man-made), and so on. 

\newglossaryentry{man-made evil}
{
name=man-made evil,
description={The pain and suffering and otherwise `bad' things which human persons do through free will},
plural=man-made evils
}


Second, we have\gls{natural evil}. This is the kind of suffering which is not caused by man, but nature doing its thing. Since we are talking about suffering when we use the term 'evil', it should be clear that natural evils are a thing, because there are some sufferings caused by nature doing as it does. Here we have things like earthquakes, disease, volcanic eruptions, tornado, and sharknados. 

\newglossaryentry{natural evil}
{
name=natural evil,
description={The pain and suffering and otherwise `bad' thingswhich are caused by natural processes of the world, not caused by human free action},
plural=natural evils
}


The stronger of the two, for our purposes, concerns natural evils, because moral evils can be easily explained as not God's doing, but man's. But, unless I specify otherwise, when I just use the term ‘evil’ or ‘evils’, I am referring to the set of both. That being said, here is the Problem of Evil Argument:
\begin{enumerate}
    \item There are evils in the world.
    \item If God exists, He is all-knowing, all-powerful, and all-good.
    \item If He is all knowing, then (a) He knows about the evil.
    \item If He is all powerful, then (b) He can stop the evil.
    \item If He is all good, then (c) He would want to stop the evil.
    \item If (a), (b), and (c), then there would not be these evils in the world.
    \item Therefore, God does not exist.
\end{enumerate}
As you might guess, the most ripe for attack, the one which the theists really want to take down, is line 5. Typically, the problems or claimed solutions to POE which keep God in the mix, give some reason why God wouldn't want to stop the evil which is in the world. But, the vast majority of those counter points tend to fall flat. 

\section{The Saints and Heroes Response}
This is the claim that God allows for evil in the world because it leads to greater goods which could not have been had otherwise. Evils are necessary for the even greater good of having saints and heroes. When we learn about these evils we are called to arms against them, we are made heroes. Take for example this story, the events happened, and this only really works as an example of man-made evils:

    \factoidbox{In the First Crusade, there were several armies. Some of them were more official and organized than others. An unofficial, arranged and poorly managed group in the first crusade was lead by a Count Emicho from Germany. Emicho's plan for this group was for them to pass from Germany to the Holy Land immediately, which was a major issue because of food scarcity during that time of the year. Rather than following this ill-conceived plan, they eventually thought that the trek to the Holy Land was too far, and noticed that the local Jewish population was a lot closer and unarmed. As a result, they went from town to town, their own towns, pillaging and slaughtering the Jewish people in the region.  The Church, to their credit, was vehemently opposed to this. They at the very least wanted the fighting outside of Christian dominated territory.  Despite this horrific time, heroes arose to fight this evil. Priests and bishops not only preached against the violence, but even hid the Jewish population in their homes, fortifying the place. Some, if they had the means, raised personal armies to protect them. They used their religious might to protect Jewish places of worship and did their best to save lives.}

If this suffering had not happened, we would not have the good of having these sorts of role models, people who would fight members of their own religion to protect people outside of that circle. We would not have the good of these heroes being in the world.

An alternative version of this, which does escape some of the problem below is called the soul-making theodicy. This is that God allows for evil in the world to refine people, to make them better, a good which could not otherwise be gotten. Some philosophers really like this soul-making theodicy, stating that the purpose of evil is to allow free beings to grow and develop into those of the "finest characteristics". The Christian philosopher John Hick proposes this sort of reasoning in his work.\autocite{Hick1} Other reasons like this are related: For example, God allows for evil as a test of faith and guides people towards God and stronger faith, again something which is said to not be able to get otherwise. This explanation for evil seems to be quite popular among classical Islamic philosophers.\autocite{Rouzati1}

The problem with this is that much of the natural evils in the world are not recorded, we are not called to arms against them, and therefore could not have served to create the greater good. (Things like genetic diseases which lead to a slow and painful death are great examples, especially when there is no possible cure).

\section{The Artful Analogy}
Some claim that the world is like God’s canvas, He is painting a picture. There are shades of good everywhere and so too are there shades of evil. The contrasting elements of good and evil make the picture more beautiful than if it was only all shades of good. The goodness of the world is a product of the harmony between these shades of good and evil. Philosophers who have liked this kind of view, or at the very least wrote something which is akin to it include: Leibniz in his Theodicy\autocite{Leibniz1}. (paragraph number 213) and Augustine in his Confessions (book 7, chapters 13-16).\autocite{Augustine1} For Augustine, evil is there to make a more beautiful, harmonious whole of creation. We can't see the whole picture because we aren't God. 

An apt analogy for this idea is to take two pictures (and for this analogy, the darker the shade, the more 'evil' it's meant to represent) and compare them. The first is just a blank white canvas, the other is a photo-realistic black-and-white picture of a rose. Which of these would you say is more beautiful? The rose or the blank canvas?

There are two major problems for this. The first is that it is hard to believe. Why would it be the case that a baby born with Tay-Sachs disease (a genetic disease which leads to a slow, painful, guaranteed death) would help the overall harmony of the world. Since God allows for so much evil and only God can see the picture, so to speak, then God must not be all-good in our sense, but in some other sense.True art is an explosion!

The second problem for this is that if God allows for the suffering to beautify the world, then God seems more like a sadist than an all-good entity. If suffering plays an artistic role in the world, then God is very close to a psychopathic artist who throws a bomb into a crowded elevator in order to admire the patterns created by the blast and the resultant splatter.

\section{The Free Will Defense}
Philosophers danced around The Free Will Defense (FWD) as an explanation for Man-Made Evil for a lot of philosophic history, but was never really explicitly given. Most think that this is a very old and maybe obvious reply, but the current and,  by far, most popular version, (even appearing in popular culture) is quite contemporary. The version which we will be discussing was given by Alvan Plantinga\autocite{Plantinga1}

As a reply to the problem of evil, FWD is also extremely popular. This is the claim that human beings are special in that we have free will. If we lacked this aspect, we would be like robots or automata and could make no choices of our own. If you are familiar with Star Trek, if we lacked free will, then we would be like Data, but with emotions. Those that go with this defense claim that  necessary consequence of having free will is the ability to do evil. Without this ability, then we would not really have free will.  A world with free will and the possibility of doing evil is better than a world without free will. God made the best world possible, so, He made one with free will and thereby one with evil (suffering). The Free Will Defense goes like this, as before the bold lines are the premises:

\subsection{1. Any world with free will is better than one without.}

This is the core intuition, or basis, for FWD. The thought here is that free will must be so valuable that any resulting suffering would be outweighed by it. Similar intuitions can be found in discussions about love/faith. For example, love is seen as a supreme good, but for a person to really love another, this must be voluntary, unforced, and originating, primarily, from the lover (or so it's claimed). This means that the person must have free will to really give love to another. A world with love is better than a loveless one. So, assuming that free will of this sort is required for love, a world with free will is better than one without.  

\subsection{2. Any world with free will is going to have some evils (suffering) in it.}

This is a build-up from the previous line and will be attacked when we look at the problems with this defense. This is, sort of, I think, based on the intuition that with free will comes the actual ability to cause suffering and, as many of us have seen, if people have the ability to do something, someone, eventually will do it. Every aspect of this line, and the previous is suspect and very much open to rebuttal. 

\subsection{3. God made the best of all possible worlds.}

Since this is a reply to POE, we will see that it assumes the existence of God, which we will give as a pass. They are, in a sense, giving excuses for God's allowing evil. This is much like how a defense lawyer will assume that their client is innocent while in court. Since God is all-good, it makes sense that, if He made a world, it would be the best world possible for Him to make. 

\subsection{4. So, God made a world with free will.}

This follows from the first and third lines of the argument. Assuming that a world with free will is going to be better than one without it, it would follow that the best world which one could make would be one with free will. With that established, God making the best possible world would be a world with free will.

\subsection{5. Therefore, God allows for evil as it's the result of free will.}

This follows from the second and fourth lines of the argument. Assuming, again, that a world with free will is going to have some suffering in it and that God made a world with free will, it follows that God must allow for the suffering in the world because that suffering is a necessary consequence of having free will. 

\subsection{Connection with previous content:}

This should remind you about the stuff we have covered in Module 4. There, just as a refresher, we covered the various ideas and arguments for and against free-will. Be aware that the Free Will Defense requires incompatibilism and, more over, it requires libertarian free will. If you want to go with this defense, then you need to believe in Libertarianism. You must physically, not just counterfactually, be able to do evil and do otherwise than what you do.  From here, we will continue down the free will defense rabbit hole and see some of the objections to this idea.
\section{A Few Initial Replies to the Free Will Defense: How good is Free Will? Do we even have it?}

There are a few problems for the Free Will Defense, and they can be broken up into two different kinds, with a few exceptions. The first concerns the actual value of free will and the second concerns whether or not we actually have it. 
\subsection{Is Free Will that Good?}

This question relates to the first line of the argument, which claimed that any world with free will is going to be better than one without it. But, is that so obvious? The suffering in the world can be so great that it would make sense to want this possibility to go away. Here is a case to think about:

    \factoidbox{Imagine, if you will, two worlds, the first is a world with all of the suffering and anguish which is present in ours. In this world, people have this Libertarian Free Will. From a first person perspective, one could not tell the difference between 'making a choice' in this world vs in the world we presently live in. The second world is one without the vast majority of the suffering which we inflict on each other. For example, WW2 never happened in this world. This decrease in suffering is because there's no Libertarian Free Will in this world. Making a choice in this world, from a first person perspective, would feel exactly the same as making a choice in the world with free will. A person could not tell the difference. Which would you chose to live in? The one with unnecessary moral evils or the one without?}

Similarly, we can there are certain moral evils which I will never get the chance to preform. For example, I will never get the opportunity to kick the pope in the shin. What if all evil acts were like that? What if people were never given the opportunity to do the evil acts or everyone had a moral character such that they would never do such things. Sure, they are possible, but we never get the chance to do them or would never do them. Am I less free because of my lacking the opportunity?
\subsection{Do We Actually Have it?}

That last note leads us into an interesting other problem. Do we actually have free will? Or do we just have some kind of illusion of it? The idea of free will is hotly debated, as we saw in Module 4. Though this is a bit of a fallacy, the majority of philosophers out there believe that people have free will, but, this is not the kind of free will necessary for the Free Will Defense. Namely, most philosophers, at least the vast majority I know, think that compatibilism is true. In such a world, it is perfectly possible that we could have done otherwise (in the counterfactual sense) but never actually do evil. More over, the choices we make, through various tests, can be shown to come from past experience and conditioning, we are programmed (so to speak). Which makes a serious problem for FWD. 
\subsection{Free Will Without Evil}

This one is related to the second kind of objection, namely, whether or not we have free will, but ties in to the second kind. One could argue that this objection gets the relevant parts of both. In this case we are asking whether it's possible to have free will without having the possibility to actually act/do evil. Much like the case which I gave before. The argument goes like this:

    It is possible for all people to have free will and yet never bring about evil.
    God can bring about any possible situation.
    Therefore, God could bring about such a situation.

The issue here for FWD is that if this is possible, there's a better world than the one we have and therefore God didn't make the best one, taking all of the wind out of the sails of FWD.  There have been some replies to this argument which are worth noting.

First, initially, people like to reject the first premise and claim that God could not have made a world like that. Saying something along the lines of "that would not really be free will!" However, going this route faces issues on its own. First, many claim that God is free but at the same time He/She does not bring about evil. So, as a result, why would it be impossible for man to be free and never bring about evil? Second, if we try to reject the first line, many of us think that God never wants us to commit evil acts. And yet, if we reject this line, it would mean that God is wishing for the impossible.

If you chose to reject the second line instead, you would be claiming that there are certain possible situations which God could not bring about. For example, it is possible for me to freely type this passage, but it's not possible for God to make it the case that I freely type this passage because then it would not have been freely done; there's a contradiction. However, this particular route does not help in this case, it just shows that there are certain possible cases which God can't bring about. This, in turn, leads us down yet another rabbit hole about the very nature of being all-powerful. But, this also doesn't help because we can look at cases where people just aren't ever given the opportunity to act in certain ways. For example, it's a very serious possibility that I will never be given the opportunity to kick the pope in the shin. Although I would not want to do this, as a matter of fact, the fact remains that I will never be given the chance to do it. God could, quite easily, it would seem, make it the case that all opportunities to commit evil acts are like mine when it comes to kicking the pope in the shin, man is just never given the chance to do them. Would you say that I am less free because of this lack? If you say no, then we can have a world without evil and free will.
\subsection{Miracles}

Another way in which God could get rid of, or at least limit, the number of evils in the world while retaining free will is to intervene more often.

People typically think that God intervenes in the world through miracles. But why does God do things which seem like minor tricks? Things like killing fig trees (one of Jesus’ miracles), producing stigmata on people’s hands, or making water into wine? Why doesn’t He jump in and prevent or stop AIDS or the entirety of WW2? All God would need to do is whisper into people's ears at certain times to prevent them from committing the truly horrible acts of history.

The reply to this is typically that if God intervened then it would not be free will… But this doesn’t really sit with the idea of miracles.

\section{The Problem of Natural Evil}
The Free Will Defense, if it can surmount the problems which we have seen previously, would be able to handle the problem of evil, but only in a restricted sense. The Free Will Defense can only handle Man-Made Evil. But, there are other kinds of evils in the world. Remember, we also have Natural Evils to contend with. Remember, also, that whatever doesn't kill the Problem of Evil only makes it stronger. This leads us to the Problem of Natural Evil (or PONE, because it 'PONEs' the Free Will Defense). There is no real connection (at least from the view of a non-believer) between free will and the plague, earthquakes, volcanic eruptions, etc. This argument is taking this into account. Rather than taking evils as any old kind of 'bad' in the world, here we will focus on natural evils, the pain and suffering caused by nature doing its thing. 
\begin{enumerate}
    \item There are natural evils in the world.
    \item If God exists, He is all-knowing, all-powerful, and all-good.
    \item If He is all knowing, then (a) He knows about the natural evils.
    \item If He is all powerful, then (b) He can stop the natural evils.
    \item If He is all good, then (c) He would want to stop the natural evils.
    \item If (a), (b), and (c), then there would not be these natural evils in the world.
    \item Therefore, God does not exist.
\end{enumerate}
This should look very familiar from the regular old Problem of Evil. The Problem of Natural Evil is far stronger because it limits the scope. It is possible, however, to use some of the responses to the Problem of Evil in response to the Problem of Natural Evil, but they are significantly weaker. 

\subsection{Saints and Heroes for The Problem of Natural Evil}

With the Saints and Heroes Response, we have that evil is in the world to make us better, to give us something to fight against. One way to think about this was that our souls are rough when we are created and the evils in the world are there to put us through the proverbial rock-tumbler. Natural Evils are there to give us something to fight against, having diseases gives us the goal of finding a cure and that process refines us. 

But this does have its issues too, much like the regular old Saints and Heroes Response. For example, it seems plausible, and in fact, likely, that the moral evils in the world are enough to spur people into action. The evils which people do to each other constantly need to be fought against and the additional natural evils seem like an extra, unneeded, addition to get the saints and heroes. God would, if She is all good, want to refine us with as little suffering as possible. 
\subsection{The Artful Analogy for The Problem of Natural Evil} 

For this reply, it was claimed that God allows for evil/suffering in the world because it makes the world more beautiful. God is all-good, in this sense, means that God is not morally perfect, but rather good as in a great artist. Following this line of thought, it's possible that the process of the environment restoring itself is beautiful in the eyes of God. Natural evils are the destruction necessary to rebuild and reshape.

This one has the same issues as before, but also one unique to it. For example, rather than resetting the already vibrant areas, why not just make another planet and start fresh? This would get the good of making something new without the suffering. Or in another case, when you have constructed something beautiful in Minecraft and get bored of it, do you plant TNT around it and blow it up or do you save it and start building in a new world? 

\subsection{The Fall Reply}

Forgive me for my lack of familiarity with the creation story in Christianity and other Abrahamic religions Some claim that the natural evils in the world are not God’s doing, but are the doing of Adam and Eve in the Garden of Eden. This causes all natural evils to actually be man-made evils, which the Free Will Defense seems to be able to handle (maybe). In eating the apple, they created all of the natural evils in the world. We can think of the natural evils in the world as the punishment for Adam and Eve's initial breaking of the rule "don't eat from this tree." This sort of response is often found coming from the Biblical literalists, those who hold that the Earth was created in 7 days and all that. 

There are a few issues with this reply, however. First, if God is all-good, then it would seem that He would need to be all-just (as in, never violating justice). Does it seem fair or correct, morally speaking, to punish the children for the actions of their parents? This seems especially wrong when the punished did not in anyway benefit from the transgression. For example, suppose that hundreds of years ago, my ancestor attempted to steal a pig. This pig ran away and back to the owner. The laws of the day say that even attempted pig stealing in punishable with the removing of the thumbs. Luckily, my ancestor was not caught. Generations down the line, it's discovered that my ancestor did this. Would it be morally permissible for them to punish me for the attempted theft, even though I had nothing to do with it? Is it OK to punish someone for the actions of their ancestor thousands of years ago?

Notice that I said "thousands of years ago". The second issue with this reply is that it requires that the story of Adam and Eve be an actual historical event. This can't be merely mytho-historic (where we say we came from). There are multiple cases where this does not line up with science and other reliable aspects of the world. But, if those can be surmounted, then this would not be an issue.
\subsection{The Demon Reply}

This reply is often cited to St. Augustine and it sort of has a feel like the phrase "the Devil made me do it". This follows from a similar line of thought as the Fall Reply, but also from a similar line of thought as the Free Will Defense. Here, like with the Fall Reply, they are trying to make a connection between these natural evils and 'moral evils' (evils by free agents (not necessarily human)). The claim is that God did not just give free will to humans, but also to certain divine beings (angels). This means that the actions of these divine entities are explained away as man-made evils, and God is not responsible for it. When Lucifer and the other Fallen rebelled, they did so freely. It's those Fallen angels which cause the 'natural' evils in the world.  Since man's free will allows for us to do evil and it's still good, these demons are in the same sort of camp. In fact, many demons are claimed to be responsible for various kinds of natural disasters.  Such as Furfur who is said to create tempests.

The responses to this sort of reply are varied. I encourage you to think of some and try them out in your heads.
\subsection{The Punishment Reply}

Although this one, also, does not rely on the Fall, it's an option. This is the claim that natural disasters are the result of people committing sins, but not necessarily the sins of your ancestors. So, if enough people eat shrimp, God will send down a tidal wave. I thought, for a long time, that this sort of thinking was historical and that no one really believed it, however, this kind of thinking/explanation can be found in very extreme and offensive sects of Christianity, such as the Westboro Baptist Church (don't look them up, they don't deserve the traffic). The reasoning here is that when a crime is committed, it's unjust for that crime to go unpunished. A just world is better than an unjust one, so a world with the kind of punishment dealt by natural evils is the better of possible options.

A quick and easy reply to this general idea runs down a similar line of reasoning as one we saw in the Fall Reply. Justice seems to require that the person who committed the crime be the one who receives the punishment. However, more often than not, these natural disasters seemingly hit indiscriminately. Those who are the most affected are often those who least deserve it. For example, the extremely poor.
\subsection{The Good-Needs-Evil Reply}

This is a very classic claim. Essentially, although the nature of good and evil are objective, there is a relation between them. Good, in this sense, is the opposite of evil. Like ‘left’ and ‘right’, one can’t logically have one without the other. So, God allows for evils because they are a logical consequence of having goods. One issue with this is that the nature of natural evils seems to be too much. Why have natural evils when moral evils will do?
\subsection{The Beneficial Laws of Nature Defense}

Called BLOND for short, this stance claims that the laws of nature result in the disasters and other sufferings caused, but they have an outweighing good. This good is that they are consistent. Having consistent, regular laws of nature give rise to the processes which lead to humans being able to evolve and have free will. This could not have come about any other way (the law of nature needed to be as they are, this is an assumption worth examining), so the natural evils are just as much a byproduct of having beings with free will as the man-made evils. There are other aspects of this which could be seen as good, for example, having the laws of nature be constant and unchanging allows for the good of scientific discovery and exploration. If there was some inconsistent aspect of these laws, eventually we would discover it and be unable to progress further. 

But there are some issues for this stance. The first problem deals with the all-powerful nature of God. This does not explain why an all-powerful God chose those laws of nature. It is perfectly conceivable that He could have made laws which (in conjunction with other things) did not lead to natural evils. Similarly, God could have made it so that natural evils don’t happen on planets where life would arise. Man-made evils, sure, but not natural ones. Intelligent beings, upon developing high enough, could discover the more naturally violent worlds, and thereby further marvel at the wonder of divine creation.

A reply to this is to say that even God is bound by the laws of nature. And, as a result, is not really all-powerful. But this also detracts from the assumptions made to get the defense started.  In a related aspect, miracles are often pointed to as God’s work, but those are violating the laws of nature, or at least bending them, so if God is bound by the Laws of Nature, how could they happen?

The second problem involves the idea of miracles. If we believe that God intervenes in the ways that He does, through miracles, then we need to ask why He engages with us in certain ways. Why turn water into wine when He could freeze the volcano? Every natural evil which are the byproduct of the laws of nature could be prevented by a being as powerful as God. If God never intervenes (as some claim, because that would hinder free will), then there would never have been any miracles. 
