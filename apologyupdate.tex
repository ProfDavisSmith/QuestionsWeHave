\chapter{Plato's Apology Translated By Harold North Fowler}
\label{platoapology}

\marginpar{17a}
How you, men of Athens, have been affected by my accusers, I do not know; but I, for my part, almost forgot my own identity, so persuasively did they talk; and yet there is hardly a word of truth in what they have said. But I was most amazed by one of the many lies that they told—when they said that you must be on your guard not to be deceived by me, \marginpar{17b} because I was a clever speaker. For I thought it the most shameless part of their conduct that they are not ashamed because they will immediately be convicted by me of falsehood by the evidence of fact, when I show myself to be not in the least a clever speaker, unless indeed they call him a clever speaker who speaks the truth; for if this is what they mean, I would agree that I am an orator—not after their fashion. Now they, as I say, have said little or nothing true; but you shall hear from me nothing but the truth. Not, however, men of Athens, speeches finely tricked out with words and phrases, \marginpar{17c} as theirs are, nor carefully arranged, but you will hear things said at random with the words that happen to occur to me. For I trust that what I say is just; and let none of you expect anything else. For surely it would not be fitting for one of my age to come before you like a youngster making up speeches. And, men of Athens, I urgently beg and beseech you if you hear me making my defence with the same words with which I have been accustomed to speak both in the market place at the bankers tables, where many of you have heard me, and elsewhere,\marginpar{17d} not to be surprised or to make a disturbance on this account. For the fact is that this is the first time I have come before the court, although I am seventy years old; I am therefore an utter foreigner to the manner of speech here. Hence, just as you would, of course, if I were really a foreigner, pardon me if I spoke in that dialect and that manner \marginpar{18a} in which I had been brought up, so now I make this request of you, a fair one, as it seems to me, that you disregard the manner of my speech—for perhaps it might be worse and perhaps better—and observe and pay attention merely to this, whether what I say is just or not; for that is the virtue of a judge, and an orator's virtue is to speak the truth.

First then it is right for me to defend myself against the first false accusations brought against me, and the first accusers, and then against the later accusations and the later accusers. \marginpar{18b} For many accusers have risen up against me before you, who have been speaking for a long time, many years already, and saying nothing true; and I fear them more than Anytus and the rest, though these also are dangerous; but those others are more dangerous, gentlemen, who gained your belief, since they got hold of most of you in childhood, and accused me without any truth, saying, “There is a certain Socrates, a wise man, a ponderer over the things in the air and one who has investigated the things beneath the earth and who makes the weaker argument the stronger.” These, men of Athens, \marginpar{18c} who have spread abroad this report, are my dangerous enemies. For those who hear them think that men who investigate these matters do not even believe in gods. Besides, these accusers are many and have been making their accusations already for a long time, and moreover they spoke to you at an age at which you would believe them most readily (some of you in youth, most of you in childhood), and the case they prosecuted went utterly by default, since nobody appeared in defence. But the most unreasonable thing of all is this, that it is not even possible {18d} to know and speak their names, except when one of them happens to be a writer of comedies. And all those who persuaded you by means of envy and slander—and some also persuaded others because they had been themselves persuaded—all these are most difficult to cope with; for it is not even possible to call any of them up here and cross-question him, but I am compelled in making my defence to fight, as it were, absolutely with shadows and to cross-question when nobody answers. Be kind enough, then, to bear in mind, as I say, that there are two classes \marginpar{18e} of my accusers—one those who have just brought their accusation, the other those who, as I was just saying, brought it long ago, and consider that I must defend myself first against the latter; for you heard them making their charges first and with much greater force than these who made them later. Well, then, I must make a defence, men of Athens, \marginpar{19a} and must try in so short a time to remove from you this prejudice which you have been for so long a time acquiring. Now I wish that this might turn out so, if it is better for you and for me, and that I might succeed with my defence; but I think it is difficult, and I am not at all deceived about its nature. But nevertheless, let this be as is pleasing to God, the law must be obeyed and I must make a defence.

Now let us take up from the beginning the question, what the accusation is from which the false prejudice against me has arisen, in which \marginpar{19b} Meletus trusted when he brought this suit against me. What did those who aroused the prejudice say to arouse it? I must, as it were, read their sworn statement as if they were plaintiffs: “Socrates is a criminal and a busybody, investigating the things beneath the earth and in the heavens and making the weaker argument stronger and \marginpar{19c} teaching others these same things.” Something of that sort it is. For you yourselves saw these things in Aristophanes' comedy, a Socrates being carried about there, proclaiming that he was treading on air and uttering a vast deal of other nonsense, about which I know nothing, either much or little. And I say this, not to cast dishonor upon such knowledge, if anyone is wise about such matters (may I never have to defend myself against Meletus on so great a charge as that!),—but I, men of Athens, have nothing to do with these things. \marginpar{19d} And I offer as witnesses most of yourselves, and I ask you to inform one another and to tell, all those of you who ever heard me conversing—and there are many such among you—now tell, if anyone ever heard me talking much or little about such matters. And from this you will perceive that such are also the other things that the multitude say about me.

But in fact none of these things are true, and if you have heard from anyone that I undertake to teach \marginpar{19e} people and that I make money by it, that is not true either. Although this also seems to me to be a fine thing, if one might be able to teach people, as Gorgias of Leontini and Prodicus of Ceos and Hippias of Elis are. For each of these men, gentlemen, is able to go into any one of the cities and persuade the young men, who can associate for nothing with whomsoever they wish among their own fellow citizens, \marginpar{20a} to give up the association with those men and to associate with them and pay them money and be grateful besides.

And there is also another wise man here, a Parian, who I learned was in town; for I happened to meet a man who has spent more on sophists than all the rest, Callias, the son of Hipponicus; so I asked him—for he has two sons—“Callias,” said I, “if your two sons had happened to be two colts or two calves, we should be able to get and hire for them an overseer who would make them \marginpar{20b} excellent in the kind of excellence proper to them; and he would be a horse-trainer or a husbandman; but now, since they are two human beings, whom have you in mind to get as overseer? Who has knowledge of that kind of excellence, that of a man and a citizen? For I think you have looked into the matter, because you have the sons. Is there anyone,” said I, “or not?” “Certainly,” said he. “Who,” said I, “and where from, and what is his price for his teaching?” “Evenus,” he said, “Socrates, from Paros, five minae.” And I called Evenus blessed, \marginpar{20c} if he really had this art and taught so reasonably. I myself should be vain and put on airs, if I understood these things; but I do not understand them, men of Athens.

Now perhaps someone might rejoin: “But, Socrates, what is the trouble about you? Whence have these prejudices against you arisen? For certainly this great report and talk has not arisen while you were doing nothing more out of the way than the rest, unless you were doing something other than most people; so tell us \marginpar{20d} what it is, that we may not act unadvisedly in your case.” The man who says this seems to me to be right, and I will try to show you what it is that has brought about my reputation and aroused the prejudice against me. So listen. And perhaps I shall seem to some of you to be joking; be assured, however, I shall speak perfect truth to you. \marginpar{21a} He was my comrade from a youth and the comrade of your democratic party, and shared in the recent exile and came back with you. And you know the kind of man Chaerephon was, how impetuous in whatever he undertook. Well, once he went to Delphi and made so bold as to ask the oracle this question; and, gentlemen, don't make a disturbance at what I say; for he asked if there were anyone wiser than I. Now the Pythia replied that there was no one wiser. And about these things his brother here will bear you witness, since Chaerephon is dead. \marginpar{21b} But see why I say these things; for I am going to tell you whence the prejudice against me has arisen. For when I heard this, I thought to myself: “What in the world does the god mean, and what riddle is he propounding? For I am conscious that I am not wise either much or little. What then does he mean by declaring that I am the wisest? He certainly cannot be lying, for that is not possible for him.” And for a long time I was at a loss as to what he meant; then with great reluctance I proceeded to investigate him somewhat as follows.

I went to one of those who had a reputation for wisdom, \marginpar{21c} thinking that there, if anywhere, I should prove the utterance wrong and should show the oracle “This man is wiser than I, but you said I was wisest.” So examining this man—for I need not call him by name, but it was one of the public men with regard to whom I had this kind of experience, men of Athens—and conversing with him, this man seemed to me to seem to be wise to many other people and especially to himself, but not to be so; and then I tried to show him that he thought \marginpar{21d} he was wise, but was not. As a result, I became hateful to him and to many of those present; and so, as I went away, I thought to myself, “I am wiser than this man; for neither of us really knows anything fine and good, but this man thinks he knows something when he does not, whereas I, as I do not know anything, do not think I do either. I seem, then, in just this little thing to be wiser than this man at any rate, that what I do not know I do not think I know either.” From him I went to another of those who were reputed \marginpar{21e} to be wiser than he, and these same things seemed to me to be true; and there I became hateful both to him and to many others.

After this then I went on from one to another, perceiving that I was hated, and grieving and fearing, but nevertheless I thought I must consider the god's business of the highest importance. So I had to go, investigating the meaning of the oracle, to all those who were reputed to know anything. And by the Dog, men of Athens \marginpar{22a} —for I must speak the truth to you—this, I do declare, was my experience: those who had the most reputation seemed to me to be almost the most deficient, as I investigated at the god's behest, and others who were of less repute seemed to be superior men in the matter of being sensible. So I must relate to you my wandering as I performed my Herculean labors, so to speak, in order that the oracle might be proved to be irrefutable. For after the public men I went to the poets, those of tragedies, and those of dithyrambs, \marginpar{22b} and the rest, thinking that there I should prove by actual test that I was less learned than they. So, taking up the poems of theirs that seemed to me to have been most carefully elaborated by them, I asked them what they meant, that I might at the same time learn something from them. Now I am ashamed to tell you the truth, gentlemen; but still it must be told. For there was hardly a man present, one might say, who would not speak better than they about the poems they themselves had composed. So again in the case of the poets also I presently recognized this, \marginpar{22c} that what they composed they composed not by wisdom, but by nature and because they were inspired, like the prophets and givers of oracles; for these also say many fine things, but know none of the things they say; it was evident to me that the poets too had experienced something of this same sort. And at the same time I perceived that they, on account of their poetry, thought that they were the wisest of men in other things as well, in which they were not. So I went away from them also thinking that I was superior to them in the same thing in which I excelled the public men.

Finally then I went to the hand-workers. \marginpar{22d} For I was conscious that I knew practically nothing, but I knew I should find that they knew many fine things. And in this I was not deceived; they did know what I did not, and in this way they were wiser than I. But, men of Athens, the good artisans also seemed to me to have the same failing as the poets; because of practicing his art well, each one thought he was very wise in the other most important matters, and this folly of theirs obscured that wisdom, so that I asked myself \marginpar{22e} in behalf of the oracle whether I should prefer to be as I am, neither wise in their wisdom nor foolish in their folly, or to be in both respects as they are. I replied then to myself and to the oracle that it was better for me to be as I am.

Now from this investigation, men of Athens, \marginpar{23a} many enmities have arisen against me, and such as are most harsh and grievous, so that many prejudices have resulted from them and I am called a wise man. For on each occasion those who are present think I am wise in the matters in which I confute someone else; but the fact is, gentlemen, it is likely that the god is really wise and by his oracle means this: “Human wisdom is of little or no value.” And it appears that he does not really say this of Socrates, but merely uses my name, \marginpar{23b} and makes me an example, as if he were to say: “This one of you, O human beings, is wisest, who, like Socrates, recognizes that he is in truth of no account in respect to wisdom.”

Therefore I am still even now going about and searching and investigating at the god's behest anyone, whether citizen or foreigner, who I think is wise; and when he does not seem so to me, I give aid to the god and show that he is not wise. And by reason of this occupation I have no leisure to attend to any of the affairs of the state worth mentioning, or of my own, but am in vast poverty \marginpar{23c} on account of my service to the god.

And in addition to these things, the young men who have the most leisure, the sons of the richest men, accompany me of their own accord, find pleasure in hearing people being examined, and often imitate me themselves, and then they undertake to examine others; and then, I fancy, they find a great plenty of people who think they know something, but know little or nothing. As a result, therefore, those who are examined by them are angry with me, instead of being angry with themselves, and say that “Socrates is a most abominable person \marginpar{23d} and is corrupting the youth.”

And when anyone asks them “by doing or teaching what?” they have nothing to say, but they do not know, and that they may not seem to be at a loss they say these things that are handy to say against all the philosophers, “the things in the air and the things beneath the earth” and “not to believe in the gods” and “to make the weaker argument the stronger.” For they would not, I fancy, care to say the truth, that it is being made very clear that they pretend to know, but know nothing. \marginpar{23e} Since, then, they are jealous of their honor and energetic and numerous and speak concertedly and persuasively about me, they have filled your ears both long ago and now with vehement slanders. From among them Meletus attacked me, and Anytus and Lycon, Meletus angered on account of the poets, and Anytus on account of the artisans and the public men, \marginpar{24a} and Lycon on account of the orators; so that, as I said in the beginning, I should be surprised if I were able to remove this prejudice from you in so short a time when it has grown so great. There you have the truth, men of Athens, and I speak without hiding anything from you, great or small or prevaricating. And yet I know pretty well that I am making myself hated by just that conduct; which is also a proof that I am speaking the truth and that this is the prejudice against me and these are its causes. And whether you investigate \marginpar{24b} this now or hereafter, you will find that it is so.

Now so far as the accusations are concerned which my first accusers made against me, this is a sufficient defence before you; but against Meletus, the good and patriotic, as he says, and the later ones, I will try to defend myself next. So once more, as if these were another set of accusers, let us take up in turn their sworn statement. It is about as follows: it states that Socrates is a wrongdoer because he corrupts the youth and does not believe in the gods the state believes in, but in other \marginpar{24c} new spiritual beings.

Such is the accusation. But let us examine each point of this accusation. He says I am a wrongdoer because I corrupt the youth. But I, men of Athens, say Meletus is a wrongdoer, because he jokes in earnest, lightly involving people in a lawsuit, pretending to be zealous and concerned about things or which he never cared at all. And that this is so I will try to make plain to you also.

Come here, Meletus, tell me: don't you consider it \marginpar{24d} of great importance that the youth be as good as possible? “I do.” Come now, tell these gentlemen who makes them better? For it is evident that you know, since you care about it. For you have found the one who corrupts them, as you say, and you bring me before these gentlemen and accuse me; and now, come, tell who makes them better and inform them who he is. Do you see, Meletus, that you are silent and cannot tell? And yet does it not seem to you disgraceful and a sufficient proof of what I say, that you have never cared about it? But tell, my good man, who \marginpar{24e} makes them better? “The laws.” But that is not what I ask, most excellent one, but what man, who knows in the first place just this very thing, the laws. “These men, Socrates, the judges.” What are you saying, Meletus? Are these gentlemen able to instruct the youth, and do they make them better? “Certainly.” All, or some of them and others not? “All.” Well said, by Hera, and this is a great plenty of helpers you speak of. But how about this?  \marginpar{25a} Do these listeners make them better, or not? “These also.” And how about the senators? “The senators also.” But, Meletus, those in the assembly, the assemblymen, don't corrupt the youth, do they? or do they also all make them better? “They also.” All the Athenians, then, as it seems, make them excellent, except myself, and I alone corrupt them. Is this what you mean? “Very decidedly, that is what I mean.” You have condemned me to great unhappiness! But answer me; does it seem to you to be so in the case of horses, that those who \marginpar{25b} make them better are all mankind, and he who injures them some one person? Or, quite the opposite of this, that he who is able to make them better is some one person, or very few, the horse-trainers, whereas most people, if they have to do with and use horses, injure them? Is it not so, Meletus, both in the case of horses and in that of all other animals? Certainly it is, whether you and Anytus deny it or agree; for it would be a great state of blessedness in the case of the youth if one alone corrupts them, and the others do them good. But, \marginpar{25c} Meletus, you show clearly enough that you never thought about the youth, and you exhibit plainly your own carelessness, that you have not cared at all for the things about which you hale me into court.

But besides, tell us, for heaven's sake, Meletus, is it better to live among good citizens, or bad? My friend, answer; for I am not asking anything hard. Do not the bad do some evil to those who are with them at any time and the good some good? “Certainly.” Is there then anyone who \marginpar{25d} prefers to be injured by his associates rather than benefited? Answer, my good man; for the law orders you to answer. Is there anyone who prefers to be injured? “Of course not.” Come then, do you hale me in here on the ground that I am corrupting the youth and making them worse voluntarily or involuntarily? “Voluntarily I say.” What then, Meletus? Are you at your age so much wiser than I at my age, that you have recognized that the evil always do some evil \marginpar{25e} to those nearest them, and the good some good; whereas I have reached such a depth of ignorance that I do not even know this, that if I make anyone of my associates bad I am in danger of getting some harm from him, so that I do this great evil voluntarily, as you say? I don't believe this, Meletus, nor do I think anyone else in the world does! \marginpar{26a} but either I do not corrupt them, or if I corrupt them, I do it involuntarily, so that you are lying in both events. But if I corrupt them involuntarily, for such involuntary errors the law is not to hale people into court, but to take them and instruct and admonish them in private. For it is clear that if I am told about it, I shall stop doing that which I do involuntarily. But you avoided associating with me and instructing me, and were unwilling to do so, but you hale me in here, where it is the law to hale in those who need punishment, not instruction.

But enough of this, for, men of Athens, this is clear, as I said, that Meletus never \marginpar{26b} cared much or little for these things. But nevertheless, tell us, how do you say, Meletus, that I corrupt the youth? Or is it evident, according to the indictment you brought, that it is by teaching them not to believe in the gods the state believes in, but in other new spiritual beings? Do you not say that it is by teaching this that I corrupt them? “Very decidedly that is what I say.” Then, Meletus, for the sake of \marginpar{26c} these very gods about whom our speech now is, speak still more clearly both to me and to these gentlemen. For I am unable to understand whether you say that I teach that there are some gods, and myself then believe that there are some gods, and am not altogether godless and am not a wrongdoer in that way, that these, however, are not the gods whom the state believes in, but others, and this is what you accuse me for, that I believe in others; or you say that I do not myself believe in gods at all and that I teach this unbelief to other people. “That is what I say, that you do not believe in gods at all.” You amaze me, Meletus! Why do you say this? \marginpar{26d} Do I not even believe that the sun or yet the moon are gods, as the rest of mankind do? “No, by Zeus, judges, since he says that the sun is a stone and the moon earth.” Do you think you are accusing Anaxagoras, my dear Meletus, and do you so despise these gentlemen and think they are so unversed in letters as not to know, that the books of Anaxagoras the Clazomenian are full of such utterances? And forsooth the youth learn these doctrines from me, which they can buy sometimes \marginpar{26e} (if the price is high) for a drachma in the orchestra and laugh at Socrates, if he pretends they are his own, especially when they are so absurd! But for heaven's sake, do you think this of me, that I do not believe there is any god? “No, by Zeus, you don't, not in the least.” You cannot be believed, Meletus, not even, as it seems to me, by yourself. For this man appears to me, men of Athens, to be very violent and unrestrained, and actually to have brought this indictment in a spirit of violence and unrestraint and rashness. For he seems, \marginpar{27a} as it were, by composing a puzzle to be making a test: “Will Socrates, the wise man, recognize that I am joking and contradicting myself, or shall I deceive him and the others who hear me?” For he appears to me to contradict himself in his speech, as if he were to say, “Socrates is a wrongdoer, because he does not believe in gods, but does believe in gods.” And yet this is the conduct of a jester.

Join me, then, gentlemen, in examining how he appears to me to say this; and do you, Meletus, answer; \marginpar{27b} and you, gentlemen, as I asked you in the beginning, please bear in mind not to make a disturbance if I conduct my argument in my accustomed manner.

Is there any human being who believes that there are things pertaining to human beings, but no human beings? Let him answer, gentlemen, and not make a disturbance in one way or another. Is there anyone who does not believe in horses, but does believe in things pertaining to horses? or who does not believe that flute-players exist, but that things pertaining to flute-players do? There is not, best of men; if you do not wish to answer, I say it to you and these others here. But answer at least \marginpar{27c} the next question. Is there anyone who believes spiritual things exist, but does not believe in spirits? “There is not.” Thank you for replying reluctantly when forced by these gentlemen. Then you say that I believe in spiritual beings, whether new or old, and teach that belief; but then I believe in spiritual beings at any rate, according to your statement, and you swore to that in your indictment. But if I believe in spiritual beings, it is quite inevitable that I believe also in spirits; is it not so? It is; for I assume that you agree, since you do not answer. But do we not think the spirits are \marginpar{27d} gods or children of gods? Yes, or no? “Certainly.” Then if I believe in spirits, as you say, if spirits are a kind of gods, that would be the puzzle and joke which I say you are uttering in saying that I, while I do not believe in gods, do believe In gods again, since I believe in spirits; but if, on the other hand, spirits are a kind of bastard children of gods, by nymphs or by any others, whoever their mothers are said to be, what man would believe that there are children of gods, but no gods? It would be just as absurd \marginpar{27e} as if one were to believe that there are children of horses and asses, namely mules, but no horses and asses. But, Meletus, you certainly must have brought this suit either to make a test of us or because you were loss as to what true wrongdoing you could accuse me of; but there is no way for you to persuade any man who has even a little sense that it is possible for the same person to believe in spiritual and divine existences and again for the same person not to believe in spirits or gods or \marginpar{28a} heroes.

Well then, men of Athens, that I am not a wrongdoer according to Meletus's indictment, seems to me not to need much of a defence, but what has been said is enough. But you may be assured that what I said before is true, that great hatred has arisen against me and in the minds of many persons. And this it is which will cause my condemnation, if it is to cause it, not Meletus or Anytus, but the prejudice and dislike of the many. This has condemned many other good men, and I think will do so; \marginpar{28b} and there is no danger that it will stop with me. But perhaps someone might say: “Are you then not ashamed, Socrates, of having followed such a pursuit, that you are now in danger of being put to death as a result?” But I should make to him a just reply: “You do not speak well, Sir, if you think a man in whom there is even a little merit ought to consider danger of life or death, and not rather regard this only, when he does things, whether the things he does are right or wrong and the acts of a good or a bad man. For according to your argument all the demigods \marginpar{28c} would be bad who died at Troy, including the son of Thetis, who so despised danger, in comparison with enduring any disgrace, that when his mother (and she was a goddess) said to him, as he was eager to slay Hector, something like this, I believe, “My son, if you avenge the death of your friend Patroclus and kill Hector, you yourself shall die;
for straightway, after Hector, is death appointed unto you;
”1he, when he heard this, made light of death and danger, \marginpar{28d} and feared much more to live as a coward and not to avenge his friends, and said, “Straightway may I die, after doing vengeance upon the wrongdoer, that I may not stay here, jeered at beside the curved ships, a burden of the earth.”. Do you think he considered death and danger?

For thus it is, men of Athens, in truth; wherever a man stations himself, thinking it is best to be there, or is stationed by his commander, there he must, as it seems to me, remain and run his risks, considering neither death nor any other thing more than disgrace.

So I should have done a terrible thing, \marginpar{28e} if, when the commanders whom you chose to command me stationed me, both at Potidaea and at Amphipolis and at Delium, I remained where they stationed me, like anybody else, and ran the risk of death, but when the god gave me a station, as I believed and understood, with orders to spend my life in philosophy and in examining myself and others, \marginpar{29a} then I were to desert my post through fear of death or anything else whatsoever. It would be a terrible thing, and truly one might then justly hale me into court, on the charge that I do not believe that there are gods, since I disobey the oracle and fear death and think I am wise when I am not. For to fear death, gentlemen, is nothing else than to think one is wise when one is not; for it is thinking one knows what one does not know. For no one knows whether death be not even the greatest of all blessings to man, but they fear it as if they knew that it is the greatest of evils. \marginpar{29b} And is not this the most reprehensible form of ignorance, that of thinking one knows what one does not know? Perhaps, gentlemen, in this matter also I differ from other men in this way, and if I were to say that I am wiser in anything, it would be in this, that not knowing very much about the other world, I do not think I know. But I do know that it is evil and disgraceful to do wrong and to disobey him who is better than I, whether he be god or man. So I shall never fear or avoid those things concerning which I do not know whether they are good or bad rather than those which I know are bad. And therefore, even if \marginpar{29c} you acquit me now and are not convinced by Anytus, who said that either I ought not to have been brought to trial at all, or since was brought to trial, I must certainly be put to death, adding that if I were acquitted your sons would all be utterly ruined by practicing what I teach—if you should say to me in reply to this: “Socrates, this time we will not do as Anytus says, but we will let you go, on this condition, however, that you no longer spend your time in this investigation or in philosophy, and if you are caught doing so again you shall die”; \marginpar{29d} if you should let me go on this condition which I have mentioned, I should say to you, “Men of Athens, I respect and love you, but I shall obey the god rather than you, and while I live and am able to continue, I shall never give up philosophy or stop exhorting you and pointing out the truth to any one of you whom I may meet, saying in my accustomed way: “Most excellent man, are you who are a citizen of Athens, the greatest of cities and the most famous for wisdom and power, not ashamed to care for the acquisition of wealth \marginpar{29e} and for reputation and honor, when you neither care nor take thought for wisdom and truth and the perfection of your soul?” And if any of you argues the point, and says he does care, I shall not let him go at once, nor shall I go away, but I shall question and examine and cross-examine him, and if I find that he does not possess virtue, but says he does, I shall rebuke him for scorning \marginpar{30a} the things that are of most importance and caring more for what is of less worth. This I shall do to whomever I meet, young and old, foreigner and citizen, but most to the citizens, inasmuch as you are more nearly related to me. For know that the god commands me to do this, and I believe that no greater good ever came to pass in the city than my service to the god. For I go about doing nothing else than urging you, young and old, not to care for your persons or your property \marginpar{30b} more than for the perfection of your souls, or even so much; and I tell you that virtue does not come from money, but from virtue comes money and all other good things to man, both to the individual and to the state. If by saying these things I corrupt the youth, these things must be injurious; but if anyone asserts that I say other things than these, he says what is untrue. Therefore I say to you, men of Athens, either do as Anytus tells you, or not, and either acquit me, or not, knowing that I shall not change my conduct even if I am \marginpar{30c} to die many times over.

Do not make a disturbance, men of Athens; continue to do what I asked of you, not to interrupt my speech by disturbances, but to hear me; and I believe you will profit by hearing. Now I am going to say some things to you at which you will perhaps cry out; but do not do so by any means. For know that if you kill me, I being such a man as I say I am, you will not injure me so much as yourselves; for neither Meletus nor Anytus could injure me; \marginpar{30d} that would be impossible, for I believe it is not God's will that a better man be injured by a worse. He might, however, perhaps kill me or banish me or disfranchise me; and perhaps he thinks he would thus inflict great injuries upon me, and others may think so, but I do not; I think he does himself a much greater injury by doing what he is doing now—killing a man unjustly. And so, men of Athens, I am now making my defence not for my own sake, as one might imagine, but far more for yours, that you may not by condemning me err in your treatment of the gift the God gave you. \marginpar{30e} For if you put me to death, you will not easily find another, who, to use a rather absurd figure, attaches himself to the city as a gadfly to a horse, which, though large and well bred, is sluggish on account of his size and needs to be aroused by stinging. I think the god fastened me upon the city in some such capacity, and I go about arousing, \marginpar{31a} and urging and reproaching each one of you, constantly alighting upon you everywhere the whole day long. Such another is not likely to come to you, gentlemen; but if you take my advice, you will spare me. But you, perhaps, might be angry, like people awakened from a nap, and might slap me, as Anytus advises, and easily kill me; then you would pass the rest of your lives in slumber, unless God, in his care for you, should send someone else to sting you. And that I am, as I say, a kind of gift from the god, \marginpar{31b} you might understand from this; for I have neglected all my own affairs and have been enduring the neglect of my concerns all these years, but I am always busy in your interest, coming to each one of you individually like a father or an elder brother and urging you to care for virtue; now that is not like human conduct. If I derived any profit from this and received pay for these exhortations, there would be some sense in it; but now you yourselves see that my accusers, though they accuse me of everything else in such a shameless way, have not been able to work themselves up to such a pitch of shamelessness \marginpar{31c} as to produce a witness to testify that I ever exacted or asked pay of anyone. For I think I have a sufficient witness that I speak the truth, namely, my poverty.

Perhaps it may seem strange that I go about and interfere in other people's affairs to give this advice in private, but do not venture to come before your assembly and advise the state. But the reason for this, as you have heard me say \marginpar{31d} at many times and places, is that something divine and spiritual comes to me, the very thing which Meletus ridiculed in his indictment. I have had this from my childhood; it is a sort of voice that comes to me, and when it comes it always holds me back from what I am thinking of doing, but never urges me forward. This it is which opposes my engaging in politics. And I think this opposition is a very good thing; for you may be quite sure, men of Athens, that if I had undertaken to go into politics, I should have been put to death long ago and should have done \marginpar{31e} no good to you or to myself. And do not be angry with me for speaking the truth; the fact is that no man will save his life who nobly opposes you or any other populace and prevents many unjust and illegal things from happening in the state. \marginpar{32a} A man who really fights for the right, if he is to preserve his life for even a little while, must be a private citizen, not a public man.

I will give you powerful proofs of this not mere words, but what you honor more,—actions. And listen to what happened to me, that you may be convinced that I would never yield to any one, if that was wrong, through fear of death, but would die rather than yield. The tale I am going to tell you is ordinary and commonplace, but true. \marginpar{32b} I, men of Athens, never held any other office in the state, but I was a senator; and it happened that my tribe held the presidency when you wished to judge collectively, not severally, the ten generals who had failed to gather up the slain after the naval battle; this was illegal, as you all agreed afterwards. At that time I was the only one of the prytanes who opposed doing anything contrary to the laws, and although the orators were ready to impeach and arrest me, and though you urged them with shouts to do so, I thought \marginpar{32c} I must run the risk to the end with law and justice on my side, rather than join with you when your wishes were unjust, through fear of imprisonment or death. That was when the democracy still existed; and after the oligarchy was established, the Thirty sent for me with four others to come to the rotunda and ordered us to bring Leon the Salaminian from Salamis to be put to death. They gave many such orders to others also, because they wished to implicate as many in their crimes as they could. Then I, however, \marginpar{32d} showed again, by action, not in word only, that I did not care a whit for death if that be not too rude an expression, but that I did care with all my might not to do anything unjust or unholy. For that government, with all its power, did not frighten me into doing anything unjust, but when we came out of the rotunda, the other four went to Salamis and arrested Leon, but I simply went home; and perhaps I should have been put to death for it, if the government had not \marginpar{32e} quickly been put down. Of these facts you can have many witnesses.

Do you believe that I could have lived so many years if I had been in public life and had acted as a good man should act, lending my aid to what is just and considering that of the highest importance? Far from it, men of Athens; nor could \marginpar{33a} any other man. But you will find that through all my life, both in public, if I engaged in any public activity, and in private, I have always been the same as now, and have never yielded to any one wrongly, whether it were any other person or any of those who are said by my traducers to be my pupils. But I was never any one's teacher. If any one, whether young or old, wishes to hear me speaking and pursuing my mission, I have never objected, \marginpar{33b} nor do I converse only when I am paid and not otherwise, but I offer myself alike to rich and poor; I ask questions, and whoever wishes may answer and hear what I say. And whether any of them turns out well or ill, I should not justly be held responsible, since I never promised or gave any instruction to any of them; but if any man says that he ever learned or heard anything privately from me, which all the others did not, be assured that he is lying.

But why then do some people love \marginpar{33c} to spend much of their time with me? You have heard the reason, men of Athens; for I told you the whole truth; it is because they like to listen when those are examined who think they are wise and are not so; for it is amusing. But, as I believe, I have been commanded to do this by the God through oracles and dreams and in every way in which any man was ever commanded by divine power to do anything whatsoever. This, Athenians, is true and easily tested. For if I am corrupting some of the young men \marginpar{33d} and have corrupted others, surely some of them who have grown older, if they recognize that I ever gave them any bad advice when they were young, ought now to have come forward to accuse me. Or if they did not wish to do it themselves, some of their relatives—fathers or brothers or other kinsfolk—ought now to tell the facts. And there are many of them present, whom I see; first Crito here, \marginpar{33e} who is of my own age and my own deme and father of Critobulus, who is also present; then there is Lysanias the Sphettian, father of Aeschines, who is here; and also Antiphon of Cephisus, father of Epigenes. Then here are others whose brothers joined in my conversations, Nicostratus, son of Theozotides and brother of Theodotus (now Theodotus is dead, so he could not stop him by entreaties), and Paralus, son of Demodocus; Theages was his brother; and \marginpar{34a} Adimantus, son of Aristo, whose brother is Plato here; and Aeantodorus, whose brother Apollodorus is present. And I can mention to you many others, some one of whom Meletus ought certainly to have produced as a witness in his speech; but if he forgot it then, let him do so now; I yield the floor to him, and let him say, if he has any such testimony. But you will find that the exact opposite is the case, gentlemen, and that they are all ready to aid me, the man who corrupts and injures their relatives, as Meletus and Anytus say. \marginpar{34b} Now those who are themselves corrupted might have some motive in aiding me; but what reason could their relatives have, who are not corrupted and are already older men, unless it be the right and true reason, that they know that Meletus is lying and I am speaking the truth?

Well, gentlemen, this, and perhaps more like this, is about all I have to say in my defence. Perhaps some one among you may be offended \marginpar{34c} when he remembers his own conduct, if he, even in a case of less importance than this, begged and besought the judges with many tears, and brought forward his children to arouse compassion, and many other friends and relatives; whereas I will do none of these things, though I am, apparently, in the very greatest danger. Perhaps some one with these thoughts in mind may be harshly disposed toward me and may cast his vote in anger. Now if any one of you is so disposed \marginpar{34d} —I do not believe there is such a person—but if there should be, I think I should be speaking fairly if I said to him, My friend, I too have relatives, for I am, as Homer has it, “not born of an oak or a rock,
”\footnote{Hom. Od. 19.163.}but of human parents, so that I have relatives and, men of Athens, I have three sons, one nearly grown up, and two still children; but nevertheless I shall not bring any of them here and beg you to acquit me. And why shall I not do so? Not because I am stubborn, Athenians, \marginpar{34e} or lack respect for you. Whether I fear death or not is another matter, but for the sake of my good name and yours and that of the whole state, I think it is not right for me to do any of these things in view of my age and my reputation, whether deserved or not; for at any rate the opinion prevails that Socrates \marginpar{35a} is in some way superior to most men. If then those of you who are supposed to be superior either in wisdom or in courage or in any other virtue whatsoever are to behave in such a way, it would be disgraceful. Why, I have often seen men who have some reputation behaving in the strangest manner, when they were on trial, as if they thought they were going to suffer something terrible if they were put to death, just as if they would be immortal if you did not kill them. It seems to me that they are a disgrace to the state and that any stranger might say that those of the Athenians who excel \marginpar{35b} in virtue, men whom they themselves honor with offices and other marks of esteem, are no better than women. Such acts, men of Athens, we who have any reputation at all ought not to commit, and if we commit them you ought not to allow it, but you should make it clear that you will be much more ready to condemn a man who puts before you such pitiable scenes and makes the city ridiculous than one who keeps quiet.

But apart from the question of reputation, gentlemen, I think it is not right \marginpar{35c} to implore the judge or to get acquitted by begging; we ought to inform and convince him. For the judge is not here to grant favors in matters of justice, but to give judgement; and his oath binds him not to do favors according to his pleasure, but to judge according to the laws; therefore, we ought not to get you into the habit of breaking your oaths, nor ought you to fall into that habit; for neither of us would be acting piously. Do not, therefore, men of Athens, demand of me that I act before you in a way which I consider neither honorable nor right nor pious, \marginpar{35d} especially when impiety is the very thing for which Meletus here has brought me to trial. For it is plain that if by persuasion and supplication I forced you to break your oaths I should teach you to disbelieve in the existence of the gods and in making my defence should accuse myself of not believing in them. But that is far from the truth; for I do believe in them, men of Athens, more than any of my accusers, and I entrust my case to you and to God to decide it as shall be best for me and for you. \marginpar{35e} I am not grieved, men of Athens, \marginpar{36a} at this vote of condemnation you have cast against me, and that for many reasons, among them the fact that your decision was not a surprise to me. I am much more surprised by the number of votes for and against it; for I did not expect so small a majority, but a large one. Now, it seems, if only thirty votes had been cast the other way, I should have been acquitted. And so, I think, so far as Meletus is concerned, I have even now been acquitted, and not merely acquitted, but anyone can see that, if Anytus and Lycon had not come forward to accuse me, he would have been fined \marginpar{36b} a thousand drachmas for not receiving a fifth part of the votes.

And so the man proposes the penalty of death. Well, then, what shall I propose as an alternative? Clearly that which I deserve, shall I not? And what do I deserve to suffer or to pay, because in my life I did not keep quiet, but neglecting what most men care for—money-making and property, and military offices, and public speaking, and the various offices and plots and parties that come up in the state—and thinking that I was really too honorable \marginpar{36c} to engage in those activities and live, refrained from those things by which I should have been of no use to you or to myself, and devoted myself to conferring upon each citizen individually what I regard as the greatest benefit? For I tried to persuade each of you to care for himself and his own perfection in goodness and wisdom rather than for any of his belongings, and for the state itself rather than for its interests, and to follow the same method in his care for other things. What, then, does such a man as I deserve? \marginpar{36d} Some good thing, men of Athens, if I must propose something truly in accordance with my deserts; and the good thing should be such as is fitting for me. Now what is fitting for a poor man who is your benefactor, and who needs leisure to exhort you? There is nothing, men of Athens, so fitting as that such a man be given his meals in the prytaneum. That is much more appropriate for me than for any of you who has won a race at the Olympic games with a pair of horses or a four-in-hand. For he makes you seem to be happy, whereas I make you happy in reality; \marginpar{36e} and he is not at all in need of sustenance, but I am needy. So if I must propose a penalty in accordance with my deserts, \marginpar{37a} I propose maintenance in the prytaneum.

Perhaps some of you think that in saying this, as in what I said about lamenting and imploring, I am speaking in a spirit of bravado; but that is not the case. The truth is rather that I am convinced that I never intentionally wronged anyone; but I cannot convince you of this, for we have conversed with each other only a little while. I believe if you had a law, as some other people have, \marginpar{37b} that capital cases should not be decided in one day, but only after several days, you would be convinced; but now it is not easy to rid you of great prejudices in a short time. Since, then, I am convinced that I never wronged any one, I am certainly not going to wrong myself, and to say of myself that I deserve anything bad, and to propose any penalty of that sort for myself. Why should I? Through fear of the penalty that Meletus proposes, about which I say that I do not know whether it is a good thing or an evil? Shall I choose instead of that something which I know to be an evil? What penalty shall I propose? Imprisonment? \marginpar{37c} And why should I live in prison a slave to those who may be in authority? Or shall I propose a fine, with imprisonment until it is paid? But that is the same as what I said just now, for I have no money to pay with. Shall I then propose exile as my penalty? Perhaps you would accept that. I must indeed be possessed by a great love of life if I am so irrational as not to know that if you, who are my fellow citizens, could not \marginpar{37d} endure my conversation and my words, but found them too irksome and disagreeable, so that you are now seeking to be rid of them, others will not be willing to endure them. No, men of Athens, they certainly will not. A fine life I should lead if I went away at my time of life, wandering from city to city and always being driven out! For well I know that wherever I go, the young men will listen to my talk, as they do here; and if I drive them away, they will themselves persuade their elders to drive me out, and if \marginpar{37e} I do not drive them away, their fathers and relatives will drive me out for their sakes.

Perhaps someone might say, “Socrates, can you not go away from us and live quietly, without talking?” Now this is the hardest thing to make some of you believe. For if I say that such conduct would be disobedience to the god and that therefore I cannot keep quiet, you will think I am jesting and will not believe me; \marginpar{38a} and if again I say that to talk every day about virtue and the other things about which you hear me talking and examining myself and others is the greatest good to man, and that the unexamined life is not worth living, you will believe me still less. This is as I say, gentlemen, but it is not easy to convince you. Besides, I am not accustomed to think that I deserve anything bad. If I had money, I would have proposed a fine, \marginpar{38b} as large as I could pay; for that would have done me no harm. But as it is—I have no money, unless you are willing to impose a fine which I could pay. I might perhaps pay a mina of silver. So I propose that penalty; but Plato here, men of Athens, and Crito and Critobulus, and Aristobulus tell me to propose a fine of thirty minas, saying that they are sureties for it. So I propose a fine of that amount, and these men, who are amply sufficient, will be my sureties. \marginpar{38c}

It is no long time, men of Athens, which you gain, and for that those who wish to cast a slur upon the state will give you the name and blame of having killed Socrates, a wise man; for, you know, those who wish to revile you will say I am wise, even though I am not. Now if you had waited a little while, what you desire would have come to you of its own accord; for you see how old I am, how far advanced in life and how near death. I say this not to all of you, \marginpar{38d} but to those who voted for my death. And to them also I have something else to say. Perhaps you think, gentlemen, that I have been convicted through lack of such words as would have moved you to acquit me, if I had thought it right to do and say everything to gain an acquittal. Far from it. And yet it is through a lack that I have been convicted, not however a lack of words, but of impudence and shamelessness, and of willingness to say to you such things as you would have liked best to hear. You would have liked to hear me wailing and lamenting and doing and saying \marginpar{38e} many things which are, as I maintain, unworthy of me—such things as you are accustomed to hear from others. But I did not think at the time that I ought, on account of the danger I was in, to do anything unworthy of a free man, nor do I now repent of having made my defence as I did, but I much prefer to die after such a defence than to live after a defence of the other sort. For neither in the court nor in war ought I \marginpar{39a} or any other man to plan to escape death by every possible means. In battles it is often plain that a man might avoid death by throwing down his arms and begging mercy of his pursuers; and there are many other means of escaping death in dangers of various kinds if one is willing to do and say anything. But, gentlemen, it is not hard to escape death; it is much harder to escape wickedness, for that runs faster than death. \marginpar{39b} And now I, since I am slow and old, am caught by the slower runner, and my accusers, who are clever and quick, by the faster, wickedness. And now I shall go away convicted by you and sentenced to death, and they go convicted by truth of villainy and wrong. And I abide by my penalty, and they by theirs. Perhaps these things had to be so, and I think they are well. \marginpar{39c} And now I wish to prophesy to you, O ye who have condemned me; for I am now at the time when men most do prophesy, the time just before death. And I say to you, ye men who have slain me, that punishment will come upon you straight-way after my death, far more grievous in sooth than the punishment of death which you have meted out to me. For now you have done this to me because you hoped that you would be relieved from rendering an account of your lives, but I say that you will find the result far different. Those who will force you to give an account will be more numerous than heretofore; \marginpar{39d} men whom I restrained, though you knew it not; and they will be harsher, inasmuch as they are younger, and you will be more annoyed. For if you think that by putting men to death you will prevent anyone from reproaching you because you do not act as you should, you are mistaken. That mode of escape is neither possible at all nor honorable, but the easiest and most honorable escape is not by suppressing others, but by making yourselves as good as possible. So with this prophecy to you who condemned me \marginpar{39e} I take my leave.

But with those who voted for my acquittal I should like to converse about this which has happened, while the authorities are busy and before I go to the place where I must die. Wait with me so long, my friends; for nothing prevents our chatting with each other \marginpar{40a} while there is time. I feel that you are my friends, and I wish to show you the meaning of this which has now happened to me. For, judges—and in calling you judges I give you your right name—a wonderful thing has happened to me. For hitherto the customary prophetic monitor always spoke to me very frequently and opposed me even in very small matters, if I was going to do anything I should not; but now, as you yourselves see, this thing which might be thought, and is generally considered, the greatest of evils has come upon me; but the divine sign did not oppose me \marginpar{40b} either when I left my home in the morning, or when I came here to the court, or at any point of my speech, when I was going to say anything; and yet on other occasions it stopped me at many points in the midst of a speech; but now, in this affair, it has not opposed me in anything I was doing or saying. What then do I suppose is the reason? I will tell you. This which has happened to me is doubtless a good thing, and those of us who think death is an evil \marginpar{40c} must be mistaken. A convincing proof of this been given me; for the accustomed sign would surely have opposed me if I had not been going to meet with something good.

Let us consider in another way also how good reason there is to hope that it is a good thing. For the state of death is one of two things: either it is virtually nothingness, so that the dead has no consciousness of anything, or it is, as people say, a change and migration of the soul from this to another place. And if it is unconsciousness, \marginpar{40d} like a sleep in which the sleeper does not even dream, death would be a wonderful gain. For I think if any one were to pick out that night in which he slept a dreamless sleep and, comparing with it the other nights and days of his life, were to say, after due consideration, how many days and nights in his life had passed more pleasantly than that night,—I believe that not only any private person, but even the great King of Persia himself \marginpar{40e} would find that they were few in comparison with the other days and nights. So if such is the nature of death, I count it a gain; for in that case, all time seems to be no longer than one night. But on the other hand, if death is, as it were, a change of habitation from here to some other place, and if what we are told is true, that all the dead are there, what greater blessing could there be, judges? For if a man when he reaches the other world, \marginpar{41a} after leaving behind these who claim to be judges, shall find those who are really judges who are said to sit in judgment there, Minos and Rhadamanthus, and Aeacus and Triptolemus, and all the other demigods who were just men in their lives, would the change of habitation be undesirable? Or again, what would any of you give to meet with Orpheus and Musaeus and Hesiod and Homer? I am willing to die many times over, if these things are true; for I personally should find the life there wonderful, \marginpar{41b} when I met Palamedes or Ajax, the son of Telamon, or any other men of old who lost their lives through an unjust judgement, and compared my experience with theirs. I think that would not be unpleasant. And the greatest pleasure would be to pass my time in examining and investigating the people there, as I do those here, to find out who among them is wise and who thinks he is when he is not. What price would any of you pay, judges, to examine him who led the great army against Troy, \marginpar{41c} or Odysseus, or Sisyphus, or countless others, both men and women, whom I might mention? To converse and associate with them and examine them would be immeasurable happiness. At any rate, the folk there do not kill people for it; since, if what we are told is true, they are immortal for all future time, besides being happier in other respects than men are here.

But you also, judges, must regard death hopefully and must bear in mind this one truth, \marginpar{41d} that no evil can come to a good man either in life or after death, and God does not neglect him. So, too, this which had come to me has not come by chance, but I see plainly that it was better for me to die now and be freed from troubles. That is the reason why the sign never interfered with me, and I am not at all angry with those who condemned me or with my accusers. And yet it was not with that in view that they condemned and accused me, but because \marginpar{41e} they thought to injure me. They deserve blame for that. However, I make this request of them: when my sons grow up, gentlemen, punish them by troubling them as I have troubled you; if they seem to you to care for money or anything else more than for virtue, and if they think they amount to something when they do not, rebuke them as I have rebuked you because they do not care for what they ought, and think they amount to something when they are worth nothing. If you do this, both I and my sons \marginpar{42a} shall have received just treatment from you.

But now the time has come to go away. I go to die, and you to live; but which of us goes to the better lot, is known to none but God.