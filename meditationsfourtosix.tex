Meditation IV


OF TRUTH AND ERROR.

1. I HAVE been habituated these bygone days to detach my mind from the senses, and I have accurately observed that there is exceedingly little which is known with certainty respecting corporeal objects, that we know much more of the human mind, and still more of God himself. I am thus able now without difficulty to abstract my mind from the contemplation of [sensible or] imaginable objects, and apply it to those which, as disengaged from all matter, are purely intelligible. And certainly the idea I have of the human mind in so far as it is a thinking thing, and not extended in length, breadth, and depth, and participating in none of the properties of body, is incomparably more distinct than the idea of any corporeal object; and when I consider that I doubt, in other words, that I am an incomplete and dependent being, the idea of a complete and independent being, that is to say of God, occurs to my mind with so much clearness and distinctness, and from the fact alone that this idea is found in me, or that I who possess it exist, the conclusions that God exists, and that my own existence, each moment of its continuance, is absolutely dependent upon him, are so manifest, as to lead me to believe it impossible that the human mind can know anything with more clearness and certitude. And now I seem to discover a path that will conduct us from the contemplation of the true God, in whom are contained all the treasures of science and wisdom, to the knowledge of the other things in the universe.

2. For, in the first place, I discover that it is impossible for him ever to deceive me, for in all fraud and deceit there is a certain imperfection: and although it may seem that the ability to deceive is a mark of subtlety or power, yet the will testifies without doubt of malice and weakness; and such, accordingly, cannot be found in God.

3. In the next place, I am conscious that I possess a certain faculty of judging [or discerning truth from error], which I doubtless received from God, along with whatever else is mine; and since it is impossible that he should will to deceive me, it is likewise certain that he has not given me a faculty that will ever lead me into error, provided I use it aright.

4. And there would remain no doubt on this head, did it not seem to follow from this, that I can never therefore be deceived; for if all I possess be from God, and if he planted in me no faculty that is deceitful, it seems to follow that I can never fall into error. Accordingly, it is true that when I think only of God (when I look upon myself as coming from God, Fr.), and turn wholly to him, I discover [in myself] no cause of error or falsity: but immediately thereafter, recurring to myself, experience assures me that I am nevertheless subject to innumerable errors. When I come to inquire into the cause of these, I observe that there is not only present to my consciousness a real and positive idea of God, or of a being supremely perfect, but also, so to speak, a certain negative idea of nothing, in other words, of that which is at an infinite distance from every sort of perfection, and that I am, as it were, a mean between God and nothing, or placed in such a way between absolute existence and non-existence, that there is in truth nothing in me to lead me into error, in so far as an absolute being is my creator; but that, on the other hand, as I thus likewise participate in some degree of nothing or of nonbeing, in other words, as I am not myself the supreme Being, and as I am wanting in many perfections, it is not surprising I should fall into error. And I hence discern that error, so far as error is not something real, which depends for its existence on God, but is simply defect; and therefore that, in order to fall into it, it is not necessary God should have given me a faculty expressly for this end, but that my being deceived arises from the circumstance that the power which God has given me of discerning truth from error is not infinite.

5. Nevertheless this is not yet quite satisfactory; for error is not a pure negation, [in other words, it is not the simple deficiency or want of some knowledge which is not due], but the privation or want of some knowledge which it would seem I ought to possess. But, on considering the nature of God, it seems impossible that he should have planted in his creature any faculty not perfect in its kind, that is, wanting in some perfection due to it: for if it be true, that in proportion to the skill of the maker the perfection of his work is greater, what thing can have been produced by the supreme Creator of the universe that is not absolutely perfect in all its parts? And assuredly there is no doubt that God could have created me such as that I should never be deceived; it is certain, likewise, that he always wills what is best: is it better, then, that I should be capable of being deceived than that I should not ?

6. Considering this more attentively the first thing that occurs to me is the reflection that I must not be surprised if I am not always capable of comprehending the reasons why God acts as he does; nor must I doubt of his existence because I find, perhaps, that there are several other things besides the present respecting which I understand neither why nor how they were created by him; for, knowing already that my nature is extremely weak and limited, and that the nature of God, on the other hand, is immense, incomprehensible, and infinite, I have no longer any difficulty in discerning that there is an infinity of things in his power whose causes transcend the grasp of my mind: and this consideration alone is sufficient to convince me, that the whole class of final causes is of no avail in physical [or natural] things; for it appears to me that I cannot, without exposing myself to the charge of temerity, seek to discover the [impenetrable] ends of Deity.

7. It further occurs to me that we must not consider only one creature apart from the others, if we wish to determine the perfection of the works of Deity, but generally all his creatures together; for the same object that might perhaps, with some show of reason, be deemed highly imperfect if it were alone in the world, may for all that be the most perfect possible, considered as forming part of the whole universe: and although, as it was my purpose to doubt of everything, I only as yet know with certainty my own existence and that of God, nevertheless, after having remarked the infinite power of Deity, I cannot deny that we may have produced many other objects, or at least that he is able to produce them, so that I may occupy a place in the relation of a part to the great whole of his creatures.

8. Whereupon, regarding myself more closely, and considering what my errors are (which alone testify to the existence of imperfection in me), I observe that these depend on the concurrence of two causes, viz, the faculty of cognition, which I possess, and that of election or the power of free choice,—in other words, the understanding and the will. For by the understanding alone, I [neither affirm nor deny anything but] merely apprehend (percipio) the ideas regarding which I may form a judgment; nor is any error, properly so called, found in it thus accurately taken. And although there are perhaps innumerable objects in the world of which I have no idea in my understanding, it cannot, on that account be said that I am deprived of those ideas [as of something that is due to my nature], but simply that I do not possess them, because, in truth, there is no ground to prove that Deity ought to have endowed me with a larger faculty of cognition than he has actually bestowed upon me; and however skillful a workman I suppose him to be, I have no reason, on that account, to think that it was obligatory on him to give to each of his works all the perfections he is able to bestow upon some. Nor, moreover, can I complain that God has not given me freedom of choice, or a will sufficiently ample and perfect, since, in truth, I am conscious of will so ample and extended as to be superior to all limits. And what appears to me here to be highly remarkable is that, of all the other properties I possess, there is none so great and perfect as that I do not clearly discern it could be still greater and more perfect. For, to take an example, if I consider the faculty of understanding which I possess, I find that it is of very small extent, and greatly limited, and at the same time I form the idea of another faculty of the same nature, much more ample and even infinite, and seeing that I can frame the idea of it, I discover, from this circumstance alone, that it pertains to the nature of God. In the same way, if I examine the faculty of memory or imagination, or any other faculty I possess, I find none that is not small and circumscribed, and in God immense [and infinite]. It is the faculty of will only, or freedom of choice, which I experience to be so great that I am unable to conceive the idea of another that shall be more ample and extended; so that it is chiefly my will which leads me to discern that I bear a certain image and similitude of Deity. For although the faculty of will is incomparably greater in God than in myself, as well in respect of the knowledge and power that are conjoined with it, and that render it stronger and more efficacious, as in respect of the object, since in him it extends to a greater number of things, it does not, nevertheless, appear to me greater, considered in itself formally and precisely: for the power of will consists only in this, that we are able to do or not to do the same thing (that is, to affirm or deny, to pursue or shun it), or rather in this alone, that in affirming or denying, pursuing or shunning, what is proposed to us by the understanding, we so act that we are not conscious of being determined to a particular action by any external force. For, to the possession of freedom, it is not necessary that I be alike indifferent toward each of two contraries; but, on the contrary, the more I am inclined toward the one, whether because I clearly know that in it there is the reason of truth and goodness, or because God thus internally disposes my thought, the more freely do I choose and embrace it; and assuredly divine grace and natural knowledge, very far from diminishing liberty, rather augment and fortify it. But the indifference of which I am conscious when I am not impelled to one side rather than to another for want of a reason, is the lowest grade of liberty, and manifests defect or negation of knowledge rather than perfection of will; for if I always clearly knew what was true and good, I should never have any difficulty in determining what judgment I ought to come to, and what choice I ought to make, and I should thus be entirely free without ever being indifferent.

9. From all this I discover, however, that neither the power of willing, which I have received from God, is of itself the source of my errors, for it is exceedingly ample and perfect in its kind; nor even the power of understanding, for as I conceive no object unless by means of the faculty that God bestowed upon me, all that I conceive is doubtless rightly conceived by me, and it is impossible for me to be deceived in it. Whence, then, spring my errors ? They arise from this cause alone, that I do not restrain the will, which is of much wider range than the understanding, within the same limits, but extend it even to things I do not understand, and as the will is of itself indifferent to such, it readily falls into error and sin by choosing the false in room of the true, and evil instead of good.

10. For example, when I lately considered whether aught really existed in the world, and found that because I considered this question, it very manifestly followed that I myself existed, I could not but judge that what I so clearly conceived was true, not that I was forced to this judgment by any external cause, but simply because great clearness of the understanding was succeeded by strong inclination in the will; and I believed this the more freely and spontaneously in proportion as I was less indifferent with respect to it. But now I not only know that I exist, in so far as I am a thinking being, but there is likewise presented to my mind a certain idea of corporeal nature; hence I am in doubt as to whether the thinking nature which is in me, or rather which I myself am, is different from that corporeal nature, or whether both are merely one and the same thing, and I here suppose that I am as yet ignorant of any reason that would determine me to adopt the one belief in preference to the other; whence it happens that it is a matter of perfect indifference to me which of the two suppositions I affirm or deny, or whether I form any judgment at all in the matter.

11. This indifference, moreover, extends not only to things of which the understanding has no knowledge at all, but in general also to all those which it does not discover with perfect clearness at the moment the will is deliberating upon them; for, however probable the conjectures may be that dispose me to form a judgment in a particular matter, the simple knowledge that these are merely conjectures, and not certain and indubitable reasons, is sufficient to lead me to form one that is directly the opposite. Of this I lately had abundant experience, when I laid aside as false all that I had before held for true, on the single ground that I could in some degree doubt of it.

12. But if I abstain from judging of a thing when I do not conceive it with sufficient clearness and distinctness, it is plain that I act rightly, and am not deceived; but if I resolve to deny or affirm, I then do not make a right use of my free will; and if I affirm what is false, it is evident that I am deceived; moreover, even although I judge according to truth, I stumble upon it by chance, and do not therefore escape the imputation of a wrong use of my freedom; for it is a dictate of the natural light, that the knowledge of the understanding ought always to precede the determination of the will. And it is this wrong use of the freedom of the will in which is found the privation that constitutes the form of error. Privation, I say, is found in the act, in so far as it proceeds from myself, but it does not exist in the faculty which I received from God, nor even in the act, in so far as it depends on him.

13. For I have assuredly no reason to complain that God has not given me a greater power of intelligence or more perfect natural light than he has actually bestowed, since it is of the nature of a finite understanding not to comprehend many things, and of the nature of a created understanding to be finite; on the contrary, I have every reason to render thanks to God, who owed me nothing, for having given me all the perfections I possess, and I should be far from thinking that he has unjustly deprived me of, or kept back, the other perfections which he has not bestowed upon me.

14. I have no reason, moreover, to complain because he has given me a will more ample than my understanding, since, as the will consists only of a single element, and that indivisible, it would appear that this faculty is of such a nature that nothing could be taken from it [without destroying it]; and certainly, the more extensive it is, the more cause I have to thank the goodness of him who bestowed it upon me.

15. And, finally, I ought not also to complain that God concurs with me in forming the acts of this will, or the judgments in which I am deceived, because those acts are wholly true and good, in so far as they depend on God; and the ability to form them is a higher degree of perfection in my nature than the want of it would be. With regard to privation, in which alone consists the formal reason of error and sin, this does not require the concurrence of Deity, because it is not a thing [or existence], and if it be referred to God as to its cause, it ought not to be called privation, but negation [according to the signification of these words in the schools]. For in truth it is no imperfection in Deity that he has accorded to me the power of giving or withholding my assent from certain things of which he has not put a clear and distinct knowledge in my understanding; but it is doubtless an imperfection in me that I do not use my freedom aright, and readily give my judgment on matters which I only obscurely and confusedly conceive. I perceive, nevertheless, that it was easy for Deity so to have constituted me as that I should never be deceived, although I still remained free and possessed of a limited knowledge, viz., by implanting in my understanding a clear and distinct knowledge of all the objects respecting which I should ever have to deliberate; or simply by so deeply engraving on my memory the resolution to judge of nothing without previously possessing a clear and distinct conception of it, that I should never forget it. And I easily understand that, in so far as I consider myself as a single whole, without reference to any other being in the universe, I should have been much more perfect than I now am, had Deity created me superior to error; but I cannot therefore deny that it is not somehow a greater perfection in the universe, that certain of its parts are not exempt from defect, as others are, than if they were all perfectly alike. And I have no right to complain because God, who placed me in the world, was not willing that I should sustain that character which of all others is the chief and most perfect.

16. I have even good reason to remain satisfied on the ground that, if he has not given me the perfection of being superior to error by the first means I have pointed out above, which depends on a clear and evident knowledge of all the matters regarding which I can deliberate, he has at least left in my power the other means, which is, firmly to retain the resolution never to judge where the truth is not clearly known to me: for, although I am conscious of the weakness of not being able to keep my mind continually fixed on the same thought, I can nevertheless, by attentive and oft-repeated meditation, impress it so strongly on my memory that I shall never fail to recollect it as often as I require it, and I can acquire in this way the habitude of not erring.

17. And since it is in being superior to error that the highest and chief perfection of man consists, I deem that I have not gained little by this day's meditation, in having discovered the source of error and falsity. And certainly this can be no other than what I have now explained: for as often as I so restrain my will within the limits of my knowledge, that it forms no judgment except regarding objects which are clearly and distinctly represented to it by the understanding, I can never be deceived; because every clear and distinct conception is doubtless something, and as such cannot owe its origin to nothing, but must of necessity have God for its author— God, I say, who, as supremely perfect, cannot, without a contradiction, be the cause of any error; and consequently it is necessary to conclude that every such conception [or judgment] is true. Nor have I merely learned to-day what I must avoid to escape error, but also what I must do to arrive at the knowledge of truth; for I will assuredly reach truth if I only fix my attention sufficiently on all the things I conceive perfectly, and separate these from others which I conceive more confusedly and obscurely; to which for the future I shall give diligent heed.

Meditation V


OF THE ESSENCE OF MATERIAL THINGS; AND, AGAIN, OF GOD; THAT HE EXISTS.

1. SEVERAL other questions remain for consideration respecting the attributes of God and my own nature or mind. I will, however, on some other occasion perhaps resume the investigation of these. Meanwhile, as I have discovered what must be done and what avoided to arrive at the knowledge of truth, what I have chiefly to do is to essay to emerge from the state of doubt in which I have for some time been, and to discover whether anything can be known with certainty regarding material objects.

2. But before considering whether such objects as I conceive exist without me, I must examine their ideas in so far as these are to be found in my consciousness, and discover which of them are distinct and which confused.

3. In the first place, I distinctly imagine that quantity which the philosophers commonly call continuous, or the extension in length, breadth, and depth that is in this quantity, or rather in the object to which it is attributed. Further, I can enumerate in it many diverse parts, and attribute to each of these all sorts of sizes, figures, situations, and local motions; and, in fine, I can assign to each of these motions all degrees of duration.

4. And I not only distinctly know these things when I thus consider them in general; but besides, by a little attention, I discover innumerable particulars respecting figures, numbers, motion, and the like, which are so evidently true, and so accordant with my nature, that when I now discover them I do not so much appear to learn anything new, as to call to remembrance what I before knew, or for the first time to remark what was before in my mind, but to which I had not hitherto directed my attention.

5. And what I here find of most importance is, that I discover in my mind innumerable ideas of certain objects, which cannot be esteemed pure negations, although perhaps they possess no reality beyond my thought, and which are not framed by me though it may be in my power to think, or not to think them, but possess true and immutable natures of their own. As, for example, when I imagine a triangle, although there is not perhaps and never was in any place in the universe apart from my thought one such figure, it remains true nevertheless that this figure possesses a certain determinate nature, form, or essence, which is immutable and eternal, and not framed by me, nor in any degree dependent on my thought; as appears from the circumstance, that diverse properties of the triangle may be demonstrated, viz, that its three angles are equal to two right, that its greatest side is subtended by its greatest angle, and the like, which, whether I will or not, I now clearly discern to belong to it, although before I did not at all think of them, when, for the first time, I imagined a triangle, and which accordingly cannot be said to have been invented by me.

6. Nor is it a valid objection to allege, that perhaps this idea of a triangle came into my mind by the medium of the senses, through my having. seen bodies of a triangular figure; for I am able to form in thought an innumerable variety of figures with regard to which it cannot be supposed that they were ever objects of sense, and I can nevertheless demonstrate diverse properties of their nature no less than of the triangle, all of which are assuredly true since I clearly conceive them: and they are therefore something, and not mere negations; for it is highly evident that all that is true is something, [truth being identical with existence]; and I have already fully shown the truth of the principle, that whatever is clearly and distinctly known is true. And although this had not been demonstrated, yet the nature of my mind is such as to compel me to assert to what I clearly conceive while I so conceive it; and I recollect that even when I still strongly adhered to the objects of sense, I reckoned among the number of the most certain truths those I clearly conceived relating to figures, numbers, and other matters that pertain to arithmetic and geometry, and in general to the pure mathematics.

7. But now if because I can draw from my thought the idea of an object, it follows that all I clearly and distinctly apprehend to pertain to this object, does in truth belong to it, may I not from this derive an argument for the existence of God? It is certain that I no less find the idea of a God in my consciousness, that is the idea of a being supremely perfect, than that of any figure or number whatever: and I know with not less clearness and distinctness that an [actual and] eternal existence pertains to his nature than that all which is demonstrable of any figure or number really belongs to the nature of that figure or number; and, therefore, although all the conclusions of the preceding Meditations were false, the existence of God would pass with me for a truth at least as certain as I ever judged any truth of mathematics to be.

8. Indeed such a doctrine may at first sight appear to contain more sophistry than truth. For, as I have been accustomed in every other matter to distinguish between existence and essence, I easily believe that the existence can be separated from the essence of God, and that thus God may be conceived as not actually existing. But, nevertheless, when I think of it more attentively, it appears that the existence can no more be separated from the essence of God, than the idea of a mountain from that of a valley, or the equality of its three angles to two right angles, from the essence of a [rectilinear] triangle; so that it is not less impossible to conceive a God, that is, a being supremely perfect, to whom existence is awanting, or who is devoid of a certain perfection, than to conceive a mountain without a valley.

9. But though, in truth, I cannot conceive a God unless as existing, any more than I can a mountain without a valley, yet, just as it does not follow that there is any mountain in the world merely because I conceive a mountain with a valley, so likewise, though I conceive God as existing, it does not seem to follow on that account that God exists; for my thought imposes no necessity on things; and as I may imagine a winged horse, though there be none such, so I could perhaps attribute existence to God, though no God existed.

10. But the cases are not analogous, and a fallacy lurks under the semblance of this objection: for because I cannot conceive a mountain without a valley, it does not follow that there is any mountain or valley in existence, but simply that the mountain or valley, whether they do or do not exist, are inseparable from each other; whereas, on the other hand, because I cannot conceive God unless as existing, it follows that existence is inseparable from him, and therefore that he really exists: not that this is brought about by my thought, or that it imposes any necessity on things, but, on the contrary, the necessity which lies in the thing itself, that is, the necessity of the existence of God, determines me to think in this way: for it is not in my power to conceive a God without existence, that is, a being supremely perfect, and yet devoid of an absolute perfection, as I am free to imagine a horse with or without wings.

11. Nor must it be alleged here as an objection, that it is in truth necessary to admit that God exists, after having supposed him to possess all perfections, since existence is one of them, but that my original supposition was not necessary; just as it is not necessary to think that all quadrilateral figures can be inscribed in the circle, since, if I supposed this, I should be constrained to admit that the rhombus, being a figure of four sides, can be therein inscribed, which, however, is manifestly false. This objection is, I say, incompetent; for although it may not be necessary that I shall at any time entertain the notion of Deity, yet each time I happen to think of a first and sovereign being, and to draw, so to speak, the idea of him from the storehouse of the mind, I am necessitated to attribute to him all kinds of perfections, though I may not then enumerate them all, nor think of each of them in particular. And this necessity is sufficient, as soon as I discover that existence is a perfection, to cause me to infer the existence of this first and sovereign being; just as it is not necessary that I should ever imagine any triangle, but whenever I am desirous of considering a rectilinear figure composed of only three angles, it is absolutely necessary to attribute those properties to it from which it is correctly inferred that its three angles are not greater than two right angles, although perhaps I may not then advert to this relation in particular. But when I consider what figures are capable of being inscribed in the circle, it is by no means necessary to hold that all quadrilateral figures are of this number; on the contrary, I cannot even imagine such to be the case, so long as I shall be unwilling to accept in thought aught that I do not clearly and distinctly conceive; and consequently there is a vast difference between false suppositions, as is the one in question, and the true ideas that were born with me, the first and chief of which is the idea of God. For indeed I discern on many grounds that this idea is not factitious depending simply on my thought, but that it is the representation of a true and immutable nature: in the first place because I can conceive no other being, except God, to whose essence existence [necessarily] pertains; in the second, because it is impossible to conceive two or more gods of this kind; and it being supposed that one such God exists, I clearly see that he must have existed from all eternity, and will exist to all eternity; and finally, because I apprehend many other properties in God, none of which I can either diminish or change.

12. But, indeed, whatever mode of probation I in the end adopt, it always returns to this, that it is only the things I clearly and distinctly conceive which have the power of completely persuading me. And although, of the objects I conceive in this manner, some, indeed, are obvious to every one, while others are only discovered after close and careful investigation; nevertheless after they are once discovered, the latter are not esteemed less certain than the former. Thus, for example, to take the case of a right-angled triangle, although it is not so manifest at first that the square of the base is equal to the squares of the other two sides, as that the base is opposite to the greatest angle; nevertheless, after it is once apprehended, we are as firmly persuaded of the truth of the former as of the latter. And, with respect to God if I were not pre-occupied by prejudices, and my thought beset on all sides by the continual presence of the images of sensible objects, I should know nothing sooner or more easily then the fact of his being. For is there any truth more clear than the existence of a Supreme Being, or of God, seeing it is to his essence alone that [necessary and eternal] existence pertains?

13. And although the right conception of this truth has cost me much close thinking, nevertheless at present I feel not only as assured of it as of what I deem most certain, but I remark further that the certitude of all other truths is so absolutely dependent on it that without this knowledge it is impossible ever to know anything perfectly.

14. For although I am of such a nature as to be unable, while I possess a very clear and distinct apprehension of a matter, to resist the conviction of its truth, yet because my constitution is also such as to incapacitate me from keeping my mind continually fixed on the same object, and as I frequently recollect a past judgment without at the same time being able to recall the grounds of it, it may happen meanwhile that other reasons are presented to me which would readily cause me to change my opinion, if I did not know that God existed; and thus I should possess no true and certain knowledge, but merely vague and vacillating opinions. Thus, for example, when I consider the nature of the [rectilinear] triangle, it most clearly appears to me, who have been instructed in the principles of geometry, that its three angles are equal to two right angles, and I find it impossible to believe otherwise, while I apply my mind to the demonstration; but as soon as I cease from attending to the process of proof, although I still remember that I had a clear comprehension of it, yet I may readily come to doubt of the truth demonstrated, if I do not know that there is a God: for I may persuade myself that I have been so constituted by nature as to be sometimes deceived, even in matters which I think I apprehend with the greatest evidence and certitude, especially when I recollect that I frequently considered many things to be true and certain which other reasons afterward constrained me to reckon as wholly false.

15. But after I have discovered that God exists, seeing I also at the same time observed that all things depend on him, and that he is no deceiver, and thence inferred that all which I clearly and distinctly perceive is of necessity true: although I no longer attend to the grounds of a judgment, no opposite reason can be alleged sufficient to lead me to doubt of its truth, provided only I remember that I once possessed a clear and distinct comprehension of it. My knowledge of it thus becomes true and certain. And this same knowledge extends likewise to whatever I remember to have formerly demonstrated, as the truths of geometry and the like: for what can be alleged against them to lead me to doubt of them ? Will it be that my nature is such that I may be frequently deceived? But I already know that I cannot be deceived in judgments of the grounds of which I possess a clear knowledge. Will it be that I formerly deemed things to be true and certain which I afterward discovered to be false ? But I had no clear and distinct knowledge of any of those things, and, being as yet ignorant of the rule by which I am assured of the truth of a judgment, I was led to give my assent to them on grounds which I afterward discovered were less strong than at the time I imagined them to be. What further objection, then, is there ? Will it be said that perhaps I am dreaming (an objection I lately myself raised), or that all the thoughts of which I am now conscious have no more truth than the reveries of my dreams ? But although, in truth, I should be dreaming, the rule still holds that all which is clearly presented to my intellect is indisputably true.

16. And thus I very clearly see that the certitude and truth of all science depends on the knowledge alone of the true God, insomuch that, before I knew him, I could have no perfect knowledge of any other thing. And now that I know him, I possess the means of acquiring a perfect knowledge respecting innumerable matters, as well relative to God himself and other intellectual objects as to corporeal nature, in so far as it is the object of pure mathematics [which do not consider whether it exists or not].

Meditation VI


OF THE EXISTENCE OF MATERIAL THINGS, AND OF THE REAL DISTINCTION BETWEEN THE MIND AND BODY OF MAN.

1. THERE now only remains the inquiry as to whether material things exist. With regard to this question, I at least know with certainty that such things may exist, in as far as they constitute the object of the pure mathematics, since, regarding them in this aspect, I can conceive them clearly and distinctly. For there can be no doubt that God possesses the power of producing all the objects I am able distinctly to conceive, and I never considered anything impossible to him, unless when I experienced a contradiction in the attempt to conceive it aright. Further, the faculty of imagination which I possess, and of which I am conscious that I make use when I apply myself to the consideration of material things, is sufficient to persuade me of their existence: for, when I attentively consider what imagination is, I find that it is simply a certain application of the cognitive faculty (facultas cognoscitiva) to a body which is immediately present to it, and which therefore exists.

2. And to render this quite clear, I remark, in the first place, the difference that subsists between imagination and pure intellection [or conception]. For example, when I imagine a triangle I not only conceive (intelligo) that it is a figure comprehended by three lines, but at the same time also I look upon (intueor) these three lines as present by the power and internal application of my mind (acie mentis), and this is what I call imagining. But if I desire to think of a chiliogon, I indeed rightly conceive that it is a figure composed of a thousand sides, as easily as I conceive that a triangle is a figure composed of only three sides; but I cannot imagine the thousand sides of a chiliogon as I do the three sides of a triangle, nor, so to speak, view them as present [with the eyes of my mind]. And although, in accordance with the habit I have of always imagining something when I think of corporeal things, it may happen that, in conceiving a chiliogon, I confusedly represent some figure to myself, yet it is quite evident that this is not a chiliogon, since it in no wise differs from that which I would represent to myself, if I were to think of a myriogon, or any other figure of many sides; nor would this representation be of any use in discovering and unfolding the properties that constitute the difference between a chiliogon and other polygons. But if the question turns on a pentagon, it is quite true that I can conceive its figure, as well as that of a chiliogon, without the aid of imagination; but I can likewise imagine it by applying the attention of my mind to its five sides, and at the same time to the area which they contain. Thus I observe that a special effort of mind is necessary to the act of imagination, which is not required to conceiving or understanding (ad intelligendum); and this special exertion of mind clearly shows the difference between imagination and pure intellection (imaginatio et intellectio pura).

3. I remark, besides, that this power of imagination which I possess, in as far as it differs from the power of conceiving, is in no way necessary to my [nature or] essence, that is, to the essence of my mind; for although I did not possess it, I should still remain the same that I now am, from which it seems we may conclude that it depends on something different from the mind. And I easily understand that, if some body exists, with which my mind is so conjoined and united as to be able, as it were, to consider it when it chooses, it may thus imagine corporeal objects; so that this mode of thinking differs from pure intellection only in this respect, that the mind in conceiving turns in some way upon itself, and considers some one of the ideas it possesses within itself; but in imagining it turns toward the body, and contemplates in it some object conformed to the idea which it either of itself conceived or apprehended by sense. I easily understand, I say, that imagination may be thus formed, if it is true that there are bodies; and because I find no other obvious mode of explaining it, I thence, with probability, conjecture that they exist, but only with probability; and although I carefully examine all things, nevertheless I do not find that, from the distinct idea of corporeal nature I have in my imagination, I can necessarily infer the existence of any body.

4. But I am accustomed to imagine many other objects besides that corporeal nature which is the object of the pure mathematics, as, for example, colors, sounds, tastes, pain, and the like, although with less distinctness; and, inasmuch as I perceive these objects much better by the senses, through the medium of which and of memory, they seem to have reached the imagination, I believe that, in order the more advantageously to examine them, it is proper I should at the same time examine what sense-perception is, and inquire whether from those ideas that are apprehended by this mode of thinking (consciousness), I cannot obtain a certain proof of the existence of corporeal objects.

5. And, in the first place, I will recall to my mind the things I have hitherto held as true, because perceived by the senses, and the foundations upon which my belief in their truth rested; I will, in the second place, examine the reasons that afterward constrained me to doubt of them; and, finally, I will consider what of them I ought now to believe.

6. Firstly, then, I perceived that I had a head, hands, feet and other members composing that body which I considered as part, or perhaps even as the whole, of myself. I perceived further, that that body was placed among many others, by which it was capable of being affected in diverse ways, both beneficial and hurtful; and what was beneficial I remarked by a certain sensation of pleasure, and what was hurtful by a sensation of pain. And besides this pleasure and pain, I was likewise conscious of hunger, thirst, and other appetites, as well as certain corporeal inclinations toward joy, sadness, anger, and similar passions. And, out of myself, besides the extension, figure, and motions of bodies, I likewise perceived in them hardness, heat, and the other tactile qualities, and, in addition, light, colors, odors, tastes, and sounds, the variety of which gave me the means of distinguishing the sky, the earth, the sea, and generally all the other bodies, from one another. And certainly, considering the ideas of all these qualities, which were presented to my mind, and which alone I properly and immediately perceived, it was not without reason that I thought I perceived certain objects wholly different from my thought, namely, bodies from which those ideas proceeded; for I was conscious that the ideas were presented to me without my consent being required, so that I could not perceive any object, however desirous I might be, unless it were present to the organ of sense; and it was wholly out of my power not to perceive it when it was thus present. And because the ideas I perceived by the senses were much more lively and clear, and even, in their own way, more distinct than any of those I could of myself frame by meditation, or which I found impressed on my memory, it seemed that they could not have proceeded from myself, and must therefore have been caused in me by some other objects; and as of those objects I had no knowledge beyond what the ideas themselves gave me, nothing was so likely to occur to my mind as the supposition that the objects were similar to the ideas which they caused. And because I recollected also that I had formerly trusted to the senses, rather than to reason, and that the ideas which I myself formed were not so clear as those I perceived by sense, and that they were even for the most part composed of parts of the latter, I was readily persuaded that I had no idea in my intellect which had not formerly passed through the senses. Nor was I altogether wrong in likewise believing that that body which, by a special right, I called my own, pertained to me more properly and strictly than any of the others; for in truth, I could never be separated from it as from other bodies; I felt in it and on account of it all my appetites and affections, and in fine I was affected in its parts by pain and the titillation of pleasure, and not in the parts of the other bodies that were separated from it. But when I inquired into the reason why, from this I know not what sensation of pain, sadness of mind should follow, and why from the sensation of pleasure, joy should arise, or why this indescribable twitching of the stomach, which I call hunger, should put me in mind of taking food, and the parchedness of the throat of drink, and so in other cases, I was unable to give any explanation, unless that I was so taught by nature; for there is assuredly no affinity, at least none that I am able to comprehend, between this irritation of the stomach and the desire of food, any more than between the perception of an object that causes pain and the consciousness of sadness which springs from the perception. And in the same way it seemed to me that all the other judgments I had formed regarding the objects of sense, were dictates of nature; because I remarked that those judgments were formed in me, before I had leisure to weigh and consider the reasons that might constrain me to form them.

7. But, afterward, a wide experience by degrees sapped the faith I had reposed in my senses; for I frequently observed that towers, which at a distance seemed round, appeared square, when more closely viewed, and that colossal figures, raised on the summits of these towers, looked like small statues, when viewed from the bottom of them; and, in other instances without number, I also discovered error in judgments founded on the external senses; and not only in those founded on the external, but even in those that rested on the internal senses; for is there aught more internal than pain ? And yet I have sometimes been informed by parties whose arm or leg had been amputated, that they still occasionally seemed to feel pain in that part of the body which they had lost,—a circumstance that led me to think that I could not be quite certain even that any one of my members was affected when I felt pain in it. And to these grounds of doubt I shortly afterward also added two others of very wide generality: the first of them was that I believed I never perceived anything when awake which I could not occasionally think I also perceived when asleep, and as I do not believe that the ideas I seem to perceive in my sleep proceed from objects external to me, I did not any more observe any ground for believing this of such as I seem to perceive when awake; the second was that since I was as yet ignorant of the author of my being or at least supposed myself to be so, I saw nothing to prevent my having been so constituted by nature as that I should be deceived even in matters that appeared to me to possess the greatest truth. And, with respect to the grounds on which I had before been persuaded of the existence of sensible objects, I had no great difficulty in finding suitable answers to them; for as nature seemed to incline me to many things from which reason made me averse, I thought that I ought not to confide much in its teachings. And although the perceptions of the senses were not dependent on my will, I did not think that I ought on that ground to conclude that they proceeded from things different from myself, since perhaps there might be found in me some faculty, though hitherto unknown to me, which produced them.

8. But now that I begin to know myself better, and to discover more clearly the author of my being, I do not, indeed, think that I ought rashly to admit all which the senses seem to teach, nor, on the other hand, is it my conviction that I ought to doubt in general of their teachings.

9. And, firstly, because I know that all which I clearly and distinctly conceive can be produced by God exactly as I conceive it, it is sufficient that I am able clearly and distinctly to conceive one thing apart from another, in order to be certain that the one is different from the other, seeing they may at least be made to exist separately, by the omnipotence of God; and it matters not by what power this separation is made, in order to be compelled to judge them different; and, therefore, merely because I know with certitude that I exist, and because, in the meantime, I do not observe that aught necessarily belongs to my nature or essence beyond my being a thinking thing, I rightly conclude that my essence consists only in my being a thinking thing [or a substance whose whole essence or nature is merely thinking]. And although I may, or rather, as I will shortly say, although I certainly do possess a body with which I am very closely conjoined; nevertheless, because, on the one hand, I have a clear and distinct idea of myself, in as far as I am only a thinking and unextended thing, and as, on the other hand, I possess a distinct idea of body, in as far as it is only an extended and unthinking thing, it is certain that I, [that is, my mind, by which I am what I am], is entirely and truly distinct from my body, and may exist without it.

10. Moreover, I find in myself diverse faculties of thinking that have each their special mode: for example, I find I possess the faculties of imagining and perceiving, without which I can indeed clearly and distinctly conceive myself as entire, but I cannot reciprocally conceive them without conceiving myself, that is to say, without an intelligent substance in which they reside, for [in the notion we have of them, or to use the terms of the schools] in their formal concept, they comprise some sort of intellection; whence I perceive that they are distinct from myself as modes are from things. I remark likewise certain other faculties, as the power of changing place, of assuming diverse figures, and the like, that cannot be conceived and cannot therefore exist, any more than the preceding, apart from a substance in which they inhere. It is very evident, however, that these faculties, if they really exist, must belong to some corporeal or extended substance, since in their clear and distinct concept there is contained some sort of extension, but no intellection at all. Further, I cannot doubt but that there is in me a certain passive faculty of perception, that is, of receiving and taking knowledge of the ideas of sensible things; but this would be useless to me, if there did not also exist in me, or in some other thing, another active faculty capable of forming and producing those ideas. But this active faculty cannot be in me [in as far as I am but a thinking thing], seeing that it does not presuppose thought, and also that those ideas are frequently produced in my mind without my contributing to it in any way, and even frequently contrary to my will. This faculty must therefore exist in some substance different from me, in which all the objective reality of the ideas that are produced by this faculty is contained formally or eminently, as I before remarked; and this substance is either a body, that is to say, a corporeal nature in which is contained formally [and in effect] all that is objectively [and by representation] in those ideas; or it is God Himself, or some other creature, of a rank superior to body, in which the same is contained eminently. But as God is no deceiver, it is manifest that He does not of Himself and immediately communicate those ideas to me, nor even by the intervention of any creature in which their objective reality is not formally, but only eminently, contained. For as He has given me no faculty whereby I can discover this to be the case, but, on the contrary, a very strong inclination to believe that those ideas arise from corporeal objects, I do not see how He could be vindicated from the charge of deceit, if in truth they proceeded from any other source, or were produced by other causes than corporeal things: and accordingly it must be concluded, that corporeal objects exist. Nevertheless, they are not perhaps exactly such as we perceive by the senses, for their comprehension by the senses is, in many instances, very obscure and confused; but it is at least necessary to admit that all which I clearly and distinctly conceive as in them, that is, generally speaking all that is comprehended in the object of speculative geometry, really exists external to me.

11. But with respect to other things which are either only particular, as, for example, that the sun is of such a size and figure, etc., or are conceived with less clearness and distinctness, as light, sound, pain, and the like, although they are highly dubious and uncertain, nevertheless on the ground alone that God is no deceiver, and that consequently he has permitted no falsity in my opinions which he has not likewise given me a faculty of correcting, I think I may with safety conclude that I possess in myself the means of arriving at the truth. And, in the first place, it cannot be doubted that in each of the dictates of nature there is some truth: for by nature, considered in general, I now understand nothing more than God Himself, or the order and disposition established by God in created things; and by my nature in particular I understand the assemblage of all that God has given me.

12. But there is nothing which that nature teaches me more expressly [or more sensibly] than that I have a body which is ill affected when I feel pain, and stands in need of food and drink when I experience the sensations of hunger and thirst, etc. And therefore I ought not to doubt but that there is some truth in these informations.

13. Nature likewise teaches me by these sensations of pain, hunger, thirst, etc., that I am not only lodged in my body as a pilot in a vessel, but that I am besides so intimately conjoined, and as it were intermixed with it, that my mind and body compose a certain unity. For if this were not the case, I should not feel pain when my body is hurt, seeing I am merely a thinking thing, but should perceive the wound by the understanding alone, just as a pilot perceives by sight when any part of his vessel is damaged; and when my body has need of food or drink, I should have a clear knowledge of this, and not be made aware of it by the confused sensations of hunger and thirst: for, in truth, all these sensations of hunger, thirst, pain, etc., are nothing more than certain confused modes of thinking, arising from the union and apparent fusion of mind and body.

14. Besides this, nature teaches me that my own body is surrounded by many other bodies, some of which I have to seek after, and others to shun. And indeed, as I perceive different sorts of colors, sounds, odors, tastes, heat, hardness, etc., I safely conclude that there are in the bodies from which the diverse perceptions of the senses proceed, certain varieties corresponding to them, although, perhaps, not in reality like them; and since, among these diverse perceptions of the senses, some are agreeable, and others disagreeable, there can be no doubt that my body, or rather my entire self, in as far as I am composed of body and mind, may be variously affected, both beneficially and hurtfully, by surrounding bodies.

15. But there are many other beliefs which though seemingly the teaching of nature, are not in reality so, but which obtained a place in my mind through a habit of judging inconsiderately of things. It may thus easily happen that such judgments shall contain error: thus, for example, the opinion I have that all space in which there is nothing to affect [or make an impression on] my senses is void: that in a hot body there is something in every respect similar to the idea of heat in my mind; that in a white or green body there is the same whiteness or greenness which I perceive; that in a bitter or sweet body there is the same taste, and so in other instances; that the stars, towers, and all distant bodies, are of the same size and figure as they appear to our eyes, etc. But that I may avoid everything like indistinctness of conception, I must accurately define what I properly understand by being taught by nature. For nature is here taken in a narrower sense than when it signifies the sum of all the things which God has given me; seeing that in that meaning the notion comprehends much that belongs only to the mind [to which I am not here to be understood as referring when I use the term nature]; as, for example, the notion I have of the truth, that what is done cannot be undone, and all the other truths I discern by the natural light [ without the aid of the body]; and seeing that it comprehends likewise much besides that belongs only to body, and is not here any more contained under the name nature, as the quality of heaviness, and the like, of which I do not speak, the term being reserved exclusively to designate the things which God has given to me as a being composed of mind and body. But nature, taking the term in the sense explained, teaches me to shun what causes in me the sensation of pain, and to pursue what affords me the sensation of pleasure, and other things of this sort; but I do not discover that it teaches me, in addition to this, from these diverse perceptions of the senses, to draw any conclusions respecting external objects without a previous [careful and mature] consideration of them by the mind: for it is, as appears to me, the office of the mind alone, and not of the composite whole of mind and body, to discern the truth in those matters. Thus, although the impression a star makes on my eye is not larger than that from the flame of a candle, I do not, nevertheless, experience any real or positive impulse determining me to believe that the star is not greater than the flame; the true account of the matter being merely that I have so judged from my youth without any rational ground. And, though on approaching the fire I feel heat, and even pain on approaching it too closely, I have, however, from this no ground for holding that something resembling the heat I feel is in the fire, any more than that there is something similar to the pain; all that I have ground for believing is, that there is something in it, whatever it may be, which excites in me those sensations of heat or pain. So also, although there are spaces in which I find nothing to excite and affect my senses, I must not therefore conclude that those spaces contain in them no body; for I see that in this, as in many other similar matters, I have been accustomed to pervert the order of nature, because these perceptions of the senses, although given me by nature merely to signify to my mind what things are beneficial and hurtful to the composite whole of which it is a part, and being sufficiently clear and distinct for that purpose, are nevertheless used by me as infallible rules by which to determine immediately the essence of the bodies that exist out of me, of which they can of course afford me only the most obscure and confused knowledge.

16. But I have already sufficiently considered how it happens that, notwithstanding the supreme goodness of God, there is falsity in my judgments. A difficulty, however, here presents itself, respecting the things which I am taught by nature must be pursued or avoided, and also respecting the internal sensations in which I seem to have occasionally detected error, [and thus to be directly deceived by nature]: thus, for example, I may be so deceived by the agreeable taste of some viand with which poison has been mixed, as to be induced to take the poison. In this case, however, nature may be excused, for it simply leads me to desire the viand for its agreeable taste, and not the poison, which is unknown to it; and thus we can infer nothing from this circumstance beyond that our nature is not omniscient; at which there is assuredly no ground for surprise, since, man being of a finite nature, his knowledge must likewise be of a limited perfection.

17. But we also not unfrequently err in that to which we are directly impelled by nature, as is the case with invalids who desire drink or food that would be hurtful to them. It will here, perhaps, be alleged that the reason why such persons are deceived is that their nature is corrupted; but this leaves the difficulty untouched, for a sick man is not less really the creature of God than a man who is in full health; and therefore it is as repugnant to the goodness of God that the nature of the former should be deceitful as it is for that of the latter to be so. And as a clock, composed of wheels and counter weights, observes not the less accurately all the laws of nature when it is ill made, and points out the hours incorrectly, than when it satisfies the desire of the maker in every respect; so likewise if the body of man be considered as a kind of machine, so made up and composed of bones, nerves, muscles, veins, blood, and skin, that although there were in it no mind, it would still exhibit the same motions which it at present manifests involuntarily, and therefore without the aid of the mind, [and simply by the dispositions of its organs], I easily discern that it would also be as natural for such a body, supposing it dropsical, for example, to experience the parchedness of the throat that is usually accompanied in the mind by the sensation of thirst, and to be disposed by this parchedness to move its nerves and its other parts in the way required for drinking, and thus increase its malady and do itself harm, as it is natural for it, when it is not indisposed to be stimulated to drink for its good by a similar cause; and although looking to the use for which a clock was destined by its maker, I may say that it is deflected from its proper nature when it incorrectly indicates the hours, and on the same principle, considering the machine of the human body as having been formed by God for the sake of the motions which it usually manifests, although I may likewise have ground for thinking that it does not follow the order of its nature when the throat is parched and drink does not tend to its preservation, nevertheless I yet plainly discern that this latter acceptation of the term nature is very different from the other: for this is nothing more than a certain denomination, depending entirely on my thought, and hence called extrinsic, by which I compare a sick man and an imperfectly constructed clock with the idea I have of a man in good health and a well made clock; while by the other acceptation of nature is understood something which is truly found in things, and therefore possessed of some truth.

18. But certainly, although in respect of a dropsical body, it is only by way of exterior denomination that we say its nature is corrupted, when, without requiring drink, the throat is parched; yet, in respect of the composite whole, that is, of the mind in its union with the body, it is not a pure denomination, but really an error of nature, for it to feel thirst when drink would be hurtful to it: and, accordingly, it still remains to be considered why it is that the goodness of God does not prevent the nature of man thus taken from being fallacious.

19. To commence this examination accordingly, I here remark, in the first place, that there is a vast difference between mind and body, in respect that body, from its nature, is always divisible, and that mind is entirely indivisible. For in truth, when I consider the mind, that is, when I consider myself in so far only as I am a thinking thing, I can distinguish in myself no parts, but I very clearly discern that I am somewhat absolutely one and entire; and although the whole mind seems to be united to the whole body, yet, when a foot, an arm, or any other part is cut off, I am conscious that nothing has been taken from my mind; nor can the faculties of willing, perceiving, conceiving, etc., properly be called its parts, for it is the same mind that is exercised [all entire] in willing, in perceiving, and in conceiving, etc. But quite the opposite holds in corporeal or extended things; for I cannot imagine any one of them [how small soever it may be], which I cannot easily sunder in thought, and which, therefore, I do not know to be divisible. This would be sufficient to teach me that the mind or soul of man is entirely different from the body, if I had not already been apprised of it on other grounds.

20. I remark, in the next place, that the mind does not immediately receive the impression from all the parts of the body, but only from the brain, or perhaps even from one small part of it, viz, that in which the common sense (senses communis) is said to be, which as often as it is affected in the same way gives rise to the same perception in the mind, although meanwhile the other parts of the body may be diversely disposed, as is proved by innumerable experiments, which it is unnecessary here to enumerate.

21. I remark, besides, that the nature of body is such that none of its parts can be moved by another part a little removed from the other, which cannot likewise be moved in the same way by any one of the parts that lie between those two, although the most remote part does not act at all. As, for example, in the cord A, B, C, D, [which is in tension], if its last part D, be pulled, the first part A, will not be moved in a different way than it would be were one of the intermediate parts B or C to be pulled, and the last part D meanwhile to remain fixed. And in the same way, when I feel pain in the foot, the science of physics teaches me that this sensation is experienced by means of the nerves dispersed over the foot, which, extending like cords from it to the brain, when they are contracted in the foot, contract at the same time the inmost parts of the brain in which they have their origin, and excite in these parts a certain motion appointed by nature to cause in the mind a sensation of pain, as if existing in the foot; but as these nerves must pass through the tibia, the leg, the loins, the back, and neck, in order to reach the brain, it may happen that although their extremities in the foot are not affected, but only certain of their parts that pass through the loins or neck, the same movements, nevertheless, are excited in the brain by this motion as would have been caused there by a hurt received in the foot, and hence the mind will necessarily feel pain in the foot, just as if it had been hurt; and the same is true of all the other perceptions of our senses.

22. I remark, finally, that as each of the movements that are made in the part of the brain by which the mind is immediately affected, impresses it with but a single sensation, the most likely supposition in the circumstances is, that this movement causes the mind to experience, among all the sensations which it is capable of impressing upon it; that one which is the best fitted, and generally the most useful for the preservation of the human body when it is in full health. But experience shows us that all the perceptions which nature has given us are of such a kind as I have mentioned; and accordingly, there is nothing found in them that does not manifest the power and goodness of God. Thus, for example, when the nerves of the foot are violently or more than usually shaken, the motion passing through the medulla of the spine to the innermost parts of the brain affords a sign to the mind on which it experiences a sensation, viz, of pain, as if it were in the foot, by which the mind is admonished and excited to do its utmost to remove the cause of it as dangerous and hurtful to the foot. It is true that God could have so constituted the nature of man as that the same motion in the brain would have informed the mind of something altogether different: the motion might, for example, have been the occasion on which the mind became conscious of itself, in so far as it is in the brain, or in so far as it is in some place intermediate between the foot and the brain, or, finally, the occasion on which it perceived some other object quite different, whatever that might be; but nothing of all this would have so well contributed to the preservation of the body as that which the mind actually feels. In the same way, when we stand in need of drink, there arises from this want a certain parchedness in the throat that moves its nerves, and by means of them the internal parts of the brain; and this movement affects the mind with the sensation of thirst, because there is nothing on that occasion which is more useful for us than to be made aware that we have need of drink for the preservation of our health; and so in other instances.

23. Whence it is quite manifest that, notwithstanding the sovereign goodness of God, the nature of man, in so far as it is composed of mind and body, cannot but be sometimes fallacious. For, if there is any cause which excites, not in the foot, but in some one of the parts of the nerves that stretch from the foot to the brain, or even in the brain itself, the same movement that is ordinarily created when the foot is ill affected, pain will be felt, as it were, in the foot, and the sense will thus be naturally deceived; for as the same movement in the brain can but impress the mind with the same sensation, and as this sensation is much more frequently excited by a cause which hurts the foot than by one acting in a different quarter, it is reasonable that it should lead the mind to feel pain in the foot rather than in any other part of the body. And if it sometimes happens that the parchedness of the throat does not arise, as is usual, from drink being necessary for the health of the body, but from quite the opposite cause, as is the case with the dropsical, yet it is much better that it should be deceitful in that instance, than if, on the contrary, it were continually fallacious when the body is well-disposed; and the same holds true in other cases.

24. And certainly this consideration is of great service, not only in enabling me to recognize the errors to which my nature is liable, but likewise in rendering it more easy to avoid or correct them: for, knowing that all my senses more usually indicate to me what is true than what is false, in matters relating to the advantage of the body, and being able almost always to make use of more than a single sense in examining the same object, and besides this, being able to use my memory in connecting present with past knowledge, and my understanding which has already discovered all the causes of my errors, I ought no longer to fear that falsity may be met with in what is daily presented to me by the senses. And I ought to reject all the doubts of those bygone days, as hyperbolical and ridiculous, especially the general uncertainty respecting sleep, which I could not distinguish from the waking state: for I now find a very marked difference between the two states, in respect that our memory can never connect our dreams with each other and with the course of life, in the way it is in the habit of doing with events that occur when we are awake. And, in truth, if some one, when I am awake, appeared to me all of a sudden and as suddenly disappeared, as do the images I see in sleep, so that I could not observe either whence he came or whither he went, I should not without reason esteem it either a specter or phantom formed in my brain, rather than a real man. But when I perceive objects with regard to which I can distinctly determine both the place whence they come, and that in which they are, and the time at which they appear to me, and when, without interruption, I can connect the perception I have of them with the whole of the other parts of my life, I am perfectly sure that what I thus perceive occurs while I am awake and not during sleep. And I ought not in the least degree to doubt of the truth of these presentations, if, after having called together all my senses, my memory, and my understanding for the purpose of examining them, no deliverance is given by any one of these faculties which is repugnant to that of any other: for since God is no deceiver, it necessarily follows that I am not herein deceived. But because the necessities of action frequently oblige us to come to a determination before we have had leisure for so careful an examination, it must be confessed that the life of man is frequently obnoxious to error with respect to individual objects; and we must, in conclusion, ac. knowledge the weakness of our nature.
