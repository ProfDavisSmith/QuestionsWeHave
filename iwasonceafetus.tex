\chapter{I Was Once a Fetus by Alexander Pruss}\autocite{Pruss1}
\label{iwasafetus}
\section{INTRODUCTION}
 
I am going to give an argument showing that abortion is wrong in exactly 
the same circumstances in which it is wrong to kill an adult.  To argue 
further that abortion is always wrong would require showing that it is 
always wrong to kill an adult or that the circumstances in which it is not 
wrong–say, capital punishment–never befall a fetus.  Such an argument 
will be beyond the scope of this paper, but since it is uncontroversial that 
it is wrong to kill an adult human being for the sorts of reasons for which 
most abortions are performed, it follows that most abortions are wrong. 
The argument has three parts, of decreasing difficulty.  The most 
difficult will be the first part where I will argue that I was once a fetus 
and  before  that  I  was  an  embryo.    This  argument  will  rest  on  simple 
considerations  of  the  metaphysics  of  identity.    The  next  part  of  the 
argument will be to show that it would have been at least as wrong to 
have killed me before I was born as it would be to kill me now.  I will 
argue for this in more than one way, but the guiding intuition is clear: if 
you kill me earlier, the victim is the same but the harm is greater since I 
am  deprived  of  more  the  earlier  I  die.    Finally,  the  easiest  part  of  the 
argument will be that I am not relevantly different from anybody else and 
the fetus that I was was not relevantly different from any other human 
fetus, and so the argument applies equally well to all fetuses. 
The advantage of this argument over others is that it avoids talking 
of personhood, except in one of the several independent arguments in 
part two. 
 
 
\section{1. I WAS ONCE A FETUS} 
The first part seems innocuous.  After all, is it not biologically evident 
that first I was an embryo, then a fetus, then a neonate, then an infant, 
then a toddler, then a child, then an adolescent, and then an adult?  Does 
not my mother talk of the time when she was “pregnant with me” and 
thereby  imply  that  it  was  I  who  was  in  her  womb  when  she  was 
pregnant?    Is  not  the  sonogram  of  my  daughter  the  sonogram  of  that 
daughter of mine who will be born?  Evident as it might be that I was 
once a fetus and given how clear it will be that abortion is wrong if I was 
once a fetus, it is obvious, however, that the opponent will have to focus 
his attack on this part of the argument.  So more needs to be said. 

About thirty years ago, nine months before I was born, a conception 
occurred.  A sperm from my father fertilized an ovum from my mother.  
Within  twenty-four  hours,  or  sooner,  a  new  organism  came    into 
existence, an organism that was neither a part of my mother nor of my 
father.  For one, this organism was genetically distinct from both.  For 
another,  this  organism’s  functioning  was  directed  towards  its  own 
benefit–selfishly, the organism colonized the womb, released hormones 
that trigger changes in the woman beneficial to the organism, and so on.  
It certainly did not behave like a body part of either my mother or my 
father.  Moreover, it clearly was not a part of my father–it need no longer 
have had any interaction with him.  But it could not really be a part of my 
mother since the genetic contribution from my father was equal to that 
from  the  mother,  so  it  was  either  a  part  of  both  or  of  neither.    Thus, 
indeed, it was not a part of either.  Besides, we can see that in the earliest 
days  of  this  organism  before  implantation,  the  organism  floated  free, 
independently seeking nutrition in my mother’s womb.  This organism 
certainly was not a part of my mother. 

Hence, we have on the scene a new individual organism, one that 
did not exist before.  Let’s give this organism a name: call it Bob.  If we 
have a camera and look at what was happening in the womb in which 
Bob is living, we will see an embryo developing, cells differentiating, a 
fetus forming, growing, and finally a birth.  If we keep watching, we see 
a neonate, then an infant, then a toddler, then a child, then an adolescent 
and then an adult.  It’s all a continuous history.  But recall what I am out 
to prove.  I am out to prove that I was once a fetus and indeed an embryo 
against an opponent who will not grant this.  My opponent will thus have 
to deny that I and Bob are one and the same entity.  He will have to say 
that “Bob” and “Alex” name two different entities, rather than being two 
names for one and the same entity at different stages of its life. 

In any case, we have on the scene Bob the embryo.  And then all 
this development happens.  I now need a simple metaphysical principle.  
If an organism that once existed has never died, then this organism still 
exists.    I  will  not  argue  for  this  principle.    Someone  who  thinks  that 
something  can  exist  at  time  A  and  not  exist  at  a  later  time  B,  without 
having ceased to exist in between, is beyond the reach of argument.  The 
crucial  question  now  is:  Has  Bob  the  embryo  ever  died?    This  is  a 
question to which the biologists can tell us the answer.  Bob’s cells have 
divided,  differentiated,  and  Bob  has  developed.    But  nowhere  in  the 
continuous history just described have we seen anything we could iden-
tify as “the death of Bob.”  In fact, the whole process is the very opposite 
of the process of death: we have a process of growth.  That embryo that 
was conceived nine months before my birth never died.  True, it ceased 
to be an embryo, and at the end of the nine months it ceased to be a fetus. 
 But this is no more a literal death than my passing from childhood to 
adolescence or from adolescence to adulthood was. 

Indeed, if Bob died, we would be mystified as to when he died.  All 
we have in its life history is a process of growth and development.  Now, 
it is true that not all deaths are alike–not all deaths involve an evident 
destruction.  For instance, some philosophers think that the right way to 
describe an amoeba’s splitting is to say that the original amoeba dies and 
from its ashes there arise two new amoebae.  Likewise, some 
philosophers  think  that  when  two  entities  merge  into  a  single  unified 
entity, the original entities perish and a new one is formed.  That in fact 
may be how we should understand the process of conception: the egg and 
sperm perish, and a new thing results.  But again, for as long as Bob has 
existed,  he  has  always,  in  fact,  been  a  single  unified  organism,  and 
nothing like that happened in Bob’s life history–Bob never split in two 
and never merged with anything else so as to lose its own identity.  If I 
were an identical twin, matters would be slightly different as an argument 
could then be made that the pre-twinning embryo has indeed perished 
when it split in two.  But that’s not what happened to Bob.  It is clear that 
Bob has not died in the prosaic way of having his organic functioning 
disrupted,  and  hasn’t  even  died  in  these  two  more  outré  ways  that 
philosophers discuss.  

Furthermore,  the  very  continuity  in  Bob’s  development  speaks 
against  the  hypothesis  that  he  died.    When  did  that  momentous  event 
happen?  When did Bob cease to exist?  Could there have been some 
moment  in  Bob’s  growth  where  one  millisecond  Bob was alive and a 
millisecond later Bob was no longer around?  Surely not. 

Therefore, it is sufficiently established that Bob, that embryo who 
came into existence nine months before my birth, has never died.  But by 
my metaphysical principle, if he has never died, he is still alive.  Where, 
then, is Bob?  But surely there is no mystery there.  Every part of Bob–
other  than  the  cells  in  the  placenta  and  the  umbilical  cord  that  were 
shedi–developed continuously into a part of me, and every part of me has 
developed ultimately out of a part of Bob.  It is thus quite futile to look 
for Bob outside of me.  If Bob is anywhere, he is right here, where I am.  
It may be true that most of the original cells in Bob are no longer around, 
but  that  does  not  stop  the  survival  of  an  organism:  organisms  replace 
their cells regularly and do not perish thereby. 

Now, Bob can’t be a mere part of my body, because all of my body 
has continuously come from Bob’s body.  Therefore, one can’t set aside 
some  special  part  of  my  body  and  say  “that  part  of  me  is  Bob.”    So, 
where is Bob?  The answer is simple: Here.  I am Bob.  That embryo has 
grown to be a fetus, then to be a neonate, then an infant, then a child, 
then an adolescent and finally an adult.  Bob is I and I am Bob.  This was 
what I was trying to establish. 

But this is a little too quick.  I just said, vaguely, that Bob is here, 
and concluded that Bob is I.  We need the following argument.  Here 
where I stand there is only one large animal–Alexander Pruss.  Bob is 
presumably right here–there is nowhere else for him to be.  Bob has been 
growing for much of his life, and so Bob is also a large animal.  The only 
large  animal  here  is  Alexander  Pruss,  and  hence  Bob  and  Alexander 
Pruss are one and the same animal.  I, thus, am Bob.  If Bob is here, and 
if no part of me is a large animal, and if Bob is a large animal, Bob and I 
must be one and the same entity.   
 
Besides, given how organic development works, it is easy to see that 
every  organ  of  mine  is  an  organ  of  Bob’s  since  Bob’s  organs  have 
developed into being my organs, and yet without any transplant 
happening.  Thus, I and Bob are organisms having all of our organs in 
common.    But  the  only  way  that  can  be  is  if  I  and  Bob  are  the  same 
organism, i.e., I am Bob.  “Bob” and “Alex” are just different names for 
one and the same being: Alexander Robert Pruss. 

There is only one way of countering this argument, and this is to 
deny that I am an animal, that I am an organism.  This response seems 
absurd on the face of it, and it is right that we should see it as absurd.  I 
am a rational animal.  But there are three seemingly plausible ways of 
making this objection work.  They are not the only ones, but they will be 
representative. 

The  first  form  that  this  objection  can  take  is  Cartesian  dualism.  
Souls and bodies are separate substances.  What I really am is a soul, a 
spiritual  substance.  The body is simply a tool that my soul owns and 
uses, much as I might use a hammer.  My body is an organism, indeed an 
animal, but I am not myself an organism or animal.  Thus, what Bob is is 
my  body:  an  animal  that  I  own.    This  dualistic  view  has  various 
paradoxical consequences.  My wife has never kissed me–she has only 
kissed Bob, my body.  You cannot touch me–you can only touch Bob.  
Likewise, rape is then a mere property crime. Making philosophical sense 
of the meaning of sexuality is a lost cause: two persons’ having sexual 
intercourse is nothing but the intercourse between the animals owned by 
each of the persons.  My body is simply my property, and so stealing one 
of my kidneys is a mere property crime–it is not stealing a part of me.  
These consequences are ethically unacceptable.  After all, the 
government can morally take away some of my property for the greater 
good  and  does  so  in  taxes.    If  my  body  were  mere  property,  then  the 
government would in principle have a right, when necessary, to extract a 
kidney  from  me  as  a  tax  payment.    Finally,  if  this  is  right,  then  the 
traditional rallying cry of abortion supporters that “it’s my body” is no 
different in principle from the silly argument that I can do whatever I like 
in my house because my house is my property. 

There is too much absurdity there, and so this Cartesian view fails.  

But  even  if  it  did  not  fail,  it  could  only  be  used  by  the  proponent  of 
abortion if he had good reason to deny that the soul substance was united 
with the embryo from conception–otherwise, the safer thing is to refrain 
from killing what might be I.  But since the soul substance is 
unobservable, no such grounds are possible, apart from revelation-based 
religious  arguments,  and  those  should  not  be  brought  into  a  secular 
societal context. 

The arguments against the Cartesian view are not arguments against 
the existence of a soul.  The Cartesian view that the soul is a separate 
substance, distinct from the body, is not the only view of the soul.  The 
Aristotelian or Thomistic view is that the soul is that which makes an 
organism to be the organism it is and to develop as it does.  Thus, the 
soul is not something over and beyond the organism–it constitutes the 
organism  as  what  it  is,  and  what  we  are  are  organisms,  organisms 
constituted by our souls.  Thus, as soon as there is a unitary organism, 
there  is  a  soul.    (Admittedly,  Aristotle  and  Thomas  believed  that  the 
conceptus  did  not  have  the same kind of soul that I do–but they were 
theorizing  in  the  absence  of  empirical  evidence  about  the  conceptus 
being an animal that continuously grows and develops into me, or else 
they were going against what they should have said by their own lights.)  
The Cartesian view is rather unpopular these days in secular circles. 
 But there is a secular version of it, that replaces body-soul duality with 
body-brain  duality:  I  am  not  my  body  and  I  am  not  an animal–I am a 
brain.  This kind of a view will not help the abortion supporter all that 
much since the brain develops relatively early in pregnancy—around six 
weeks after conception.  But in fact the most trenchant objections against 
the  “I  am  a  soul”  view  can  be  made  against  the  “I  am  a  brain”  view.  
Only in the course of brain surgery can my wife kiss me if I am a brain.  
Rape, still, is only a property crime.  My kidneys are not parts of me but 
mere  property,  and  hence  can  be  expropriated  by  the  government  if 
necessary. 

And there is a further objection.  My brain developed out of earlier 
cells guided by the genetic information already present in the embryo. 
There was, first, a neural tube, and earlier there were precursors to that.  
Brain development was gradual, cells specializing more and more and 
arranging themselves.  At which point did I come to exist?  And why 
should the cells that were the precursors of the brain cells not be counted 
as having been the same organ as the brain, albeit in inchoate form?  If 
so, then perhaps I was there from conception, even on this view. 

The third response to my argument is that I am not my body or my 
brain, but what I am is my body’s intellectual functioning.  This response 
requires a metaphysical answer.  On this view, I do not think.  Rather, I 
am nothing else than thought itself, or more precisely, I am nothing else 
than a process of thinking.  We would do well to reject this view just 
because it contradicts the commonsensical fact that we think.  But we can 
also reject this view for a deeper reason.  If I am a particular process of 
thought, then it follows that if that process of thought were not to have 
occurred, I would not have existed.  Thus, when asleep, I do not exist.  
Moreover, were I not to have engaged in the processes of thought that I 
have engaged in over my lifetime, but instead were I to have engaged in 
different processes of thought, then I would not have existed–there would 
then have been a different process of thought, and hence someone else, if 
what I am is the process of thought that I am.  It follows that we cannot 
think otherwise than we do because our very identity is defined by the 
process of thought we engage in.  This fatalism, this deprivation of free 
will, is unacceptable. 

As I said, there are views of who I am that compete with the view 
that I am an animal and that are not the same as these three, but they tend 
to be variations of these three.  For instance, some think that what I am is 
a whole made up of two parts, a Cartesian soul and a body-animal.  This 
view  is  open  to  the  simple  objection  that  two  interacting  parts  do  not 
automatically make for a whole.  Moreover, there is the objection that 
surely I think, and yet my soul thinks, and since I am not a part of me, it 
follows absurdly that there are two thinkers here: I and my soul. 

We see thus that I am Bob.  I was once an embryo and a fetus.  The 
embryo or fetus that was there was just I–in an earlier stage of my life.  
This completes the first and hardest step of the argument. 

An  objection.    In  the  first  two  weeks  or  so  after  conception,  the 
blastocyst was not an individual, and hence in particular is not the same 
individual as I am, because it was capable of twinning–of splitting into 
two or more individuals–which in fact it does in about once every 260 
cases.  While what is normally called “abortion” is not likely to be done 
at  this  time  since  the  woman  at  this  time  rarely  knows  herself  to  be 
pregnant,  nonetheless  there  are  abortifacients  that  act  this  early–for 
instance  the  IUD,  Emergency  Birth  Control  or  the  Pill  in  those  cases 
where these act through an abortifacient effect–and hence the question is 
not merely of theoretical interest. 

This objection rests on the false principle that if it is merely possible 
that an organism will split in the future, then we do not have a genuine 
individual on the scene.  But this is plainly false: amoebae are certainly 
individuals,  but  they  are  capable  of  splitting.    What  happens  to  the 
individuality  when  they  split  is  disputed  by  philosophers.    One  might 
hold that the old amoeba continues existing as one of the two new ones, 
but  we  simply  do  not  know  which  one.    Or  one  might  hold,  more 
plausibly, that the old perishes and a new one comes to be in its place.  In 
the latter case, if I had had an identical twin, then I would have come to 
exist about two weeks after conception, not at conception, and the human 
being who came to exist at conception would no longer be alive. 

But if we have an amoeba  in front of us for a period of time during 
which it does not split, then it is the same amoeba, the same organism, 
over all of this time.  This judgment is unaffected even should we learn 
that the amoeba could have split during this period of time, just as our 
judgment that someone is alive is unaffected by learning that she could 
have died yesterday.  As long as the amoeba does not in fact split, it is 
one and the same individual as we had on the scene earlier. 

One might argue that if one could know in the first two weeks that 
twinning  was  going  to  occur,  then  one  would  thereby  know  that  the 
conceived  embryo  would  cease  to  exist  at  two  weeks  of  age,  and  one 
could abort it earlier, since one would not be depriving it of a long and 
meaningful  life.    Whether  this  argument  is  correct  or  not–and  I  am 
inclined to think it is not, since I think how good the life that one is being 
deprived of should not affect whether it is wrong for someone deprive 
one  of  it–it  does  not  matter  in  practice.    We  just  cannot  tell  at  the 
moment.  And as in 259 out of 260 cases twinning will not occur, one 
needs to act on the presumption that it will not in fact occur. 

\section{2. IF I WAS A FETUS, IT WOULD HAVE BEEN WRONG TO KILL THAT FETUS}

There  are  several  paths  to  the  conclusion  of  the  second  part  of  the 
argument, that if I was once a fetus (or an embryo for that matter), then it 
would  have  been  wrong  to  kill  that  fetus,  under  exactly  the  same 
circumstances under which it would be wrong to kill me now. 

The most powerful argument is to look at what is wrong with killing 
me  now.    Killing  me  now  is  a  paradigmatic  crime-with-a-victim,  the 
victim being me.  What would make killing me now wrong is the harm it 
would do to me: it would deprive me, who am juridically innocent, of 
life, indeed of the rest of my life.  Now, consider the hypothetical killing 
of the fetus that I once was.  This killing would have exactly the same 
victim as killing me now would.  Moreover, the harm inflicted on the 
victim  would  have  been  strictly  greater,  in  the  sense  that  any  harm 
inflicted on me by killing me now would likewise have been inflicted on 
me by killing me when I was a child.  I am now 29 years old.  Suppose 
that left to nature’s resources, I would die at 65.  Then, killing me now 
would deprive me of years 29 through 65 of my life.  However, killing 
me when I was a fetus would also deprive me of years 29 through 65 of 
my  life–as  well  as  the  years  from  the  moment  of  the killing up to 29.  
Given that murder is a crime whose wrongness comes from the harm to 
the  victim,  it  is  clear  that  when  the  victim  is  the  same,  and  the  harm 
greater, killing is if anything more wrong. 

Of course, there may be circumstances in which it is acceptable to 
kill  me  now.    It  might  be  that  under  some  circumstances  capital 
punishment is justified.  If so, then it might be acceptable to kill the fetus 
under  the  same  circumstances.    However,  it  is  also  clear  that  the 
circumstances involved in capital punishment do not apply in the case of 
the fetus.  Whether there are any other circumstances in which it would 
be acceptable to kill me now is a question that is beyond the scope of this 
paper, although I believe that the answer is basically negative.ii  In any 
case, we see that the wrongfulness of killing me when I was a fetus is at 
least as great as the wrongfulness of killing me now in relevantly similar 
circumstances.  Thus, my moral status when I was a fetus with respect to 
being killed is the same, or more favorable to me than, my status now. 

The reason for the “more favorable than now” option is that we have
an intuition that it is particularly wrong to kill people earlier.  Although 
there may be no duty thus to sacrifice one’s life, we see nothing irrational 
in an older person sacrificing his life for a younger on the grounds that 
the older has literally less to lose by death.  When I was a fetus, I had 
more to lose by death than I do now.  Thus, to have killed me then would, 
strictly speaking, have been to inflict a greater harm. 

Observe that nothing is said here about whether I was a person when 
I was a fetus.  That issue is irrelevant.  Whether I was a person then or 
not,  killing  me  would  have  had  the  same  victim  and  involved  greater 
harm as killing me now.  Observe that if I was not a person when I was a 
fetus, then the harm in killing me then would have been even greater than 
if I was a person then.  For killing me when I was not a person would 
thus have deprived me of all of my personhood as lived out on earth, and 
this radical deprivation would have been a greater crime than killing me 
now which would not deprive me of ever having had a personhood lived 
out on earth. 

That  said,  an  independent  argument  shows  that  in  fact  I  was  a 
person when I was a fetus.  This gives a second argument for why killing 
a fetus is wrong, and it is the only argument I give that depends on issues 
of  personhood.    The argument turns on the metaphysical notion of an 
“essential  property.”    The  essential  property  of  a  being  is  a  property 
which that being cannot lack as long as that being exists.  For instance, 
many philosophers think that being a horse is an essential property of a 
horse.    If  you  take  a  horse  like  Silver  Blaze  and  modify  it  to  such  a 
degree that it is no longer a horse, Silver Blaze will cease to exist and 
something  else  will  come  to  exist  in  his  place.    Being  material  is  an 
essential property of a rock: it could not exist without being material. 
Now,  it  is  likewise  plausible  that  being  a  person  is  an  essential 
property of every person.  If someone were a person and if personhood 
were removed from her, she would cease to exist.  If this is correct, then 
the fetus that I was truly was a person since I am a person.  If the fetus 
that I was were not a person, then it would be the case that I could have 
existed without being a person–which is impossible. 

Even  more  plausibly,  it  is  an  essential  property  of  me  to  have  a 
property that I will call human dignity.  Human dignity is a property of
me that makes it wrong for another human being to set out to kill meiii 
when I am juridically innocent.  As before, I leave capital punishment as 
an open question.  Human dignity is an essential property: it is part of the 
essence of who I am. Were I to lack this intrinsic dignity, I would not be 
myself; I would not exist.  But if human dignity understood in this way is 
an essential property and I have it, then the fetus that I was also had it–
otherwise it wouldn’t be an essential property. 

Finally, there is a very different argument for the wrongfulness of 
killing  the fetus that I was, based on John Rawls’s concept of justice.  
Even though I take this concept to be incorrect, the more bases on which 
our argument can rest, the better the argument. Rawls bids us to imagine 
that we do not know which role in society we fill–imagining this is called 
entering under the “veil of ignorance.”  What kind of a society I would I 
come  up  with,  and  what  kinds  of  rules  would  I  rationally  devise  on 
selfish grounds, if I did not know which role in this society I am going to 
live in?  Rawls says that that kind of society is the just society, and its 
rules are the rules of justice.  In such a society, for instance, we would 
forbid racism because under the veil of ignorance we would not know 
whether we would end up having the role of victim or inflicter of racism, 
and we would not want to take the risk of being on the victim.  Likewise, 
we would prohibit the murder of adults. 

Would we forbid the killing of fetuses?  This question depends on 
just how much we are to be ignorant of under the veil of ignorance.  If we 
know  that  we  are  not  fetuses,  then  we  might  not  forbid  the  killing  of 
fetuses when it is convenient to non-fetuses because we would have no 
selfish  reason  to  prohibit  it.    So,  is  the  fact  of  us  not  being  fetuses 
something that is under the veil of ignorance or not?  Well, we must be 
careful  not to take too much out from under the veil.  For instance, if 
racism is to end up being deemed unjust, our race must lie under the veil. 
Moreover,  even  our  being  conscious  must  fall  under  the  veil–thereby 
showing how much the veil is just a figure of speech since we cannot 
really be ignorant of our consciousness.  The reason our being conscious 
must fall under the veil is that otherwise we might well enact that it is 
right to kill the unconscious for the sake of the conscious–to use the man 
in a coma for medical experiments, say.  But at the same time, we cannot
put  too  much  under  the  veil.    We  had  better  have  an  awareness  of 
ourselves  as  human  since  otherwise  our  “just  society”  will  end  up 
prohibiting  all  killing  of  animals,  and  this  would  make  even  most 
vegetarian  farming  wrong  because  of  the  moles  and  voles  and  other 
animals killed in the process of farming, as someone has once argued. 

So,  where  do  we  draw  the  line?    I  would  propose  this  simple 
criterion.  Under the veil, we are aware of which social roles it would be 
logically possible for us to fill, but not aware which of those roles we do 
in fact fill.  It would not be logically possible for me to fill the role of a 
mole in the ground–I would not be myself then.  So I know, even under 
the veil, that I am not a mole.  However, it plainly is logically possible for 
me to fill the role of a fetus–it is possible because I did fill the role of a 
fetus once!   Thus, whether I am a fetus or not is something that must fall 
under the veil of ignorance, and hence the killing of fetuses will end up 
being  prohibited  in  exactly  the  same  way  as  that  of  adults:  we  just 
wouldn’t want to take the risk that we might end up being a fetus that is 
being killed.  Hence, justice requires a prohibition on killing fetuses in 
exactly the way in which it requires a prohibition on killing adults. 

Later,  Rawls  modified  his  criterion  by  talking  of  an  unselfish 
caretaker  for  someone  making  the  decision  under  a  veil  of  ignorance 
about what role her charge would fill.  This takes care of the problem that 
we can hardly be ignorant of whether we are conscious, while a caretaker 
can be ignorant of whether her charge is conscious.  But it does not affect 
the  rest  of  the  argument.    The  caretaker  needs  to be ignorant of some 
properties of her charge, such as the charge’s profession in life, but not of 
others, such as that her charge is not an insect.  Again, I would suggest 
that  a  natural  way  to  draw  the  line  is  apt  to  make  the  caretaker  be 
ignorant of whether her charge is a fetus or not.  For at the very least the 
caretaker should be ignorant as to which of the roles her charge could fill 
she  in  fact  fills,  and  certainly  her  charge  could  fill the role of a fetus.  
And if that were so, the caretaker, truly loving whoever is entrusted into 
her care, would not want to take the risk of enacting a system whereby 
her charge could be killed. 

\section{3. IF IT WAS WRONG TO KILL ME WHEN I WAS A FETUS, IT WAS WRONG TO KILL ANYONE WHEN HE IS A FETUS} 
If you cut me, do I bleed any more than the next guy?  No.  I was not and 
am  not  special.    If  it  was  wrong  to  kill  me  when  I was a fetus, it was 
likewise wrong to kill anyone else when he was a fetus.  

It might be argued that there are some special differences between 
the fetus that I was, which we have seen it would have been wrong to 
  kill  me  when  I was a fetus, it was 
likewise wrong to kill anyone else when he was a fetus.  
It might be argued that there are some special differences between 
the fetus that I was, which we have seen it would have been wrong to 
kill, and some other fetuses.  For instance, I was wanted.  But that I was 
wanted  did  not  anywhere  enter  into  my  arguments  against  killing  me 
when I was a fetus.  It is wrong to kill me now no matter whether I am 
wanted  by  others  or  not.    Killing  me  earlier,  I  have  argued,  is  not 
significantly different from killing me now, and so whether I was wanted 
or not is irrelevant. 

A  different  objection  would  be  that,  as  far  as  I  know,  I  did  not 
endanger my mother’s life.  However, my arguments would continue to 
apply even if I did: the fetus needs to be protected at least to the extent to 
which we would protect an adult under relevantly similar circumstances.  
If  the  fetus  endangers  the  mother’s  life,  it  does  so    unintentionally.  
Whether  it  is  acceptable  to  kill  the  fetus  under  those  circumstances 
depends  on  whether  it  would  be  acceptable  to  kill  me  now  were  I  to 
endanger my mother’s life unintentionally. As I announced, the aim of 
this paper is limited: it is to argue that killing fetuses is wrong under the 
same circumstances under which it is wrong to kill adults, but it is not the 
paper of the paper to discuss the circumstances, if any, under which it is 
permissible to kill adults.  I think it would not be acceptable to kill me 
were I endangering my mother’s life unintentionally: I will simply say in 
support  of  this  that  were  I  alone  in  a  space  capsule,  three  days  from 
rescue, with my mother, with only enough air for 1.5 days each, it would 
not be acceptable for my mother or her agent to kill me. 

A yet different objection is: I was a healthy fetus, but some others 
are not.  The wrong in killing me when I was a fetus would have been 
depriving me of a meaningful and long future life.  But what if the fetus 
cannot be expected to have such a life?  Again, I respond that the purpose 
of this paper is limited: I am not going to settle issues of euthanasia here. 
 It is acceptable to kill such a fetus only if it is acceptable to kill an adult 
 who  cannot  be  expected  to  have  a  meaningful  and  long  future  life.  
Again, I think it is not acceptable to kill an adult under such 
circumstances.    Human  life  is  intrinsically  worthwhile  and  always 
meaningful.  But this is not a paper about euthanasia.  If I have shown 
that  the  fetus  is  worthy  of  at  least  the  same  respect  as  an  adult  in 
comparable circumstances, I have done my task.