\chapter{The Mind-Body Problem by Tim Crane}\autocite{Crane1}
\label{mindbodyprob}
\newcounter{ff}
\setcounter{ff}{\thefootnote}
\setcounter{footnote}{0}
The mind-body problem is the problem of explaining how our mental states, events
and processes—like beliefs, actions and thinking—are related to the physical states,
events and processes in our bodies. A question of the form, ‘how is A related to B?’
does not by itself pose a philosophical problem. To pose such a problem, there has to
be something about A and B which makes the relation between them seem
problematic. Many features of mind and body have been cited as responsible for our
sense of the problem. Here I will concentrate on two: the fact that mind and body
seem to interact causally, and the distinctive features of consciousness.

A long tradition in philosophy has held, with René Descartes, that the mind
must be a non-bodily entity: a soul or mental substance. This thesis is called
‘substance dualism’ (or ‘Cartesian dualism’) because it says that there are two kinds
of substance in the world, mental and physical or material. One reason for believing
this is the belief that the soul, unlike the body, is immortal. Another reason for
believing it is that we have free will, and this seems to require that the mind is a
non-physical thing, since all physical things are subject to the laws of nature.

To say that the mind (or soul) is a mental substance is not to say that the mind is
made up of some non-physical kind of stuff or material. The use of the term
‘substance’ is rather the traditional philosophical use: a substance is an entity which
has properties and persists through change in its properties. A tiger, for instance, is a
substance, whereas a hurricane is not. To say that there are mental substances—
individual minds or souls—is to say that there are objects which are non-material or
non-physical, and these objects can exist independently of physical objects, like a
person’s body. These objects, if they exist, are not made of non-physical ‘stuff’: they
are not made of ‘stuff’ at all.

But if there are such objects, then how do they interact with physical objects?
Our thoughts and other mental states often seem to be caused by events in the world
external to our minds, and our thoughts and intentions seem to make our bodies
move. A perception of a glass of wine can be caused by the presence of a glass of
wine in front of me, and my desire for some wine plus the belief that there is a glass
of wine in front of me can cause me to reach towards the glass. But many think that
all physical effects are brought about by purely physical causes: physical states of
my brain are enough to cause the physical event of my reaching towards the glass.
So how can my mental states play any causal role in bringing about my actions?

Some dualists react to this by denying that such psychophysical causation really
exists (this view is called ‘epiphenomenalism’). Some philosophers have thought
that mental states are causally related only to other mental states, and physical states
are causally related only to other physical states: the mental and physical realms
operate independently. This ‘parallelist’ view has been unpopular in the 20th
century, as have most dualist views. For if we find dualism unsatisfactory, there is
another way to answer the question of psychophysical causation: we can say that
mental states have effects in the physical world precisely because they are, contrary
to appearances, physical states\autocite{Lewis1}. This is a *monist* view, since it
holds that there is *one* kind of substance, physical or material subtance. Therefore
it is known as ‘physicalism’ or ‘materialism’.

Physicalism comes in many forms. The strongest form is the form just
mentioned, which holds that mental *states* or *properties* are identical with
physical states or properties. This view, sometimes called the ‘type-identity theory’,
is considered an empirical hypothesis, awaiting confirmation by science. The model
for such an identity theory is the identification of properties such as the heat of a gas
with the mean kinetic energy of its constituent molecules. Since such an
identification is often described as part of the *reduction* of thermodynamics to
statistical mechanics, the parallel claim about the mental is often called a ‘reductive’
theory of mind, or ‘reductive physicalism’\autocite{Lewis2}.

Many philosophers find reductive physicalism an excessively bold empirical
speculation. For it seems committed to the implausible claim that all creatures who
believe that grass is green have one physical property in common—the property
which is identical to the belief that grass is green. For this reason (and others) some
physicalists adopt a weaker version of physicalism which does not have this
consequence. This version of physicalism holds that all particular objects and events
are physical, but allows that there are mental properties which are not identical to
physical properties. (Davidson\autocite{Davidson1} is one inspiration for such views.) This kind of
view, ‘non-reductive physicalism’, is a kind of dualism, since it holds there are two
kinds of property, mental and physical. But it is not *substance* dualism, since it
holds that all substances are physical substances.

Non-reductive physicalism is also sometimes called a ‘token-identity theory’
since it identifies mental and physical particulars or tokens, and it is invariably
supplemented by the claim that mental properties *supervene* on physical
properties. Though the notion can be refined in many ways, supervenience is
essentially a claim about the dependence of the mental on the physical: there can be
no difference in mental facts without a difference in some physical facts (see Kim\autocite{Kim1}; Horgan\autocite{Horgan1}).

If the problem of psychophysical causation was the whole of the mind-body
problem, then it might seem that physicalism is a straightforward solution to that
problem. If the only question is, ‘how do mental states have effects in the physical
world?’, then it seems that the physicalist can answer this by saying that mental
states are identical with physical states.

But there is a complication here. For it seems that physicalists can only propose
this solution to the problem of psychophysical causation if mental causes are
identical with physical causes. Yet if properties or states are causes, as many
reductive physicalists assume, then non-reductive physicalists are not entitled to this
solution, since they do not identify mental and physical properties. This is the problem of mental causation for non-reductive physicalists. (See Davidson\autocite{Davidson2},
Crane\autocite{Crane2}, Jackson\autocite{Jackson1}).

However, even if the physicalist can solve this problem of mental causation,
there is a deeper reason why there is more to the mind-body problem than the
problem of psychophysical interaction. The reason is that, according to many
philosophers, physicalism is not the *solution* to the mind-body problem, but
something which gives rise to a version of that problem. They reason as follows: we
know enough to know that the world is completely physical. So if the mind exists, it
too must be physical. However, it seems hard to understand how certain aspects of
mind—notably consciousness—could just be physical features of the brain. How can
the complex subjectivity of a conscious experience be produced by the grey matter
of the brain? As McGinn\autocite{McGinn1} puts it, neurones and synapses seem ‘the wrong
kind’ of material to produce consciousness. The problem here is one of intelligibility:
we know that the mental is physical, so consciousness must have its origins in the
brain; but how can we make sense of this mysterious fact?

Thomas Nagel dramatised this in a famous paper\autocite{Nagel1}. Nagel says that
when a creature is conscious, there is something it is *like* to be that creature: there
is something it is like to be a bat, but there is nothing it is like to be a stone. The heart
of the mind-body problem for Nagel is the apparent fact that we cannot understand
*how* consciousness can just be a physical property of the brain, even though we
know that in some sense physicalism is true (see also Chalmers\autocite{Chalmers1}).

Some physicalists respond by saying that this problem is illusory: if physicalism
*is* true, then consciousness is just a physical property, and it simply begs the
question against physicalism to wonder whether this *can* be true (see Lewis\autocite{Lewis3}).
But Nagel’s criticism can be sharpened, as it has been by what Frank Jackson calls
the ‘knowledge argument’ (Jackson\autocite{Jackson2}; see also Robinson\autocite{Robinson1}). Jackson argues
that even if we knew all the physical facts about, say, pain, we would not ipso facto
know what it is like to be in pain. Someone omniscient about the physical facts
about pain would learn something new when they learn what it is like to be in pain.

Therefore there is some knowledge—knowledge of what it is like—which is not
knowledge of any physical fact. So not all facts are physical facts. (For physicalist
responses to Jackson’s argument see Lewis\autocite{Lewis4}; Dennett\autocite{Dennett1}; Churchland\autocite{Churchland1}.)

In late twentieth century philosophy of mind, discussions of the mind-body
problem revolve around the twin poles of the problem of psychophysical causation
and the problem of consciousness. And while it is possible to see these as
independent problems, there is nonetheless a link between them, which can be
expressed as a dilemma: if the mental is not physical, then how we make sense of its
causal interaction with the physical? But if it is physical, how can we make sense of
the phenomena of consciousness? These two questions, in effect, define the
contemporary debate on the mind-body problem.
\setcounter{footnote}{\theff}