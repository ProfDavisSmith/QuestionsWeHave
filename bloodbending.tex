\chapter{The Moral Status of Bloodbending by Davis Smith}
\label{bloodbending}
\section{Introduction}
\newcounter{fd}
\setcounter{fd}{\thefootnote}
\setcounter{footnote}{0}

During their travels around the world of Avatar, our heroes have encountered many people with strange or unique bending abilities. For example, they encountered Combustion Man, who could make things explode with his mind.\footnote{First seen in “The Headband,” (23:24-23:48) but his abilities are displayed for the first time in “The Beach” (15:21-16:16).} None, however, are more troubling and ethically interesting than Hama, a waterbender from the Southern Tribe who can, on a full moon night, bend the very blood in a person’s veins and manipulate their body to do her bidding.\autocite{puppetmaster} Such a violation of a person’s control over themselves would, rightfully, be the stuff of nightmares. However, are all cases of this bloodbending wrong? To answer this question, we turn to two well-established theories in Ethics, an area of Philosophy which, among other things, tries to answer the question “what is the moral thing to do?” Almost as if by design, bloodbending works as a fantastic example of how these theories determine the morality of an action and how they can come to radically different conclusions on a single case.

The episode gives us four different situations in which bloodbending is used and each can serve as examples for the different ways one can evaluate an action to determine its morality. In all four cases, the bender is forcing a person to do something against their will. So, if our moral intuition or the theory gives different responses to the question “was that the right thing to do?”, then it is not the manipulation of a person which makes that difference. Because the cases are so different, if an ethical theory gives the same response to all four cases, then the manipulation is the cause.

\section{The Ethical Theories}

To start us off, I would like to give a basic overview of the different ways in which one could classify ethical theories, theories about whether some action is right or wrong. First, we can classify them according to what they use to make their determination. For this, there are two natural categories. On one side, we have theories which make their moral evaluation according to the consequences, the results of the action. These are called Consequentialist theories. The ultimate command for a Consequentialist theory is to leave the world a better place. Theories in this family differ in how they determine ‘better’ and in how they measure it, but they all share that primary command. On the other side, we have Non-Consequentialist theories. As the name implies, these theories are the opposite of the Consequentialist ones. They hold that the consequences are irrelevant to the morality of an action. Some actions, according to these theories, are just wrong, regardless of the outcome.

We could also classify them according to whether they are concerned with individual instances of an action or kinds of actions. Act-Type theories start by asking about the kind of action it is. These theories tend to move from general principles or commands about categories of actions and use those to determine moral status of an individual case. For example, one could have the general principle “lying is always wrong” or the command “don’t lie” and from that determine whether some case of lying is wrong (it is, according to that principle). On the other hand, some theorists think that this approach is too impersonal and not down-to-Earth. This is where we get the Act-Token theories. Rather than moving from some general principle to make a judgement, Act-Token theories make their evaluations by just looking at the case in isolation. These theories have no problem claiming that most of the time lying is wrong, but they are more than willing to make exceptions and give reasons for those exceptions.

These two different ways of classifying ethical theories gives us four different, generic, types of theories. In this paper, we will be focusing in on the two major ones, Utilitarianism (a Consequentialist Act-Token theory) and Kantianism (a Non-Consequentialist Act-Type theory). Because of their radically different stances regarding the measurement of morality and what is to be measured, these theories serve as a wonderful foundation to explore the moral status of bloodbending.

\section{The Utilitarian Account} 

Jeremy Bentham (1748-1832)\autocite{Bentham1}, John Stuart Mill (1806-1873)\autocite{Mill1}, and Harriot Taylor Mill (1807-1858)\footnote{Harriot Taylor Mill’s husband was J.S. Mill and by his own words, she was heavily influential in his formulations and refinements of Utilitarianism.} built off each other and formalized a theory of morality called Utilitarianism. This account of the permissibility of actions is Consequentialist in nature and, though it is possible for this account to use act-types, Mill and Bentham both certainly preferred focusing on act-tokens. These two features, almost immediately, give us a possible answer to the question “is bloodbending always wrong?”, namely, “no.” In addition to this, when we think about the more general question about whether it is permissible to violate a person’s autonomy, this theory would say “sometimes.” In fact, because it is an Act-Token theory, it will always allow for some exceptions to this kind of general rule.

There is one principle, one overarching metric, according to Utilitarianism, which cannot have exceptions, and which determines the morality of an action; The Principle of Utility.\autocite[pg.12]{Bentham1} This says that the action you should do is the one which results in the greatest amount of happiness (or well-being) once the suffering (sadness, pain) has been subtracted from it. It should be noted, however, that the doer’s interests and the effects on them are not treated as especially important. Your well-being is treated equally to all other human (and potentially non-human) souls. It is part of the core command of this theory that you think of others and ensure that, all else being equal, their lives are made better by your actions. Morally speaking, this theory has it that all beings are equal in so far as their ability to suffer or experience happiness is equal.\autocite[pg.245]{Bentham1}

The world of Avatar has no shortage of examples of Utilitarian style thinking, seemingly, getting morality correct, outside of cases of bloodbending. To start off, in Book 1, Episode 5, we first encounter one of my favorite side characters, the Cabbage Merchant. As he is attempting to enter the city of Omashu, we see the guards, without regard for the merchant’s desires or the effects of having a cabbage merchant in the city, use earthbending to launch his cart into a canyon.\autocite[02:53-03:07]{Omashu1} Given what we know about the Cabbage Merchant and the results of this, we can say that the guards were wrong in launching the cart. In that same episode, we find Team Avatar preparing to use the delivery system of the city as a slide, for their amusement. In doing so, they cause a lot of damage to various people’s property and they cause even further distress to the Cabbage Merchant.\autocite[04:54-07:29]{Omashu1} This destruction and the suffering, for lack of a better word, caused certainly outweighs the enjoyment had by the team. It follows from this that Utilitarianism would say that what they did was wrong. 

In Book 1, Episode 11, we encounter rival tribes (with radically different personalities) each seeking passage through a dangerous canyon.\autocite[00:00-23:37]{Divide1} The guide warns them that bringing food into the canyon will attract the carnivorous and dangerous animals. Both tribes end up bringing food into the canyon, believing that the other will. Given the risk that they are putting the people, collectively, in by doing so, we can see that this, like before, outweighs the benefit of not going a day (or so) without food. Utilitarianism, as a result, says that the tribes did something wrong here.  To finish off our set of examples, we turn to Book 2, Episode 2, where Iroh is deciding about whether to make tea out of a strange plant.\autocite[03:14-03:53, 06:11-07:06]{Lovers1} As he put it, this plant is either “delectable tea or deadly poison.” If we suppose that it is a 50-50 shot, we need to ask whether the temporary enjoyment from such a “heartbreaking” tea is equal to the suffering caused by his death. Since we know the negatives which would come from Iroh’s death, we can say rather confidently, using Utilitarian reasoning, that Iroh making the tea was not worth the risk.

\section{The Kantian Account}

Some people do not like this Utilitarian account. They think that it is missing something fundamental to the morality of actions. For example, Utilitarianism does not have a respect for personhood built in. Showing a person respect and dignity, for a Utilitarian, is seen as secondary. If you can get the best results while being respectful, do so, but the main driving force is to make the world a better place. This is where we get Kantianism, named after its inventor Immanuel Kant (1724-1804).\autocite{Kant1} Kant was a Non-consequentialist, meaning that he held that the consequences have no bearing on the morality of actions. Kant’s ethical theory, also, is based around act-types rather than tokens. This puts Kantianism in direct opposition to Utilitarianism.

For Kant, and the Kantians, there are many moral imperatives, commands, which we need to follow to in order to act morally. These imperatives act categorically according to the type of action it is. For example, you have simple commands like “don’t make false statements,” “don’t cheat,” and “don’t steal.” For this theory, we don’t need to wait until the dust settles to know whether we did the right thing, rather we just need to know what kind of action it was and what the imperative associated with it is.

Determining the imperative associated with a given act-type can be quite difficult. Luckily, however, according to Kant, they all reduce to one simple command. This is the Categorical Imperative.  Kant, himself, gives four different formulations of the Categorical Imperative and claims that they all mean essentially the same thing. There is no small debate about whether they do, but it is certainly clear that two of the four mean the same as the other two, so the debate is really about whether the two formulations get at the same thing. That being said, the most relevant formulation to our question about the moral status of bloodbending is the Formula from Humanity.\autocite[pg. 38]{Kant1} This imperative tells us to treat all of humanity, ourselves included, as an end in and of themselves and never merely as a means to an end. Within that formulation, there is a central concept which we need to be clear on before we can use the theory to evaluate bloodbending. This is the idea of using someone as a mere-means.

We all have goals and aspirations which we seek to accomplish in our lives. For Aang, it is to master all four elements and bring balance to the world. To accomplish these goals, we often need to use each other by getting help or through normal exchanges, such as Aang using Katara, Toph, and Zuko to learn the different styles of bending. In these cases, we are using them as a means to our end but we are not necessarily using them merely as a means. We take the other person’s goals and aspirations into account when we use them and, thereby, help them achieve their goals. Using someone merely as a means is a different matter entirely. In these cases, we do not take their goals into consideration, we trick, coerce, or force them to do something which they otherwise would not consent to. Kant believed that our autonomy (our free will) and our rationality were the most valuable things in the world. Any action which hindered or depreciated a person’s autonomy or rationality, regardless of the consequences, is wrong. In lying, for example, we are intentionally deceiving a rational autonomous agent for the betterment of ourselves or another. This is using them merely as a means because we are not appreciating their great worth.

As with the Utilitarian account, Avatar has several examples which we could apply Kantian thinking to in order to make a moral judgement. In the very first episode of the series, Aang lies about being the Avatar. In doing so, Aang is misinforming Katara and Sokka for his benefit, as he later admits.\autocite[11:31-11:52]{Iceberg1} In such a simple case, we can see that Kant would say that Aang did something wrong. In a similar situation, in Book 1, Episode 4, Team Avatar arrives at the island of Kyoshi, named after one of Aang’s past lives.\autocite[06:05-07:35]{Warriors1} There, Aang chooses to be honest about being the Avatar. Despite the consequences of this, namely Zuko finding out and going to the island, Kantianism says that Aang did the right thing, because him lying would have been wrong.

For other examples, in Book 1, Episode 9, Team Avatar encounters a band of pirates who have, through “high risk trading,” acquired a waterbending scroll, with forms and moves which Katara and Aang can learn to forward their quest to master the element.\autocite[07:04-10:00]{Scroll1} As the story progresses, we learn that Katara stole the scroll from the pirates. Even though the scroll was stolen by them, stealing from a thief is still stealing. In doing so, Katara used them as a mere means and violated the Categorical Imperative. This means that Kant would say that her theft was morally wrong. The guilt for the action is on Katara, just as the initial theft of the scroll by the pirates is on the pirates. Similarly, in Book 3, Episode 5, a few Fire Nation soldiers spot Team Avatar and the proceed to send a message to the Fire Lord.\autocite[03:46-04:20, 08:09-08:56]{Beach1} As the messenger hawk flies away, Combustion Man steals the scroll. Kant and those who think like him would say that Combustion Man did something wrong because he is stealing and the Categorical Imperative strictly forbids theft.

\section{The Theories Applied to Bloodbending}

As you may have guessed from my examples, both Kantianism and Utilitarianism truly shine when put to the test, used in cases, and the World of Avatar has no shortage of examples. For many of the moral cases in Avatar, Kantianism and Utilitarianism give remarkably different responses. In Book 2, Episode 3, in order to help the citizens of Omashu flee the Fire Nation invaders, Team Avatar and the populace use a small octopus-like creature to fake an epidemic, called ‘penta-pox’.\autocite[08:48-10:57]{Return1} Doing so was an affective and non-violent way to get out of the city and Utilitarianism would approve. Kantianism, on the other hand, holds that any form of deception, which is a kind of lying, is wrong. In Book 3, Episode 1, Team Avatar and their cohorts are on a stolen Fire Nation ship.\autocite[06:58-08:00]{Awakening1} The act of stealing this ship would be seen as fine according to Utilitarianism but be seen as morally wrong according to Kantianism. Similarly, in the same episode, they are impersonating Fire Nation soldiers.\autocite[08:46-10:00]{Awakening1} This is deceitful and is lying. Kantianism gives a clear and absolute prohibition to lying, meaning that they say this is wrong. But Utilitarians disagree. These fundamental disagreements become even more pronounced when we apply the theories to the four cases of bloodbending seen in The Puppetmaster.

The, chronologically, first case of bloodbending is Hama using it to escape from prison. Hama is in prison after being captured by the Fire Nation during one of their raids. Hama, feeling the power of the full moon, realizes that living creatures, like elephant-rats and humans, have blood in their veins and that blood has water. Using her waterbending, Hama learns to manipulate that blood in the elephant rats and use them as puppets. This is not, however, enough to make her escape. On another full moon, Hama uses bloodbending to force a guard to open her cell and thereby escape from the prison.\autocite[18:04-19:33]{puppetmaster} The moral question which this scene raises and which these theories need to answer is whether it was permissible for Hama to use her skill to escape.

Looking at this prison escape, we see that Utilitarianism says that Hama did the right thing. In fact, she should have continued to use bloodbending to free the other Southern Water Tribe benders as well (which she may have). Doing so, in this case, results in the best outcome given her options, freedom to pursue her happiness and (in freeing the others) their happiness and, according to Utilitarianism, the right action is the one with the best consequences given your options. Many of us will likely agree with this assessment. Morally speaking, it was wrong for the Fire Nation to imprison those benders in the first place, so it seems only right to use her skills to escape.

Kantianism, on the other hand, holds that Hama did something wrong, regardless of the consequences. Hama bloodbends the guard of the prison and thereby forces him to release her. In doing this, she is using the guard merely as a means to an end (namely, escape from prison). She does not care about the goals or aspirations of the guard, rather she is forcing him to do something which he would not do otherwise. Kant and those who think like him would likely say that there is nothing wrong with her escaping from prison, rather it was wrong for her to use bloodbending to get do it. This is in stark contrast to Utilitarianism, which says that the results are all that matter, the steps used to get there are not relevant (unless there is a better way). Kant does not think that the ends justify the means, how you achieve a goal is just as important, if not more important, than what you achieved.

For the next case, we find Hama, significantly older, settled down in a Fire Nation Village. She is still a strong waterbender and still knows how to bloodbend. On full moon nights, Hama uses this ability to force people to leave their homes and walk deep into the forest, where she has a cave and she chains them to the walls, making them her prisoners.\autocite[15:12-16:49, 17:10-18:03]{puppetmaster} Her motivation is vengeance, not just against the soldiers who imprisoned her or against the soldiers which raided her tribe, but rather against the entirety of the Fire Nation. In her mind, all Fire Nation citizens are guilty and worthy of her wrath. For this case, the questions which needs answering is whether it was permissible for Hama to use her skill to have vengeance in this way.

Utilitarianism would say that Hama kidnapping these citizens, regardless of the method used, is wrong. We likely will agree with this assessment too. Hama’s actions are certainly causing more harm than good. The people of the village are terrified and those who she as imprisoned are experiencing similar pain and suffering to her own imprisonment years before. There are very little, if any, positive results from her kidnapping of the villagers. Since the suffering outweighs the happiness caused, Hama’s actions are wrong, according to Utilitarianism. The Kantians agree with this assessment, but for different reasons. According to Kantianism, Hama is clearly compelling people to do something which they would not normally do, namely be chained up in a cave. She is not taking their goals and aspirations into account and is using them as a mere-means. We need to note that the pain, suffering, and emotional turmoil which Hama is causing does not enter into the Kantian framework. That would be consequentialist and Utilitarian thinking, which is not Kantian. Hama’s actions here are wrong solely because of the violation of their autonomy.  

In the third case, we have Hama and Katara using waterbending to fight each other. Katara, a skilled waterbender in her own right, uses some of the skills which she learned from Hama to pull water from the surrounding plant life. Hama switches tactics and uses bloodbending to take control of Aang and Sokka’s bodies, using them to fight for her against Katara. Hama then forces Sokka to raise his Space-sword and fly towards Aang.\autocite[22:08-22:53]{puppetmaster} If nothing is done, Aang will die. The question here is whether it was permissible for Hama to use bloodbending to force Sokka to attempt to kill Aang.

Utilitarianism gives a similar response to The Kidnapping, claiming that Hama was doing something wrong. This is because, had Hama succeeded, she would have killed the Avatar, which would have stopped him from defeating the Fire Lord and ending the war. Not only that, but it would have allowed the destruction of the Earth Kingdom at the Fire Lord’s hands, because Aang would not have been there to stop it. All of the pain and suffering which Aang will have prevented would equally fall on Hama’s shoulders. All of those extreme factors make for a very clear Utilitarian response; Hama was doing something wrong. This example, in particular, leads to a possible response. Hama was ignorant of these facts, does that enter into it? For Utilitarianism, a person’s knowledge does not change the morality of an action. The Principle of Utility does not have room for the state of mind the doer is in. If a person is ignorant of the potential consequences of their actions, we may hold them less responsible, we may place less blame on them for their wrongdoing, but that does not make what they did any less wrong.

For this case, Kant would agree with the Utilitarians and say that Hama was in the wrong. Again, different reasoning is used. Killing a person (either yourself or another), especially for your own benefit, is the ultimate case of using them as a mere-means. In doing so, you are removing their rationality and autonomy from the world and, at the same time, violating that autonomy because you are removing all of their abilities. This reasoning, again, is very different than the kind which the Utilitarian would use. The Utilitarian does not think that killing a person is always wrong, rather they work on a case-by-case basis. In choosing whether to kill the Fire Lord, the Utilitarian would likely say that killing him would be a better choice than merely removing his bending, as there would be less net suffering in the world because of it. The Kantian would disagree, killing a person is always wrong.

For the final case, we pick up right where the previous left off. Sokka, sword raised, is flying towards Aang. Katara is the only one who can save them, and she only has one option, use bloodbending on Hama to stop her. This would save our heroes and subdue Hama. Katara, realizing this, uses bloodbending for the first time and defeats Hama, who is captured by her former prisoners and taken away. As she leaves, Hama says “my work is done. Congratulations, Katara. You’re a bloodbender.”\autocite[22:53-23:43]{puppetmaster} The ethical question which this case gives is whether it is permissible to use bloodbending to save a life.

The Kantians and the Utilitarians, yet again, disagree. Utilitarianism has it that Katara did the right thing. If she had not bloodbended Hama, The Avatar would have died. All of the good which Aang would cause in the world would not happen. In fact, in a sense, all of those positives act in favor of saving him, by any means. In bloodbending Hama, Katara is partially responsible for all of the good which Aang would cause thereafter. All of those factors make the response pretty clear. Katara may feel bad and cry because of what she had to do to save Aang’s life, but those feelings are a drop in the ocean compared to the alternatives. For this case, unlike The Attempt to Kill Aang, Katara knew all of those factors. Though this does not change the actual moral rectitude of the action, it should change how much praise or blame we place on her. Katara knowing this makes it so that we should not blame her for bloodbending, in fact, we should thank and congratulate her for it. If Katara were a good Utilitarian, she would feel good for saving Aang’s life and not sadness for how she had to do it. Since Utilitarianism is a Consequentialist stance, it has that the ends justify the means. Bloodbending, in this case, is a means to saving the Avatar’s life, which more than justifies its use.

The Kantians, on the other hand, will say that Katara using bloodbending to save Aang is wrong. This case mirrors very closely a thought experiment which Kant himself encountered. Suppose that your roommate is in the shower and you hear a knock at the door. You open it to find an axe-murderer, weapon out, asking about the whereabouts of your roommate. You have a choice. You could lie to save your friend or you could tell them the truth. In telling them the truth, they will die at the hands of the axe-murderer; but, Kant argued, you cannot lie to them. If you lie to the axe-murderer, you are using them merely as a means to an end and the results of that lie are on you, morally speaking. Whereas, if you tell the truth, you are not using them merely as a means and the results of their actions are not on you.\autocite[346-350]{Beck1} \footnote{There are more than 23 cases of lying or deception in the three seasons of the show and all would be counted as immoral according to Kantianism. There are 23 cases in the first two seasons (books) alone.} This distinction could be generalized to a difference between killing and letting die. Killing a person is always wrong but letting a person to die at the hands of another (keeping your own hands clean, so to speak) is fine.\footnote{A disagreement about killing vs letting die is even in one of the final episodes of Avatar. In Sozin’s Comet Part 2, while Aang is contacting his past lives, he speaks with Avatar Kyoshi. Kyoshi describes her encounter with Chin the Conqueror, to which Aang replies “You didn't really kill Chin. Technically, he fell to his own doom because he was too stubborn to get out of the way.” And then Kyoshi says “Personally, I don't really see the difference, but I assure you, I would have done whatever it took to stop Chin.” Since Kyoshi does not see a difference between killing and letting die, we could, arguably, claim that she is more Utilitarian than Kantian. \cite[32:59-33:13]{Sozen1}} Katara’s use of bloodbending is very similar to the axe-murderer case. In violating Hama’s autonomy, the results of the action are on Katara and she did something morally wrong. If she had done nothing, on the other hand, the results of Aang’s death would not be on her, morally speaking and she would not have done something wrong. This, again, points to a radically different way of thinking about morality. The Utilitarian does not see a distinction between killing and letting a person die, because both have the same results. The Utilitarian, also, would say that the results of Aang’s death would, at least partially, be on Katara because she could have prevented it.

But, what do these theories say about violating a person’s autonomy? In all four cases, we have situations where a bender is forcing another to do something which they would not otherwise do. In all four cases, we see a person’s autonomy is being violated. Utilitarianism gives mixed results for these cases. As a result, this means that Utilitarianism gives us situations where a person’s autonomy should be violated. Of course, most of the time, we should not force people or coerce them into doing things that they would not otherwise want, but, according to this theory, there are always exceptions. According to Utilitarianism, you should violate another’s autonomy when your failure to do so would result in less than optimal results. Most of the time, violating a person’s autonomy is not the best course of action, but there are going to be cases where it is what you need to do.   

For all of these examples, Kantianism gives the same response, bloodbending is morally wrong. This means that the theory says that a person’s autonomy should never be violated, through manipulation or through bloodbending. Could there be a case where Kantianism says that it is permissible? Such a case would require a person to voluntarily, in good faith and without deception, wish to be bloodbended. Even then, there could be a sense in which that person, in acting on that wish, is using themselves merely as a means to an end.
\section{Conclusion}

In this paper, we have used the two major theories in Ethics to figure out the moral status of bloodbending. According to Utilitarianism, sometimes using the ability is wrong and other times it is the right thing to do. The evaluation has nothing to do with bloodbending itself, rather the judgement is made on the basis of the consequences of the action. The Utilitarian will claim that Hama escaping from prison was permissible, Hama kidnapping the villagers was wrong, Hama attempting to kill Aang was wrong, and Katara saving Aang’s life was permissible. The Kantian, on the other hand, will make their evaluation based on the nature of bloodbending itself, regardless of the consequences. They claim that the prison escape was wrong, the kidnapping was wrong, attempting to kill Aang was wrong, and Katara saving the Avatar was wrong. In each case, the bender is violating a person’s autonomy which is wrong.  So, what is the moral status of bloodbending? The answer to this question depends on the answer to a bigger question: Which theory is more accurate, Utilitarianism or Kantianism?

\setcounter{footnote}{\thefd}