\chapter*{Preface: Why Study Philosophy?}
Philosophy, put lightly, is thinking really hard about something. It is not just your world view or that of another, but it is also the mother of all other subjects. At its core, philosophy is critical thinking taken to a ridiculous extreme (you will read this again later on). The most famous for this is Socrates. Although philosophy has taken this system of reasoning to a formal extreme, in our day to day lives, we think and argue with each other using a less formal, but very similar system. Philosophers ask themselves and others perplexing questions which they want the answer to. 

There was a time, long ago, when all subjects were a branch of philosophy, stemming from very basic questions and seeking the answers to them. As time went on, philosophy splintered into various fields which we know today, taking on different standards for evidence and reasoning. Philosophy, as it is today, retains the highest standard for reasoning and argumentation for the stances. But, though the alternative standards, like those found in science, are practically valuable, they can't answer certain kinds of questions, which is where philosophy remains alive and well.  
\section{Questions Philosophers are Prone to Ask:}

Here are some examples of questions philosophers are prone to think about and ask (yes, headaches are common). We will be covering the introduction to many of these as the course progresses.
\begin{earg}
    \item[]What is the meaning of life? Is there one? 
    \item[]What am I? Am I some soul thing or just matter?
    \item[]What makes actions right or wrong? Is it relative or is there some hard and fast rule?
    \item[]Is there such a thing as God?
    \item[]How does time work? Is there such a thing as time?
    \item[]What is ‘free will’? Do we have it? Does free will even make sense?
    \item[]What is it to know something? How does that differ from opinion?
\end{earg}
We will touch lightly on aspects of the meaning of life in Module \ref{ch.modtwo}. Souls and other aspects are part of the Mind-Body Problem which is Module \ref{ch.modthree}. Whether morality is relative is in Module \ref{ch.modseven}. Moral rectitude is in Module \ref{ch.modeight}. Arguments for and against the existence of God is in Module \ref{ch.modfive}. The Free Will Debate is in Module \ref{ch.modFour}. And, finally, theories of knowledge are in Module \ref{ch.modsix}. 

There are other questions which we will ponder as the course progresses, such as the moral status of animals and those who haven't been born yet and various contemporary debates and issues (including the nature of justice and equality). I am willing to wager that at least one question or topic in this class will get you really interested and excited about Philosophy and make you want to take more courses on that subject. 
\section{Why Study Philosophy?}

Like I just said, philosophy is critical thinking taken to an extreme... But why should the average person be concerned with it? When a person wants to be able to lift heavier objects, they train by lifting weights. Training with heavier weights (eventually) makes the smaller amount that they need to carry easier. The mind is no different. Working with an extreme, outlandish, and hard question, knowing how to think about them and the rules for good critical thinking will make debate and reasoning in every day contexts easier.

There are other, potentially more practical or financial, reasons to study Philosophy. Many of you might think that the humanities, like English, Art, and Philosophy, aren't the best way to make a living. Your parents might say something like ``O, you want to get a major in Philosophy? Great, there's a new Philosophy factory opening up down the street!" in a sarcastic way. This common thinking is because many don't see the value in Philosophy. For example, in 2015, in his Presidential bid, Marco Rubio regularly said things along the lines of ``we need more welders, less philosophers!"\autocite{philervwelder2} He attempted to support this claim by implying or outright stating that welders make more money than those who got degrees in Philosophy.

Both of these claims are incorrect. On average, people with philosophy degrees make more money (yearly) than those who entered the welding trade.\autocite{weldersvphilers} When we compare the average pay for people with Philosophy degrees to other majors, we also see that those with this major make more money than any other humanities degree.\autocite{philermoney} Holders of this degree make more money than even the average business management degree holder and some STEM fields. 

But why is this? Getting a degree in Philosophy doesn't make you `pre-packaged' with the skills to enter a particular job (like going to a trade school), rather a degree in Philosophy and the classes which are entailed by it give you a more important, overarching set of skills, namely creative problem solving. Employers, the world over, are seeking those who can be given a problem and come up with creative or innovated ways of solving it. The courses you take in Philosophy give you significantly harder problems than those you would be presented with in those situations and this makes those problems so much easier for you to handle. Many famous CEOs and company founders got degrees in Philosophy. 
