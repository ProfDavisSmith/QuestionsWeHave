\chapter{Plato's Apology}
\label{platoapology}
I do not know, men of Athens, how my\marginpar{17}
accusers affected you; as for me, I was almost
carried away in spite of myself, so persuasively
did they speak. And yet, hardly anything of
what they said is true. Of the many lies they
told, one in particular surprised me, namely
that you should be careful not to be deceived
by an accomplished speaker like me. That they
were not ashamed to be immediately proved\marginpar{b}
wrong by the facts, when I show myself not to
be an accomplished speaker at all, that I
thought was most shameless on their
part—unless indeed they call an accomplished
speaker the man who speaks the truth. If they
mean that, I would agree that I am an orator,
but not after their manner, for indeed, as I say, practically nothing they said was true. From me you\marginpar{c}
will hear the whole truth, though not, by Zeus, gentlemen, expressed in embroidered and stylized
phrases like theirs, but things spoken at random and expressed in the first words that come to mind,
for I put my trust in the justice of what I say, and let none of you expect anything else. It would
not be fitting at my age, as it might be for a young man, to toy with words when I appear before
you.

One thing I do ask and beg of you, gentlemen: if you hear me making my defence in the same
kind of language as I am accustomed to use in the market place by the bankers' tables, where many
of you have heard me, and elsewhere, do not be surprised or create a disturbance on that account.
The position is this: this is my first appearance in a lawcourt, at the age of seventy; I am therefore\marginpar{d}
simply a stranger to the manner of speaking here. Just as if I were really a stranger, you would
certainly excuse me if I spoke in that dialect and manner in which I had been brought up, so too
my present request seems a just one, for you to pay no attention to my manner of speech—be it\marginpar{18}
better or worse—but to concentrate your attention on whether what I say is just or not, for the
excellence of a judge lies in this, as that of a speaker lies in telling the truth.

It is right for me, gentlemen, to defend myself first against the first lying accusations made
against me and my first accusers, and then against the later accusations and the later accusers.
There have been many who have accused me to you for many years now, and none of their\marginpar{b}
accusations are true. These I fear much more than I fear Anytus and his friends, though they too
are formidable. These earlier ones, however, are more so, gentlemen; they got hold of most of you
from childhood, persuaded you and accused me quite falsely, saying that there is a man called
Socrates, a wise man, a student of all things in the sky and below the earth, who makes the worse
argument the stronger. Those who spread that rumour, gentlemen, are my dangerous accusers, for\marginpar{c}
their hearers believe that those who study these things do not even believe in the gods. Moreover,
these accusers are numerous, and have been at it a long time; also, they spoke to you at an age
when you would most readily believe them, some of you being children and adolescents, and they
won their case by default, as there was no defence.

What is most absurd in all this is that one cannot even know or mention their names unless one
of them is a writer of comedies. Those who maliciously and slanderously persuaded you—who\marginpar{d}
also, when persuaded themselves then persuaded others—all those are most difficult to deal with:
one cannot bring one of them into court or refute him; one must simply fight with shadows, as it
were, in making one's defence, and cross-examine when no one answers. I want you to realize too
that my accusers are of two kinds: those who have accused me recently, and the old ones I
mention; and to think that I must first defend myself against the latter, for you have also heard their
accusations first, and to a much greater extent than the more recent.\marginpar{e}

Very well then. I must surely defend myself and attempt to uproot from your minds in so short
a time the slander that has resided there so long. I wish this may happen, if it is in any way better\marginpar{19}
for you and me, and that my defence may be successful, but I think this is very difficult and I am
fully aware of how difficult it is. Even so, let the matter proceed as the god may wish, but I must
obey the law and make my defence.

Let us then take up the case from its beginning. What is the accusation from which arose the
slander in which Meletus trusted when he wrote out the charge against me? What did they say\marginpar{b}
when they slandered me? I must, as if they were my actual prosecutors, read the affidavit they
would have sworn. It goes something like this: Socrates is guilty of wrongdoing in that he busies
himself studying things in the sky and below the earth; he makes the worse into the stronger
argument, and he teaches these same things to others. You have seen this yourselves in the comedy
of Aristophanes, a Socrates swinging about there, saying he was walking on air and talking a lot\marginpar{c}
of other nonsense about things of which I know nothing at all. I do not speak in contempt of such
knowledge, if someone is wise in these things—lest Meletus bring more cases against me—but,
gentlemen, I have no part in it, and on this point I call upon the majority of you as witnesses. I
think it right that all those of you who have heard me conversing, and many of you have, should
tell each other if anyone of you has ever heard me discussing such subjects to any extent at all.\marginpar{d}
From this you will learn that the other things said about me by the majority are of the same kind.

Not one of them is true. And if you have heard from anyone that I undertake to teach people\marginpar{e}
and charge a fee for it, that is not true either. Yet I think it a fine thing to be able to teach people
as Gorgias of Leontini does, and Prodicus of Ceos, and Hippias of Elis. Each of these men can1
go to any city and persuade the young, who can keep company with anyone of their own fellow-
citizens they want without paying, to leave the company of these, to join with themselves, pay\marginpar{20}
them a fee, and be grateful to them besides. Indeed, I learned that there is another wise man from
Paros who is visiting us, for I met a man who has spent more money on Sophists than everybody
else put together, Callias, the son of Hipponicus. So I asked him—he has two sons—"Callias," I
said, "if your sons were colts or calves, we could find and engage a supervisor for them who would
make them excel in their proper qualities, some horse breeder or farmer. Now since they are men,\marginpar{b}
whom do you have in mind to supervise them? Who is an expert in this kind of excellence, the
human and social kind? I think you must have given thought to this since you have sons. Is there
such a person," I asked, "or is there not?" "Certainly there is," he said. "Who is he?" I asked,
"What is his name, where is he from? and what is his fee?" "His name, Socrates, is Evenus, he
comes from Paras, and his fee is five minas." I thought Evenus a happy man, if he really possesses\marginpar{c}
this art, and teaches for so moderate a fee. Certainly I would pride and preen myself if I had this
knowledge, but I do not have it, gentlemen.

One of you might perhaps interrupt me and say: "But Socrates, what is your occupation? From
where have these slanders come? For surely if you did not busy yourself with something out of the
common, all these rumours and talk would not have arisen unless you did something other than
most people. Tell us what it is, that we may not speak inadvisedly about you." Anyone who says\marginpar{d}
that seems to be right, and I will try to show you what has caused this reputation and slander.
Listen then. Perhaps some of you will think I am jesting, but be sure that all that I shall say is true.
What has caused my reputation is none other than a certain kind of wisdom. What kind of wisdom?
Human wisdom, perhaps. It may be that I really possess this, while those whom I mentioned just
now are wise with a wisdom more than human; else I cannot explain it, for I certainly do not\marginpar{e}
possess it, and whoever says I do is lying and speaks to slander me. Do not create a disturbance,
gentlemen, even if you think I am boasting, for the story I shall tell does not originate with me, but
I will refer you to a trustworthy source. I shall call upon the god at Delphi as witness to the
existence and nature of my wisdom, if it be such. You know Chairephon. He was my friend from
youth, and the friend of most of you, as he shared your exile and your return. You surely know the\marginpar{21}
kind of man he was, how impulsive in any course of action. He went to Delphi at one time and
ventured to ask the oracle—as I say, gentlemen, do not create a disturbance—he asked if any man
was wiser than I, and the Pythian replied that no one was wiser. Chairephon is dead, but his
brother will testify to you about this.

Consider that I tell you this because I would inform you about the origin of the slander. When
I heard of this reply I asked myself: "Whatever does the god mean? What is his riddle? I am very\marginpar{b}
conscious that I am not wise at all; what then does he mean by saying that I am the wisest? For
surely he does not lie; it is not legitimate for him to do so." For a long time I was at a loss as to his
meaning; then I very reluctantly turned to some such investigation as this: I went to one of those
reputed wise, thinking that there, if anywhere, I could refute the oracle and say to it: "This man
is wiser than I, but you said I was." Then, when I examined this man—there is no need for me to
tell you his name, he was one of our public men—my experience was something like this: I\marginpar{c}
thought that he appeared wise to many people and especially to himself, but he was not. I then
tried to show him that he thought himself wise, but that he was not. As a result he came to dislike
me, and so did many of the bystanders. So I withdrew and thought to myself: "I am wiser than this\marginpar{d}
man; it is likely that neither of us knows anything worthwhile, but he thinks he knows something
when he does not, whereas when I do not know, neither do I think I know; so I am likely to be
wiser than he to this small extent, that I do not think I know what I do not know." After this I
approached another man, one of those thought to be wiser than he, and I thought the same thing,
and so I came to be disliked both by him and by many others.\marginpar{e}

After that I proceeded systematically. I realized, to my sorrow and alarm, that I was getting
unpopular, but I thought that I must attach the greatest importance to the god's oracle, so I must
go to all those who had any reputation for knowledge to examine its meaning. And by the dog,
gentlemen of the jury—for I must tell you the truth—I experienced something like this: in my
investigation in the service of the god I found that those who had the highest reputation were\marginpar{22}
nearly the most deficient, while those who were thought to be inferior were more knowledgeable.
I must give you an account of my journeyings as if they were labours I had undertaken to prove
the oracle irrefutable. After the politicians, I went to the poets, the writers of tragedies and
dithyrambs and the others, intending in their case to catch myself being more ignorant then they.
So I took up those poems with which they seemed to have taken most trouble and asked them what
they meant, in order that I might at the same time learn something from them. I am ashamed to tell\marginpar{b}
you the truth, gentlemen, but I must. Almost all the bystanders might have explained the poems
better than their authors could. I soon realized that poets do not compose their poems with
knowledge, but by some inborn talent and by inspiration, like seers and prophets who also say
many fine things without any understanding of what they say. The poets seemed to me to have had\marginpar{c}
a similar experience. At the same time I saw that, because of their poetry, they thought themselves
very wise men in other respects, which they were not. So there again I withdrew, thinking that I
had the same advantage over them as I had over the politicians.

Finally I went to the craftsmen, for I was conscious of knowing practically nothing, and I knew\marginpar{d}
that I would find that they had knowledge of many fine things. In this I was not mistaken; they
knew things I did not know, and to that extent they were wiser than I. But, gentlemen of the jury,
the good craftsmen seemed to me to have the same fault as the poets: each of them, because of his
success at his craft, thought himself very wise in other most important pursuits, and this error of
theirs overshadowed the wisdom they had, so that I asked myself, on behalf of the oracle, whether
I should prefer to be as I am, with neither their wisdom nor their ignorance, or to have both. The\marginpar{e}
answer I gave myself and the oracle was that it was to my advantage to be as I am.
As a result of this investigation, gentlemen of the jury, I acquired much unpopularity, of a kind that
is hard to deal with and is a heavy burden; many slanders came from these people and a reputation\marginpar{23}
for wisdom, for in each case the bystanders thought that I myself possessed the wisdom that I
proved that my interlocutor did not have. What is probable, gentlemen, is that in fact the god is
wise and that his oracular response meant that human wisdom is worth little or nothing, and that
when he says this man, Socrates, he is using my name as an example, as if he said: "This man
among you, mortals, is wisest who, like Socrates, understands that his wisdom is worthless." So\marginpar{b}
even now I continue this investigation as the god bade me—and I go around seeking out anyone,
citizen or stranger, whom I think wise. Then if I do not think he is, I come to the assistance of the
god and show him that he is not wise. Because of this occupation, I do not have the leisure to
engage in public affairs to any extent, nor indeed to look after my own, but I live in great poverty
because of my service to the god.

Furthermore, the young men who follow me around of their own free will, those who have
most leisure, the sons of the very rich, take pleasure in hearing people questioned; they themselves\marginpar{c}
often imitate me and try to question others. I think they find an abundance of men who believe they
have some knowledge but know little or nothing. The result is that those whom they question are
angry, not with themselves but with me. They say: "That man Socrates is a pestilential fellow who
corrupts the young." If one asks them what he does and what he teaches to corrupt them, they are\marginpar{d}
silent, as they do not know, but, so as not to appear at a loss, they mention those accusations that
are available against all philosophers, about "things in the sky and things below the earth," about
"not believing in the gods" and "making the worse the stronger argument;" they would not want
to tell the truth, I'm sure, that they have been proved to lay claim to knowledge when they know
nothing. These people are ambitious, violent and numerous; they are continually and convincingly
talking about me; they have been filling your ears for a long time with vehement slanders against
me. From them Meletus attacked me, and Anytus and Lycon, Meletus being vexed on behalf of\marginpar{e}
the poets, Anytus on behalf of the craftsmen and the politicians, Lycon on behalf of the orators,
so that, as I started out by saying, I should be surprised if I could rid you of so much slander in so
short a time. That, gentlemen of the jury, is the truth for you. I have hidden or disguised nothing.\marginpar{24}
I know well enough that this very conduct makes me unpopular, and this is proof that what I say
is true, that such is the slander against me, and that such are its causes. If you look into this either
now or later, this is what you will find.\marginpar{b}

Let this suffice as a defence against the charges of my earlier accusers. After this I shall try
to defend myself against Meletus, that good and patriotic man, as he says he is, and my later
accusers. As these are a different lot of accusers, let us again take up their sworn deposition. It
goes something like this: Socrates is guilty of corrupting the young and of not believing in the gods
in whom the city believes, but in other new spiritual things. Such is their charge. Let us examine
it point by point.\marginpar{c}

He says that I am guilty of corrupting the young, but I say that Meletus is guilty of dealing
frivolously with serious matters, of irresponsibly bringing people into court, and of professing to
be seriously concerned with things about none of which he has ever cared, and I shall try to prove
that this is so. Come here and tell me, Meletus. Surely you consider it of the greatest importance\marginpar{d}
that our young men be as good as possible? —Indeed I do.

Come then, tell the jury who improves them. You obviously know, in view of your concern.
You say you have discovered the one who corrupts them, namely me, and you bring me here and
accuse me to the jury. Come, inform the jury and tell them who it is. You see, Meletus, that you
are silent and know not what to say. Does this not seem shameful to you and a sufficient proof of
what I say, that you have not been concerned with any of this? Tell me, my good sir, who improves
our young men? —The laws.\marginpar{e}

That is not what I am asking, but what person who has knowledge of the laws to begin
with?—These jurymen, Socrates.

How do you mean, Meletus? Are these able to educate the young and improve
them?—Certainly.

All of them, or some but not others?—All of them.

Very good, by Hera. You mention a great abundance of benefactors. But what about the
audience? Do they improve the young or not?—They do, too.\marginpar{25}

What about the members of Council?—The Councillors, also.
But, Meletus, what about the assembly? Do members of the assembly corrupt the young, or
do they all improve them?—They improve them.

All the Athenians, it seems, make the young into fine good men, except me, and I alone corrupt
them. Is that what you mean?—That is most definitely what I mean.

You condemn me to a great misfortune. Tell me: does this also apply to horses do you think?\marginpar{b}
That all men improve them and one individual corrupts them? Or is quite the contrary true, one
individual is able to improve them, or very few, namely the horse breeders, whereas the majority,
if they have horses and use them, corrupt them? Is that not the case, Meletus, both with horses and
all other animals? Of course it is, whether you and Anytus say so or not. It would be a very happy
state of affairs if only one person corrupted our youth, while the others improved them.

You have made it sufficiently obvious, Meletus, that you have never had any concern for our\marginpar{c}
youth; you show your indifference clearly; that you have given no thought to the subjects about
which you bring me to trial.

And by Zeus, Meletus, tell us also whether it is better for a man to live among good or wicked
fellow-citizens. Answer, my good man, for I am not asking a difficult question. Do not the wicked
do some harm to those who are ever closest to them, whereas good people benefit
them?—Certainly.

And does the man exist who would rather be harmed than benefited by his associates? Answer,\marginpar{d}
my good sir, for the law orders you to answer. Is there any man who wants to be harmed? —Of
course not.

Come now, do you accuse me here of corrupting the young and making them worse
deliberately or unwillingly?—Deliberately.

What follows, Meletus? Are you so much wiser at your age than I am at mine that you
understand that wicked people always do some harm to their closest neighbors while good people\marginpar{e}
do them good, but I have reached such a pitch of ignorance that I do not realize this, namely that
if I make one of my associates wicked I run the risk of being harmed by him so that I do such a
great evil deliberately, as you say? I do not believe you, Meletus, and I do not think anyone else\marginpar{26}
will. Either I do not corrupt the young or, if I do, it is unwillingly, and you are lying in either case.
Now if I corrupt them unwillingly, the law does not require you to bring people to court for such
unwilling wrongdoings, but to get hold of them privately, to instruct them and exhort them; for
clearly, if I learn better, I shall cease to do what I am doing unwillingly. You, however, have
avoided my company and were unwilling to instruct me, but you bring me here, where the law
requires one to bring those who are in need of punishment, not of instruction.

And so, gentlemen of the jury, what I said is clearly true: Meletus has never been at all\marginpar{b}
concerned with these matters. Nonetheless tell us, Meletus, how you say that I corrupt the young;
or is it obvious from your deposition that it is by teaching them not to believe in the gods in whom
the city believes but in other new spiritual things? Is this not what you say I teach and so corrupt
them? —That is most certainly what I do say.

Then by those very gods about whom we are talking, Meletus, make this clearer to me and to
the jury: I cannot be sure whether you mean that I teach the belief that there are some gods—and
therefore I myself believe that there are gods and am not altogether an atheist, nor am I guilty of\marginpar{c}
that—not, however, the gods in whom the city believes, but others, and that this is the charge
against me, that they are others. Or whether you mean that I do not believe in gods at all, and that
this is what I teach to others. —This is what I mean, that you do not believe in gods at all.

You are a strange fellow, Meletus. Why do you say this? Do I not believe, as other men do,
that the sun and the moon are gods?—No, by Zeus, jurymen, for he says that the sun is stone, and\marginpar{d}
the moon earth.

My dear Meletus, do you think you are prosecuting Anaxagoras? Are you so contemptuous
of the jury and think them so ignorant of letters as not to know that the books of Anaxagoras of
Clazomenae are full of those theories, and further, that the young men learn from me what they
can buy from time to time for a drachma, at most, in the bookshops, and ridicule Socrates if he
pretends that these theories are his own, especially as they are so absurd? Is that, by Zeus, what\marginpar{e}
you think of me, Meletus, that I do not believe that there are any gods? —That is what I say, that
you do not believe in the gods at all.

You cannot be believed, Meletus, even, I think, by yourself. The man appears to me,
gentlemen of the jury, highly insolent and uncontrolled. He seems to have made this deposition
out of insolence, violence and youthful zeal. He is like one who composed a riddle and is trying
it out: "Will the wise Socrates realize that I am jesting and contradicting myself, or shall I deceive
him and others?" I think he contradicts himself in the affidavit, as if he said: "Socrates is guilty of\marginpar{27}
not believing in gods but believing in gods," and surely that is the part of a jester!

Examine with me, gentlemen, how he appears to contradict himself, and you, Meletus, answer
us. Remember, gentlemen, what I asked you when I began, not to create a disturbance if I proceed
in my usual manner.\marginpar{b}

Does any man, Meletus, believe in human activities who does not believe in humans? Make
him answer, and not again and again create a disturbance. Does any man who does not believe in
horses believe in horsemen's activities? Or in flute-playing activities but not in flute-players? No,
my good sir, no man could. If you are not willing to answer, I will tell you and the jury. Answer
the next question, however. Does any man believe in spiritual activities who does not believe in
spirits?—No one.\marginpar{c}

Thank you for answering, if reluctantly, when the jury made you. Now you say that I believe
in spiritual things and teach about them, whether new or old, but at any rate spiritual things
according to what you say, and to this you have sworn in your deposition. But if I believe in
spiritual things I must quite inevitably believe in spirits. Is that not so? It is indeed. I shall assume
that you agree, as you do not answer. Do we not believe spirits to be either gods or the children
of gods? Yes or no?—Of course.\marginpar{d}

Then since I do believe in spirits, as you admit, if spirits are gods, this is what I mean when
I say you speak in riddles and in jest, as you state that I do not believe in gods and then again that
I do, since I do believe in spirits. If on the other hand the spirits are children of the gods, bastard
children of the gods by nymphs or some other mothers, as they are said to be, what man would
believe children of the gods to exist, but not gods? That would be just as absurd as to believe the
young of horses and asses, namely mules, to exist, but not to believe in the existence of horses and\marginpar{e}
asses. You must have made this deposition, Meletus, either to test us or because you were at a loss
to find any true wrongdoing of which to accuse me. There is no way in which you could persuade
anyone of even small intelligence that it is possible for one and the same man to believe in spiritual
but not also in divine things, and then again for that same man to believe neither in spirits nor in
gods nor in heroes.\marginpar{28}

I do not think, gentlemen of the jury, that it requires a prolonged defence to prove that I am
not guilty of the charges in Meletus' deposition, but this is sufficient. On the other hand, you know
that what I said earlier is true, that I am very unpopular with many people. This will be my
undoing, if I am undone, not Meletus or Anytus but the slanders and envy of many people. This
has destroyed many other good men and will, I think, continue to do so. There is no danger that\marginpar{b}
it will stop at me.

Someone might say: 'Are you not ashamed, Socrates, to have followed the kind of occupation
that has led to your being now in danger of death?" However, I should be right to reply to him:
"You are wrong, sir, if you think that a man who is any good at all should take into account the risk
of life or death; he should look to this only in his actions, whether what he does is right or wrong,
whether he is acting like a good or a bad man." According to your view, all the heroes who died
at Troy were inferior people, especially the son of Thetis who was so contemptuous of danger\marginpar{c}
compared with disgrace. When he was eager to kill Hector, his goddess mother warned him, as I
believe, in some such words as these: "My child, if you avenge the death of your comrade,
Patroclus, and you kill Hector, you will die yourself, for your death is to follow immediately after
Hector's." Hearing this, he despised death and danger and was much more afraid to live a coward
who did not avenge his friends. "Let me die at once," he said, "when once I have given the\marginpar{d}
wrongdoer his deserts, rather than remain here, a laughing-stock by the curved ships, a burden
upon the earth." Do you think he gave thought to death and danger?

This is the truth of the matter, gentlemen of the jury: wherever a man has taken a position that
he believes to be best, or has been placed by his commander, there he must I think remain and face
danger, without a thought for death or anything else, rather than disgrace. It would have been a
dreadful way to behave, gentlemen of the jury, if, at Potidaea, Amphipolis and Delium, I had, at
the risk of death, like anyone else, remained at my post where those you had elected to command\marginpar{e}
had ordered me, and then, when the god ordered me, as I thought and believed, to live "the life of
a philosopher, to examine myself and others, I had abandoned my post for fear of death or anything
else. That would have been a dreadful thing, and then I might truly have justly been brought here
for not believing that there are gods, disobeying the oracle, fearing death, and thinking I was wise\marginpar{29}
when I was not. To fear death, gentlemen, is no other than to think oneself wise when one is not,
to think one knows what one does not know. No one knows whether death may not be the greatest
of all blessings for a man, yet men fear it as if they knew that it is the greatest of evils. And surely
it is the most blameworthy ignorance to believe that one knows what one does not know. It is
perhaps on this point and in this respect, gentlemen, that I differ from the majority of men, and if\marginpar{b}
I were to claim that I am wiser than anyone in anything, it would be in this that as I have no
adequate knowledge of things in the underworld, so I do not think I have. I do know, however, that
it is wicked and shameful to do wrong, to disobey one's superior, be he god or man. I shall never
fear or avoid things of which I do not know, whether they may not be good rather than things that
I know to be bad. Even if you acquitted me now and did not believe Anytus, who said to you that
either I should not have been brought here in the first place, or that now I am here, you cannot\marginpar{c}
avoid executing me, for if I should be acquitted, your sons would practise the teachings of Socrates
and all be thoroughly corrupted; if you said to me in this regard: "Socrates, we do not believe
Anytus now; we acquit you, but only on condition that you spend no more time on this
investigation and do not practise philosophy, and if you are caught doing so you will die;" if, as
I say, you were to acquit me on those terms, I would say to you: "Gentlemen of the jury, I am\marginpar{d}
grateful and I am your friend, but I will obey the god rather than you, and as long as I draw breath
and am able, I shall not cease to practise philosophy, to exhort you and in my usual way to point
out to anyone of you whom I happen to meet: Good Sir, you are an Athenian, a citizen of the
greatest city with the greatest reputation for both wisdom and power; are you not ashamed of your
eagerness to possess as much wealth, reputation and honours as possible, while you do not care
for nor give thought to wisdom or truth or the best possible state of your soul?" Then, if one of you\marginpar{e}
disputes this and says he does care, I shall not let him go at once or leave him, but I shall question
him, examine him and test him, and if I do not think he has attained the goodness that he says he
has, I shall reproach him because he attaches little importance to the most important things and
greater importance to inferior things. I shall treat in this way anyone I happen to meet, young and
old, citizen and stranger, and more so the citizens because you are more kindred to me. Be sure\marginpar{30}
that this is what the god orders me to do, and I think there is no greater blessing for the city than
my service to the god. For I go around doing nothing but persuading both young and old among
you not to care for your body or your wealth in preference to or as strongly as for the best possible
state of your soul, as I say to you: "Wealth does not bring about excellence, but excellence makes\marginpar{b}
wealth and everything else good for men, both individually and collectively."

Now if by saying this I corrupt the young, this advice must be harmful, but if anyone says that
I give different advice, he is talking nonsense. On this point I would say to you, gentlemen of the
jury: "Whether you believe Anytus or not, whether you acquit me or not, do so on the
understanding that this is my course of action, even if I am to face death many times." Do not\marginpar{c}
create a disturbance, gentlemen, but abide by my request not to cry out at what I say but to listen,
for I think it will be to your advantage to listen, and I am about to say other things at which you
will perhaps cry out. By no means do this. Be sure that if you kill the sort of man I say I am, you
will not harm me more than yourselves. Neither Meletus nor Anytus can harm me in any way; he
could not harm me, for I do not think it is permitted that a better man be harmed by a worse;
certainly he might kill me, or perhaps banish or disfranchise me, which he and maybe others think\marginpar{d}
to be great harm, but I do not think so. I think he is doing himself much greater harm doing what
he is doing now, attempting to have a man executed unjustly. Indeed, gentlemen of the jury, I am
far from making it defence now on my own behalf, as might be thought, but on yours, to prevent
you from wrongdoing by mistreating the god's gift to you by condemning me; for if you kill me\marginpar{e}
you will not easily find another like me. I was attached to this city by the god—though it seems
a ridiculous thing to say—as upon a great and noble horse which was somewhat sluggish because
of its size and needed to be stirred up by a kind of gadfly. It is to fulfill some such function that
I believe the god has placed me in the city. I never cease to rouse each and everyone of you, to
persuade and reproach you all day long and everywhere I find myself in your company.

Another such man will not easily come to be among you, gentlemen, and if you believe me you\marginpar{31}
will spare me. You might easily be annoyed with me as people are when they are aroused from a
doze, and strike out at me; if convinced by Anytus you could easily kill me, and then you could
sleep on for the rest of your days, unless the god, in his care for you, sent you someone else. That
I am the kind of person to be a gift of the god to the city you might realize from the fact that it does
not seem like human nature for me to have neglected all my own affairs and to have tolerated this
neglect now for so many years while I was always concerned with you, approaching each one of\marginpar{b}
you like a father or an elder brother to persuade you to care for virtue (aretë). Now if I profited
from this by charging a fee for my advice, there would be some sense to it, but you can see for
yourselves that, for all their shameless accusations, my accusers have not been able in their
impudence to bring forward a witness to say that I have ever received a fee or ever asked for one.\marginpar{c}
I, on the other hand, have a convincing witness that I speak the truth, my poverty.

It may seem strange that while I go around and give this advice privately and interfere in
private affairs, I do not venture to go to the assembly and there advise the city. You have heard me
give the reason for this in many places. I have a divine or spiritual sign which Meletus has
ridiculed in his deposition. This began when I was a child. It is a voice, and whenever it speaks it
turns me away from something I am about to do, but it never encourages me to do anything. This\marginpar{d}
is what has prevented me from taking part in public affairs, and I think it was quite right to prevent
me. Be sure, gentlemen of the jury, that if I had long ago attempted to take part in politics, I should
have died long ago, and benefited neither you nor myself. Do not be angry with me for speaking
the truth; no man will survive who genuinely opposes you or any other crowd and prevents the
occurrence of many unjust and illegal happenings in the city. A man who really fights for justice\marginpar{e}
must lead a private, not a public, life if he is to survive for even a short time.

I shall give you great proofs of this, not words but what you esteem, deeds. Listen to what\marginpar{32}
happened to me, that you may know that I will not yield to any man contrary to what is right, for
fear of death, even if I should die at once for not yielding. The things I shall tell you are
commonplace and smack of the lawcourts, but they are true. I have never held any other office in
the city, but I served as a member of the Council, and our tribe Antiochis was presiding at the time
when you wanted to try as a body the ten generals who had failed to pick up the survivors of the
naval battle. This was illegal, as you all recognized later. I was the only member of the presiding\marginpar{b}
committee to oppose your doing something contrary to the laws, and I voted against it. The orators
were ready to prosecute me and take me away; and your shouts were egging them on, but I thought
I should run any risk on the side of law and justice rather than join you, for fear of prison or death,
when you were engaged in an unjust course.

This happened when the city was still a democracy. When the oligarchy was established, the
Thirty summoned me to the Hall, along with four others, and ordered us to bring Leon from\marginpar{c}
Salamis, that he might be executed. They gave many such orders to many people, in order to
implicate as many as possible in their guilt. Then I showed again, not in words but in action, that,
if it were not rather vulgar to say so, death is something I couldn't care less about, but that my
whole concern is not to do anything unjust or impious. That government, powerful as it was, did\marginpar{d}
not frighten me into any wrongdoing. When we left the Hall, the other four went to Salamis and
brought in Leon, but I went home. I might have been put to death for this, had not the government
fallen shortly afterwards. There are many who will witness to these events.

Do you think I would have survived all these years if I were engaged in public affairs and,\marginpar{e}
acting as a good man must, came to the help of justice and considered this the most important
thing? Far from it, gentlemen of the jury, nor would any other man. Throughout my life, in any
public activity I may have engaged in, I am the same man as I am in private life. I have never come
to an agreement with anyone to act unjustly, neither with anyone else nor with anyone of those
who they slanderously say are my pupils. I have never been anyone's teacher. If anyone, young or\marginpar{33}
old, desires to listen to me when I am talking and dealing with my own concerns, I have never
begrudged this to anyone, but I do not converse when I receive a fee and not when I do not. I am
equally ready to question the rich and the poor if anyone is willing to answer my questions and\marginpar{b}
listen to what I say. And I cannot justly be held responsible for the good or bad conduct of these
people, as I never promised to teach them anything and have not done so. If anyone says that he
has learned anything from me, or that he heard anything privately that the others did not hear, be
assured that he is not telling the truth.

Why then do some people enjoy spending considerable time in my company? You have heard
why, gentlemen of the jury, I have told you the whole truth. They enjoy hearing those being
questioned who think they are wise, but are not. And this is not unpleasant. To do this has, as I say,\marginpar{c}
been enjoined upon me by the god, by means of oracles and dreams, and in every other way that
a divine manifestation has ever ordered a man to do anything. This is true, gentlemen, and can
easily be established.

If I corrupt some young men and have corrupted others, then surely some of them who have
grown older and realized that I gave them bad advice when they were young should now
themselves come up here to accuse me and avenge themselves. If they were unwilling to do so\marginpar{d}
themselves, then some of their kindred, their fathers or brothers or other relations should recall it
now if their family had been harmed by me. I see many of these present here, first Crito, my
contemporary and fellow demesman, the father of Critoboulos here; next Lysanias of Sphettus, the
father of Aeschines here; also Antiphon the Cephisian, the father of Epigenes; and others whose
brothers spent their time in this way; Nicostratus, the son of Theozotides, brother of Theodotus,
and Theodotus has died so he could not influence him; Paralios here, son of Demodocus, whose\marginpar{e}
brother was Theages; there is Adeimanttls, son of Ariston, brother of Plato here; Acantidorus,
brother of Apollodorus here.

I could mention many others, some one of whom surely Meletus should have brought in as
witness in his own speech. If he forgot to do so, then let him do it now; I will yield time if he has\marginpar{34}
anything of the kind to say. You will find quite the contrary, gentlemen. These men are all ready
to come to the help of the corruptor, the man who has harmed their kindred, as Meletus and Anytus
say. Now those who were corrupted might well have reason to help me, but the uncorrupted, their
kindred who are older men, have no reason to help me except the right and proper one, that they
know that Meletus is lying and that I am telling the truth.

Very well, gentlemen of the jury. This, and maybe other similar things, is what I have to say\marginpar{b}
in my defence. Perhaps one of you might be angry as he recalls that when he himself stood trial
on a less dangerous charge, he begged and pleaded and implored the jury with many tears, that he
brought his children and many of his friends and family into court to arouse as much pity as he
could, but that I do none of these things, even though I may seem to be running the ultimate risk.
Thinking of this, he might feel resentful toward me and, angry about this, cast his vote in anger.\marginpar{c}
If there is such a one among you—I do not deem there is, but if there is—I think it would be right
to say in reply: My good sir, I too have a household and, in Homer's phrase, I am not born "from
oak or rock" but from men, so that I have a family, indeed three sons, gentlemen of the jury, of
whom one is an adolescent while two are children. Nevertheless, I will not beg you to acquit me\marginpar{d}
by bringing them here. Why do I do none of these things? Not through arrogance, gentlemen, nor
through lack of respect for you. Whether I am brave in the face of death is another matter, but with
regard to my reputation and yours and that of the whole city, it does not seem right to me to do
these things, especially at my age and with my reputation. For it is generally believed, whether it
be true or false, that in certain respects Socrates is superior to the majority of men. Now if those\marginpar{e}
of you who are considered superior, be it in wisdom or courage or whatever other virtue makes
them so, are seen behaving like that, it would be a disgrace. Yet I have often seen them do this sort
of thing when standing trial, men who are thought to be somebody, doing amazing things as if they
thought it a terrible thing to die, and as if they were to be immortal if you did not execute them.\marginpar{35}
I think these men bring shame upon the city so that a stranger, too, would assume that those who
are outstanding in virtue among the Athenians, whom they themselves select from themselves to
fill offices of state and receive other honours, are in no way better than women. You should not
act like that, gentlemen of the jury, those of you who have any reputation at all, and if we do, you\marginpar{b}
should not allow it. You should make it very clear that you will more readily convict a man who
performs these pitiful dramatics in court and so makes the city a laughingstock, than a man who
keeps quiet.

Quite apart from the question of reputation, gentlemen, I do not think it right to supplicate the
jury and to be acquitted because of this but to teach and persuade them. It is not the purpose of a
juryman's office to give justice as a favour to whoever seems good to him, but to judge according\marginpar{c}
to law, and this he has sworn to do. We should not accustom you to perjure yourselves, nor should
you make a habit of it. This is irreverent conduct for either of us.

Do not deem it right for me, gentlemen of the jury, that I should act towards you in a way that
I do not consider to be good or just or pious, especially, by Zeus, as I am being prosecuted by
Meletus here for impiety; clearly, if I convinced you by my supplication to do violence to your\marginpar{d}
oath of office, I would be teaching you not to believe that there are gods, and my defence would
convict me of hot believing in them. This is far from being the case, gentlemen, for I do believe
in them as none of my accusers do. I leave it to you and the god to judge me in the way that will
be best for me and for you.

[The jury now gives its verdict of guilty, and Meletus asks for the penalty of death.]

There are many other reasons for my not being angry with you for convicting me, gentlemen
of the jury, and what happened was not unexpected. I am much more surprised at the number of
votes cast on each side, for I did not think the decision would be by so few votes but by a great
many. As it is, a switch of only thirty votes would have acquitted me. I think myself that I have\marginpar{e}
been cleared on Meletus' charges, and not only this, but it is clear to all that, if Anytus and Lycon \marginpar{36}
had not joined him in accusing me, he would have been fined a thousand drachmas for not
receiving a fifth of the votes.

He assesses the penalty at death. So be it. What counter-assessment should I propose to you,\marginpar{b}
gentlemen of the jury? Clearly it should be a penalty I deserve, and what do I deserve to suffer or
to pay because I have deliberately not led a quiet life but have neglected what occupies most
people: wealth, household affairs, the position of general or public orator or the other offices, the
political clubs and factions that exist in the city? I thought myself too honest to survive if I
occupied myself with those things. I did not follow that path that would have made me of no use
either to you or to myself, but I went to each of you privately and conferred upon him what I say
is the greatest benefit, by trying to persuade him not to care for any of his belongings before caring
that he himself should be as good and as wise as possible, not to care for the city's possessions\marginpar{c}
more than for the city itself, and to care for other things in the same way. What do I deserve for
being such a man? Some good, gentlemen of the jury, if I must truly make an assessment according
to my deserts, and something suitable. What is suitable for a poor benefactor who needs leisure
to exhort you? Nothing is more suitable, gentlemen, than for such a man to be fed in the\marginpar{d}
Prytaneum, much more suitable for him than for anyone of you who has won a victory at Olympia5
with a pair or a team of horses. The Olympian victor makes you think yourself happy; I make you
be happy. Besides, he does not need food, but I do. So if I must make a just assessment of what
I deserve, I assess it at this: free meals in the Prytaneum.

When I say this you may think, as when I spoke of appeals to pity and entreaties, that I speak
arrogantly, but that is not the case, gentlemen of the jury; rather it is like this: I am convinced that\marginpar{e}
I never willingly wrong anyone, but I am not convincing you of this, for we have talked together\marginpar{37}
but a short time. If it were the law with us, as it is elsewhere, that a trial for life should not last one
but many days, you would be convinced, but now it is not easy to dispel great slanders in a short
time. Since I am convinced that I wrong no one, I am not likely to wrong myself, to say that I\marginpar{b}
deserve some evil and to make some such assessment against myself. What should I fear? That I
should suffer the penalty Meletus has assessed against me, of which I say I do not know whether
it is good or bad? Am I then to choose in preference to this something that I know very well to be
an evil and assess the penalty at that? Imprisonment? Why should I live in prison, always subjected
to the ruling magistrates the Eleven? A fine, and imprisonment until I pay it? That would be the\marginpar{c}
same thing for me, as I have no money. Exile? for perhaps you might accept that assessment.

I should have to be inordinately fond of life, gentlemen of the jury, to be so unreasonable as
to suppose that other men will easily tolerate my company and conversation when you, my fellow
citizens, have been unable to endure them, but found them a burden and resented them so that you
are now seeking to get rid of them. Far from it, gentlemen. It would be a fine life at my age to be\marginpar{d}
driven out of one city after another, for I know very well that wherever I go the young men will
listen to my talk as they do here. If I drive them away, they will themselves persuade their elders
to drive me out; if I do not drive them away, their fathers and relations will drive me out on their\marginpar{e}
behalf.

Perhaps someone might say: But Socrates, if you leave us will you not be able to live quietly,
without talking? Now this is the most difficult point on which to convince some of you. If I say
that it is impossible for me to keep quiet because that means disobeying the god, you will not
believe me and will think I am being ironical. On the other hand, if I say that it is the greatest good
for a man to discuss virtue every day and those other things about which you hear me conversing
and testing myself and others, for the unexamined life is not worth living for man, you will believe
me even less.\marginpar{38}

What I say is true, gentlemen, but it is not easy to convince you. At the same time, I am not
accustomed to think that I deserve any penalty. If I had money. I would assess the penalty at the
amount I could pay, for that would not hurt me, but I have none, unless you are willing to set the
penalty at the amount I can pay, and perhaps I could pay you one mina of silver. So that is my
assessment.\marginpar{b}

Plato here, gentlemen of the jury, and Crito and Critobulus and Apollodorus bid me put the
penalty at thirty minae, and they will stand surety for the money. Well then, that is my assessment,
and they will be sufficient guarantee of payment.

[The jury now votes again and sentences Socrates to death.]

It is for the sake of a short time, gentlemen of the jury, that you will acquire the reputation and
the guilt, in the eyes of those who want to denigrate the city, of having killed Socrates, a wise man,
for they who want to revile you will say that I am wise even if I am not. If you had waited but a
little while, this would have happened of its own accord. You see my age, that I am already\marginpar{c}
advanced in years and close to death. I am saying this not to all of you but to those who
condemned me to death, and to these same jurors I say: Perhaps you think that I was convicted for
lack of such words as might have convinced you, if I thought I should say or do all I could to avoid
my sentence. Far from it. I was convicted because I lacked not words but boldness and\marginpar{d}
shamelessness and the willingness to say to you what you would most gladly have heard from me,
lamentations and tears and my saying and doing many things that I say are unworthy of me but that
you are accustomed to hear from others. I did not think then that the danger I ran should make me
do anything mean, nor do I now regret the nature of my defence. I would much rather die after this
kind of defence than live after making the other kind. Neither I nor any other man should, on trial\marginpar{e}
or in war, contrive to avoid death at any cost. Indeed it is often obvious in battle that one could
escape death by throwing away one's weapons and by turning to supplicate one's pursuers, and
there are many ways to avoid death in every kind of danger if one will venture to do or say\marginpar{39}
anything to avoid it. It is not difficult to avoid death, gentlemen of the jury, it is much more
difficult to avoid wickedness, for it runs faster than death. Slow and elderly as I am, I have been
caught by the slower pursuer, whereas my accusers, being clever and sharp, have been caught by
the quicker, wickedness. I leave you now, condemned to death by you, but they are condemned\marginpar{b}
by truth to wickedness and injustice. So I maintain my assessment, and they maintain theirs. This
perhaps had to happen, and I think it is as it should be.

Now I want to prophesy to those who convicted me, for I am at the point when men prophesy
most, when they are about to die. I say gentlemen, to those who voted to kill me, that vengeance
will come upon you immediately after my death, a vengeance much harder to bear than that which\marginpar{c}
you took in killing me. You did this in the belief that you would avoid giving an account of your
life, but I maintain that quite the opposite will happen to you. There will be more people to test
you, whom I now held back, but you did not notice it. They will be more difficult to deal with as
they will be younger and you will resent them more. You are wrong if you believe that by killing
people you will prevent anyone from reproaching you for not living in the right way. To escape\marginpar{d}
such tests is neither possible nor good, but it is best and easiest not to discredit others but to
prepare oneself to be as good as possible. With this prophecy to you who convicted me, I part from
you.

I should be glad to discuss what has happened with those who voted for my acquittal during
the time that the officers of the court are busy and I do not yet have to depart to my death. So,
gentlemen, stay with me awhile, for nothing prevents us from talking to each other while it is\marginpar{e}
allowed. To you, as being my friends, I want to show the meaning of what has occurred. A
surprising thing has happened to me, judges—you I would rightly call judges. At all previous times
my familiar prophetic power, my spiritual manifestation frequently opposed me, even in small
matters, when I was about to do something wrong, but now that, as you can see for yourselves, I\marginpar{40}
was faced with what one might think, and what is generally thought to be, the worst of evils, my
divine sign has not opposed me, either when I left home at dawn, or when I came into court, or at
any time that I was about to say something during my speech. Yet in other talks it often held me
back in the middle of my speaking, but now it has opposed no word or deed of mine. What do I
think is the reason for this? I will tell you. What has happened to me may well be a good thing, and\marginpar{b}
those of us who believe death to be an evil are certainly mistaken. I have convincing proof of this,
for it is impossible that my familiar sign did not oppose me if I was not about to do what was right.

Let us reflect in this way, too, that there is good hope that death is a blessing, for it is one of
two things: either the dead are nothing and have no perception of anything, or it is, as we are told,\marginpar{c}
a change and a relocating for the soul from here to another place. If it is complete lack of
perception, like a dreamless sleep, then death would be a great advantage. For I think that if one
had to pick out that night during which a man slept soundly and did not dream, put beside it the
other nights and days of his life, and then see how many days and nights had been better and more\marginpar{d}
pleasant than that night, not only a private person but the great king would find them easy to count
compared with the other days and nights. If death is like this I say it is an advantage, for all eternity
would then seem to be no more than a single night. If, on the other hand, death is a change from
here to another place, and what we are told is true and all who have died are there, what greater\marginpar{e}
blessing could there be, gentlemen of the jury? If anyone arriving in Hades will have escaped from
those who call themselves judges here, and will find those true judges who are said to sit in
judgement there, Minos and Radamanthus and Aeacus and Triptolemus and the other demi-gods\marginpar{41}
who have been upright in their own life, would that be a poor kind of change? Again, what would
one of you give to keep company with Orpheus and Musaeus,Hesiod and Homer? I am willing to
die many times if that is true. It would be a wonderful way for me to spend my time whenever I
met Palamedes and Ajax, the son of Telamon, and any other of the men of old who died through
an unjust conviction, to compare my experience with theirs. I think it would be pleasant. Most
important, I could spend my time testing and examining people there, as I do here, as to who
among them is wise, and who thinks he is, but is not.

What would one not give, gentlemen of the jury, for the opportunity to examine the man who\marginpar{b}
led the great expedition against Troy, or Odysseus, or Sisyphus, and innumerable other men and
women one could mention. It would be an extraordinary happiness to talk with them, to keep
company with them and examine them. In any case, they would certainly not put one to death for\marginpar{c}
doing so. They are happier there than we are here in other respects, and for the rest of time they
are deathless, if indeed what we are told is true.

You too must be of good hope as regards death, gentlemen of the jury, and keep this one truth
in mind, that a good man cannot be harmed either in life or in death, and that his affairs are not
neglected by the gods. What has happened to me now has not happened of itself, but it is clear to
me that it was better for me to die now and to escape from trouble. That is why my divine sign did\marginpar{d}
not oppose me at any point. So I am certainly not angry with those who convicted me, or with my
accusers. Of course that was not their purpose when they accused and convicted me, but they
thought they were hurting me, and for this they deserve blame. This much I ask from them: when
my sons grow up, avenge yourselves by, causing them the same kind of grief that I caused you, if
you think they care for money or anything else more than they care for virtue, or if they think they\marginpar{e}
are somebody when they are nobody. Reproach them as I reproach you, that they do not care for
the right things and think they are worthy when they are not worthy of anything. If you do this, I
shall have been justly treated by you, and my sons also.

Now the hour to part has come. I go to die, you go to live. Which of us goes to the better lot
is known to no one, except the god.\marginpar{42}