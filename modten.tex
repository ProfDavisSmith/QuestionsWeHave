\part{What about duties to animals and future generations?}
\label{ch.modten}
\addtocontents{toc}{\protect\mbox{}\protect\hrulefill\par}
\stepcounter{chapcount}
\chapter{Part \thechapcount: Vegetarianism/Veganism}\setcounter{seccount}{1}

Vegetarianism, in a nutshell, is the stance or practice of abstaining from the consumption of animal products. This is an umbrella term which encompasses several different variations on the stance. For example, you have ovo-lacto-vegetarianism, which says that eating eggs and milk is fine, but the flesh of the animal is off limits; there's lacto-vegetarianism, which says that milk and dairy products are fine, but eggs and flesh are a no go; ovo-vegetarianism says that eggs are cool, but milk and fleh are off-limits. There are several other varieties aside, for pretty much any kind of animal product which you can think of, there's likely a flavor of vegetarian which says that it's off limits. Because of the kind of creatures we are, we can't totally cut out all foods which come from a living thing (this would say that plants are off limits too), so the overarching theme for vegetarianism is that the food consumed and (maybe) the products worn/used must be predominately plant based.  The most extreme version of vegetarianism is veganism. This is the stance that all animal products, and even insect products, like honey, are not to be eaten. With the stance clearly laid out, we need to discuss the why, remember the why is always the most important part.
\section{Part \thechapcount.\theseccount: Why are People Vegetarian?}\stepcounter{seccount}  

Some people out there, like some of my mother's side of the family, think that vegetarianism is a new-fangled thing, something `hippy-dippy'. But, this idea and the practices which surround it are very old. Many ancient Greeks practiced it and many religions and cultures have vegetarianism built in because of the moral implications. Plato believed that the consumption of meat did not belong in a civil society because it caused violence in the soul. Pythagoras of Samos, who you may remember from a math course, and his disciples refused to even talk to hunters or butchers. Some very old customs in Asia have vegetarianism built in on religious grounds (also, by extension moral grounds). For similar reasons as Plato and other Greeks, some early Christian sects practiced vegetarianism. Leonardo da Vinci, because of his concern for animal welfare, abstained from the consumption of animal flesh. Mary Shelley, the author of the world-renowned book Frankenstein, because of the deep appreciation for the beauty in nature which came from the art movements of her time, refused to take part in the killing of an animal, and thereby refused to eat meat. And then, also, you have me. I have been vegetarian for 20+ years, this was initially because of concern for animal welfare, the realization that what I was eating once was a thing which could feel pain, but this evolved over time. I started looking into my family history and a solid chunk of my family are/were vegetarian and they did not have various health problems with tend to appear in my family (diabetes, mostly). But, I am not saying that this is healthy for everyone, just how it is for my family. Abstaining from animal products for health reasons is a relatively new concept and there is some evidence to back up the claims that limiting the consumption of meat is good for you, but it's far from common. Looking at the examples I just gave and the current trends, we see that people are vegetarian for moral reasons. But, what are those reasons?

With the exception of some stances which may or may not entail a vegetarian diet, by far the most common reason for people to adopt this stance comes from some moral intuition. In this case, the intuition comes from a concern for animal welfare. But (and this might seem obvious), what makes eating meat violating to animal welfare? Put into ordinary terms, this stems from the idea that animals suffer, they feel pain. The most appropriate moral stance to take on this sort of consideration is consequentialism. Consequentialism says that causing more pain than necessary for some greater good is wrong. The next aspect comes from the realization that the process of killing and butchering animals for food does cause more suffering than necessary for some greater good. So, the process of killing and eating animals for food is morally wrong. Since Consequentialism does not see a distinction between killing and letting die (I don't), eating the animals is a cause for their death, so you are, in a sense, causing this suffering, which the reduced consumption would have prevented. So, eating meat is wrong.  

\subsection{The Animal Suffering Argument} 
\begin{earg}
    \item[1] Animals feel pain.
    \item[2] Causing pain when it lacks a pleasure which outweighs it is always morally wrong.
    \item[3] Killing an animal for food causes more pain than pleasure (killing animals for food is a case of (2))
    \item[4] Therefore, killing animals for food is morally wrong.
\end{earg}

It should be noted that the first line of the argument is not necessary, it's sort of built in to the third line, but it makes it clearer for what Descartes is going to be replying to. This argument serves as a good backdrop for discussing whether or not we should all be vegetarian/vegan. Morally speaking, we are in a situation where, though we come from meat eating ancestors, we don't need to do that anymore to survive, so it might be time to remove it.  As is always the case, if a person makes a claim, they need to back it up with reasons. For example, if someone was to claim that Peter Quill dating Gamora\footnote{Guardians of the Galaxy part 1} is morally wrong, we would ask for some reasons. If someone claimed that the earth is flat, we would ask for proof, reasons, same as in the case of it being round.  The Animal Suffering Argument is a set of reasons for us to be vegetarian/vegan and the reasons are structured in a valid way. Since the argument is valid, we need to look at the facts of the case. Now, what we need to do is see what objections the opposing side may have (but if they want to say that everyone should not be vegan, then they need to make an argument for this).

\section{Part \thechapcount.\theseccount: Descartes' Reply}\stepcounter{seccount}

Descartes, who you may remember from this class, had an interesting reply to this argument. As I mentioned in the previous section, this line was not necessary but it's mostly there to help you see the reply which Descartes is making. Basically, Descartes claims that animals don't feel pain. Without the first line, Descartes would be presented as claiming that killing and eating animals for food does not cause unnecessary suffering; not because we `need it to survive', but rather because it does not cause suffering.  

\subsection{The Mind-Body Problem and Ethics}

Remember that Descartes was our case of a substance dualist. He held that there were two kinds of things in the world, minds and bodies. Suffering, pain, emotions in general, are all mental, not physical (though the reason we think others have these is because of the physical responses given). If you are a physicalist, of any kind, then you might not find this persuasive.\footnote{ It is possible, however, to be a physicalist but claim that only the human brain or greater is complex enough for consciousness to emerge or some such like that.} From substance dualism, we get that suffering or any other kind of experience, positive or negative, is something which requires a mind/soul to have.  If Descartes were to say that animals suffer/feel pain, then he would need to say that they had souls/mental substances. This was a step too far for Descartes. Descartes thought that humans had souls, but animals did not. Going outside of what Descartes thought, we could get reason to think this as a reply to a particular version of the Problem of Natural Evil (concerning evolution), though Descartes might not have given this thought:
\begin{earg}
    \item[1] The process of evolution is necessary to get beings such as us with free will and moral aspects.
    \item[2] The process of evolution, if animals have souls, would result is an incalculable amount of pain and suffering.
    \item[3] An all good god would not allow for this sort of thing.
    \item[4] There is an all-good god who would want beings with free will.
    \item[5] So, animals don't have souls. 
\end{earg}
When I was in undergrad, there was a seminar which I took on the problem of evil as it relates to evolution, this was an argument which appeared there (put in my own way with my own thoughts) Since Descartes has his dualism, we can get that since animals don't have souls, they don't suffer. To make this point clear, Descartes did not argue this way and it was an argument I made up for this content, nothing more.

An injured animal, for Descartes, is nothing more than a broken machine. If animals do not feel pain, then there would be no suffering caused by the killing and eating of them. Poof, all moral reasons for not eating meat go away (when it comes to the animals themselves).  
\section{Part \thechapcount.\theseccount: Some Replies to this Objection}\stepcounter{seccount}

Here I am going to give a few of the replies which people could give to Descartes' idea that animals don't feel pain. These come in three different flavors, each of which we have seen in various modules throughout this course, not necessarily the Mind-Body Problem.
\subsection{Physicalism:}

Remember what was said about Physicalism. This was in the Mind-Body Problem. This is the stance, roughly, that there aren't things like souls. This puts a hole into Descartes' reply to the Animal Suffering Argument. The physicalist agrees with Descartes that animals don't have these mental substances, but neither do humans. They claim that there are no mental substances, only physical ones. This points a whole in the core assumption in Descartes' argument. Whatever way the physicalist has to explain mental phenomena, such as pain, will likely apply equally will to sufficiently complex animals.  One of those is clearly flawed.
\subsubsection{Related Objection:}

Following this line of thinking, concerning the Mind-Body Problem, if we look really closely at the dualist stance, we see that there's nothing built into it which entails that non-human animals don't have souls and that human persons are the only ones with them. Descartes snuck this assumption into the stance and there's no reason to think that it's correct. There are, potential, arguments for it, bringing in things like the Free Will Debate or other assumptions, but these are equally warranted to pose objections to them. Similarly, there's nothing in dualism which states that souls have these properties and there's only one kind of soul. There could be a series of different kinds of souls with different experiential capabilities, much like there are different brains with different levels of complexity. Humans, as far as we know, just so happen to have the highest quality soul, so to speak, and even some humans lack those. 
\subsection{Darwinism:}

This objection comes in part from the arguments for and against the existence of God. Darwinism, evolution, gave us a way to explain the complex structures in the world without the need for a divine architect. But, it also gives us a way of working backwards to get that animals feel pain.  This does not necessarily require physicalism, but that stance fits nicer into the idea here.  Creatures on this planet evolved from a common ancestor which had certain features, these features remained because it was adventitious. Having the capacity to feel pain would be adventitious for creatures to have and the beings which evolved from the common ancestor with that trait would have retained it. So, at least some animals feel pain, in virtue of the relevant features they share with humans, like certain kinds of nerve endings and brain structures.

It could be said that a soul evolved in conjunction with these complex brains necessary for pain, which would mean that they came in degrees, or it could be an addition to a physicalist model. Either way, we get this.
  
\subsection{Utilitarianism:}

This particular reply is mostly here to point out how unintuitive the idea which Descartes is proposing is. Objections like this, in Philosophy, are called the ``incredulous stare." These are cases where the stance is internally consistent, we can reason and debate through it, but it's so wild and out-there that we just don't know how to reply aside from staring at them in disbelief.  When we think about the rightness and wrongness of our actions, often, the pain and suffering we expect to cause fits into the reasoning. For example, Jeremy Bentham once wrote ``The question is not, Can they reason?, nor Can they talk? but, Can they suffer?".\autocite{Bentham1}  Though it may be true that they can't suffer to the degree which we can, it seems that they can still suffer to some degree. In many people’s calculations of the consequences of actions, animal suffering does play a role. Though it may be seen as more basic (bodily) than the ones humans can experience, they still count for something. As a result, a sufficient amount of suffering could outweigh the bodily pleasure which the fried bacon gives. 
\section{Part \thechapcount.\theseccount: To Get Out of These Worries}\stepcounter{seccount}

If you wish to claim, as Descartes does, that animals don't feel pain and avoid these worries, you will need to do a few things. First, you will need to reject the idea that species can change over time (also called evolution), and in doing that (to stay consistent), you will need to reject the idea that things change over time, which is easier for some than others. Second, you will need to say that all people who think about animal suffering in their moral reasoning are wrong, which is harder for everyone. You can do this by either accepting Kantianism (which has another way of getting at the animal suffering argument later) or by claiming that animals aren't worth adding to the moral figuring (either way, me kicking an unowned dog is morally OK). And third, you will need to be a dualist, but one which rejects the idea that animals have souls and, because of that, you will need to claim that there is no doggie/kitty heaven.

\section{Part \thechapcount.\theseccount: Speciesism and Anti-Speciesism}\stepcounter{seccount}

Speciesism is another way to get out of the argument, basically meaning that you can have as much meat as you want and not be a moral monster. Speciesism is a fancy term and it should remind you of terms like ‘sexism’ and ‘racism’. It has a very similar origin in how we use it. For example, racism, roughly, is treating one group differently than another group solely in virtue of their race (assuming all else is equal). For example, it would be racist to put a Mexican person in jail for jay-walking for 6 months (and that's the only offense) while only putting a white person in jail for the same crime for 1 month. Similarly, sexism, roughly, is treating one group differently than another solely in virtue of their sex. For example, (assuming that all qualifications are equal) it would be sexist to give a man a promotion over a woman, solely because he is a man.\footnote{These examples also apply in reverse, it would be sexist to give a woman that promotion merely on the grounds that she is a woman. We would need to find other qualifying factors which should be taken into account, otherwise flip a coin.} Speciesism follows a similar line of thought. Speciesism, roughly, is treating one group differently solely in virtue of their species. For example, there used to be a Comedy Central show called Ugly Americans in which the core premise is that the various monsters (zombies, vampires, demons, etc) are real and are trying to integrate into New York. Refusing to offer a job to a demon solely because they are a demon would be speciesist. Almost any Science Fiction or Fantasy story where there are more than one intelligent species could be used to make examples, all following the same line of reasoning as racism and sexism. But, speciesism does not just apply to these strange, unreal, stories, but it also applies to the real world as well. Speciesism would tell us to favor the interests of a human child over an ape, solely because they are human, no other reason is necessary. Though there may be other reasons, those reasons are extra. There are some reasons to be a speciesist, and many people think that reasons like these are persuasive: 
\subsection{Arguments for Speciesism}

For this, if these arguments are persuasive, then you are being speciesist. Applying this to the Animal Suffering Argument, they claim that the suffering of an animal only counts morally when humans are not involved or aren't benefited. If the animal suffering benefits humans, then the benefit to humans is all that matters. 
\subsubsection{Christianity/Abrahamic Stances:}

This does not just apply in Christianity, but also in most religions which have a similar historical origin (Islam, Judeism, and others). In such belief systems, you have a passage or belief that God created man and gave them dominion over the world, as in God created humans with the explicit feature that they are greater than all other beings, and as a result all humans are worth more than animals. Morally speaking, no amount of animal suffering (if they actually have it) counteracts human pleasure. But this stance does have its issues, as we will see.
\begin{earg}
    \item[1] God gave man dominion over the animals.
    \item[2] If (1), then they aren't equal.
    \item[3] If they aren't equal, then animal suffering, if beneficial to humans, doesn't count morally.
    \item[4] Therefore, animal suffering, if beneficial to humans, doesn't count morally. 
\end{earg}
This argument, however, does have its issues. For example, this assumes the existence of God, which is debatable, but it also assumes that dominion means that we can use them as we see fit without moral consideration. This might not be the case. For example, just because I own a dog, I have dominion over them, but that doesn't mean that I can treat the puppy poorly.  
\subsubsection{Tu Quoque:}

This is a Latin term for ``you also". Think about it this way, imagine that a parent is a smoker and gets really angry when they catch their 21+ year old child smoking; a natural response is for the person to say ``well, you smoke, why can't I?" Basically, if you do it, I can too. This is a fallacy, just because another being does something, it doesn't make it OK. In the animal kingdom, we don't see other animals treating other species without favorism towards their own (as in, we see animal favoring members of their own species over members of different ones). This raises the question ``why shouldn't we favor our own, solely because they are our own?" 

The issue with this line of thinking is that it's a fallacy, the core root of the reasoning doesn't generalize well. For example, (this is an example given to me by former military students) in some cases, the enemy combatants in war will not treat our fallen soldiers with respect. The question here is ``are we allowed to treat their fallen disrespectfully?" Unless you are particularly hardened, the reply is ``no, we are better than that." And that's the point, we are better than that, we need to hold ourselves to a higher moral standard.
\subsubsection{The Children Analogy:}

This is an example that I tend to give as an example in Feminist Ethics, but it works quite well here. There, it's used to show that bias can be a morally good thing and being impartial wrong (Feminist Ethics tends to have a level of bias built in as a good thing). This example tends to work best for people who are/were the primary care giver to children (the mom/Mr. Mom).

    \factoidbox{Suppose that your child and their friend are swimming in a lake, due to the cold water (common in Washington), they both begin to have difficulty breathing as they swim, resulting in them starting to drown. You can only save one, who do you save?}

For people who raised children, this tends to be a very easy answer ``I save my kid." Feminist Ethics uses examples like these to argue against that impartiality is a feature of correct moral thinking, but here, we can use it to work in another way. People tend to favor their own children when their interests conflict with others. We don’t see this as morally wrong. A very similar case can be seen in favoring members of your species, why is that seen as wrong? For example, lets look at this case, the Chimp or Baby Case:

    \factoidbox{Suppose that a fire has broken out in an apartment building. Everyone has gotten out with the exception of a small child and a chimp. We will suppose that the child has certain disabilities which makes them the mental equivalent of the chimp. Yet again, you can only save one. Is it wrong for you to save the chimp?}

In this case, if you say that it is wrong to save the chimp, then you are being speciesist. 
\subsection{Anti-Speciesism}

Anti-speciesism, on the other hand, does not imply that when choosing between a dog and a baby to save, you should flip a coin, but rather you should make your choice based on factors other than whether it is a member of your species, like the capacity for pain. When it comes to moral considerations, the species of the creature does not play a part. This is central to Consequentialist style ethical thinking, You can also think of it as that the painful experience of one creature should be weighed the same as the same experience had by another creature. Humans are not special in the grand scheme. Rather, we are only special in that, as far as we are aware, we are the only beings which can experience certain kinds of mental suffering. So, in the Chimp or Baby Case, the anti-speciesist would say that the two lives are equal, but in the case of choosing between a worm and a human baby, you would choose the human baby, because that creature is the kind of thing which has a greater ability to experience happiness and contribute goods.  It is certainly possible, however, to have an ample-enough number of creatures to outweigh the value of a human, for example, it might be possible that 5 chimps equal 1 human, or some other conversion system, depending on the context of the case. 

\section{Part \thechapcount.\theseccount: Animal Rights}\stepcounter{seccount}
Here we are going to be talking about rights. But, rather than talking about human rights, which is an entirely different course I teach, we will be looking at whether or not there's such a thing as `animal rights'. Now, I don't know whether or not this is just a story or it actually happened; but regardless it makes a good story: Wesley Hohfield tried to make law students carefully figure out the different ways that the word ‘right’ was used in American law. This made a lot of people angry and his students even tried to get him to lose his job as Chairman. 

The lesson from this is that, since law students don’t like the analysis of the various kinds of rights which we can have (passive, positive, legal, moral, etc), we should not be shocked that similar break downs are not liked by politicians, writers, and political theorists. Some have a vested interest in not using the term `right' in various contexts, because it carries a certain weight, while others have a vested interest in applying the term in cases where it does not apply, for the same reasons. Some politicians can even have a vested interest in keeping that kind of talk (about rights) as meaningless as possible.

'Rights’ is a term which gets thrown around a lot and often in inconsistent ways (for example the `right to own a gun' does not seem like the same sort of thing as the `right to freedom of expression'). But, for our discussions here, we will use this basic condition, one which is seemingly universal:

\begin{center}If something has a right to X (either do something or have something), then others have a moral duty to ensure that they have X.\end{center}

This basically says that if you have a right, then others have the moral duty to make sure that it's not infringed upon. For example, if you have the right to vote, then I and all other people have the moral duty to make sure that you can vote (it's morally wrong for us to let your right be taken away). Similarly, if you have the right to life, all other people have the moral duty to make sure that you don't get killed. If you have the right to a fair trial, then other people have the duty to make sure that you get it if needed.

But all of these examples apply exclusively to people, beings of a certain intelligence and maturity level. Some people want to extend the notion of rights beyond people and to non-human animals. Such people are often found in animal rights movements. For example, some might claim that animals, like people, have a right to life. This would mean that all others have a moral duty to ensure that they don't get killed or at the very least they have a moral duty to not kill them. Others might claim that animals have a right to a certain standard of living (like people, but this one is not commonly talked about). This would mean that people have the moral duty to not destroy their habitat. Other examples can be made and the extent of the moral duty would depend on the actual range of the right, same with that of human rights. 
\subsection{Some criticisms of animal rights}

There have always been, to some degree or another, in various cultures around the world, people who think that animals have at least some rights. They could argue that rights come in levels. If a being has certain features, then it has rights and the rights of those beings below it on the level. Human persons would be at the top of the chart, so to speak, so we get all of the rights, But, say, chimps and dolphins get some but not the right to vote or some such. This would be much like the features of person-hood found in the Abortion Debate Module. That being said, there are some criticisms directed towards the very notion of Animal Rights which would apply regardless of the system for determining which rights would otherwise be had. 
\subsection{Rights Imply Responsibilities:}

If animals have rights, then I have a moral duty to see that they have whatever the right is. This comes from the definition of rights which we are using for this section. But, at the same point, when we look at the various uses of the term `rights', we see that it also implies something else. In this case, having rights gives us responsibilities. For example, if you have the right to own a gun, then you have the responsibility to use it properly and in the right kind of cases or if you have the right to vote, then you have the responsibility to know the details about what you are voting on (if you do so). If you can't handle the responsibilities which come with a right, then you don't have that right.  We can’t say animals have rights because they lack the responsibilities, or the ability to fulfill those responsibilities, which come with those rights. 
\subsubsection{A Reply}

To this idea, many people will point to how some humans lack the ability to take on the responsibilities of having various rights and yet still have them. For example, some claim that people in a permanent vegetative state have the right to life, but they lack the ability to take on the responsibilities of life itself. Children who cannot develop past a certain point due to genetic diseases or some such are still said to have rights despite not having the moral ability to have moral duties. We even have various systems in place which determine whether a human has certain rights or not depending on their intellectual abilities. Animals are no different. 

\section{Part \thechapcount.\theseccount: Direct Vs Indirect Duties to Animals}\stepcounter{seccount}
As I mentioned in the onset of the last little section, when a thing has rights, we have a duty to ensure that they get or have the ability to do whatever the right is. Immanuel Kant, who you know from Kantianism, held that animals do not have rights, as they are not people, which means that we don't have direct duties to them, but rather we have indirect duties to them. When a being has a right to something, others have the duty to them directly  to ensure that they get what ever the right implies. But, not all duties are direct, sometimes we have duties because of other duties. For example, (though Kant may not like this), I have the duty to feed my children, which means that I also have the duty to make money and buy them food. My duties there are indirect because I have them to complete other duties. My duty not to lie to you is a direct duty because it's one step removed from a person, you. My duties to (certain) animals are indirect because they are (at least) two or more steps removed from a person. My obligation to not kick your dog doesn't come from the animal, but rather it comes from you. I have a duty to you and that duty entails a duty to your animal. 

\begin{center}It is wrong to harm certain animals because we have a duty to ourselves or other humans to not harm them.\end{center}

For Kant, needlessly harming animals is wrong, not because of the pain, but because of how this is damaging to the character of the person doing it which we have a duty to ourselves/others to not harm. Maintaining certain character traits is a duty we have to ourselves because it leads to us better following the categorical imperative.  According to Kant and a lot of popular culture, a person who beats/tortures animals is more likely to beat/torture humans, which would be using them as mere-means. Maintaining your character as an upright person prevents you from falling short on your duties to other people.
\subsection{Some Replies to Indirect Duties}

This notion of indirect duties, or that we have duties at all, seems to run into criticisms here and there. For example, when I give students certain ethical problems concerning the environment, they want to say that we don't have duties (meaning that it's morally wrong for me to fail to do the thing) at all, saying that doing it is going above and beyond. But there are other, more common criticisms which can be put against Kant's ideas here. 
\subsubsection{Implausibility:}

If you in the Ethics module (Module \ref{ch.modeight}) did not find Kantianism particularly pursuasive or you thought that it was loony-tunes and found the Consequentialist line of thought more appealing, then you will need to handle the Animal Suffering Argument in some way, but you don't need to think that we have duties to animals. Consequentialist thinking does not have room for the ideas and reasoning which Kant gives.

For the Utilitarian and other Consequentialists, the idea that morality rests on duties, indirect or otherwise, is just missing the point. 

This is because they think that morality rests on the consequences of the action, not the character of the action or some other thing about human autonomy or some such. Morally speaking, rights are there because of the consequences of people having them, they don't come pre-built in, which leads to animals potentially getting them too. 
\subsubsection{Untested and Unconfirmed Empirical Claim:}

You have likely heard of cases where particularly heinous serial killers started with animals first. This might make you think that the disposition to harm animals is a slippery-slope to harming people. But, this is, frankly, an empirical claim, it's the sort of thing which scientists would need to perform tests and confirm. As far as I am aware, there is no data which links the two. But this does link into an interesting discussion of a certain kind of fallacy. Lottery groups frequently fill the airwaves with pictures and stories of people winning the lottery (we will see this sort of fallacy again when we talk about political philosophy). These grab our attention and make us think that the winners are in, some way, close to us. This makes us think that winning the lottery is very common. But, in fact, it's quite rare.

Similarly, when it comes to serial killers starting with animals, we all know stories of this happening and it makes us think that it's common for people who harm animals to become serial killers, but it is be quite rare.1 Does a willingness to treat animals as mere-means lead to a willingness to treat humans as a mere-means? Are farmers who beat their horses in drawing their carts more likely to beat their wives/husbands?
\stepcounter{chapcount}
\chapter{Part \thechapcount: What Are Our Duties to Future Generations?}\setcounter{seccount}{1}
When I say ‘future people’, I mean people who aren’t born yet, who, unless something terrible happens, will exist a few generations down the road. People who aren’t even a twinkle in their pappy’s eyes yet. Although I can’t be sure that any one of them will exist, I can be sure that some will exist. One way to think about this interesting point is that there, more than  likely, will be people in the future, but I can't say with certainty who those people will be, what individuals will make up the collection of people out there. We can point to things like chaos theory1 to explain why we can't know who those people will be.  Now, our question is ``do those people, people who don't exist, but likely will, have rights? Are they the sort of things worthy of our moral consideration? Do we need to be concerned about their welfare?"

Some of you, I am willing to wager, will say that this is a no-duh sort of question, but, like with the Abortion Debate, we need to look at the reasons. Just as in that case, both sides will say that it's obvious. Some say, the consequentialists being the strongest voices there, that we have the same obligations to future people as we do currently existing people. Just because they don't exist, they will, so we need to ensure that the best future is there for them. Others will say that we don't have any obligations to future people. Here we could find the non-consequentialists. They would say that, for example, the being must exist in order to be treated as a mere-means; so we don't have obligations to them. We can only have duties to contemporaneous persons. But this debate and difference leads us to this interesting topic:
\section{Part \thechapcount.\theseccount: Intergenerational Justice/Ethics}\stepcounter{seccount}

Trying to figure out the morality of actions concerning future people is an area of philosophy called ‘intergenerational ethics’. When it comes to their rights and the duties we have to them, this is `intergenerational justice'. Since the consequentialist is not too much of a fan of rights and doesn't, fundementally, think about morality in terms of duties, the consequentialist will tend to work in the more general intergenerational ethics, while the non-consequentialist will tend to work in the more particular intergenerational justice. However, there are certain powers which the current generation has over the future generations which are not had by them over us. These powers add variables into the typical equations which we would use for the morality of actions which lead to both epistemic and metaphysical questions which need to be addressed in order to figure out what the right course of action is. There are three such powers, with the third being the greatest and most perplexing problem, especially for the consequentialist.
\subsection{Limiting choices:}

The current generation can set up a system which would be very costly for the future generations to change and, essentially, force them to continue with that system. For example, what if we made various choices which placed the future generations into an extreme economic debt, this would force them to maintain certain policies in order to pay off said debt. Or, on a smaller scale, what if we bought houses which were tied to familial wealth with a 200 year long repayment plan? This would force the future generations to live in that house, unless they got very wealthy rather quickly. Future generations are not able to place that sort of burden on us, they can't force us to do anything (with the exception of if time-travel happens, then they could). But the limitations don't need to just be economic, they can also, for example, be intellectual. In the case of intellectual limitations, we can greatly advance in one direction which would make the alternatives woefully under-researched, making switching to the alternatives very hard because it would require the future generation to back-track and start the research from scratch. Here are two examples: 


\thoughtex{The Green Dictators}{Suppose that we know that the future generation will need to continue pursuing alternative energy sources and not transition back to coal/oil power for electricity/transportation. As a result, the current generation signs multiple treaties which make it very difficult for the countries to back out and have extreme penalties for refusing to comply with this green agenda.}{GreenDictators.jpg}{A green city overlooked by a dictator made out of plants.}

\thoughtex{The Cheapest Route}{Suppose that we build an infrastructure, roads, electrical lines, waterways, etc. with the easiest materials and power-sources available to progress very quickly and have great advancements. This infrastructure gets so ingrained that the transition to other sources of power and other infrastructure methods is very expensive and underdeveloped compared to where it would have been.}{CheapestRoute.jpg}{A contemporary looking city with many cars and garbage everywhere.}


For both of these cases, we have constrained the future generations to take a path which they did not choose for themselves. They could not have given consent or a voice in the process. So, we could, maybe, say that they were coerced into the system. But, some might say that we did the right thing when it comes to The Green Dictators and the wrong thing in The Cheapest Route. This difference must be because of the consequences. But, we are still forcing another group to follow our choices, much like the Cultural Imperialist. As an interesting side note, regardless of the path we choose, we are still imposing our wishes onto the future generations.
\subsection{Unidirectional Benefit:}

This is a sort of reiteration of the `Limiting choices' ability which our generation has on the future ones. It is possible for the current generation to benefit themselves at the expense of the future generation and the future generation will experience all of the cost and, in some cases, none of the benefits. In such cases, the current generation will enact policies or engage in behaviors which will benefit them in the short term, be costly in the long term, and be long dead by the time those bills come due.  The Cheapest Route case works for this example if we add in that the benefits of the rapid expansion are less than the costs to the future generations. But there are other cases:

\thoughtex{The Origins of the Automobile (this is sort-of fictional)}{When automobiles (cars) were first becoming a product which the average American could buy, there were three different generic kinds. First, they had the steam-powered car. There were several companies producing these and they had the benefits of being familiar technology as well as able to go long distances. They weren't bought up as much because they didn't have the `get up and go' and the speed of the other options. Second, there were the electric cars. Women consumers really liked these because they were just a push-button to get started, but they did not have the speed nor could they go long distances. And finally, there were the gasoline cars. These, with the introduction of the electric starter, had the quick start, the speed, and the ability to go long distances. Men, who were the predominate buyers, really liked these. That generation collectively mostly going with gasoline cars greatly benefited them,  but it resulted in great costs to the future generations, in the form of Global Warming.}{ElectricCars.jpg}{Several women in a old-timey electric car.}

\thoughtex{Multi-generational Mortgages}{In some areas of the world, there are multi-generational mortgages 100+ year plans (the most famous are Japan’s 100 year mortgages, but Sweden has 105 year plans). In such cases, the future generations will be forced to pay the debt off of the house or the loan with the current generation getting the money from the loan. This will also force those generations to live in those houses until the house is paid off and able to be sold, some cases they are forced to sell the home to pay it off.}{GenerationalMortgage.jpg}{Several generations of a family crammed into a relatively small house.}


As I said previously, there are three, general, powers which the current generation has over the future generations which they don't have towards us. But this third problem deserves a page all to itself, because it leads us into a very complex and mind-bending problem, especially for certain kinds of consequentialists.

\subsection{Power to bring them into being:}

Not only do we have the power, without any resistance from them, to essentially make them our slaves long after we have taken to dust, but we have the power to actually create them. We have the power to influence and alter what individuals will come into being and how many of them will come into being. We could, though unlikely, completely destroy ourselves, making no future people possible, or we could have a giant baby-boom which makes a ton of them. Even little things, which are totally coincidental or seemingly unrelated can result in an entirely different crop of future people being produced than if it hadn't happened. Take these three cases:

\thoughtex{The Black-Out Baby Boom}{In certain areas in New York, there was a time when the places would shut down because everyone was watching Friends. On one such day, a young grad student and his wife were settling down to watch and there was a power outage, so, they grab a few blankets, light the candles… 9 months later, a baby is born. This was, to be exact, the August 14th Northeast Blackout in 2003.}{BlackoutBabyBoom.jpg}{A man and woman cuddled up under a blanket with candles.}
	
\thoughtex{The Iron Maiden Baby Boom}{In Brazil and several South American Countries, when Iron Maiden did their world tour, on Flight 666, there was a spike in births in the cities which they visited 9 months later, in order of their visit. (This is a bit exaggerated, but Woodstock had a similar effect.)}{IronMaiden.jpg}{Iron Maiden's Eddie looking over a large concert.}
	
\thoughtex{The Sport Event Baby Boom}{In 2005, the Red Sox beat the Cardinals in a match after a 85 year losing streak. This resulted in a spike in the number of births roughly 9 months later. FC Barcelona's win in the 2009 UEFA Champions League semi-finals caused a 16.1\% jump in the birth rate in Barcelona 9 months later.\autocite{baby2} This was directly attributed to the victory. On Nov. 2nd 2016, the Chicago Cubs, a Baseball team, won the world series after a 108 year losing streak.\autocite{baby1} There was a lot of celebrations in the city. As a result, roughly 40 weeks later, there was a massive increase in births.}{SoccerVictory.jpg}{People celebrating the victory of their favorite sports team.}

When we are talking about intergenerational ethics, we do run into a bit of a mess. Though Consequentialism does have the better methods for handling these sort of cases, it falls flat in a few regards. The reason is that, for all of the ethical theories we have discussed, there is always an individual, or group of persons, which is/are being affected. Keeping with Consequentialism, since it has the easiest time handling these cases, we will say that an action is wrong only if it makes the well-being of a person/group of persons worse than otherwise. For example, when we are talking about current people, we say that something is wrong when they are in a worse situation than some other action which could have been taken. When thinking about future generations, sometimes, the actions we take directly cause them to exists. Those actions, so long as their lives are better than not existing, can't be said to make their lives worse off than otherwise. But, we still, sometimes, want to say that the acting generation did something wrong. This is a paradox, each line seems intuitive, but in combination, they can't work. Put clearly, we have The Non-Identity Problem.
\section{Part \thechapcount.\theseccount: The Non-Identity Problem}\stepcounter{seccount}
As I said in the previous page, there are three, general, powers which the current generation has over the future generations which they don't have towards us. There powers, collectively, lead to a uniquely difficult situation to think about morally. But, the third power, the power to bring them into being, generates a very complex and mind-bending problem, especially for certain kinds of consequentialists. 
The Non-Identity Problem
\begin{earg}
    \item[] An action is wrong only if it makes a person (morally relevant being) worse off than otherwise.
    \item[] An action which brings a person into existence such that they couldn’t have been made otherwise and their life is at least marginally better than never being born, can’t have made them worse off than otherwise.
    \item[] There are some actions which bring such a being into existence and are still wrong.
\end{earg}
Since this is a paradox, each of these lines needs to seem intuitive. So, I will give some examples or arguments for the first two and in doing so, show that the third line is intuitive, just makes sense or seems true. 
\subsection{An action is wrong only if it makes a person (morally relevant being) worse off than otherwise.}

This is the first line of the paradox and to make it seem intuitive, I could reuse ton of the examples which I have already given in the cases concerning Utilitarianism (consequentialism), but for this one, I have chosen to given another example. This example is quasi-historical, as in all of the events took place, but there were other factors involved (which, frankly, only make the actions more wrong). This particular line, however, does have an element of a person-oriented nature to Ethics. It's basically saying that there needs to be a person (current or future) who is made worse by the action for it to be wrong. If there isn't a person (future or otherwise) made worse off, then the action isn't wrong. 
\subsubsection{Historical Example (Ethics concerning contemporaneous peoples)}

    \factoidbox{George Forde\footnote{Yes, George Forde was a real person, in real life, he was a preacher.} was a farmer in Ireland in 1845. Due to the oppressive practices of the British, he and his family survived on a regular potato crop and the milk of a single cow. This was quite common for his region and in fact, this diet made the people in his region healthier in several regards than their British counterparts. Then the potato famine struck. This was a blight which rotted the potatoes and made them inedible. Removing almost all of the food sources in the region. Other countries and regions throughout Europe were also affected, however, due to relief measures, it did not escalate very far. The British, for a time, did similar, and the Irish were doing quite well. But, a few years into the blight, the British powers saw this as an act of God and removed all support and relief measures.\footnote{For a more rounded explanation for the ending of support, the British Government saw this, in private, as a way to reform the moral character of the Irish people and used a sanctimonious cover to end the aid (which was, through convoluted machinations, paying for itself).} This lead to Forde contracting diseases due to malnourishment and starving to death, like approximately 1 million others.}

The British ending their relief programs and aid lead to mass starvation and death. They could have, easily, continued those aid programs and saved more than a million people from a slow death. The vast majority of people will say that the British were wrong in ending their aid. This is, in part, because we know the results of it looking back and, even in that time period, they could have looked at how the other countries were handling the blight. We can generalize this, as we have done in other parts of this class, to any action, which, if it follows, makes this line of the paradox seem correct. 
\subsection{An action which brings a person into existence such that they couldn’t have been made otherwise and their life is at least marginally better than never being born, can’t have made them worse off than otherwise.}

I know that this line of the paradox is a bit long and seemingly convoluted, but this is because I need to give some examples for it to make sense. Essentially it is saying that a person can't be made worse than otherwise by an action which was necessary for their creation, so long as their life is better than never being born. Because of the precariousness of a person's existence, as can be seen in my above examples, there are several events (and the actions/situations which caused them) which need to go just right for any person to exist. However, if those events don't happen just right then they will not exist. This means that if we choose to take a different route than the one which will result in a person, their `good life points' would be 0, rather than whatever they would have been had we chosen to take the other route which would have made them. So long as their life is better than 0, they could not have been made worse by the choice. Take this as an example:

    \thoughtex{Bio-dome Bob}{Bob is a person living on Earth in a bio-dome in the far off future, breathing recycled air, unable to go outside because the O-zone layer has long since been whipped out. He eats rationed food because farming is impossible in this environment. Earth has essentially become Venus. Bob’s life is not the best, but there are enough pleasures to make it better than non-existing. In this future, the past generations chose to continue with the non-green policies, not caring about global warming or the environmental effects. Bob’s ancestors, in particular, were coal-mine executives, who were able to ensure that their kids meet certain people because of the profits. Bob could not have existed if the green policies were enacted.}{BioDomeBob.jpg}{A man in a bubble on a barren-looking plot of land.}

We can give Bob's life a score, this is an arbitrary choice which we can use as a metric for later, Bob's life has 10 good-life points. In order to exist, the previous generations needed to not enact those green policies. If the previous generations had gone with the green policies, then Bob's good-life points would have been 0 (because he would not have existed). So, since 10 > 0, Bob's life is not worse than it would have been otherwise. As more generations take place, I can generalize this to all people in the future because of Chaos Theory and things like the Butterfly Effect. This means that for an action which makes a person who couldn't have been otherwise and has a life better than non-existence, can't have made that person worse off than otherwise. 
There are some actions which bring such a being into existence and are still wrong.

This is our third line to the paradox and it seems like the most intuitive of the bunch, but maybe less intuitive than the first line. For this one, we can look at several examples to justify it. But, really, it's just the underlying intuition behind things like wanting to prevent global warming and why we would think that the previous generations in the Bio-dome Bob case did something wrong. To start on potential ways out of this paradox, take this case which is based on my Bio-dome Bob case:

    \thoughtex{Sunshine Sally}{Sally is a person living on Earth in the forest 100 years from now, breathing air from the trees, happily playing outside. She eats food grown from her garden and can drink the water from the tap. Sally’s life is awesome. In this future, the past generations chose to change their policies to green ones, caring deeply about the environmental effects and global warming.  Sally’s ancestors were pioneering wind-farmers, the move to green policies caused them to make a bunch of money and be able to send their kids to different places and meet different people.}{SunshineSally.jpg}{A forested city with moss and trees covering all of the buildings.}

We can say that Sally's life is, roughly, 10x better than Bob's. If you were to choose which life you would like to be born into, you would choose one like Sally's over one like Bob's. In this case, we will give Sally's life a score of 100 good-life points. But, same as Bob, if the previous generation had chosen differently, she would not have existed. And in this case, she would have had a good-life score of 0, because she was never born. 

\section{Part \thechapcount.\theseccount: Two Potential Ways Out (there are problems with each)}\stepcounter{seccount}

If we take Sunshine Sally and Bio-dome Bob as our token cases of people in these two far distant futures, we can’t say that the people individually were worse off than otherwise, we can only say that they are better off than otherwise. But, which do we choose? Neither is wrong (given our situation). The trick is to say that what makes something wrong is not, necessarily, making an individual worse, but rather making the collection, the group, worse. Sometimes this is a group of only one person, in which case it’s the same individualistic intuition, other times it’s more than one. So, for this, we are going to reject the first line of the paradox, but not become non-consequentialists, rather say that we need to take the collective well-being. 
\subsection{Averagism}

One way is to take the overall average of the future people and compare the averages. In Sunshine Sally’s world, the average person has a better life than the average person in Bio-dome Bob’s world, so we can say that this abstract, average person, is worse off than otherwise and that makes the choice the wrong one. But, there is an issue with this sort of account for the outcomes, although it works for cases like Bob's and Sally's, it gets the wrong answer for more down-to-Earth cases, like those concerning inequalities. 

    \factoidbox{Suppose that the average across the board for life-points is 100 (Sally's life) in some future, but this is because the top 1 percent of the population has all the points, while the lower groups have very few, if any, points, maxing out at 10 (Bob's life). But it’s a far worse world than one where everyone in fact had 100 (Sally's world). To drive this home, imagine that you needed to choose which world you would want to be born into, by chance. Would you take the bet on a world with  a 1\% chance of having a good life or a pretty close to 100\% chance of having a good life?\footnote{This case/thought experiment is a variation on John Rawls' Veil of Ignorance thought experiment. I generalized his thought experiment to concern future generations.}}

The vast majority of people, if they are sensible, would not choose to take the 1\% bet, that's just crazy. So, we can say that Sally's world is better. But, at the same point, what if I just increased the majority's max by just a few points? This would increase the average by a tiny amount, but make it better than Sally's world. This is an issue because a sensible person still would not take that bet. 
\subsection{Totalism}

Another way we could go about this is to have it just be the net total of the people in each world and say that the abstract group “Future People” is better or worse off because of that. In Sunshine Sally’s world, the total is higher than the total in Bio-dome Bob’s world, so, that’s what makes it wrong. This is called Totalism. But this one too does have its issues as well. For this case, the more people in the world, the better the world would look from an outsider's perspective, even though the individual lives in it are far worse.

    \factoidbox{Imaging that you have two possible futures, one where A) there are 1 million people each with 100 good-life points (a million Sallys) and another where B) there are 10 million people each with 10 good-life points (10 million Bobs). By totalism, we could not tell the difference between these two futures, and, in fact, if I changed B so that everyone had 20 good life points, totalism would claim that it’s the better world, even though everyone, individually, is worse off than in A.}

As before, we would not want to take this bet, if I were to give you a choice to gamble on where to be born and I told you to choose which of these two worlds you would `roll the dice' in, you would certainly choose A, even if I made B have everyone with 20 points rather than 10. This is the sort of issue which we saw with the Utility Monster Objection in the past.