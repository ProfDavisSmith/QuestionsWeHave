\part{Is Morality Culturally Relative?}
\label{ch.modseven}
\addtocontents{toc}{\protect\mbox{}\protect\hrulefill\par}
\chapter{The Challenge of Cultural Relativism by James Rachels}\autocite{Rachels1}
\label{challengerelativism}
\newcounter{fz}
\setcounter{fz}{\thefootnote}
\setcounter{footnote}{0}

\factoidbox{``Morality  differs  in  every  society,  and  is  a  convenient  term  for 
socially  approved habits.” \autocite{RuthBenedict1}}

\section{2.1 How Different Cultures Have Different Moral Codes} 
Darius, a king of ancient Persia, was intrigued by the variety of cultures 
he  encountered  in  his  travels.  He  had  found,  for  example,  that  the 
Callatians  (a  tribe  of  Indians)  customarily  ate  the  bodies  of  their  dead 
fathers.  The  Greeks,  of  course,  did  not  do  that—the Greeks  practiced 
cremation and regarded the funeral pyre as the natural and fitting way to 
dispose  of  the  dead.  Darius  thought  that  a  sophisticated  understanding 
of  the  world  must  include  an  appreciation  of  such  differences  between 
cultures. One day, to teach this lesson, he summoned some Greeks who 
happened  to  be  present  at  his  court  and  asked  them  what  they  would 
take  to  eat  the  bodies  of  their  dead  fathers.  They  were  shocked,  as 
Darius knew they would be, and replied that no amount of money could 
persuade them to do such a thing. Then Darius called in some 
Callatians,  and  while  the  Greeks  listened  asked  them  what  they  would 
take to burn their dead fathers' bodies. The Callatians were horrified and 
told Darius not even to mention such a dreadful thing.

This  story,  recounted  by  Herodotus  in  his  History  illustrates  a  recurring 
theme in the literature of social science: Different cultures have different 
moral  codes.  What  is  thought  right  within  one  group  may  be  utterly 
abhorrent  to the members  of another  group, and  vice versa. Should  we 
eat  the  bodies  of  the  dead  or  burn  them?  If  you  were  a  Greek,  one 
answer  would  seem  obviously  correct;  but  if  you  were  a  Callatian,  the 
opposite would seem equally certain. 

It  is  easy  to  give  additional  examples  of  the  same  kind.  Consider  the 
Eskimos.  They  are  a  remote  and  inaccessible  people.  Numbering  only 
about  25,000,  they  live  in  small,  isolated  settlements  scattered  mostly 
along  the  northern  fringes  of  North  America  and  Greenland.  Until  the 
beginning  of  the  20th  century,  the  outside  world knew  little  about  them. 
Then explorers began to bring back strange tales. 

Eskimos customs turned out to be very different from our own. The men 
often  had  more  than  one  wife,  and  they  would  share  their  wives  with 
guests,  lending  them  for  the  night  as  a  sign  of  hospitality.  Moreover, 
within  a  community,  a  dominant  male  might  demand  and  get  regular 
sexual access to other men's wives. The women, however, were free to 
break  these  arrangements  simply  by  leaving  their  husbands  and  taking 
up  with  new  partners—free,  that  is,  so  long  as  their  former  husbands  
chose  not  to  make  trouble.  All  in  all,  the  Eskimo  practice  was  a  volatile 
scheme that bore little resemblance to what we call marriage. 

But  it  was  not only  their  marriage  and sexual  practices  that  were 
different.  The  Eskimos also  seemed  to  have less  regard for  human  life. 
Infanticide,  for  example,  was  common.  Knud  Rasmussen,  one  of  the 
most famous early explorers, reported that be met one woman who bad 
borne  20  children  but had  killed  10  of  them  at  birth.  Female  babies,  he 
found,  were  especially  liable  to  be  destroyed,  and  this  was  permitted 
simply at the parents' discretion, with no social stigma attached to it. Old 
people  also,  when  they  became  too  feeble  to  contribute  to  the  family, 
were left out in the snow .to die. So there seemed to be, in this society, 
remarkably little respect for life. 

To the general public, these were disturbing revelations. Our own way of 
living  seems  so  natural  and  right  that  for  many  of  us  it  is  hard  to 
conceive of  others  living  so  differently.  And  when  we  do  hear  of  such 
things, we tend immediately to categorize those other peoples as 
``backward"  or  ``primitive."  But  to  anthropologists  and  sociologists,  there 
was nothing particularly surprising about the Eskimos. Since the time of 
Herodotus,  enlightened  observers  have  been  accustomed  to  the  idea 
that  conceptions  of  right  and  wrong  differ  from  culture  to  culture.  If  we 
assume that our ideas of right and wrong will be shared by all peoples as 
all times, we are merely naive.

\section{2.2 Cultural Relativism} 
To  many  thinkers,  this  observation—``Different  cultures  have  different 
moral  codes"— has  seemed  to  be  the  key  to  understanding  morality. 
The idea of universal truth in ethics, they say, is a myth. The customs of 
different societies are all that exist. These customs cannot be said to be 
``correct" or ``incorrect," for that implies we have an independent standard 
of  right  and  wrong  by  which  they  may  be  judged.  But  there  is  no  such 
independent standard; every standard is culture-bound. The great 
pioneering  sociologist  William  Graham  Sumner, writing in  1906,  put  the 
point like this: 
\factoidbox{The  ``right"  way  is  the  way  which  the  ancestors  used  and  which 
has  been handed down. The tradition  is its own warrant. It  is not 
held subject to verification by experience. The notion of right is in 
the folkways. It is not outside of them, of independent origin, and 
brought to test them. In the folkways, whatever is, is right. This is 
because they are traditional, and therefore contain in themselves 
the  authority  of  the  ancestral  ghosts.  When  we  come  to  the 
folkways we are at the end of our analysis}

This line of thought has probably persuaded more people to be skeptical 
about  ethics  than  any  other  single  thing.  Cultural  Relativism,  as  it  has 
been called, challenges our ordinary belief in the objectivity and 
universality  of  moral  truth.  It  says,  in  effect,  that  there  is  not  such  thing 
as universal truth in ethics; there are only the various cultural codes, and 
nothing more. Moreover, our own code has no special status; it is merely 
one among many. 

As we shall see, this basic idea is really a compound of several different 
thoughts.  It  is  important  to  separate  the  various  elements  of  the  theory 
because,  on  analysis,  some  parts  turn  out  to  be  correct,  while  others 
seem  to  be mistaken.  As  a  beginning,  we may  distinguish  the  following 
claims, all of which have been made by cultural relativists: 

\begin{enumerate}
\item[1] Different societies have different moral codes.
\item[2] There  is  no  objective  standard  that  can  be  used  to  judge 
one societal code better than another.
\item[3] The moral code of our own society has no special status; it 
is merely one among many. 
\item[4] There  is no ``universal truth"  in ethics; that  is, there are no 
moral truths that hold for all peoples at all times. 
\item[5] The moral code of a society determines what is right within 
that society; that is, if the moral code of a society says that 
a  certain  action  is  right,  then  that  action  is  right,  at  least 
within that society. 
\item[6] It  is  mere  arrogance  for  us  to  try  to  judge  the  conduct  of 
other  peoples.  We  should  adopt  an  attitude  of  tolerance 
toward the practices of other cultures.
\end{enumerate}

Although  it  may  seem  that  these  six  propositions  go  naturally  together, 
they  are  independent  of  one  another,  in  the  sense  that  some  of  them 
might  be  false  even  if  others  are  true.  In  what  follows,  we  will  try  to 
identify what  is  correct in Cultural Relativism, but  we will  also be 
concerned to expose what is mistaken about it.

\section{2.3 The Cultural Differences Argument} 
Cultural Relativism is a theory about the nature of morality. At first blush 
it  seems  quite  plausible.  However,  like  all  such  theories,  it  may  be 
evaluated  by  subjecting  it  to  rational  analysis;  and  when  we  analyze 
Cultural Relativism we find that it is not so plausible as it first appears to 
be. 

The first thing we need to notice is that at the heart of Cultural Relativism 
there  is  a  certain  form  of  argument.  The  strategy  used  by  cultural 
relativists  is  to  argue  from  facts  about  the  differences  between  cultural 
outlooks to a conclusion about the status of morality. Thus we are invited 
to accept this reasoning: 
\begin{enumerate}
\item[1] The Greeks believed it was wrong to eat the dead, 
whereas  the  Callatians  believed  it  was  right  to  eat  the 
dead. 
\item[2] Therefore,  eating  the  dead  is  neither  objectively  fight  nor 
objectively  wrong.  It  is  merely  a  matter  of  opinion,  which 
varies from culture to culture. 
\end{enumerate}
Or, alternatively:
\begin{enumerate} 
\item[1] The  Eskimos  see  nothing  wrong  with  infanticide,  whereas 
Americans believe infanticide is immoral. 
\item[2] Therefore, infanticide is neither objectively right nor 
objectively  wrong.  It  is  merely  a  matter  of  opinion,  which 
varies from culture to culture. 
\end{enumerate}
Clearly,  these  arguments  are  variations  of  one  fundamental  idea  They 
are both special cases of a more general argument, which says: 
\begin{enumerate}
\item[1] Different cultures have different moral codes. 
\item[2] Therefore,  there  is  no  objective  ``truth"  in  morality.  Right 
and  wrong  are  only  matters  of  opinion,  and  opinions  vary 
from culture to culture. 
\end{enumerate}
We may call this  the Cultural Differences  Argument. To many people, it 
is persuasive. But from a logical point of view, is it sound?

It  is  not  sound.  The  trouble  is  that  the  conclusion  does  not  follow  from 
the  premise— that  is,  even  if  the  premise  is  true,  the  conclusion  still 
might  be  false.  The  premise  concerns  what  people  believe.  In  some 
societies,  people  believe  one  thing;  in  other  societies,  people  believe 
differently.  The  conclusion,  however,  concerns  what  really  is  the  case. 
The trouble is that this sort conclusion does not follow logically from this 
sort of premise. 

Consider  again  the  example  of  the  Greeks  and  Callatians.  The  Greeks 
believed  it  was  wrong  to  eat  the  dead;  the  Callatians  believed  it  was 
right. Does it follow, from the mere fact that they disagreed, that there is 
no objective truth in the matter? No, it does not follow; for it could be that 
the practice was objectively right (or wrong) and that one or the other of 
them was simply mistaken. 

To make the point clearer, consider a different matter In some societies, 
people  believe  the  earth  is  flat  In  other  societies,  such  as  our  own,
people believe  the  earth  is  (roughly)  spherical.  Does  it  follow,  from  the 
mere  fact  that  people  disagree,  that  there  is  no  ``objective  truth"  in 
geography?  Of  course  not;  we  would  never  draw  such  a  conclusion 
because we realize that, in their beliefs about the world, the members of 
some societies might simply be wrong. There is no reason to think that if 
the world is round everyone must know it. Similarly, there is no reason to 
think that if there is moral truth everyone must know it. The fundamental 
mistake in the Cultural Differences Argument is that it attempts to derive 
a substantive conclusion about a subject from the mere fact that people 
disagree about it. 

This is a simple point of logic, and it is important not to misunderstand it. 
We are not saying (not yet, anyway) that the conclusion of the argument 
is false. It is still an open question whether the conclusion is true or false. 
The  logical  point  is  just  that  the  conclusion  does  not  follow  from  the 
premise.  This  is  important,  because  in  order  to  determine  whether  the 
conclusion is true, we need arguments in its support. Cultural Relativism 
proposes  this  argument,  but  unfortunately  the  argument  turns  out to  be 
fallacious. So it proves nothing.  

\section{2.4 The Consequences of Taking Cultural Relativism Seriously} 
Even  if  the  Cultural  Differences  Argument  is  invalid,  Cultural  Relativism 
might still be true. What would it be like if it were true? 

In the passage quoted above, William Graham Sumner summarizes the 
essence of Cultural Relativism. He says that there is no measure of right 
and wrong other than the standards of one's society: ``The notion of right 
is  in  the  folkways.  It  is  not  outside  of  them,  of  independent  origin,  and 
brought to test them. In the folkways, whatever is, is right." Suppose we 
took this seriously. What would be some of the consequences? 
\subsection{1. We could no longer say that the customs of other societies are morally 
inferior to our own.} 
This, of course, is one of the main points stressed by 
Cultural  Relativism. We  would  have to  stop condemning  other  societies 
merely  because  they  are  ``different:'  So  long  as  we  concentrate  on 
certain  examples,  such  as  the  funerary  practices  of  the  Greeks  and 
Callatians, this may seem to be a sophisticated, enlightened attitude. 

However,  we  would  also  be  stopped  from  criticizing  other,  less  benign 
practices. Suppose a society waged war on its neighbors for the purpose 
of taking slaves. Or suppose a society was violently anti-Semitic and its 
leaders  set  out  to  destroy the  Jews.  Cultural  Relativism  would  preclude 
us  from  saying that  either  of  these  practices  was  wrong. We  would  not 
even be able to say that a society tolerant of Jews is better than the anti-
Semitic society, for  that would imply some  sort of transcultural standard 
of  comparison.  The  failure  to  condemn  these  practices  does  not  seem 
enlightened;  on  the  contrary,  slavery  and  anti-Semitism  seem  wrong 
wherever they occur. Nevertheless, if we took Cultural Relativism 
seriously,  we  would  have  to  regard  these social  practices  as  also 
immune from criticism. 

\subsection{2. We could decide whether actions are right or wrong just by consulting 
the  standards of  our  society.}
Cultural  Relativism  suggests  a  simple test 
for  determining  what  is  right  and  what  is  wrong:  All  one  need  do  is  ask 
whether  the  action  is  in  accordance  with  the  code  of  one's  society. 
Suppose in 1975, a resident of South Africa was wondering whether his 
country's policy of apartheid—a rigidly racist system—was morally 
correct.  All  he  has  to  do  is  ask  whether  this  policy  conformed  to  his 
society's  moral  code.  If  it  did,  there  would  have  been  nothing  to  worry 
about, at least from a moral point of view. 

This  implication  of  Cultural  Relativism  is  disturbing  because  few  of  us 
think that our society's code is perfect; we can think of ways it might be 
improved. Yet Cultural Relativism would not only forbid us from criticizing 
the  codes  of  other  societies;  it  would  stop  us  from  criticizing  our  own. 
After  all,  if  right  and  wrong  are  relative  to  culture,  this  must  be  true  for 
our own culture just as much as for other cultures. 

\subsection{3. The idea of moral progress is called into doubt.}
Usually, we think that at  least  some  social  changes  are  for  the  better.  (Although,  of  course, 
other  changes  may  be  for  the  worse.)  Throughout  most  of  Western 
history the place of women in society was narrowly circumscribed. They 
could  not  own  property;  they  could  not  vote  or  hold  political  office;  and 
generally they were under the almost absolute control of their husbands. 
Recently  much  of  this  has  changed,  and  most  people  think  of  it  as 
progress.

If  Cultural  Relativism  is  correct,  can  we  legitimately  think  of  this  as 
progress? Progress means replacing a way of doing things with a better 
way.  But  by  what  standard  do  we  judge  the  new  ways as  better?  If  the 
old ways were in accordance with the social standards of their time, then 
Cultural  Relativism  would  say  it  is  a  mistake  to  judge  them  by  the 
standards of a different time. Eighteenth-century society was, in effect, a 
different society from the one we have now. To  say that  we have made 
progress implies  a  judgment  that  present-day society  is better,  and that 
is  just  the  sort  of  transcultural  judgment  that,  according  to  Cultural 
Relativism, is impermissible. 

Our  idea  of  social  reform  will  also  have  to  be  reconsidered.  Reformers 
such as Martin Luther King, Jr., have sought to change their societies for 
the better. Within the constraints imposed by Cultural Relativism, there is 
one way this might be done. If a society is not living up to its own ideals, 
the  reformer  may  be  regarded  as  acting  for  the  best:  The  ideals  of  the 
society  are  the  standard  by  which  we  judge  his  or  her  proposals  as 
worthwhile. But the ``reformer" may not challenge the ideals themselves, 
for those ideals are by definition correct. According to Cultural 
Relativism,  then,  the  idea  of  social  reform  makes  sense  only  in  this 
limited way. 

These three consequences of Cultural Relativism have led many 
thinkers  to reject it as implausible on  its face. It does make  sense, they 
say,  to  condemn  some  practices,  such  as  slavery  and  anti-Semitism, 
wherever  they  occur.  It  makes  sense  to  think  that  our  own  society  has 
made some moral progress, while admitting that it is still imperfect and in 
need  of  reform.  Because  Cultural  Relativism says  that  these  judgments 
make no sense, the argument goes, it cannot be right. 

\section{2.5 Why There Is Less Disagreement Than It Seems} 
The original impetus for Cultural Relativism comes from the observation 
that cultures differ dramatically in their views of right and wrong. But just 
how much do they differ? It is true that there are differences. However, it 
is easy to overestimate  the extent of those differences, Often, when we 
examine  what  seems  to  be  a  dramatic  difference,  we  find  that  the 
cultures do not differ nearly as much as it appears. 

Consider a culture in which people  believe it is wrong to eat cows. This 
may even be a poor culture, in which there is not enough food; still, the 
cows are not to be touched. Such a society would appear to have values 
very  different  from  our  own.  But  does  it?  We  have  not  yet  asked why 
these  people  will not  eat  cows. Suppose  it  is because  they believe  that 
after death the souls of humans inhabit the bodies of animals, especially 
cows, so  that  a  cow  may be  someone's  grandmother.  Now do  we want 
to  say  that  their  values  are  different  from  ours?  No;  the  difference  lies 
elsewhere. The difference is in our belief systems, not in our values. We 
agree that we shouldn't eat Grandma; we simply disagree about whether 
the cow is (or could be) Grandma. 

The point is that many factors work together to produce the customs of a 
society. The  society's values are  only one  of them. Other matters, such 
as the religions and factual beliefs held by its members, and the physical 
circumstances  in  which  they  must  live,  are  also  important.  We  cannot 
conclude, then, merely because customs differ, that there is a 
disagreement about values. The difference in customs may be 
attributable to some other aspects of social life. Thus there may be less 
disagreement about values than there appears to be. 

Consider  again the  Eskimos,  who  often  kill  perfectly  normal  infants, 
especially girls. We do not approve of such things; a parent who killed a 
baby  in  our  society  would  be  locked  up.  Thus  there  appears  to  be  a 
great  difference  in  the  values  of  our  two  cultures.  But  suppose  we  ask 
why  the  Eskimos  do  this.  The  explanation  is  not  that  they  have  less 
affection  for  their  children  or  less  respect  for  human  life.  An  Eskimo 
family will always protect its babies if conditions permit. But they live in a 
harsh environment, where food is in short supply. A fundamental 
postulate  of  Eskimos  thought  is:  ``Life  is  hard,  and  the  margin  of  safety 
small.” A family may want to nourish its babies but be unable to do so. 
As  in many ``primitive" societies,  Eskimo mothers will nurse  their  infants 
over a much longer period of time than mothers in our culture. The child 
will  take  nourishment  from  its  mother's  breast  for  four  years,  perhaps 
even longer. So even in the best of times there are limits to the number 
of  infants  that  one  mother  can  sustain.  Moreover,  the  Eskimos  are  a 
nomadic  people—unable  to  farm,  they  must  move  about  in  search  of 
food.  Infants  must  be carried,  and  a  mother  can  carry only  one  baby in 
her parka as she travels and goes about her outdoor work. Other family 
members help whenever they can. 

Infant girls are more readily disposed of because, first, in this society the 
males are the primary food providers—they are the hunters, according to 
the traditional division of labor—and it is obviously important to maintain 
a  sufficient  number  of  food  providers.  But  there  is  an  important  second 
reason as well. Because the hunters suffer a high casualty rate, the adult 
men who die prematurely far outnumber the women who die early. Thus 
if  male  and  female  infants  survived  in  equal  numbers,  the  female  adult 
population would greatly outnumber the male adult population. 
Examining the available statistics, one writer concluded that ``were it not 
for female infanticide...there  would  be  approximately  one-and-a-half 
times  as  many  females  in  the  average Eskimo  local  group  as  there are 
food-producing males." 

So  among  the  Eskimos,  infanticide  does  not  signal  a  fundamentally 
different  attitude  toward  children.  Instead,  it  is  a  recognition  that  drastic 
measures  are  sometimes  needed  to  ensure  the  family's  survival.  Even 
then, however, killing the baby is not the first option considered. 
Adoption  is  common;  childless  couples  are  especially  happy  to  take  a 
more fertile couple's ``surplus." Killing is only the last resort. I emphasize 
this  in  order  to  show  that  the  raw  data  of  the  anthropologists  can  be 
misleading;  it  can  make  the  differences  in  values  between  cultures 
appear  greater  than  they  are.  The  Eskimos'  values  are  not  all  that 
different from our values. It is only that life forces upon them choices that 
we do not have to make. 

\section{2.6 How All Cultures Have Some Values in Common} 
It  should  not  be  surprising  that,  despite  appearances,  the  Eskimos  are 
protective  of  their  children.  How  could  it  be  otherwise?  How  could  a 
group survive that did not value its young? It is easy to see that, in fact, 
all cultural groups must protect their infants: 
\begin{enumerate}
\item[1] Human  infants  are  helpless  and  cannot  survive  if  they are 
not given extensive care for a period of years. 
\item[2] Therefore,  if  a  group  did  not  care  for  its  young,  the  young 
would  not  survive,  and  the  older  members  of  the  group 
would  not  be  replaced.  After  a  while  the  group  would  die 
out. 
\item[3] Therefore,  any  cultural  group  that  continues  to  exist  must 
care for its young. infants that are not cared for must be the 
exception rather than the rule. 
\end{enumerate}
Similar reasoning shows that other values must be more or less 
universal. Imagine what it would be like for a society to place no value at 
all on truth telling. When one person spoke to another, there would be no 
presumption at all that he was telling the truth for he could just as easily 
be speaking falsely. Within that society, there would be no reason to pay 
attention  to  what  anyone  says.  (I  ask  you  what  time  it  is,  and  you  say 
``Four  o'clock:'  But  there  is  no  presumption  that  you  are  speaking  truly; 
you  could  just  as  easily  have  said  the  first  thing  that  came  into  your 
head. So I have no reason to pay attention to your answer; in fact, there 
was no point  in my asking you in the first place.) Communication would 
then  be  extremely  difficult,  if  not  impossible.  And  because  complex 
societies  cannot  exist  without  communication  among  their  members, 
society would become impossible. It follows that in any complex society 
there  must  be  a  presumption  in  favor  of  truthfulness.  There  may  of 
course be exceptions to  this rule: There may be  situations in which it is 
thought to be permissible to lie. Nevertheless, there will be exceptions to 
a rule that is in force in the society.
 
Here  is  one  further  example  of  the  same  type.  Could  a  society  exist  in 
which  there  was  no  prohibition  on  murder?  What  would  this  be  like? 
Suppose people were free to kill other people at will, and no one thought 
there  was anything  wrong  with  it.  In  such a  ``society,"  no  one  could feel 
secure.  Everyone  would  have  to  be  constantly  on  guard.  People  who 
wanted to survive would have to avoid other people as much as 
possible.  This  would  inevitably  result  in  individuals  trying  to  become  as 
self-sufficient  as  possible— after  all,  associating  with  others  would  be 
dangerous. Society on any large scale would collapse. Of course, people 
might  band  together  in  smaller  groups  with  others  that  they  could  trust 
not  to  harm  them.  But  notice  what  this  means:  They  would  be  forming 
smaller  societies  that  did  acknowledge  a  rule  against  murder:  The 
prohibition of murder, then, is a necessary feature of all societies. 

There  is  a  general  theoretical  point  here,  namely,  that  \emph{there  are  some 
moral  rules  that  all  societies  will  have  in  common,  because  those  rules 
are necessary for society to exist.} The rules against lying and murder are 
two  examples.  And  in  fact,  we  do  find  these  rules  in  force  in  all  viable 
cultures. Cultures may differ in what they regard as legitimate exceptions 
to  the  rules,  but  this  disagreement  exists  against  a  background  of 
agreement on the larger issues. Therefore, it is a mistake to 
overestimate the amount of difference between cultures. Not every moral 
rule can vary from society to society. 
\section{2.7 Judging a Cultural Practice to Be Undesirable} 
In  1996,  a  17-year-old  girl  named  Fauziya Kassindja  arrived  at  Newark 
International  Airport  and  asked  for  asylum.  She  had  fled  her  native 
country  of  Togo,  a  small  west  African  nation,  to  escape  what  people 
there call excision. 

Excision is a permanently disfiguring procedure that is sometimes called 
``female  circumcision," although it bears little resemblance to the Jewish 
ritual.  More  commonly,  at  least  in Western newspapers,  it is  referred  to 
as  ``genital  mutilation."  According  to  the World  Health  Organization,  the 
practice  is  widespread  in  26  African  nations,  and  two  million  girls  each 
year  are  ``excised."  In  some  instances,  excision  is  part  of  an  elaborate 
tribal ritual, performed in small traditional villages, and girls look forward 
to  it  because  it  signals  their  acceptance  into  the  adult  world.  In  other 
instances, the practice is carried out by families living in cities on young 
women who desperately resist. 

Fauziya  Kassindja  was  the  youngest  of  five  daughters  in  a  devoutly 
Muslim  family.  Her  father,  who  owned  a  successful  trucking  business, 
was opposed to excision, and he was able to defy the tradition because 
of his wealth. His first four daughters were married without being 
mutilated.  But  when  Fauziya  was  16,  he  suddenly  died.  Fauziya  then 
came under the authority of his father, who arranged a marriage for her 
and prepared to have her excised. Fauziya was terrified, and her mother 
and oldest sister helped her to escape. Her mother, left without 
resources, eventually had to formally apologize and submit to the 
authority of the patriarch she had offended. 

Meanwhile, in America, Fauziya was imprisoned for  two years while the 
authorities decided what to do with her. She was finally granted asylum, 
but  not  before  she  became  the  center  of  a  controversy  about  how 
foreigners should regard the cultural practices of other peoples. A series 
of articles in the New York Times encouraged the idea that excision is a 
barbaric  practice  that  should  be  condemned.  Other  observers  were 
reluctant  to  be  so  judgmental—live  and  let  live,  they  said;  after  all,  our 
practices probably seem just as strange to them. 

Suppose we are inclined to say that excision is bad. Would we merely be 
applying  the  standards  of  our  own  culture?  If  Cultural  Relativism  is 
correct,  that  is  all  we  can  do,  for  there  is  no  cultural-neutral  moral 
standard to which we may appeal. Is that true? 

\section{Is There a Culture-Neutral Standard of Right and Wrong?} 
There  is,  of  course,  a  lot  that  can  be  said  against  the  practice  of 
excision. Excision is painful and it results in the permanent loss of sexual 
pleasure. Its short-term effects include hemorrhage, tetanus, and 
septicemia. Sometimes the woman dies. Longterm effects include 
chronic infection, scars that hinder walking, and continuing pain. 

Why, then, has it become a widespread social practice? It is not easy to 
say. Excision has no obvious social benefits. Unlike Eskimo infanticide, it 
is  not  necessary  for  the  group's  survival.  Nor  is  it  a  matter  of  religion. 
Excision  is  practiced  by  groups  with  various  religions,  including  Islam 
and Christianity, neither of which commend it. 

Nevertheless, a number of reasons are given in its defense. Women who 
are incapable of sexual pleasure are said to be less likely  to be 
promiscuous; thus there will be fewer unwanted pregnancies in 
unmarried women. Moreover, wives for whom sex is only a duty are less 
likely  to  be  unfaithful  to  their  husbands;  and  because  they  will  not  be 
thinking  about  sex,  they  will  be  more  attentive  to  the  needs  of  their 
husbands  and  children.  Husbands,  for  their  part,  are  said  to  enjoy  sex 
more  with  wives  who  have  been  excised.  (The  women's  own  lack  of 
enjoyment  is  said  to  be  unimportant.)  Men  will  not  want  unexcised 
women,  as  they  are  unclean  and  immature.  And  above  all,  it  has  been 
done since antiquity, and we may not change the ancient ways. 

It would be easy, and perhaps a bit arrogant, to ridicule these 
arguments. But we may notice an important feature of this whole line of 
reasoning:  it  attempts  to  justify  excision  by  showing  that  excision  is 
beneficial— men,  women,  and their  families  are  all  said  to be  better  off 
when women are excised. Thus we might approach this reasoning, and 
excision itself, by asking which is true: Is excision, on the whole, helpful 
or harmful? 

Here,  then,  is  the  standard  that  might  most  reasonably  be  used  in 
thinking  about  excision:  We  may  ask  whether  the  practice  promotes  or 
hinders the welfare of the people whose lives are affected by it. And, as 
a corollary, we may ask if there is an alternative set of social 
arrangements that would do a better job of promoting their welfare. If so, 
we may conclude that the existing practice is deficient. 

But  this  looks  like  just  the  sort  of  independent  moral  standard  that 
Cultural Relativism says cannot exist. It is a single standard that may be 
brought  to  bear  in  judging  the practices  of  any  culture,  at  any  time, 
including our own. Of course, people will not usually see this principle as 
being ``brought in from the outside" to judge them, because, like the rules 
against lying and homicide, the welfare of its members is a value internal 
to all viable cultures. 

\section{Why Thoughtful People  May Nevertheless  Be Reluctant to Criticize 
Other Cultures.}

Although  they  are  personally  horrified  by  excision,  many  thoughtful 
people  are  reluctant  to  say  it  is  wrong,  for  at  least  three  reasons.  First, 
there  is  an  understandable  nervousness  about  ``interfering  in  the  social 
customs of other peoples." Europeans and their cultural descendents in 
America have a shabby history of destroying native cultures in the name 
of Christianity and Enlightenment, not to mention self-interest. Recoiling 
from  this  record,  some  people  refuse  to  make  any  negative  judgments 
about  other  cultures,  especially  cultures  that  resemble  those  that  have 
been  wronged  in  the  past.  We  should  notice,  however,  that  there  is  a 
difference between (a) judging a cultural practice to be morally deficient 
and (b) thinking that we should announce the fact, conduct a campaign, 
apply diplomatic pressure, or send in the army to do something about it. 
The  first  is  just  a  matter  of  trying to  see  the world clearly,  from  a  moral 
point  of  view.  The  second  is  another  matter  altogether.  Sometimes  it 
may be right to ``do something about it," but often it will not be. 

People  also  feel,  rightly  enough,  that  they  should  be  tolerant  of  other 
cultures. Tolerance is, no doubt, a virtue—a tolerant  person is willing to 
live  in  peaceful  cooperation  with  those  who  see  things  differently.  But 
there is nothing in the nature of tolerance that requires you to say that all 
beliefs,  all  religions,  and  all  social  practices  are  equally  admirable.  On 
the contrary, if you did not think that some were better than others, there 
would be nothing for you to tolerate. 

Finally,  people  may  be  reluctant  to  judge  because  they  do  not  want  to 
express  contempt  for  the  society  being  criticized.  But  again,  this  is 
misguided: To condemn a particular practice is not to say that the culture 
is  on  the  whole  contemptible  or  that  it  is  generally  inferior  to  any  other 
culture,  including  one's  own.  It  could  have  many  admirable  features.  In 
fact, we should expect this to be true of most human societies— they are 
mixes of good and bad practices. Excision happens to be one of the bad 
ones.

\setcounter{footnote}{\thefz}

\stepcounter{chapcount}
\chapter{Part \thechapcount: What is Moral Relativism?}\setcounter{seccount}{1}
\section{Part \thechapcount.\theseccount: Going Meta}\stepcounter{seccount}

You might have heard the term ‘meta’ before. For example, when I was in undergrad, I was explaining my Epistemology course to some friends, saying something like “it’s the study of knowledge, we are learning about knowledge”, and one person in the group said “that’s so meta!”. The term ‘meta’ comes from Greek, meaning ‘over’, and has there a similar sense to our prefix ‘trans-’ (as in ‘translate’ or ‘transfer’), which comes from Latin and has the same sense as the Greek. In English, also, there are two senses of ‘over’, the first means ‘across’, like to carry something over to another, and the second means ‘above’, as in to put something above another. ‘Meta-’ and ‘trans-’ differ in the same way. With very few exceptions (‘metaphor’, which comes from the Greek for ‘carry over’), the prefix ‘meta-’ means that you are doing something above, over, the area in question and the prefix ‘trans-’ means that you are doing something across, over, certain boundaries. In the case of `translate', this comes from the Latin verb `transfero', meaning ``I carry over", that verb is also where we get the English verb `transfer'.

More explicitly, when \gls{meta-} is attached to a field of study, it refers to a field of study one step more abstract than it. One way to think about this is that I am now asking questions about the questions in the field. This is not just true for fields in philosophy, though once the level gets high enough, it becomes an area of philosophy, but it’s also true for any science or subject you pick. Here is a table of some examples:

\noindent
\begin{tabular}{p{1in}|p{1.5in}|p{1.2in}|p{1.5in}}
Subject&Questions &Meta-Subject&Meta-Questions\\\hline
Biology&Concerning living organisms&Metabiology&What exactly is living? 

What exactly is an organism?\\\hline
Physics&Concerning matter and its actions in space and time&Metaphysics&Is matter all that exists? 

What is space? 

What is time?\\ \hline
Linguistics&Concerning language, meaning, and structure in humans&Metalinguistics&How does meaning relate to truth and intentions? 

How do words pick out things? \\
\end{tabular}

\newglossaryentry{meta-}
{
  name=meta-,
  description={a prefix which changes the level of abstraction for the questions in the field, namely by moving it up. This means, roughly, `asking questions about the questions in...'}
}


When you ‘go meta’, you are taking yourself out of the sphere of the field and starting to ask questions about the very nature of it. Due to the nature of philosophy and philosophers in general, we are far more inclined to go meta on a topic. So, for any field in philosophy, there’s likely a well-trotted path for the various meta-questions and sometimes a well-trotted path for the meta-meta-questions.

The various stances which can be generated once you choose to go meta on a topic tend to fall into nice, neat, easy to classify packets. The first level has two different kinds of stances, \Gls{realism} and Anti-Realism (sometimes called \Gls{nihilism}, which is what I prefer). Realism claims that the facts about the subject are real, hence the name, while the Anti-Realist claims that they aren't real. And then, depending on what the stance says about various meta-level questions, they are further subdivided. In the Realist side, there are two general stances which get our attention, Relativism and Objectivism.

\newglossaryentry{realism}
{
  name=realism,
  description={The stance that there are facts in a given topic. In other words, the stance that the facts are `real'}
}

\newglossaryentry{nihilism}
{
  name=nihilism,
  description={The stance that there are no facts in a given topic. In other words, the stance that the facts aren't `real', there is nothing true or accurate to be said about it. The name for this stance comes from the Latin word `nihil' meaning `nothing'}
}


There are different kinds of Relativisms and Objectivisms (the typical subdivisions of Realism) out there, and those tend to be more particular to the field and there may be different kinds of Nihilisms aside from Error Theory and Expressivism, but those go well beyond the scope of this class. Towards the end of this module, I will explain those stances in the Nihilism branch for Meta-Ethics in more detail.

\section{Part \thechapcount.\theseccount: The Start of Moral Relativism}\stepcounter{seccount}

Moral Relativism is a part of a family of stances, all of which say that the truth of something, or everything, is relative (generally called \Gls{relativism}). What is relative or to what it is relative all depends on the stance. This sort of idea is found in several areas of philosophy and the notion (applied differently) is found in physics. This, in its broad sense, is the stance that there is no absolute truth (objective truth), but rather the truth about things varies from person to person or from culture to culture, depending on the version of relativism which is held.

Relativism comes in many different flavors. The first is that everything is relative, this is global relativism. The other is that only certain things are relative, this is limited relativism. From this, we have to ask what are they relative to? These are either to your culture or to you individually.

\newglossaryentry{relativism}
{
  name=relativism,
  description={The stance that there are facts in a given topic but those facts are variable, they change according to the culture, context, or individual. In philosophy, generally, the facts change according to either what the individual believes or the culture in question believes, depending on the form of relativism held}
}


\Glspl{moral relativism} is a limited form of relativism. It does come in two flavors, Moral Individual Relativism (what is moral is relative to the doer, and there is no way for another person to say that what they did is/was morally wrong) and Moral Cultural Relativism (what is moral is relative to the culture and there is no way of saying that they are wrong from an outsider's perspective). One can also say that Moral Relativism is the stance that there is no objective basis for claiming that a particular action, or a class of actions is right or wrong, permissible or impermissible. Or, that there is no fact of the matter regarding whether actions are right or wrong. Whether something is, in fact, right or wrong is relative to the culture or to the person. Some classic examples of this are killing babies, cannibalism, and funeral rights.

\newglossaryentry{moral relativism}
{
  name=moral relativism,
  description={The stance that there are facts about morality, that is there are standards of right and wrong, but those facts are variable, they change according to the beliefs of culture or individual in question (which depends on the form of Moral Relativism claimed to be true)}
plural=Moral Relativism
}


\section{Part \thechapcount.\theseccount: The Cultural Difference Argument}\stepcounter{seccount}
Often, people argue for moral relativism by pointing to examples, it is often claimed that due to the differences in moral beliefs, there must not be an objective, absolute morality. If you remember from my charts, the moral relativism still holds that there are moral truths, but those truths are relative to the culture. Cultural Moral Relativism is by far the most common stance found in popular culture, so that's the one we will point to, but individual moral relativism has similar problems as well as others unique to it. Here is a classical example used to argue for Cultural Moral Relativism:

    \factoidbox{Darius, a king of ancient Persia, was intrigued by the variety of cultures he encountered in his travels. The Callatians (a tribe of Indians) customarily ate the bodies of their dead fathers. The Greeks practiced cremation. Darius thought that an understanding of the world must include an appreciation of such differences. One day, he brought some Greeks who happened to be present and asked them what they would take to eat the bodies of their dead fathers. They were shocked and replied that no amount of money could persuade them to do such a thing. Then Darius called in some Callatians, and while the Greeks listened asked them what they would take to burn their dead fathers' bodies. The Callatians were horrified and told Darius not even to mention such a dreadful thing. The two people had a very similar reaction to the opposite actions.\autocite{Herodotus1}} 

From this, we can see that two cultures, when the encountered the customs of the other were equally mortified. but sometimes, even the most `socially aware' of us will be mortified by customs found in other cultures. Take this example:  

    \factoidbox{In some Inuit cultures, there is a common practice know as ‘the dip’ (my translation). When a child is born in the dead of winter, it is common for the mother to carve out a hole in the ice and place the child in the water. This kills the child instantly (infanticide). This is done purely at the parents’ discretion and there is no negative stigma about it. Old people also, when they became too feeble to contribute to the family, were left out in the snow to die.}

The lesson from these two stories (the second of which I have seen it used as examples enough for me to think it's true) is that different cultures have different moral codes. What is seen as just fine in one culture is seen as horrifying in another. For a fictional case, take this example:

    \factoidbox{For most of the world, the elderly are treated with respect and cared for until their natural death. The Kaelon people are different. In this society, the prevailing view is that it's the duty of the `elderly' to leave their tasks to the next generation, that forcing the next generation to care for the elderly is cruel,  and that having death come for a person  seemingly randomly is heartless. So, when a Kaelon turns 60 years old, they undergo the Resolution. In it, a great party is thrown, celebrating their life and accomplishments, and afterwards, the Kaelon commits suicide. Living past this point is seen as greedy, their time is up and their accomplishments after this point are seen as stolen from the next generation. This is expected of all Kaelons, refusing to kill yourself at the allotted time will cause even family members to be ashamed of you. If a Kaelon seeks asylum to avoid the Resolution, the Kaelons will declare war on the other culture in order to kill the person.\autocite{HalfaLife}}

Many have looked at examples like this and have thought that there must not be an actual objective truth about morality, that it must merely be opinion, either held by a person or collectively as a group. This leads us to:

\section{Part \thechapcount.\theseccount: The Cultural Differences Argument}\stepcounter{seccount}

These examples, and my not so subtle  hints, point to a very interesting argument. It seems, given how I have framed it, the most socially conscious stance would be one which puts all cultural customs on equal footing, to not have some objective measure which can be used to discriminate against a culture. So, using this observation, we get this argument:
\noindent
\begin{tabular}{p{2.75in}|p{2.75in}}
Cultural Differences Argument&Disagreement Argument\\\hline
    Different cultures have different moral codes.&There is disagreement about morality.\\
    Therefore, there is no objective ``truth" about morality.&Therefore, there's no objective fact of the matter about morality.
\end{tabular}

But, does this show that the next module in this class is going to be gibberish? Well, no. Though there are some moral relativists out there in philosophy today, there are very few, and no philosopher (no really good philosopher) is a moral relativist because of this argument. There are many problems with it which are worth exploring. 

\subsection{Problem 1: It isn't a good argument}

For an argument to be good, philosophers have a pretty high standard. The conclusion needs to follow from the premises necessarily. There can be no case where the premises are true and the conclusion false (this is called ‘validity’). If an argument is not valid, then it is not sound. The first line of the argument concerns what people believe while the conclusion concerns what is actually the case. The issue with this sort of reasoning can be seen using a similar sort of method as the Limitation of the Conclusion problems in the arguments for the existence of God. If you think that The Cultural Differences Argument is accurate, then you will, by the same reasoning, need to think that these arguments are accurate: 
\noindent
\begin{tabular}{p{2.75in}|p{2.75in}}
Shape of the Earth Argument&Global Warming Argument\\\hline
    Different cultures/people have different beliefs about the shape of the earth.& Different people have different beliefs about global warming.\\
    Therefore, there is no fact of the matter about the shape of the earth.&Therefore, there's no fact of the matter about global warming.\\
\end{tabular}

This argument follows the exact same reasoning as the cultural differences argument. If it works for one then it should work for the other. But, we can also look at the spirit of the argument, and attack that as well. 

\subsection{Problem 2: Cultural Oppression}

Though the argument may not be any good, all that shows is that the argument isn't good. It does not show that morality is objective. But, if we look at the last example I gave, involving the Kaelons, we see something interesting. If morality is relative to a culture, there's nothing stopping one culture, morally speaking, from oppressing people in their culture and oppressing people in other cultures. For example, if there are no objective moral facts, then there would be nothing wrong with the Kaelons declaring war on another culture for allowing someone to receive asylum. Similarly, there would be nothing wrong with the Kaelons forcing members of their culture to unwillingly commit suicide. This sort of case can be found in history as well, with the sear number of examples being too numerous to list. This is a problem which will appear again later in this class, because it's more obvious inside another argument.  

\section{Part \thechapcount.\theseccount: The Cultural Imperialism Argument}\stepcounter{seccount}

The Cultural Differences Argument is not the only argument in favor of Cultural Moral Relativism, there are several others. Most fall to the same sort of objections which we will see later, others have problems of their own. The Cultural Imperialism Argument is a little more advanced, claims to prove this sort of relativism, but it, equally, has issues.  Here is the introduction:

\subsection{Cultural Imperialism}

Throughout history, we have seen cases of one culture forcing beliefs and moral stances onto another. For example, we have the forced religious conversions of the Aztecs by the Spanish, we have the boarding schools which the US put Native American children in, and in China, we have the Uyghurs being put into `reeducation camps' (this may still be happening, this is true as of 2018). All of these examples have/had the same purpose: changing the practices, religions, and moral ideologies of the people.  

Cultural Imperialism is the reason for these behaviors, why cultures force themselves on others in this way (and sometimes in even more violent ways). Cultural Imperialism is the stance your culture as all of the right answers, morally speaking, your culture is perfect, and because of this, you are justified in forcing other cultures to change (typically to your own). But, looking at the number of examples from history, we can see that this imperialism is wrong, it should not be done. This can be seen both from an insider's and an outsider's perspective. Naturally, this perspective on Imperialism leads easily to Cultural Moral Relativism, because it seems never to be right to judge the values of one culture against the values of another. Put more formally, we get this argument:

\subsection{The Cultural Imperialism Argument}

\subsubsection{Cultural Imperialism is morally wrong. (Intuition)}
This follows from the examples which I have given, many argue that this sort of behavior is wrong and should not be done at all. The main reason for the Prime Directive in Star Trek is because of the base line principle, you need to let cultures' internal matters be internal matters. 
\subsubsection{If Cultural Imperialism is morally wrong, then it is wrong to judge the values of one culture against the values of another. (Consequence of it being wrong)}
This, in turn, is a consequence of the wrongness of Cultural Imperialism (the first line). Basically, if it was OK to make judgments about other cultures' practices, then it would be OK to act on those judgments. Acting on those judgments would be Cultural Imperialism, which we said was wrong.   
\subsubsection{If it is wrong to judge the values of one culture against the values of another, then no person is ever justified in criticizing the moral norms of another culture. (Consequence of the consequence)}
This is a consequence of the wrongness of the judgments about a culture. Criticizing a culture's practices, in either word or action, is wrong by the same reasoning as Cultural Imperialism. The exact reasoning is found in the previous line. 
\subsubsection{If no person is ever justified in criticizing the moral norms of another culture, there are no non-relative moral truths (they are all relative to culture). (Consequence)}
This is the final line of support for the conclusion. If there were an absolute, objective, moral rule, or set of rules, then it would be possible for a person to be correctly justified in criticizing another culture's norms. All they would need to do is point to the objective rules and show that the culture is in violation of them. But, we have shown that a person can never be justified in this way, so, it would seem that there must not be an absolute moral standard (they are relative to the culture).
\subsubsection{Therefore, there are no non-relative moral truths (they are all relative to culture).}
Now, despite my build-up and explanation for this, there is a major issue with this argument, both in the way it's constructed (though, technically valid) and the spirit behind it. For example, I can use this argument, exact same premises, to show that unicorns exist, 2+2=5, and anything else I could want. 

\subsection{The Problem with the Cultural Imperialism Argument}

As I mentioned above, I can use this argument, not the structure, but the argument itself, to prove that mermaids exist. Any argument with this capability will be valid, but it will not be sound, there will be a factual error in it somewhere. For this one, let's try and work out the factual error, and here it's fairly easy to spot (hint, look at the first line and the last line):


\noindent
\begin{tabular}{p{2.75in}|p{2.75in}}
First Line&Last Line\\\hline
Cultural Imperialism is morally wrong.&There are no non-relative moral truths.
\end{tabular}

The problem with this argument is that it's contradictory. You are saying that the same is both true and false. In the first line, we are saying that Cultural Imperialism is wrong, which is not contradictory on its own (if you think that it can be OK, then you have included other aspects into it). Similarly, the last line says that all moral truths are relative. Again, not contradictory on its own. But, in combination, the lines conflict. The first line is a non-relative moral truth and the last line says that those don't exist. This also attacks the spirit of the argument, in that the core premises/intuition which build it up come from the idea that Cultural Imperialism is wrong, which contradicts the point it's trying to prove. 

There are a few ways out for the Relativist. First, they could accept that some cases of Cultural Imperialism are fine, making it not always wrong. For example, we can see that the Allies entering WW2 to stop certain behaviors was Cultural Imperialism, but that wasn't wrong. The issue with this is that if we go down this rabbit hole, we find that the cases where Cultural Imperialism is permissible are cases of objective moral truths, which limits the initial scope of Objectivism, but contradicts Relativism. 

Another way out, which limits Cultural Moral Relativism further, is to say that there's a distinction between internal and external behaviors. Internal behaviors are those which only apply to members of the same culture, while external behaviors are those which involve members of different cultures. Cultural Imperialism is an external behavior. The Relativist here can claim that the morality of internal behaviors is relative while the morality of external behaviors is objective. This has its own issues as well, mostly dealing with cases like the Kaelons which I gave before, and can be easily seen in that case. 

\section{Part \thechapcount.\theseccount: Tests to Tell Whether You are \emph{actually} a Relativist}\stepcounter{seccount}
These are some tests which you can preform on yourself or on others, or on groups, which will tell you your actual feelings about moral relativism. There are three tests total, sometimes these tests will work individually, sometimes jointly (depending on the case), but they can be very useful. These are based on \emph{Why I am an Objectivist about Ethics (And Why You Are, Too)} 
by David Enoch\autocite{Enoch1}.

\subsection{The First Test: The Spinach Test}

This is a pretty straight forward test for whether you actually are a moral relativist. But, as a fair warning, philosophy kills jokes. As an example, and where the test gets its name, lets look at the following joke:

    \thoughtex{The Spinach Test}{A kid hates spinach, later, he says that he’s glad he hates spinach. When asked why, he says “because if I liked it, I would eat it; and it’s yucky!”}{Spinach.jpg}{A young child refusing to eat his spinach.}

More often than not, we will take this as funny. And, in a lecture format, if I present it right, I will get a laugh. But, the test is not whether or not you laugh at this joke, but rather whether you laugh at jokes of this form/structure, with different things put in the place of `spinach'. For example, let's try it with entertainment and other foods:

    \factoidbox{A kid hates watching golf. He says that he’s glad he hates watching golf because “if I liked it, I would watch it, and it’s boring!”

    A metal-head says that he’s glad he hates country. When asked why, he replies ‘because if I liked country, I would listen to country, and that’s just bad music.’

    George H.W. Bush hates broccoli and says that he’s glad he hates broccoli. The reason? “Because if I liked broccoli, I would eat it, and it’s gross.”}

These are the same, if presented right, they will get a laugh. But, will this joke work for anything which I plug into the joke's formula? What if I plug in something which is more absolute? Take these examples:

    \factoidbox{A 20th century man believes that the Earth is round. He claims that he’s glad that he’s a 20th century man because “If I grew up in the first century, I would have believed that the Earth is flat, and that's just wrong.”

    A 20th century man believes that women should have the right to vote. He claims that he’s glad that he’s a 20th century man because “If I grew up in the 19th century, I would have believed that women shouldn’t vote, and this is wrong.”

    An Iraqi woman who grew up in the 2000s believes that women are not equal to men. She claims that she’s glad about this because “if I grew up in the 60s and 70s, I would have believed that we were equal, and this is wrong.”}

Something is very different here. If I tried to present these as a joke, even with the same tones and inflections, I would not get a laugh (and I don't, I have tried). But, what's the difference between these? What sort of things get a laugh and what sort of things don't? Well, the difference between these cases is seemingly unrelated, but it's the only one which works (they are related). The relative things, the things which are merely preference, those get a laugh when plugged into the joke, the things which aren't relative, the objective things, those don't get a laugh.  So, this test is basically if you get a laugh, it's relative, if you don't, then it's objective. The vast, vast majority of the time, we do not laugh about morality. So, we, at least at some level, think that morality is objective. 

\subsection{The Second Test: The Disagreement Test}

Often, in real life, we get into disagreements with people. Sometimes these are serious, like whether refugees should be allowed in the country or whether global warming is occurring. Other times they are just silly, like whether broccoli is yucky or whether dark chocolate is better than milk. The second test for relativism about a subject concerns the nature of such disagreements. For this test, you need to  imagine, or actually get into, a disagreement about something. Ask yourself what the disagreement feels like. For example, take these examples:
\factoidbox{
\begin{tabular}{p{2.5in}|p{2.5in}}
Disagreements A&Disagreements B\\\hline
A disagreement about whether the Earth is flat.&A disagreement about whether pineapple should go on pizza.\\\hline
A disagreement about a mathematical principle.&A disagreement about how enjoyable Buffy the Vampire Slayer was.\\\hline
A disagreement about whether refugees should be allowed in the country.&A disagreement about the direction of a toilet paper roll.\\\hline
A disagreement about whether slavery was bad.&A disagreement about the best metal band.\\\hline
A disagreement about whether God exists.&A disagreement about flavors of ice cream.\\\hline
A disagreement about what to do in the trolley problem.&A disagreement about a movie (how enjoyable it was).\\\hline
\end{tabular}
 }

These disagreements are very different. Think about what it’s like to be in those kinds of disagreements. In the case of broccoli or chocolate, you are trying to state your own preference and then, maybe, trying to get the listener to come to their senses and change theirs. However, when we are engaged in a disagreement about, say, global warming, the point of the argument is not to change the other person’s mind or to state how you feel. Rather, it is to get at the truth. Looking above at the examples, do the ones in the left column feel the same? Do they all feel like facts? Similarly, do the ones in the right column all feel the same? Do they feel like preferences? If the feeling of the disagreements in the left column all feel the same, then you are an objectivist about morality, in your heart.  

So, for this test about whether you are a relativist about some subject, you need to ask yourself what it feels like to be in a disagreement about the subject. Imagine that you are having a disagreement with someone about the morality of abortion. How does that feel? Does it feel like you are having a discussion about pineapple and green-olive pizza? Or does it feel like you are having a discussion about global warming?

Yet again, very rarely do we feel like disagreements about morality are disagreements about preference. When we have those debates, it feels like they are about facts.
\subsection{The Third Test: The What-If Test}

Like the other two, this one is a bit different. It involves what are called counterfactuals. They are called this because they are counter to the facts. We use these sort of statements all the time without realizing it. Any time you ask yourself what would happen if something were the case, you are using a counterfactual. These are insanely useful in philosophy as well as in science. There's an entire cottage industry in philosophy dedicated to coming up with how we use, accept, and reject these kinds of statements. That being said, since we use them so much, your gut intuition will suffice. For this test, we will present the subjects as yes-no `what if' style questions. Such as, ‘what if I drove a 110MPH on a 60MPH highway, would that be dangerous?’ It is not true that I drive that fast, but I am trying to figure out what would be the case if I did. Here are some examples of what-if style questions which show this test at work.
\factoidbox{
If the vast majority of people believed that politicians were actually aliens, would they still be human?

If the vast majority of people believed that global warming wasn't happening, would it still?

If the vast majority of people thought that gender discrimination was fine, would it still be wrong? 

If the vast majority of people thought that the Earth was flat, would it still be round? 

If the vast majority of people thought that Philosophy was pointless, would it still be valuable? 

If the vast majority of people thought that pineapple on pizza was delicious, would it still be gross? 

If the vast majority of people wore top hats and thought they were cool, would they still be out of fashion?

If the vast majority of people thought that Buffy was the worst show ever, would it still be great?

If the vast majority of people thought that sweet potatoes were tasty, would they still be gross?

If the vast majority of people thought that spinach was gross, would it still be tasty? 
}
 

For each of these, there should be a yes or no answer, one which you can easily give. In some versions of this, I have presented it as ``all people" rather than ``the vast majority of people", but that has lead to some confusion. As before, the normal responses to the questions in the left column are all `yes', while the answers to the questions in the right are normally `no'. This test, if the question is formulated correctly, is the most definitive, I think. It points out the fact that objective truths are not up-to-us, they will be the case regardless of whether or not people discover them. 

The steps to preform this test on yourself are pretty straight forward. First, take a stance about something which you think is true. For example, you can have that slavery is bad, that pineapple on pizza is good, that the Kaelons shouldn't force their elderly to commit suicide, that the Earth is round, or something like that. Next, you imagine a case where the vast majority of people believe the opposite. And, finally, ask yourself whether, in this strange world, you would still be correct.  If the question is phrased correctly (it's possible to phrase them so that you will get, consistently, the opposite answer) and if you get the answer `yes', then the subject matter is objective and if you get the answer `no', then the subject is relative.

\section{Part \thechapcount.\theseccount: Consequences of Taking Moral Relativism Seriously}\stepcounter{seccount}

So far, we have seen evidence that we don't actually think that moral relativism is correct and we have seen that the arguments for the stance aren't any good (it's possible that there are others, and likely are others, but they will all, likely, fall you the same issues). But, like with the arguments for the existence of God, an objection to the argument doesn't show that it's conclusion is false. There are some true or false things which just can't be proven one way or the other (for example, ``the first digit of TREE(3) is 1" is either true or  false, but due to the size of the number, it's just impossible for us to prove one way or the other). Now, to actually disprove Cultural Moral Relativism we need to attack it head on, not just the supporting evidence. To do this, we will use a method which you might recognize from science (science got it from Philosophy). We will assume that relativism is true and show that it does not line up with the world, or that the consequences of it are so outlandish we can dismiss it easily. There are, at the heart of Moral Relativism, 3 problems which we will address (and more, but 3 is enough). 

\subsection{The Criticism Problem}

If we remember back to the Cultural Imperialism Argument, one of the core reasons it works is because we want to say that criticism of other societies (through word or action) is morally wrong. Though this does contradict Cultural Moral Relativism, that stance does have a variant to it built in. Namely:
\begin{center}
We could no longer say that the customs of other societies are morally inferior to our own
\end{center}
In other words, if moral relativism is correct, we can't say, truthfully/correctly, that any culture has `got it wrong', we can't say, in that way, that any cultural behavior is wrong. Sometimes, this seems fine, like in cases where doing so leads to the wrong sort of Cultural Imperialism. Other times, however, it leads to some very outlandish conclusions like:
\factoidbox{
We can't say that a culture torturing and/or killing other groups systematically is morally wrong.

We can't say that intolerance is bad.

We can't say that cultures which engage in hateful acts towards others are morally wrong.

We can't say that tolerance is good.}

The argument for this conclusion rests on the idea that if we accept moral relativism, there would be no gauge, objective standard, to measure morality.
\begin{enumerate}
    \item If MCR is true, then morality is relative to culture. (This is basically the definition)
    \item If morality is relative to the culture, then there is no standard to gauge the morality of a culture.
    \item If there is no such standard, there is no way to compare one culture to another (morally speaking)
    \item If there is no way to compare them, then there is no way to say that the Nazis were bad.
    \item So, if MCR is true, then there is no way to say that the Nazis were bad.
    \item But, the Nazis were bad!
    \item Therefore Moral Cultural Relativism is false.
\end{enumerate}
The way out, for the relativist, is to claim that the Nazis and every other culture which has engaged in horrible acts weren't bad. This can be quite the pill to swallow. 

\subsection{The Sheeple Problem}

Some of you may have heard the term `sheeple' before. It's a mash-up of the word `sheep' and the word `people'. In this case, if we take Moral Cultural Relativism seriously, thinking for yourself about morality is always morally wrong. This is not directly because of some universal moral rule, which would contradict Moral Relativism, rather it's because if something is a moral rule, then not abiding by it is morally wrong, and since every culture will have moral rules, not blindly following them will be wrong. Rather than speaking in negatives, we have this statement: 
\begin{center}
We could decide whether actions are right or wrong just by consulting the standards of our society.
\end{center}
Moral Cultural Relativism gives us a very simple test to tell whether an act is right or wrong. Just look at the cultural standard. Also, at the same point, trying to change those standards, whatever they may be, is breaking those standards, and thereby wrong. This means that, for example: Suppose that a person in the deep south (Pre-Civil War) was curious about whether slavery was permissible. All they would need to do is ask whether it fit with the code of the society. If it did, then they would be OK with having slaves. Another way to think about this is to imagine that God, or whoever, once the criteria for being a culture are met, sends down a book with the rules. Doing anything other than blindly following those rules is wrong, regardless of the culture. This means that every civil rights movement in history, in every culture, was morally wrong. This includes Black Rights Movements and Women's Rights Movements.

But, no cultural system is ever perfect and they have room to improve. Not only can’t we criticize other cultural codes, but we can’t criticize our own. However, claiming that every civil rights movement in history was morally wrong is absolutely outrageous, so Moral Cultural Relativism must be incorrect.  

This leads us to the following argument:
\begin{enumerate}
    \item If MCR is true, then morality is relative to culture. (This is basically the definition)
    \item If morality is relative to the culture, then whether an action is right or wrong is determined by your cultural norms.
    \item If those are determined by the norms, then trying to change those norms is always morally wrong.
    \item So, if MCR is true, then trying to change moral norms is always morally wrong.
    \item But, trying to change moral norms is sometimes morally right.
    \item Therefore, Moral Cultural Relativism is false.
\end{enumerate}
The way out for the relativist here is to claim that moral change is always wrong. Basically, claim that Martin Luther King Jr. was about as immoral as they come.

\subsection{The Progress Problem}

Many of us, especially some of the more `woke' members of our society, will claim that some moral change is for the better. But, the question quickly becomes ``what is `better'?" When we use this term, we are comparing one thing to another. We say that ``pineapple and green-olive pizza is better than pepperoni" or ``Pushing Daisies is better than The Big Bang Theory" or ``carrots are better than potatoes", in the context of healthiness. But, when we say that some moral change is for the better, what are we comparing and what are we using to make the comparison?

When we claim that some moral change is for the better, we are comparing where we were, morally, and where we are, morally. To make this comparison, if it's correct, we are using an absolute moral standard, a universal standard. We mean that we have moved closer to getting morality right. We call this progress. If we take Moral Cultural Relativism seriously, there's no moral standard, there's nothing to use to compare where we are and where we were morally speaking. We can't, correctly, claim that there's progress.
\begin{center}
The idea of moral progress is called into doubt
\end{center}
This means that not only can't we compare other cultures, but we can't even compare the same culture to itself at a different time. This leads us to the following argument:
\begin{enumerate}
    \item If MCR is true, then morality is relative to culture. (This is basically the definition)
    \item If morality is relative to the culture, then there is no standard to the morality of a culture.
    \item If there is no such standard, there is no way to compare one culture to another (morally speaking)
    \item Different people at different times, if the norms changed, are, for all intents and purposes, different cultures.
    \item So (from 1, 2, 3, and 4), if MCR is true, then there is no way to compare people from one time to another (morally).
    \item If there is no way to compare people from different times, there can be no way for us to say that we are better than we were (made progress).
    \item So, if MCR is true, then there is no way to say that we have made progress.
    \item But, we have progressed (at least in some areas).
    \item Therefore, Moral Cultural Relativism is false.
\end{enumerate}
The way out for the relativist is to claim that we haven't made progress, in any way. We are no better than we ever were. 
\stepcounter{chapcount}
\chapter{Part \thechapcount: So, If Moral Relativism is Wrong, What Next?}\setcounter{seccount}{1}

If you are willing to say that moral cultural relativism is wrong, and, given the general responses to the various problems with moral relativism as well as what can be expected from the tests, you likely are willing to say that it's wrong, we have some problems. Namely, what is moral? Who decides who's correct in a moral conflict? Where do moral truths come from? Here I will cover some of the basic questions which former relativists tend to give and the general responses (sometimes, the response will need to be more particular).

\section{Part \thechapcount.\theseccount: If culture's don't choose, what is morality?}\stepcounter{seccount}

There are a few ways to go about replying to this question. Morality/ethics centers around `should' questions of a certain kind. When we ask questions like these, for example ``what should I do?", ``should you call a doctor?", ``should you get the assignment in on time?",  there are two different senses, which may or may not overlap, depending on context and relevance. Something is moral/ethical when it's the correct answer to the question ``what should I do?", when we aren't talking about practical cases (a practical case is one where you are asking about the correct way to perform some task, like changing the oil on your car). The cultural moral relativist would claim that the correct answer to the question in non-practical cases (for example, should I flip the switch in the trolley problem?) will depend on your culture and the norms associated with it. The moral objectivist, a moral realist who is also a non-relativist, would claim that there's an actual answer to this question, which is not determined by your culture. The next module of this class covers some of the ways which philosophers have answered how to answer that question, without being relativist. Ethics is, at it's core, trying to find the correct answer to this question.

You will notice that when I give examples of cultural norms, most of the time, these are cases where the culture got the answer wrong. Ethical theories are basically hypotheses about the correct way to get the answer to the question. Some theories point to absolute, objective duties which a person has (typically, it's wrong to be irrational; acting in a certain way is irrational, therefore those ways of acting are wrong). Others point to the well-being of those affected (if an action makes the affected better off than otherwise, that's the right action). Others still point to some exemplar of morality, some person who has the perfect character and then asks ``what would that person do?". You may have heard that kind of thinking before with `WWJD'.  

\section{Part \thechapcount.\theseccount: Who decides who's right in moral conflict?}\stepcounter{seccount}

A moral conflict is a case where two or more people/groups disagree about the answer to the ``What should I do?" question, in the relevant sense. The moral cultural relativist has a simple way to answer it, almost too simple, ``if it's two groups in the same culture, the cultural norms settles the conflict, if it's two cultures, there's no real conflict." But this just does not seem right, if you recall the last page, were this true, all civil rights movements would wrong, because the cultural norms would settle the conflict in favor of the oppressing group. Very few of us would want this. So, that's out, but is there a non-cultural way to get the answer? This is the quest for the moral realist. It's not going to be based on belief (as the relativist points to).

Recall the Cultural Differences Argument. This argument relies on moral conflicts/disagreements. But, do we always settle debates like that? There may be disagreement about something, anything (for that matter), but it doesn't follow that there's no fact of the matter. If there's a conflict about math, we don't say that there's no answer, we consult the rules of mathematics (as they were discovered, not invented). When there's a conflict in science, we don't say that there's no answer, we perform experiments and discover the truth. For the moral realist, settling moral conflicts is more like settling a conflict in math or science than one in art. Realists perform experiments and consult the rules of morality to settle the debates (and the experiments either further support the rules or give examples of amendments which should be made). The quest to get the non-relativist rule to settle moral conflicts is hard, but it does not seem impossible, we do make progress in it. There's even a scientific-style method for figuring out which ethical theories should be applied and/or how they should be amended (as we will see in the next module).

In short, no one decides who's right in a moral conflict, it's just a fact that one is right and the other is wrong.

\section{Part \thechapcount.\theseccount: Where do moral truths come from?}\stepcounter{seccount}

Moral truths are the answers to the ``what should I do?" question (in the relevant sense). The moral cultural relativist has a simple, again almost too simple, answer ``from the beliefs of the culture." Yet again, how can we possibly say that a culture has the wrong answer to the question when the culture chooses the answer? Like the issues before, this just can't be right. So, the question for the realist is just that, where do they come from? 

For the realist, this is like asking ``where do scientific truths come from?" or ``where do mathematical truths come from?". Some moral realists point to God and say that all truths flow through Him, moral truths included, saying that God made them (an easy way to get all-knowing and all-good). Others point to the sort of creatures we are and our place in nature, these give us moral truths. Others still point to abstract notions of well-being or rationality.  And even more say that moral truths are things which just could not have been otherwise, they just always are, they didn't ``come from" anything. There is some debate about where they come from.

But, don't lose heart! There's just as much debate about where mathematical and scientific truths come from. For example, in the case of math, we could claim that the truths are necessary, they could not have been otherwise. We could claim, with just as much evidence, that the truths are constructs of the definitions of the terms which we are using overlapping in consistent ways (1+2=3 because of the definitions of the terms and operations). And we could also claim that the truths came from God.  As it turns out, in philosophy, the most common stance (for where mathematical truths come from and where moral truths come from) is that they always were and always will be.

\section{Part \thechapcount.\theseccount: What about forcing them onto another culture?}\stepcounter{seccount}

This is a hard problem, because one of the core intuitions behind the Moral Relativist, and one of the big reasons why the stance is contradictory, is that it’s wrong to force your morals onto another. The Moral Objectivist will claim that under the right circumstances, it’s morally required that you force another to change. To some, this feels wrong. This feeling, often, gets traced back to the horrors of history, which I used as examples for the intuition behind the Cultural Imperialism Argument, where one group imposed themselves on another. At the same point, we need to look at history as a whole. There have been times when it was actually correct for one group to come in and force a change. In those cases, the imposing groups had real morality on their side and in the cases where it was wrong, we can say that they were mistaken about morality. For example, the Nazis in WW2 were just wrong about morality and the groups which came in were right. So, the Moral Objectivist just needs to be sure that their moral theory is getting the right answer in the case. This is an epistemological worry for the objectivist, not an ethical, meta-ethical, or metaphysical one. 

Interestingly, when we look at our own history, we find things which we are ashamed of, things which we condemn the previous generation for. How and when this is correct is easy for us to see. Since different people at different times are, basically, different cultures, the exact same, or a similar, thought process applies when condemning or appreciating the behaviors of another culture. Kwame Appiah in What Will Future Generations Condemn Us For?\autocite{Appiah1} gives us some compelling tests for when future generations will be ashamed of our practices. These same tests apply to the behaviors of other cultures and tell us when it may be OK to force a change. 

\section{Part \thechapcount.\theseccount: Moral Nihilism}\stepcounter{seccount}

Moral Relativism and Moral Objectivism (Realism) are not the only meta-ethical stances, there are two others worth noting (though this does go beyond the necessary scope of this class). Like Moral Relativism, these stances deny that there are absolute moral truths. But, unlike Relativism, these theories claim that they aren't relative. They claim that there's no truth to them at all, regardless of context. Theories which do this fall into a category called `Moral Nihilism'. Like Relativism, Nihilism comes as a family and you can pick and choose accordingly. Skepticism is most like Nihilism in how the family tree is organized. 

Moral Nihilism, like Moral Relativism, is limited, it only makes claims about moral claims. Like Relativism, if you find one absolute, objective truth, then Global Nihilism is false, but for Limited Nihilism, you would need to find it in the context it's limited to. The two major theories which call Moral Nihilism home are `Error Theory' and `Expressivism'. Of the two, Error Theory is the strongest, but it can be quite counter-intuitive. 

\subsection{Error Theory}

Error Theory, applied differently, is a concept and stance which can be found in other areas of philosophy, though I have personally mostly encountered it in Metaphysics. It is a lot like Skepticism, but rather than saying that we will never know about the thing, it says that we will always be mistaken.  For Meta-Ethics, this is the stance that all of our moral judgments will be mistaken. We will always be in error, hence the name. This is true even if we were to randomly chose an answer. For the \glspl{error theory}, we will ask questions about a subject, sincerely try to answer them, but will always get the wrong answer. There is no right answer.  Applied in other fields, it claims that all of our judgments in those fields will always be in error. They come to this conclusion from three assumptions, or starting points, which get them off the ground:

First, they need to have the standard-issue feature of a moral nihilistic stance, this is that morality is not a feature of the world. In other words, there are no moral facts. Some could say that morality is a useful fiction or say that morality is self-contradictory or some other means. The second feature is a bit more precise to this stance. This is the claim that no moral judgments will ever be correct/accurate.  This follows, using some fairly simple reasoning, from the first feature. For our judgments to be accurate/correct, they must correctly/accurately depict something. However, because of the core premise to Moral Nihilism, there's nothing for the judgment to correctly depict, so it can never be accurate/correct. The third aspect follows suit, they claim that it's pretty obvious that people try to use moral judgments, we act on them, and we think about them, but those judgments will always be in error, hence the name. Since our judgments can never be correct, they will always be wrong. The error theorist is making a pretty big claim. They aren’t just trying to bash on social policies and individual actions. They are going all out and claiming that morality is a fiction.

\newglossaryentry{error theory}
{
  name=error theory,
  description={A general claim that all other claims within a field of study or context will always be mistaken, try as we might to say something truthful, we cannot because there are no truths in that area or context},
plural=Error Theorist
}


They are essentially the atheists of ethics. Atheists hold that there are no religious features in the world, the error theorist holds the same about ethical features. The atheist holds that, try as you might, all religious doctrines are mistaken because (for example) God does not exist. The error theorist holds that, try as you might, ethical judgments/beliefs are all mistaken because morality does not exist.

\subsubsection{For Error Theory to be Correct}

All error theorists hold that the basic mistake in moral thinking is that it depends on some absolute or objective truth about morality. This truth applies to all of us, regardless of who or what we are. These are what some refer to this as ‘objective morality’ and ‘categorical reasons’.

To get error theory off of the ground, they need to convince us of 2 things. First, morality really does depend on these absolute standards (they would need to come up with the correct moral theory, examples are in Module \ref{ch.modeight}). Second, supposing that it does, they have to show us that at least one of these is false, which would mean that the correct theory to describe morality is fundamentally and irreparably flawed. So, for the Error Theorist to prove themselves, they need to solve all problems in Ethics and then show that the stance which did it couldn't be right. This is quasi-contradictory.

\subsection{Expressivism}

Expressivism could, in principle, be a stance in other areas of philosophy and maybe, for some of them, it could make a solid stance, but I have not encountered it so much in my time and, when I have, it has almost exclusively been in Ethics. As I mentioned before, given the core flaws in trying to prove Error Theory (in the case of Ethics, in other fields this issue is not present), 
\Gls{expressivism} is the stronger of the two. As it is a Nihilistic stance, it does come with two assumptions, but it differs from Error Theory in the last aspect.

\newglossaryentry{expressivism}
{
  name=expressivism,
  description={A general claim that all other claims within a field of study or context are not intended to and do not express anything true or false, rather those claims are expressing emotions, commands, or questions. In ethics, saying something is wrong, according to this stance, is the same as saying something like `boo' and saying that something is right is the same as saying `yay'},
plural=expressivist
}


First, as with the Error Theorist, the Expressivist claims that morality is not a feature of the world. But, it doesn't say that it's a useful fiction or anything like that. It also claims, but the same reasoning, that moral judgments are never accurate. But, it does not say that they are always wrong. I know that this last bit seems odd, so let me explain: The Moral Objectivist (Realist), Relativist, and Error Theorist all share one thing in common: When we make moral claims, we are, at least, trying to describe the world, pick out some feature in it. The Expressivist is different: They claim that when we make moral claims our intent is to express something, not describe. They think that our ethical statements/judgments aren't the sort of things which can be true/false, correct/incorrect, or accurate/inaccurate. Take these two examples: 
\noindent
\begin{tabular}{p{2.75in}|p{2.75in}}
Example A&Example B\\\hline
    Murder is wrong&Blue is a color
\end{tabular}

Although it may look like these examples are stating the same sort of thing, and, in fact, were we to  do a sentence/semantic diagram for these examples, they would look freakishly similar, according to the Expressivist, these are fundamentally different sorts of sentences. In linguistics, profanity is often claimed to not add any meaning to a sentence, but rather is an emotional or rhetorical expression; the Expressivist says, basically, that moral claims, like (A), are the same sort of thing, they are only emotionally/rhetorically meaningful. In the case of (B), it is expressing something true about the world, there's a collection of things, colors, and it's putting `blue' in that collection (by one theory) or there's a thing in the world, `blue', and assigning a feature to `is a color' (by another theory, this is what I really work in). But, in the case of (A), we aren’t assigning a feature to murder, according to the Expressivist. Rather it is like we are saying something like:
\begin{earg}
    \item[]BOO murder!
    \item[]Don’t murder!
    \item[]Let’s not murder!
    \item[]Wouldn’t it be nice if we didn’t murder?
\end{earg}
Those sorts of statements aren’t the kinds of things which are true or false, they are more like actions or commands or questions (for interesting content on statements as actions, check out J.L. Austin's How To Do Things With Words). If you have ever been to a highly emotionally charged (peaceful) protest, people screaming statements which boil down to ``this is wrong" will feel more like ``boo this thing" rather than something based on an intellectual process, which is evidence in support of this sort of stance. 

\subsection{Three Concerns for Expressivism:}

There are three major worries for Expressivism. Although, I will say, Expressivism is the most popular, it is by far the weaker of the two stances and it will need some time in the gym, so to speak, to really hold its own. The first worry for the Expressivist concerns how we really reason about morality. Take the following arguments:

    \begin{earg}
    \item[]All cases of hurting people are immoral.
    \item[]Torture hurts people.
    \item[]Therefore, torture is immoral.
   \end{earg}

\begin{earg}
    \item[]All men are mortal.
    \item[]Socrates is a man.
    \item[]Therefore, Socrates is mortal.
\end{earg}

In fact, we use the exact same style of reasoning in both cases, agree or disagree with it, the logic is exactly the same. But, according to the Expressivist, the first argument looks like this:
\begin{earg}
    \item[]Hurting people - YUCKY
    \item[]Torture hurts people.
    \item[]Therefore, BOO torture. 
\end{earg}
We reason about morality logically, not emotionally (sometimes we do both, but when we are thinking clearly, emotions play a supporting role, they aren't the main character). According to Expressivism, however, we wouldn't do this, it would only be emotional. This observation, if correct, means that Expressivism can't be accurate. 

The second concern for Expressivism comes from Amoralism. An amoralist is a person who sincerely makes moral claims but is completely unmoved into action by them. These people might be out there, this would be like Data from Star Trek, in that they would not feel the emotions associated with ethical claims. If they do exist, this is a serious problem for the Expressivist. Expressivism claims that moral claims are emotional expressions. But, emotional expressions, most of the time, move us to action (even if we don’t actually act, we still kind of want to). How can a person sincerely make a moral claim, which is supposed to be an emotional expression, without being moved to action?  Basically, if a person can sincerely make a moral claim without being moved by it, then Expressivism is incorrect. 

The final concern for the Expressivist concerns how we actually make these moral judgments, much like the first. The absolutist (objectivist/realist), the relativist, and the error theorist disagree on just about everything, but they do agree on one thing, that moral claims are, at least, trying to describe the world. The Expressivist denies this. They have to paraphrase the moral claims to not have them attempt to describe the world. But, if this were really the case, why wouldn’t we just express our emotions and not conceal them? There are many statements, which are ethical in nature, which the Expressivist can't handle, many of which I have heard people make in the real world, for example:
\factoidbox{
    I’m not sure whether torture is moral, but I think people smarter than me would know.
    
    There’s a difference between being moral, being required, and being praise worthy.
    
    How much we punish should match how bad the crime was.
    
    If war is immoral, then generals are not as good as they seem.}

If we read these like the other stances would, then we have no problems, all are clearly understandable, but if we read them as the Expressivist thinks they should be, we are up a creek. We could all be lying or mistaken, but when we think about what we are trying to do when we make these statements, we see that we are really trying to describe the world. 