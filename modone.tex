\part{What is Philosophy?}
\label{ch.modone}
\addtocontents{toc}{\protect\mbox{}\protect\hrulefill\par}

\chapter{Part \thechapcount: What Philosophy Is and Isn't}
Philosophy is not just a way of life, it is not just some outlook which a person takes on. It is also not the product of some really deep thinking. Philosophic thought is not philosophy. One does not have a philosophy. That's not how it works. `My philosophy on that is...' is like saying `my science on that is...' What they should say is `my belief is that...' or `I have thought long and hard about it and I think...' Philosophy is critical thinking taken to the ridiculous extreme. Since it is, at its core, thinking, one does not simply study philosophy. Philosophy is an activity, it's something that you do. Doing philosophy involves finding a stance on some issue and then coming up with reasons for that stance. This could be a stance that you hold or it could be the stance that your opposition holds in some debate.\footnote{This would be putting yourself in `their shoes', so to speak, so that you can see where they are coming from and thereby exploit the flaws in their stance.} In order to really truly prove your point, you will want to know why another might believe otherwise and have the skills to show the errors in that reasoning. Engaging in philosophy is how you get the skills to find the errors in the reasoning. 

Since philosophic activities tend to go really abstract really quickly and since the questions which philosophers tend to think about are very difficult to answer, some have claimed that Philosophy is a pointless endeavor. This is, however, not accurate. All fields of study are defined, roughly, in terms of the questions which they are trying to answer or have the ability to answer. The questions in Philosophy tend to be those which other fields can't answer. In those cases, most of the time, the standards are very high, higher than other fields can match. Sometimes, those standards are so high that some claim that they are not possible to answer. Again, this is not accurate. There are answers, even if those answers are difficult to get. Doing philosophy tends to involve the following activities:


\begin{tabular}{p{2in}|p{3.5in}}\hline
Philosophy Involves&Example\\\hline
Resolving Confusion &Noticing that people misunderstand something because of a word-choice.\\\hline
Unmasking Assumptions &Seeing that you/another hold a position for a reason which was not stated.\\\hline
Revealing Presuppositions &Seeing that there are unexamined/unexplained jumps in reasoning or an aspect of the reasoning which is hidden.\\\hline
Distinguishing Importance &Figuring out which things you need to do vs which things you want to do.\\\hline
Testing Positions &Putting yourself in another's shoes to see whether their stance makes sense or to see whether your own stance holds water.\\\hline
Correcting Distortions &Fixing/identifying misrepresented stances or facts, seeing that you are being misinformed/lied to.\\\hline
Looking for Reasons &Asking yourself why a person would think/do what they are thinking/doing\\\hline
Examining World-Views &Seeing whether another's explanation for something or some course of behavior is correct/accurate to the world itself. People can have wrong opinions.\\ \hline
Questioning Conceptual Frameworks &Seeing whether a particular way of carving the world, a way of thinking about it, is better or worse than another way. We may think that the world is one way, but is it?
\end{tabular}

\section{Part \thechapcount.\theseccount: How Does One Do Philosophy?}\stepcounter{seccount}
\label{s:p1.1}
As a kind of inquiry, philosophy is aimed at establishing knowledge and understanding. Even when you can't get exact and precise knowledge about a topic, you can often find interesting things to learn (for example, why you can't get that knowledge). So, rational inquiry (critical thinking, doing philosophy) may be interesting and fruitful even when we don't get straight-forward answers. Once we raise a philosophical issue, a question or puzzle, whether about the nature of justice or about the nature of reality, we want to ask what can be said for or against the various possible answers to our question. For example, if my question is `when is it morally permissible to lie?', I look at cases where it seems OK and cases where it just seems wrong. I look at the \textbf{reasons} why they seem that way.

At this point, I am making \glspl{argument}. These are my sets of reasons for a stance. Some arguments give us better reasons or accepting their \glspl{conclusion} (what is being argued for) than others. Once we have an argument, we want to evaluate the reasoning it offers. So, if you want to know what philosophers do, this is a pretty good answer: philosophers formulate and evaluate arguments. We look at the reasons for some stance and figure out whether those reasons actually hold water, so to speak.  Your introduction to philosophy should be as much a training in how to do philosophy as it is a chance to become acquainted with the views of various philosophers. To that end, you should carefully study the sections below on arguments. I personally approach teaching philosophy from the perspective that there's still a lot going on, I show you the relevance of philosophy to questions which we are puzzled with today (like Artificial Intelligence, the Abortion Debate, the Existence of God, and so on). Through this, you will become accuanted with philosophers, both historical and contemporary, but this is more of a byproduct than a goal.

Once we have some philosophical position to think about, we want to ask what arguments can be made for or against it (\textbf{formulate}). We then want to examine the quality of the arguments (\textbf{evaluate}). Evaluating flawed arguments, figuring out why they are wrong, often leads us to other, stronger, arguments and the process of formulating, clarifying, and evaluating arguments continues.

This circular method of question and answer (make an argument, evaluate it, make a stronger one, and repeat) is known as a \gls{dialectic}. A dialectic looks a lot like debate, but a big difference is in the goals of the two activities. The goal of a debate is to win by persuading an audience that your position is right and your opponent’s is wrong.

\newglossaryentry{dialectic}
{
name=dialectic,
description={the art of investigating and determining the truth of some stance or opinion},
plural=dialectics
}


A dialectic, on the other hand, is aimed at inquiry. The goal is to learn something new about the issue under discussion. Unlike debate, in a dialectic your sharpest critic is your best friend. Critical evaluation of your argument brings new evidence and reasoning to light. The person you disagree with on a philosophical issue is often the person you stand to learn the most from (and this doesn’t necessarily depend on which of you is closer to the truth of the matter).

A common dialectic used in Philosophy is sometimes referred to as the Socratic Method after the famous originator of this systematic style of inquiry.\footnote{It should be noted that this style of dialectic is not unique to Socrates, the Greeks, or even the western world. Socrates just so happened to become the most famous practitioner of it. In what is today India, the philosopher/monk Moggaliputta-tissa (327–247 BCE) engaged in this style of investigation. See \citetitle{Katha} for examples. In the Americas, especially Mesoamerica, the Nahuas have tlamatinime (singular tlamatini) who regularly engage in this kind of dialectic. `Tlamatini' literally translates as `knower of things' but, given the role they fill in society and the manner in which they fill that role, `philosopher' is likely a better translation.} For this module, the reading is one of the more famous of Plato’s dialogues, namely The Apology. This will give you a good sense for how the Socratic Method works. Then watch for how the Socratic Method is deployed throughout the rest of the course.

Doing philosophy boils down to thinking really hard, but in a structured and organized way. The thinking must be critical and comprehensive. The critical nature of philosophy in general makes some people uneasy. A person gives reasons for a stance and then a good philosopher will ask questions about those reasons and judge whether they `hold water', so to speak. We are all uncomfortable with being proven wrong or being made to look like a fool (if the philosopher isn't gentle about it), but being wrong is a necessary aspect of learning, growing, and becoming better. In pre-colonial Nahuatl societies, there was/is a role called `tlamatini', philosopher. When early anthropologists during the colonial period asked the Nahuas to describe a good tlamatini, they replied with:\footnote{It should be noted that while the pronoun used in these passages is masculine, being a tlamatini was not and is not a gendered role, there have been women tlamatini and they have been well acclaimed for their successes.} 

\factoidbox{The wise man: a light, a torch, a stout torch that does not smoke.

A perforated mirror, a mirror pierced on both sides.

His are the black and red ink, his are the illuminated manuscripts, he studies the illuminated manuscripts.

He himself is writing and wisdom.

He is the path, the true way for others.

He directs people and things; he is a guide in human affairs.

Teacher of truth, he never ceases to admonish.

He makes wise the countenances of others; to them he gives a face; he leads them to develop it.

He opens their ears; he enlightens them.

He puts a mirror before others, he makes them prudent, cautious; he causes a face to appear on them.

He attends to things; he regulates their path, he arranges and commands.

He applies his light to the world.

Thanks to him people humanize their will and receive a strict education.\autocite[p. 10-11]{Leon-Portilla1}}

Similarly,  Bernardino de Sahag\'un, directly interviewing Nahuas who lived prior to colonization and remember the time, records the following description of a good tlamatini and a bad tlamatini (likely an amalgamation of many different people's accounts):

\factoidbox{
    The good [tlamatini] is a physician, a person of trust, a counselor; an instructor worthy of confidence, deserving of credibility, deserving of faith; a teacher. [He is] an advisor, a counselor, a good example; a teacher of prudence, of discretion; a light, a guide who lays out one’s path, who goes accompanying one. [He is] reflective, a confessor, deserving to be considered a physician, to be taken as an example. He bears responsibility, shows the way, makes arrangements, establishes order. He lights the world for one; he knows the land of the dead; he is dignified, unreviled. He is relied upon, acclaimed by his descendants, confided in, trusted -- very congenial. 
    
    The bad  [tlamatini] is a stupid physician, silly, decrepit, pretending to be a person of trust, a counselor, advised. He is vainglorious, vainglory is his; he is a pretender to wisdom . . ., vain, -- discredited. He is a socerer, a soothsayer, a medicineman, a [public robber.] A soothsayer, a deluder, he deceives, confounds, causes ills, leads into evil...\autocite[Book 10 page 29-30]{Sahagun1}}

In both of these passages, we see that the role of a philosopher (tlamatini) is to advise, to council; the philosopher is to make people prudent, cautious. But why is that? Through dialectics and arguments, the philosopher exposes where people went wrong in their reasoning and directs them back on the right path, this makes them careful so that they don't go astray.\footnote{Sadly, much of the philosophical work which a tlamatini did was done through interpersonal conversations, not written down. This was true also for most of classical Greek Philosophy (like that of Socrates) and for most of classical Chinese Philosophy (like that of Confucius). What we do have from the pre-colonial period shows that the dominant style for \emph{written} Philosophy was \emph{poetry}, which makes some aspects of analysis difficult.\autocite[p. 36-37]{McLeod1}  As far as I am aware, the Nahuas did not have someone like Plato who passionately wrote down everything which happened in these exchanges.} Personally, I love the line ``he puts a mirror before others, he makes them prudent, cautious.'' This is a wonderful depiction of what Socrates and other philosophers do when grappling with a problem. We expose the faults in the reasoning or illuminate the good traits in the reasoning. When discussing something with another, we expose the faults or goods in their reasoning. Similarly, in both of these passages, we see that a philosopher (tlamatini) is interested in \emph{truth}. We want to discover the truth of some topic and then illuminate others to that truth. Arguments serve both of those desires. Not only can they be used to discover the truth but they are also used to expose others to it. 

Before we dive too deep into the nature of arguments, however, given the kind of world we live in today, it is worth taking a look at what philosophers mean by `truth'.

\subsection{Truth}
\label{s:p1.1.1}
Since both science and philosophy are mainly concerned with getting at knowledge and understanding about the world (though the kind of knowledge may be argued to be different), it is natural to think that both are after the truth about things. There are some interesting and some confused challenges to the idea that philosophy and science are truth oriented. But for now let’s assume that rational inquiry is truth oriented and address a couple of questions about truth. Let’s focus on just these two:
\begin{earg}
    \item[\ex{truth1}] What is it for a claim to be true?
    \item[\ex{truth2}] How do we determine that a claim is true?
\end{earg}
It’s important to keep these two questions separate. Questions about how we know whether something is true (like the second question) are epistemic questions. Epistemic questions, which we will return to in a later module, are questions about knowledge, beliefs, reasons, or, to a certain extent, faith. But the question of what it is for something to be true (the first question) is not an epistemic issue. The truth of a claim is quite independent of how or whether we know it to be true. Questions about whether or not something is true are, more or less metaphysical questions.

For example, consider these two claims:

\begin{earg}
    \item[] There's extraterrestrial life.
    \item[] There's no extraterrestrial life. 
\end{earg}


I assume we don’t know which of one of these is true, but surely one of them is, for any sensible claim that you make, it's either true or false. Whichever of these claims is true, its being true doesn’t depend in any way on whether or how we know it to be true. So, there is a correct answer to the question ``is there extraterrestrial life?" but as it sits, we don't know what that answer is. There are many truths that will never be known or believed by anyone, and appreciating this is enough to see that the truth of a claim is not relative to belief, knowledge, proof, or any other epistemic notion.\footnote{There are certain facts within mathematics which are not knowable, this could be because of the physical limitations of the computations or because of the nature of the question which the fact is an answer to. For example, there is a first digit to Graham's number, but it is so large that it's physically impossible to determine.} Similarly, unless you hold certain philosophic positions about certain things (we will see this in the Relativism/Meta-Ethics Module), just because you believe something, that does not make it true. How much weight you give a belief also doesn't change how true it is. Unlike somethings which you may encounter, like knowledge, certainty, belief, and so on, truth does not come in degrees (and this is regardless of the stances you take). 

But then what is it for a claim to be true? The ordinary everyday notion of truth would have it that a claim is true if the world is the way the claim says it is. And this is pretty much all we are after for this class. When we make a claim, we represent some part of the world as being a certain way. If how my claim represents the world fits with the way the world is, then my claim is true. Truth, then, is correspondence, or good fit, between what we assert and the way things are. There are other accounts of truth out there, which do better or worse jobs at various things (for example, this account can't make ``Harry Potter is a wizard"\autocite{HarryPotter1}, ``Stepan Arkadyevich Oblonsky's wife is mad at him"\autocite{AnnaKarenina}, or ``Sherlock Holmes is a detective"\autocite{Sherlock} true, because those beings don't exist), but this works for our purposes.  

\subsection{Truth and Meaning}
\label{s:p1.1.2}

A potential confusion about truth comes from confusing a sentence with the meaning behind it. This is where we get into a very basic introduction into Philosophy of Language. This area of philosophy deals with things such as meaning, reference, and so forth (Philosophy of Language is what I mostly work in, it deals with semantics while Linguistics deals with syntax, but the fields overlap quite a bit). Even when we aren't trying very hard, we can use words and sentences in a ton of different ways and there seems to be some vagueness or ambiguity in natural languages. For example, I can easily make examples where sentences might have two or more different interpretations, like double or triple entendres. 
\begin{earg}
\item[\ex{ambi1}] GOP grills the IRS chief over lost emails.
\item[\ex{ambi2}] If your dog poops, you must put it in the trash can.
\item[\ex{ambi3}] A woman gives birth in the UK every 48 seconds.
\item[\ex{ambi4}] People wanted for pickling and canning.
\end{earg}
All four of the above examples have at least 2 different interpretations and, in each case, one seems true while the other seems false. In Example \ref{ambi1}, it could be interpreted to mean that the GOP are having an IRS chief BBQ over a server or it could mean that they are harshly questioning the chief. In Example \ref{ambi2}, one way to understand this is that it's telling you to put your dog in the garbage can, while another is telling you to put the poop therein. In Example \ref{ambi3}, we could say that there's a single woman who is having a baby every 48 seconds (she must be tired) or that some woman, not necessarily the same woman, is having a child in that time. In Example \ref{ambi4}, it could be that we have a cannibal looking to store their meats for the winter using canning and fermenting techniques or it could be that a food fermentation business is looking to hire on some more people in those departments. Given these examples, and how often we misunderstand each other on a daily basis,  it might be tempting, therefore, to think that the truth of a sentence must be relative to its interpretation.  In an even more robust example, imagine the following:

\factoidbox{Suppose that we all collectively switched the sorts of things the words `dog' and `cat' pick-out. So, the word `dog' now is used to point out the meowing critters and the word `cat' refers to the barking ones.  In this case, it would seem that the sentence `dogs are canines' would be false and `dogs are feline' would be true. If we, again, flipped the meanings of the words `feline' and `canine', we would get that `dogs are canines' as true, but for a totally different reason than it was originally.}

But does this make truth open for interpretation? Well, no, but in a sense, yes. When we look at a language, from one perspective, we notice that things like words and sentences are nothing more than characters on a screen/page, random collections of sounds, or certain structured gestures (in the case of sign languages). Those things, on their own, don't have meaning. There must be something extra, beyond the sound and signs, which has the meaning.  Philosophers call the meaning behind a sentence a proposition. A proposition, itself, is not a sentence or a word, rather it's the meaning behind the sentences.


\begin{center}
\begin{tabular}{p{2in}|p{1.5in}|p{2in}}\hline
Sentence &Language &Proposition\\\hline
Snow is white &English &that snow is white\\\hline
Schnee ist weiss &German &that snow is white\\\hline
Nix alba est &Latin &that snow is white\\\hline
La neige est blanche &French &that the snow is white\\\hline
\end{tabular}
\end{center}

All of the above examples are different sentences, made clear because they are in different languages, but they all express the same proposition. The truth of a sentence is relative to the truth of the proposition attached to it, propositions are the things which are true or false, and then the truth of the proposition is determined by its correspondence with reality. Translators, at least the good ones, often take the sentence in one language, figure out the proposition connected to it, and then express the same proposition in the necessary language.\footnote{Sometimes there can be cases where a language has the ability to express a proposition which is not possible to express in another without adding something to that language or giving information beyond what was in the original statement. For example, take Cicero's Epistulae ad Familiares 9.22, the examples Cicero gives in the letter of profanity and why they make sense are  not translatable without explaining some aspect of Latin phonology and semantics.}

So the proposition expressed by a sentence is not itself a linguistic thing. Propositions themselves don't have meaning, but rather they are the meaning behind a sentence. For a bit of language to be open to interpretation is for us to be able to attach different propositions (meanings) to it. But the meanings themselves are not open to further interpretation. And it is the proposition, what is meant by the sentence, which makes the statements, sentences, true or false. So, when I speak of arguments consisting of premises (the supporting evidence and their arrangement in the argument), I am talking about the core meaning behind the sentence, not the sentence itself. If we misinterpret the sentence, then we haven’t yet gotten on to the claim being made and hence probably don’t fully understand the argument. Getting clear on just what an argument says is critical to the dialectical process.

\subsection{Simplified Breakdown}

Even if you are exceptionally bright, you probably found the last couple paragraphs rather challenging. That’s OK. You might work through them again more carefully and come back to it in a day or two if it’s still a struggle. The path to becoming a better critical thinker is more like mountain climbing than a walk in the park, but with this crucial difference: no bones get broken when you fall off an intellectual cliff. So you are always free to try to scale it again. We can sum up the key points of the last few paragraphs as follows:
\begin{earg}
    \item[] We use sentences, bits of language, to express propositions.
    \item[] The proposition, what is meant by the sentence, represents the world as being some way.
    \item[] The proposition is true when it represents the world in a way that corresponds to how the world is.
    \item[] Truth, understood as correspondence between a claim (a proposition) and the way the world is, is not relative to meaning, knowledge, belief, or opinion.
\end{earg}
Hopefully we now have a better grip on what it is for a claim to be true. A claim is true just when it represents things as they are. As is frequently the case in philosophy, the real work here was just getting clear on the issue. Once we clearly appreciate the question at hand, the answer seems pretty obvious. So now we can set aside the issue of what truth is and turn to the rather different issue of how to determine what’s true.

\subsection{The connection}

Arguments are how philosophers, and scientists, and the rest of us get at the truth. We are using them to structure our reasons and prove that the conclusion is true.

\section{Part \thechapcount.\theseccount: Arguments}\stepcounter{seccount}
\label{s:p1.2}


The common-sense, everyday, way to tell whether a claim is true or false is to look at the reasons for or against it. Sometimes our observations give us good reasons. For example, I have a good reason for thinking my bicycle has a flat tire when I see the tire sagging on the rim or hear air hissing out of the tube. But often the business of identifying and evaluating reasons is a bit more involved. Logic is the business of identifying and evaluating reasons. You do this all the time. You give yourself reasons to choose one shirt over another for a job interview, you reason your way through making a choice about which classes to take, you believe certain things based on evidence, and you give reasons for why something isn't your fault (when you make excuses). In all of these cases and more, you are making \textbf{arguments}.

This is very different from how you might use the term ‘argument'. In everyday language, we sometimes use the word ‘argument’ to talk about belligerent shouting matches. If you and a friend have an argument in this sense, things are not going well between the two of you. Logic is not concerned with such teeth-gnashing and hair-pulling. They are not arguments, in our sense; they are just disagreements. For a humorous example of how arguments and disagreements differ, take a look at this funny exchange from Monty Python:\autocite{ArgumentClinic}

\factoidbox{\begin{multicols}{2}
Man: Is this the right room for an argument?\\
Other Man:(John Cleese) I've told you once.\\
Man: No you haven't.\\
Other Man: Yes I have.\\
M: When?\\
O: Just now.\\
M: No you didn't!\\
O: Yes I did!\\
M: You didn't!
\ldots\\
O: Oh I'm sorry, is this a five minute argument, or the full half hour?\\
M: Ah! (taking out his wallet and paying) Just the five minutes.\\
O: Just the five minutes. Thank you.\\
O: Anyway, I did.\\
M: You most certainly did not!\\
O: Now let's get one thing quite clear: I most definitely told you!\\
M: Oh no you didn't!\\
O: Oh yes I did!
\ldots\\
M: Oh look, this isn't an argument!\\
(pause)\\
O: Yes it is!\\
M: No it isn't!\\
(pause)\\
M: It's just contradiction!\\
O: No it isn't!\\
M: It IS!\\
O: It is NOT!
\ldots\\
M: (exasperated) Oh, this is futile!!\\
(pause)\\
O: No it isn't!\\
M: Yes it is!\\
(pause)\\
M: I came here for a good argument!\\
O: AH, no you didn't, you came here for an argument!\\
M: An argument isn't just contradiction.\\
O: Well! it CAN be!\\
M: No it can't!\\
M: \emph{An argument is a connected series of statements intended to establish a proposition.}\\
O: No it isn't!\\
M: Yes it is! ‘tisn't just contradiction.\\
O: Look, if I \emph{argue} with you, I must take up a contrary position!\\
M: Yes but it isn't just saying ‘no it isn't'.\\
O: Yes it is!
\ldots\\
M: No it ISN'T! \emph{Argument is an intellectual process.} Contradiction is just the automatic gainsaying of anything the other person says.\\
O: It is NOT!\\
M: It is!
\ldots 
\end{multicols}}

An argument is a reason for taking something to be true. \Glspl{argument} are made out of two or more claims, one of which is a \gls{conclusion}. The conclusion is the claim the argument purports to give a reason for believing. The other claims are the \glspl{premise}. The premises of an argument taken together are offered as a reason for believing its conclusion. In the above exchange, the person looking for an argument claims that ``An argument is a collected series of statements to establish a definite proposition". This is a close approximation to how philosophers use the term, but a better, more exact definition is:


\newglossaryentry{argument}
{
name=argument,
description={A connected series of sentences, divided into \gls{premise}s and \gls{conclusion}},
plural=arguments
}

\newglossaryentry{premise}
{
name=premise,
description={A sentence in an \gls{argument} other than the \gls{conclusion}, often indicated with a premise indicator},
plural=premises
}

\newglossaryentry{conclusion}
{
name=conclusion,
description={The sentence which an argument is intended to support or prove, often indicated with a conclusion indicator.},
plural=conclusions
}

\begin{center}
An argument is a collected series of propositions intended to establish others in the series.
\end{center}
The propositions (statements, in this case) which support others are our premises. The one being supported is the conclusion. That `intent' bit is a heavy hitter as we will soon see.

Some arguments provide better reasons for believing their conclusions than others. In case you have any doubt about that, consider the following examples:


\noindent
\begin{tabular}{p{2.75in}|p{2.75in}}\hline
Example \exarg{linecook1}: &Example \exarg{linecook2}:\\\hline
Sam is a line cook. &Sam is a line cook.\\
Line cooks generally have good kitchen skills. &Line cooks generally aren't paid well.\\
Therefore, Same can probably cook well &Therefore, Sam is probably a billionaire. 
\end{tabular}

\noindent\begin{tabular}{p{2.75in}|p{2.75in}}\hline
Example \exarg{boston3}: &Example \exarg{boston4}:\\\hline
Boston is in Massachusetts. &Boston is in California.\\
Massachusetts is east of the Rockies.&California is west of the Rockies.\\
Therefore, Boston is east of the Rockies.&Therefore, Boston is west of the Rockies.
\end{tabular}

The premises in Example \ref{linecook1} provide pretty good support for thinking Sam can cook well. That is, assuming the premises in the first argument are true, we have a good reason to think that its conclusion is true (at this stage, we are not interesting in whether the premises are actually true, only the structure). Whether or not the argument is any good depends on how well they establish the conclusion, relative to the intent. So, we can say that the reasoning in Example 1 is pretty good (at least). The premises in Example \ref{linecook2} give us no reason to think Sam is a millionaire, let alone a billionaire. So whether or not the premises of an argument support its conclusion is a key issue.

Looking at Examples \ref{boston3} and \ref{boston4}, we see some similarities. The intent is different (in the first two, the goal was to get something with likelihood, in these, the intent is to get something with certainty), but we seem not to like the second (Example \ref{boston4}) and the first seems OK (Example \ref{boston3}). The main problem with Example \ref{boston4} is very different than the problem with \ref{linecook2}. And, in fact, the issue is an entirely different animal. With Example \ref{linecook2}, the issue was with the structure, here, the issue is with the truth. Notice, the structure of the arguments, Examples \ref{boston3} and \ref{boston4}, are exactly the same, there's no difference in how they are being reasoned.  If its premises were true, then we would have a good reason to think the conclusion is true (the best reason, 100\% certainty reasoning). That is, the premises do support the conclusion. But the first premise of the second argument just isn’t true. Boston is not in California. So the latter pair of arguments suggests another key issue for evaluating arguments. Good arguments have premises which support the conclusion, the best arguments have true premises.

That is pretty much it. The best arguments are those which have true premises that, when taken together, support its conclusion. So, evaluating an argument involves just these two essential steps:
\begin{enumerate}
    \item Determine whether or not the premises are true.
    \item Determine whether or not the premises support the conclusion (that is, whether we have grounds to think the conclusion is true if all of the premises are true).
\end{enumerate}
Often, figuring out whether the premises of an argument are true involves looking at further arguments for those premises individually. An argument might be the last link in a long chain of reasoning. In this case, the quality of the argument depends on the whole chain. Really high-quality philosophy papers (I don't expect these in this class) often involve an argument and further arguments for each of the premises. And since arguments can have multiple premises, each of which might be supported by further arguments, evaluating one argument might be more involved yet, since its conclusion is really supported by a rich network of reasoning, not just one link and then another. While the potential for complication should be clear, the basic idea should be pretty familiar. Think of the little kid who would constantly ask their parent “why?” after being an explanation. Even at a young age we understood that the reasons for believing one thing can depend on the reasons for believing a great many other things.

However involved the network of reasons supporting a given conclusion might be, it seems that there must be some starting points. That is, it seems there must be some reasons for believing things that don’t themselves need to be justified in terms of further reasons. There needs to be some sort of bedrock, ground level, foundation at the bottom. Otherwise the network of supporting reasons would go on without end, and the kid would happily ask `why' until the end of time. The problem here is about how do we tell where the ultimate foundations of knowledge and justified belief. This is a big epistemological issue and we will return to it later in the course. For now, let’s consider one potential answer we are already familiar with. In the sciences, our complex chains of reasoning seem to proceed from the evidence of the senses.

We think that evidence provides the foundation for our edifice of scientific knowledge. Sounds great for science, but where does this leave philosophy? Does philosophy entirely lack evidence on which its reasoning can be based? Philosophy does have a kind of evidence to work from and that evidence is provided by philosophical problems. When we encounter a problem in philosophy this often tells us that the principles and assumptions that generate that problem can’t all be correct. This might seem like just a subtle clue that leaves us far from solving the big mysteries. But clues are evidence just the same. Sensory evidence by itself doesn’t tell us as much about the nature of the world as we’d like to suppose. Scientific evidence provides clues, but there remains a good deal of problem solving to do in science as well as in philosophy.

So we can assess the truth or falsity of the premises of an argument by examining evidence or by evaluating further argument in support of the premises. Now we will turn to the other step in evaluating arguments and consider the ways in which premises can support or fail to support their conclusions. The question of support is distinct from the question of whether the premises are true. When we ask whether the premises support the conclusions we are asking whether we’d have grounds for accepting the conclusion assuming the premises are true. In answering this question we will want to apply one of two standards of support: deductive validity or inductive strength.

\subsection{Deductive Validity}
\label{s:p1.2.1}
There are two kinds of arguments which we deal with, one is \glspl{deductive argument} and the other is inductive arguments, here we are going to be concerned with deduction. When you are dealing with this kind of argument, the standard for the `goodness' of the argument is validity. An argument counts as deductive whenever it is aiming at this standard of support. Deductive validity is the strictest standard of support we can uphold. It is also the one which is used in the vast majority of philosophy (with the exception of some, very contemporary, fields).\footnote{This would be Experimental Philosophy, which involves testing philosophic intuitions. This can be useful when you are trying to get a common person, average Joe view.} In a deductively \gls{valid} argument, the truth of the premises guarantees the truth of the conclusion. This standard is not concerned with whether the premises are actually true, that is a different standard. Basically, if you assume that the premises are true, the conclusion must be true. Here are two equivalent definitions of deductive validity:

\newglossaryentry{valid}
{
name=valid,
description={A property of arguments where the truth of the premises guarantee the truth of the conclusion; i.e. it is impossible for the premises to be true and the conclusion false}
}

\begin{tabular}{p{1in}|p{4.5in}}
Deductive&Definition\\\hline
D &A valid argument is an argument where if its premises are true, then its conclusion must be true.\\\hline
D* &A valid argument is an argument where it is not possible for all of its premises to be true and its conclusion false.
\end{tabular}

That (D*) standard is the more formal way, and exact way, of claiming (D). Here are a few examples of deductively valid arguments:


\noindent
\begin{tabular}{p{2in}|p{2in}|p{2in}}
Example \exarg{socrates1} &Example \exarg{mammals2} &Example \exarg{raining3}\\\hline
If Socrates is a man, then Socrates is mortal &All primates are mammals &If it's wet outside, then it's raining\\
Socrates is a man &All humans are primates &It's not raining\\
Therefore, Socrates is mortal &Therefore, all humans are mammals &Therefore, it's not wet outside.
\end{tabular}

If you think about these two examples for a moment, it should be clear that there is no possible way for the premises to all be true and the conclusion false. The truth of the conclusion is guaranteed by the truth of the premises. \ref{raining3}, on the other hand, should have given you pause. Yes, that argument is valid, but it's not \gls{sound}. Soundness involves whether or not the premises are in fact true, validity concerns the structure of the argument. If an argument is sound, then it's valid, but not the other way around. In contrast, the following arguments are not valid:

\newglossaryentry{sound}
{
name=sound,
description={A property of arguments that holds if the argument is valid and has all true premises}
}


\noindent
\begin{tabular}{p{2in}|p{2in}|p{2in}}
Example \exarg{socrates4} &Example \exarg{cookies5} &Example \exarg{raining6}\\\hline
If Socrates the Cat is a man, then Socrates the Cat is mortal &Billy or Sally stole cookies from the jar &If it's wet outside, then it's raining\\\hline
Socrates the Cat is mortal &Billy stole cookies &It's not wet outside\\\hline
Therefore, Socrates the Cat is man &Therefore, Sally didn't steal cookies &Therefore, it's not raining
\end{tabular}

These examples, again, are not valid, they have a flaw in their reasoning, in arguing this way, you will have an error which will lead you astray. To see why, it might require a bit of imagination, but it's a reasonably simple test. Imagine a case where the premises are true and the conclusion is false. If you can't do it, then the argument is valid, if you can, then there's a flaw in the reasoning. Think of this test as like a trial by worst-case scenario. For Example \ref{socrates4}, Socrates the Cat is obviously mortal, and \emph{were} the cat a man, he still would be mortal, but that doesn't mean that a cat is a man. For Example \exarg{cookies5}, it's perfectly possible that both Billy and Sally were naughty, bad, kids and stole some cookies from the jar, just because one did it doesn't mean that the other didn't.\footnote{Some languages, like Latin, have two different words to express the different kinds of disjunctions (or-statements). There are inclusive disjunctions, which allow for both to be true and there are exclusive disjunctions, which only allow for only one of them to be true. English uses both but does not have a built in way to tell the difference between them consistently. Context is your best bet for this, but also, sometimes `either... or...' is used for exclusive and just `... or...' is used for inclusive.} For example \ref{raining6}, this might be easier to imagine for people who have spent time in the desert: In those sorts of regions, there's a phenomenon which I know as sun-showers. This is where the sun is still shining, the ground is perfectly dry, but there's rain coming down from the sky. So, it's perfectly possible for it to be raining and yet not be wet outside.
\factoidbox{
Deductive validity is the gold standard for an argument. Soundness is the platinum standard.}

The deductive arguments we’ve looked at here are pretty intuitive. We only need to think about whether the conclusion could be false even if the premises were true. But most deductive arguments are not so obvious.These example arguments only use one logical rule and only two supporting premises; most of the ones which are really powerful use several logical rules/forms in conjunction with each other and several more lines of supporting evidence.  Logic is the science of deductive validity. For a class on this subject, check out PHIL\&120, which is the barebones, no fluff, mathematics of Philosophy. Philosophy has made some historic advances in logic over the past few centuries, with great advancements happening in the last century.  If you want to see some of the practical side of their good work, what it's actually been used for, just look at your computer. The background coding for that machine, I know because I worked with it for a time and I know the history behind it, is all based on the logical structures which philosophy has used forever.

\subsection{Inductive Strength}
\label{s:p1.2.2}

In the previous section, we looked at deductive arguments and their standards are validity and soundness, but those aren't the only kinds of arguments which are used, other fields use a different kind of argumentation, \glspl{inductive argument}. The standard for the goodness here is strength, not validity. Like with deductive arguments, an argument counts as inductive when it's shooting for this kind of support for the conclusion. Inductive strength is a weaker standard of support we can shoot for. That being said, it's also the standard which is found in science.\footnote{As well as in Experimental Philosophy.} An inductively strong argument is one where the premises make the conclusion more likely, give probabilistic support. Strength, again, is not concerned with the truth of the premises, that's a different standard still. Here are some examples of inductive arguments:

\newglossaryentry{deductive argument}
{
name=deductive argument,
description={An argument such that the intent behind it is to guarantee the conclusion. There is no ‘wiggle room'. In general, deductive arguments move from broad general principles and to more particular instances.},
plural=deductive arguments
}

\newglossaryentry{inductive argument}
{
name=inductive argument,
description={An argument such that the intent behind it is \emph{merely} to make the conclusion more likely, assuming the truth of the premises. There is some ‘wiggle room'. In general, inductive arguments move from a collection of particular instances and then use those to support the truth of some general principle},
plural=inductive arguments
}

\noindent
\begin{tabular}{p{2in}|p{2in}|p{2in}}
Example 1: &Example 2: &Example 3:\\\hline
Sam is a line cook &Most things cancerous to mice are cancerous to humans &Oscar was born in North America\\
Line cooks generally can cook well &Various chemicals in tobacco products are cancerous to mice &Oscar was not born in Mexico\\
Therefore, Sam can probably cook well &Therefore, various chemicals in tobacco products are probably cancerous to humans &Therefore, Oscar was probably born in the USA
\end{tabular}

Examples 1 and 2 seem like pretty decent arguments. The premises do support the conclusion (we are looking at inductive strength here). But, none of the above arguments are, in fact, valid. It's perfectly possible for the premises to be true and the conclusion false. Sam could be a brand new cook hired because he’s the manager’s son who has never cooked in his life. The chemicals in tobacco products could be cancerous to mice but not humans.\footnote{In this case, we can only look at the information provided in the premises, same with validity, so if you disagree with this sentence, you would be correct, but that's bringing information not given as support in the argument.} Many arguments give us good reasons for accepting their conclusions even if their premises being true fails to completely guarantee the truth of the conclusion.  The intent behind all of these arguments is different. The point of these arguments is not to guarantee the conclusion, but rather to make them more likely to be true.  We judge how good such an argument is according to how strong it is.  Unlike validity, strength is a bit more wishy-washy, it comes in degrees, which is why I use the terms `probable' and `improbable' in the below definitions:

\begin{tabular}{p{1in}|p{4.5in}}
Inductive&Definition\\\hline
I &An inductively strong argument is an argument in which if its premises are true, its conclusion is probably true.\\
I* &An inductively strong argument is an argument in which it is improbable that its conclusion is false given that its premises are true.
\end{tabular}

As in the case of validity, when we say that an argument is strong, we are only claiming that if the premises are true then the conclusion is likely to be true. Corresponding to the notion of deductive soundness, an inductive argument that is both strong and has true premises is called a cogent inductive argument. Unlike the case with deductively sound arguments, it is possible for an inductively cogent argument to have true premises and a false conclusion. When we are asking about the validity of an argument, we are asking whether it's possible for the premises to be true and the conclusion false. When we are asking about the strength of an argument, we are asking about the probability of the conclusion being false if we assume that the premises are true.  Possibility does not come in degrees, it's either possible or impossible. Probability does come in degrees. In the simplest case, inductive reasoning involves inferring that something is generally the case from a pattern observed in a limited number of cases.\footnote{One way to think about this is that deduction goes from general to particular and induction goes from particular to general.}

    Suppose we conducted a poll of 1000 Seattle voters. The results showed that 600 of them claimed to be Democrats. We could inductively infer that 60\% of the voters in Seattle are Democrats. The results of the poll give a pretty good reason to think that around 60\% of the voters in Seattle are Democrats. But the results of the poll don’t guarantee this conclusion. It is possible that only 50\% of the voters in Seattle are Democrats and Democrats were, just by luck, over represented in the 1000 cases we considered, but it may not be probable.

There are a few factors which tell us how strong an inductive argument is. One is how much evidence we have looked at before inductively generalizing. Our inductive argument above would be stronger is we drew our conclusion from a poll of 100,000 Seattle voters, for instance. And it would be much weaker if we had only polled 100. Similarly, if we were trying to figure out the political stances of all of Washington but solely looked at Enumclaw, we would get a radically different view about our standings than if we had looked at Seattle and Enumclaw.

Also, the strength of an inductive argument depends on how much of the evidence represents the amount in reality. So our inductive argument will be stronger if we randomly select our 1,000 voters from the Seattle phone book than if they are selected from the Ballard phone book (Ballard being a notably liberal neighborhood within Seattle).

So far, we’ve only discussed inductive generalization, where we identify a pattern in a limited number of cases and draw a more general conclusion about a broader class of cases. Inductive argument comes in other varieties as well. In the example we started with about Sam the line cook, we inductively inferred a prediction about Sam based on a known pattern in a broader class of cases. Argument from analogy is another variety of inductive reasoning that can be quite strong. The strength here is in how many commonalities the two cases have, we will see an argument like this in the section concerning the existence of God.

\section{Part \thechapcount.\theseccount: Common Argument Structures In Philosophy}\stepcounter{seccount}
\label{s:p1.3}

Since Philosophy relies on arguments to get at the facts about the world, like science, but we go with the gold-standard for arguments, validity, often arguments will come in certain forms, structures. These forms can be just on their own, or in combinations with others. If you structure your arguments in these ways or you make arguments out of combinations of them, you are guaranteed to have a valid argument, soundness is a different story. 

\subsection{Modus Ponens}
\label{s:p1.3.1}

This is another easy and intuitive argument structure. It follows a very similar model to cause-and-effect. If I knock over the glass, it will break; I knocked over the glass; so it broke. Here, I have an if-then statement, which can be phrased in several different ways, I affirm the antecedent (the if-part), and thereby get the consequent (the then-part). But, that doesn't mean it's fool-proof. For example, there's a very common logical fallacy called ``affirming the consequent", this is where you have an if-then statement, you affirm the then-part, and thereby think you get the antecedent (the if-part). This is not valid, it's not a good structure. For example, take this quick argument (which is Modus Ponens done properly):

\factoidbox{If my car won't start, then there's something wrong with the battery terminals. My car won't start. So, there must be something wrong with the battery terminals.}

Using cause-and-effect again, there can be many different possible causes for some event. One of them definitely caused the event, but you can't say with certainty which one caused it just because it happened. For example, if an apple is too heavy, it will fall off a tree. Yes, when an apple is too heavy it will fall from a tree, but loads of other things can cause it to fall, so you can't say that it's weight caused it to fall, it could have been the wind.

\subsection{Modus Tollens}
\label{s:p1.3.2}

Modus Tollens is like Modus Ponens in reverse and negated. In this case, you have an if-then statement, you have that the then-part is false, and so you get that the if-part must be false. This makes sense in the case of cause-and-effect too. You know that one thing would cause another and you know that that other thing didn't happen, so you know that the first thing didn't happen. For example, if I sleep in, I will be late, I wasn't late, so I must not have slept in. 

Again, this is a pretty simple idea, but it's easily misused. The fallacy here is called `denying the antecedent'. This, again, is not good reasoning; you have an if-then statement, you have that the if-part is false, so you think that the then-part must be false. But, just because something wasn't caused by one thing, it doesn't mean that it wasn't caused by another. For example, if God exists, then humans exist; God doesn't exist; therefore, man doesn't exist. Poof, all atheists  just disappeared, right? No, they didn't. This is not a valid argument. 

Modus Tollens, properly, we will mostly see in this class, as it's the easiest for the sort of teaching style which I employ. For example, take the following:

\factoidbox{Ethical Egoism is the stance that the morality of an action is determined by how much it benefits the doer personally. By this theory, an action is moral when it benefits me and immoral when it does not. So, if Ethical Egoism is correct, then donating to charity, which will in no way benefit me, is always morally wrong. But, it seems obvious that the most moral actions we can do are selfless (in this case, donating when it won't benefit me). So, Ethical Egoism is incorrect.}
    
This particular argument structure was taken to the extreme by the Indian Philosopher Moggaliputta-tissa (327-247 BCE) who would ask questions to the `opposing side' with the goal of them assenting to a conditional and its antecedent but rejecting its consequent. In doing so, Moggaliputta-tissa would catch them in a contradiction. You see, if a person accepts a conditional (If A, then B) and its antecedent (A), then by Modus Ponens, they would need to also accept the consequent (B). At the same time, if the person accepts a conditional but rejects its consequent, then by Modus Tollens, they must reject its antecedent.\footnote{As I mentioned in a footnote before, see \citetitle{Katha} for examples. Also, as fate would have it, `MT' is a common acronym or short-hand for `Modus Tollens' and Moggaliputta-tissa's transliterated initials are also `MT'.}    
\subsection{Disjunctive Syllogism}
\label{s:p1.3.3}

This is by far the most simple argument structure used in Logic and it's probably the hardest to misuse (think that you have a valid argument, but don't). This does not mean that I have seen bad formulations of this which I will show, but first, here is a story:
\factoidbox{Years ago, while I was living in Arizona, I had some car trouble. Periodically, my car just would not start, or it spontaneously let me start it after a few attempts, seemingly randomly. I eventually called my local mechanic and had them take it to a shop. The mechanic, maybe knowing I was a philosophy professor, said the following (and this is a real quote): ``Either it's your ECM or it's your fuse-box. We tested and it's not your fuse-box. So, the issue must be your ECM."}

Basically, a disjunctive syllogism takes 2 possible options, has that one of them is false, and thereby gets the other. This can be used in various different ways, for example, you could have 3 possible options, show that one of them is false and thereby get that it must be one of the other 2. This is a sort of process of elimination sort of argument structure. This is a very simple, easy, argument structure, but it's still possible to structure it poorly. For example, I have seen people take two options, show that one of them is true and thereby claim that the other is false. This is not a good structure. Disjunctions, in English, without context, don't allow for this sort of move. It is possible for both options to be true. There are some cases where this works, but to play it safe, and stay on the windy side of validity, assume that both can be correct. 

\subsection{Hypothetical Syllogism}
\label{s:p1.3.4}

This is the next, basic, argument structure which I will cover with you here. This argument structure does not require you to have anything other than conditionals, you don't need any proven facts. But, at the end, you don't get any proven facts out either, all you get are conditionals. We often use this sort of reasoning when we are thinking about what will happen next if we go through some possible scenario. For example, take the following:

\factoidbox{If you give a mouse a cookie, he will ask for a glass of milk. If he asks for a glass of milk, then he will ask for a straw. Therefore, if you give a mouse a cookie, then he will ask for a straw.} 

When the antecedent of one conditional is the same as the consequent of another, you can collapse them into one conditional by taking out the middle man. This is a driving force for the slippery slope fallacy, but in that case, the fallacy is not in the reasoning itself, but rather in the truth of the conditionals which it employs. Often in arguments, this is used to shorten the work of Modus Ponens or Modus Tollens (below), but it can be used all on its own to get a point across. 

\subsection{Combining the Structures}

Arguments aren't always merely 3 lines long, sometimes they are far longer and more involved. For example, take the following perfectly valid argument:

\factoidbox{If I eat a ton of food this Thanksgiving, then I will get yelled at by my doctor. If I get yelled at by my doctor, then I will need to work out. I did eat a ton of food this Thanksgiving. Therefore, I will need to work out.}

In this case, I am using several different structures together to get my conclusion. I am using Hypothetical Syllogism and Modus Ponens all at once to get my answer. Similarly, I can have an argument like this:

\factoidbox{If I need to get a math credit, then I will either need to take PHIL\&120 or a college level math course. I need to get a math credit and I really don't want to take a college level math course, so I will need to take PHIL\&120.} 

Here, I am using both Modus Ponens and Disjunctive Syllogism to get the conclusion. And yes, PHIL\&120 is a math credit, but it can be more difficult than other options. All of these examples come from PHIL\&120. 

\section{Part \thechapcount.\theseccount: Fallacies and Biases}\stepcounter{seccount}
\label{s:p1.4}

A fallacy is just a mistake in reasoning. Humans are not nearly as rational as we’d like to suppose. In fact we are so prone to certain sorts of mistakes in reasoning that philosophers and logicians refer those mistakes by name. For now I will discuss just one by name but in a little detail. Watch for explanations of other fallacies over the course of the class. For pretty thorough catalogue of logical fallacies, I’ll refer to you The Fallacy Files.\autocite{Curtis1} There is also a section in the appendix of this textbook which lists and explains various informal fallacies. Formal fallacies are cases where the structure of the argument seems fine, but it actually relies on an improper move (outlined when I covered the common structures). Informal fallacies are when improper or incorrect reasons are used in the argument, which are outlined in the module I linked to. Here are some examples of fallacies which are rather egregious and should never be seen in any work, in philosophy or elsewhere: 
\subsection{Ad Hominem Fallacy}

“Ad hominem” is Latin for “against the man.” It is the name for the fallacy of attacking the proponent of a position rather than critically evaluating the reasons offered for the proponent’s position. The reason ad hominem is a fallacy is just that the attack on an individual is simply not relevant to the quality of the reasoning offered by that person. Attacking the person who offers an argument has nothing to do whether or not the premises of the argument are true or support the conclusion. Ad hominem is a particularly rampant and destructive fallacy in our society. What makes it so destructive is that it turns the cooperative social project of inquiry through conversation into polarized verbal combat. This fallacy makes rational communication impossible while it diverts attention from interesting issues that often could be fruitfully investigated.

\factoidbox{Here is a classic example of ad hominem: A car salesman argues for the quality of an automobile and the potential buyer discounts the argument with the thought that the person is just trying to earn a commission. There may be good reason to think the salesman is just trying to earn a commission. But even if there is, this is irrelevant to the evaluation of the reasons the salesman is offering. The reasons should be evaluated on their own merits.}

Notice, it is easy to describe a situation where it is both true that the salesman is just trying to earn a commission and true that he is making good arguments. Consider a salesman who is not too fond of people and cares little for them except that they earn a commission for him. Otherwise he is scrupulously honest and a person of moral integrity. In order to reconcile himself with the duties of a sales job, he carefully researches his product and only accepts a sales position with the business that sells the very best. He then sincerely delivers good arguments for the quality of his product, makes lots of money, and dresses well. This salesman must have been a philosophy major. The customer who rejects his argument on the ad hominem grounds that he is just trying to earn a commission misses an opportunity to buy the best. The moral of the story is just that the salesperson’s motive is logically independent of the quality of his argument.
\subsection{Strawman Fallacy}

The strawman fallacy is one which I encounter occasionally in student works but you really will see it commonly in the political sphere (much to our misfortune, regardless of the side you hold). A strawman is like a scarecrow, a mock façade of a person used as a distraction or a trick. The strawman in the fallacy is not a façade of the person making the argument or claim, per se, but rather it is a façade of the claim or argument. Often, when we are explaining a view or defending our own views, we will need to explain the opposing side and give their arguments for their stance. This is so that we will be able to explain why our own is more reasonable or better. Putting your opponent's stance in your own words is not strawmanning, in fact I encourage you to (because then you will need to put yourself in their shoes and potentially find flaws in your own stance). This becomes strawmanning, however, when the argument or stance which you are attributing to your opponent is not their argument or stance. You have left out key details, misrepresented their findings or claims, or imposed your own stance (which they reject) onto them in order to make them seem absurd. Strawmanning another person's stance doesn't only harm your quest for the right answer to a problem (because you have removed the view of a person (who I am assuming is reasonable) from the discussion) but it also harms others who could take what you say seriously. In the case of politics, if the party or group you support presents the opposing side as having absurd or outlandish views which could never work or have some fundamental flaw which anyone could see, often, rather than thinking that they must have missed something, you will take it as gospel and move on to spread this misinformation. The vast majority of the time, however, the absurd or outlandish view was actually a strawman of the original. The original, if the people behind it are reasonable, would lack the fundamental flaw or have some explanation about how the flaw was handled in connection with other policies. 

The way to avoid this fallacy is to paint all arguments for any claim in the best possible light. If there is a subtle flaw in the argument which could easily be overlooked, make the patch yourself (even and especially if you disagree with the conclusion to it). For example, in this class, when we talk about the first cause argument for the existence of God, I present the argument without a `fatal' flaw which is seen in the ordinary presentation of it because it was easy to see and fix; I did not want to present you with a strawman.  If a stance seems absurd, do some digging into the actual stance and see whether the person presenting it to you missed or intentionally left out something. For example, suppose that a study showed that some program, which would remove a service from the private sector and made it public (have the government provide the service rather than a collection of private companies), would cost the tax payers 2.6 trillion USD. They present this as an absurd amount of money and then show how much your taxes would increase by. What they fail to mention, however, is that the current cost for that program on tax payers is closer 3.6 trillion USD, but rather than this money going to the government, it goes to the private companies. Others may strawman the cost by inflating it by ignoring the basic fact that many middlemen would not exist in the proposed system and the lack of them would reduce the cost. Doing a little deeper digging will save you from this fallacy. 

\section{Part \thechapcount.\theseccount: Critical Thinking}\stepcounter{seccount}

This page roughly finishes up our crash course on logic and critical thinking, but we will be seeing more logical forms and critical thinking structures as the course progresses. Most of it was dedicated to logic, and if you are more interested in that, I strongly recommend that you take PHIL\&120. But, for good critical thinking, there are some further standards which are worth noting. These standards follow from the others which we have discussed, but are a wee-bit more practical. These standards are Consistency and Coherency, but there are other aspects which come in when philosophy goes empirical.

\subsection{Consistency}

The first, basic, bare-bones standard for good critical thinking, and thereby good philosophy, is that the thinking must be consistent. This is more basic than saying that it's logical or that it lines-up with reality. For some train of thought to be consistent, it needs to lack two different things. First, it can't have any contradictions. A contradiction is a case where some proposition must be both true and false (at the same time) in order for your reasoning to work. In everyday life, it is (hopefully) very rare for us to encounter a contradiction either in our own or someone else's thinking, but there are cases where they show up, such as in conspiracy theories, and we only notice them well after the stance was explained. Take, for example, a conspiracy theory which has that both the Earth is the center of the universe and that gravity is just the Earth moving upwards at an ever increasing rate. This might not seem too contradictory on the face of it, but once you start digging into what must be true for both of those statements to be true, you will find that some proposition must be both true and false at the same time. There are other more blunt examples, such as:.\footnote{The example involving driving in New York comes from the Futurama episode ``The Lesser of Two Evils", season 2, episode 11. The example involving the equality of animals is from George Orwell's Animal Farm.} 

\begin{tabular}{p{2.5in}|p{2.5in}}
Contradictory &Non-Contradictory\\\hline
The non-existent ghosts stole the painting &The ghosts stole the painting\\\hline
Bobby Joe committed the crime in New York (in person) while in LA. &Bobby Joe made it seem like he was in LA in order to cover up the crime.\\\hline
No one drives in New York because there is too much traffic.&The traffic in New York is so bad that most people walk.\\\hline
All animals are equal, but some are more equal than others. &Either all animals are equal or they aren't all equal.\\\hline
I make my own choices, with my wife's permission. &What I ultimately choose should be a joint decision between my wife and I.\\\hline
\end{tabular}

The Non-Contradictory column is there to show you how, most of the time, there is an easy way to resolve the contradiction, but there are other cases, which we will see in this class, where the contradiction is more deep seeded and can't be so easily resolved (like the contradictions in Moral Relativism). In order to avoid these, make sure that all of the propositions you are using are consistent, that they are always true or false (depending) throughout your thinking.

The second requirement for consistent thinking is that there aren't any equivocations. This is where you use the same statement or word/phrase in two or more parts of your reasoning and your reasoning relies on them meaning the same thing, but they don't mean the same thing in the two different instances. For example, take a look at these arguments:
\begin{earg}
   \item[] Nothing is better than God.
    \item[]A cheese sandwich is better than nothing.
    \item[]Therefore, a cheese sandwich is better than God.
\end{earg}
In this argument, the term `nothing' is being used in two different ways. In order for us to accept the two lines of this argument, individually, we have different meanings in mind. In the first case, `nothing' means that something along the lines of `it is not the case that there is exists a being such that that being..." or ``no being". So, the proper way to understand this first line is ``it is not the case that there exists a being such that that being is better than God." The second line of the argument uses a different sense of the word `nothing', namely ``not having anything", so the correct way to understand this is that it means ``a cheese sandwich is better than not having anything." Taking these two together, the argument looks like this:
\begin{earg}
    \item[]There isn't anything better than God.
    \item[]Having a cheese sandwich is better than not having anything.
    \item[]Therefore, ...
\end{earg}
As you should be able to see, the flow of the argument isn't there any more, the equivocation was the glue holding it together. Some other examples can be in quick reasoning like:
\begin{earg}
    \item[]André the Giant is so called because of his great size.
    \item[]André René Roussimoff is so called because that's what his mother named him.
    \item[]André the Giant is André René Roussimoff.
    \item[]Therefore, André René Roussimoff is so called because of his great size.
\end{earg}
That last line should seem off to you. This is because there is equivocation in this reasoning. In the first two lines we are talking about the names, but in the third we are talking about what the names pick out in the world. The equivocation is in the shift between talking about the word and the person. In this case, the phrase `so called' shifts the reference from the person being referenced to the phrase used to reference them. Here is an example of this same argument made clear by removing the equivocation: 

\begin{earg}
    \item[]André the Giant is called `André the Giant' because of his great size.
    \item[]André René Roussimoff is called `André René Roussimoff' because that's what his mother named him.
    \item[]André the Giant is André René Roussimoff.
    \item[]Therefore, André René Roussimoff is called `André the Giant' because of his great size.
\end{earg}

As you can see, simply being clear about what we mean by the phrase `so called' prevents us from thinking that André René Roussimoff is called `André René Roussimoff' because of his great size, rather André René Roussimoff is called `André the Giant' because of his great size, which does not have any problems. 
\subsection{Coherency}

The next basic requirement for good critical thinking is that it needs to be coherent. Coherency is not the same as consistency, though there is a fair bit of overlap. Each premise, each proposition used in the reasoning need to relate to each other in a reasonable way. There can't be any strange or unorthodox jumps in the argument. You can think of this like a spider web, each of the strands is connected together and it would not be as strong if some of those strands were removed. For example, take this argument:
\begin{earg}
    \item[]If the moon landing was fake, then the Government did so to deceive us.
    \item[]If the Government did so to deceive us, then it was to make us lose faith in our religion.
    \item[]The moon landing was fake.
    \item[]Therefore, the Government faked it in order to make us lose faith in our religion.
\end{earg}
So, I am not going to make an argument that the moon landing wasn't fake (we really did go to the moon), rather, assuming that it was fake, it does make sense that it would have been faked to deceive us in some way. The real issue with this argument is in the second line. Does it really make sense that the Government would deceive us to make us lose faith in this way? There is a jump in the reasoning. Much more evidence and premises are required to show a connection between the assumption that the Government deceived us with the moon landing and that the purpose of it had something to do with religion. There are many other closer, more relevant potential reasons which need to be discounted first. For example, that the Government did so in order to reaffirm a sense of exceptionalism.

\subsection{Empirical Thinking}

There was a time in which most philosophic thought was empirical, but as the sciences diversified, that became less common (though the trend is swinging back that way in philosophy). Empirical thinking is reasoning which relates to or explains the outside world. In the other sciences, you may have heard a distinction between empirical (applied) and theoretical. Theoretical science is the arm-chair, hypothetical models which are thought about and debated and later tested using empirical methods (like experiments). Empirical science is the science which actually does the tests and uses the materials in question. There are two features to reasonable empirical thinking, which the other sciences should take note of and explain from the beginning. These are that the thinking must be adequate and applicable.

Empirical thinking is adequate when all of the cases you are trying to explain are accounted for. There shouldn't be too many exceptions to the account you are giving and those exceptions should be easily accounted for. For example, although this story is apocryphal, when Galileo presented his findings about there being objects which orbit something other than the Earth, one person claimed that it was because of interference in the telescope. So, they tried it in a different location, and they got the same result. Eventually, the objector said something like ``my hypothesis works so long as you don't look through a telescope". The hypothesis that all objects orbit the Earth wasn't adequate because there were cases which it could not explain and those exceptions could not be easily accounted for. 

Empirical thinking is applicable when there isn't anything in the explanation which doesn't relate back to experience and evidence or data gained from testing in the relevant environment and the explanation/hypothesis is useful (as in, it leads or gives way to further understanding and can be used to make other explanations). Many examples of empirical thoughts failing in this regard can be found when you look at the claims and tests involving pyramid power, magic crystals, feng shui, various vitamins and supplements seen on Dr. Oz, and even copper/magnetic bracelets claimed to treat arthritis. The explanations for each of these contain claims which either can't be demonstrated with experiments or don't relate to experience. For a more scientific example, take this quote from the famous physicist Richard Feynman:
\factoidbox{
    The next question was — what makes planets go around the sun? At the time of Kepler some people answered this problem by saying that there were angels behind them beating their wings and pushing the planets around an orbit.\autocite{physicallaw}}

 The hypothesis that the angels are making the planets move isn't applicable because it doesn't relate back to experience and evidence. If, in some strange world, we could see the angels sweating and pushing the planets really hard, then it would be. However, we can't see the angels and if someone were to claim that they are there, they are just invisible, the hypothesis would be even more inapplicable.

\chapter{Plato's Apology Translated By Harold North Fowler}
\label{platoapology}

\marginpar{17a}
How you, men of Athens, have been affected by my accusers, I do not know; but I, for my part, almost forgot my own identity, so persuasively did they talk; and yet there is hardly a word of truth in what they have said. But I was most amazed by one of the many lies that they told—when they said that you must be on your guard not to be deceived by me, \marginpar{17b} because I was a clever speaker. For I thought it the most shameless part of their conduct that they are not ashamed because they will immediately be convicted by me of falsehood by the evidence of fact, when I show myself to be not in the least a clever speaker, unless indeed they call him a clever speaker who speaks the truth; for if this is what they mean, I would agree that I am an orator—not after their fashion. Now they, as I say, have said little or nothing true; but you shall hear from me nothing but the truth. Not, however, men of Athens, speeches finely tricked out with words and phrases, \marginpar{17c} as theirs are, nor carefully arranged, but you will hear things said at random with the words that happen to occur to me. For I trust that what I say is just; and let none of you expect anything else. For surely it would not be fitting for one of my age to come before you like a youngster making up speeches. And, men of Athens, I urgently beg and beseech you if you hear me making my defence with the same words with which I have been accustomed to speak both in the market place at the bankers tables, where many of you have heard me, and elsewhere,\marginpar{17d} not to be surprised or to make a disturbance on this account. For the fact is that this is the first time I have come before the court, although I am seventy years old; I am therefore an utter foreigner to the manner of speech here. Hence, just as you would, of course, if I were really a foreigner, pardon me if I spoke in that dialect and that manner \marginpar{18a} in which I had been brought up, so now I make this request of you, a fair one, as it seems to me, that you disregard the manner of my speech—for perhaps it might be worse and perhaps better—and observe and pay attention merely to this, whether what I say is just or not; for that is the virtue of a judge, and an orator's virtue is to speak the truth.

First then it is right for me to defend myself against the first false accusations brought against me, and the first accusers, and then against the later accusations and the later accusers. \marginpar{18b} For many accusers have risen up against me before you, who have been speaking for a long time, many years already, and saying nothing true; and I fear them more than Anytus and the rest, though these also are dangerous; but those others are more dangerous, gentlemen, who gained your belief, since they got hold of most of you in childhood, and accused me without any truth, saying, “There is a certain Socrates, a wise man, a ponderer over the things in the air and one who has investigated the things beneath the earth and who makes the weaker argument the stronger.” These, men of Athens, \marginpar{18c} who have spread abroad this report, are my dangerous enemies. For those who hear them think that men who investigate these matters do not even believe in gods. Besides, these accusers are many and have been making their accusations already for a long time, and moreover they spoke to you at an age at which you would believe them most readily (some of you in youth, most of you in childhood), and the case they prosecuted went utterly by default, since nobody appeared in defence. But the most unreasonable thing of all is this, that it is not even possible {18d} to know and speak their names, except when one of them happens to be a writer of comedies. And all those who persuaded you by means of envy and slander—and some also persuaded others because they had been themselves persuaded—all these are most difficult to cope with; for it is not even possible to call any of them up here and cross-question him, but I am compelled in making my defence to fight, as it were, absolutely with shadows and to cross-question when nobody answers. Be kind enough, then, to bear in mind, as I say, that there are two classes \marginpar{18e} of my accusers—one those who have just brought their accusation, the other those who, as I was just saying, brought it long ago, and consider that I must defend myself first against the latter; for you heard them making their charges first and with much greater force than these who made them later. Well, then, I must make a defence, men of Athens, \marginpar{19a} and must try in so short a time to remove from you this prejudice which you have been for so long a time acquiring. Now I wish that this might turn out so, if it is better for you and for me, and that I might succeed with my defence; but I think it is difficult, and I am not at all deceived about its nature. But nevertheless, let this be as is pleasing to God, the law must be obeyed and I must make a defence.

Now let us take up from the beginning the question, what the accusation is from which the false prejudice against me has arisen, in which \marginpar{19b} Meletus trusted when he brought this suit against me. What did those who aroused the prejudice say to arouse it? I must, as it were, read their sworn statement as if they were plaintiffs: “Socrates is a criminal and a busybody, investigating the things beneath the earth and in the heavens and making the weaker argument stronger and \marginpar{19c} teaching others these same things.” Something of that sort it is. For you yourselves saw these things in Aristophanes' comedy, a Socrates being carried about there, proclaiming that he was treading on air and uttering a vast deal of other nonsense, about which I know nothing, either much or little. And I say this, not to cast dishonor upon such knowledge, if anyone is wise about such matters (may I never have to defend myself against Meletus on so great a charge as that!),—but I, men of Athens, have nothing to do with these things. \marginpar{19d} And I offer as witnesses most of yourselves, and I ask you to inform one another and to tell, all those of you who ever heard me conversing—and there are many such among you—now tell, if anyone ever heard me talking much or little about such matters. And from this you will perceive that such are also the other things that the multitude say about me.

But in fact none of these things are true, and if you have heard from anyone that I undertake to teach \marginpar{19e} people and that I make money by it, that is not true either. Although this also seems to me to be a fine thing, if one might be able to teach people, as Gorgias of Leontini and Prodicus of Ceos and Hippias of Elis are. For each of these men, gentlemen, is able to go into any one of the cities and persuade the young men, who can associate for nothing with whomsoever they wish among their own fellow citizens, \marginpar{20a} to give up the association with those men and to associate with them and pay them money and be grateful besides.

And there is also another wise man here, a Parian, who I learned was in town; for I happened to meet a man who has spent more on sophists than all the rest, Callias, the son of Hipponicus; so I asked him—for he has two sons—“Callias,” said I, “if your two sons had happened to be two colts or two calves, we should be able to get and hire for them an overseer who would make them \marginpar{20b} excellent in the kind of excellence proper to them; and he would be a horse-trainer or a husbandman; but now, since they are two human beings, whom have you in mind to get as overseer? Who has knowledge of that kind of excellence, that of a man and a citizen? For I think you have looked into the matter, because you have the sons. Is there anyone,” said I, “or not?” “Certainly,” said he. “Who,” said I, “and where from, and what is his price for his teaching?” “Evenus,” he said, “Socrates, from Paros, five minae.” And I called Evenus blessed, \marginpar{20c} if he really had this art and taught so reasonably. I myself should be vain and put on airs, if I understood these things; but I do not understand them, men of Athens.

Now perhaps someone might rejoin: “But, Socrates, what is the trouble about you? Whence have these prejudices against you arisen? For certainly this great report and talk has not arisen while you were doing nothing more out of the way than the rest, unless you were doing something other than most people; so tell us \marginpar{20d} what it is, that we may not act unadvisedly in your case.” The man who says this seems to me to be right, and I will try to show you what it is that has brought about my reputation and aroused the prejudice against me. So listen. And perhaps I shall seem to some of you to be joking; be assured, however, I shall speak perfect truth to you. \marginpar{21a} He was my comrade from a youth and the comrade of your democratic party, and shared in the recent exile and came back with you. And you know the kind of man Chaerephon was, how impetuous in whatever he undertook. Well, once he went to Delphi and made so bold as to ask the oracle this question; and, gentlemen, don't make a disturbance at what I say; for he asked if there were anyone wiser than I. Now the Pythia replied that there was no one wiser. And about these things his brother here will bear you witness, since Chaerephon is dead. \marginpar{21b} But see why I say these things; for I am going to tell you whence the prejudice against me has arisen. For when I heard this, I thought to myself: “What in the world does the god mean, and what riddle is he propounding? For I am conscious that I am not wise either much or little. What then does he mean by declaring that I am the wisest? He certainly cannot be lying, for that is not possible for him.” And for a long time I was at a loss as to what he meant; then with great reluctance I proceeded to investigate him somewhat as follows.

I went to one of those who had a reputation for wisdom, \marginpar{21c} thinking that there, if anywhere, I should prove the utterance wrong and should show the oracle “This man is wiser than I, but you said I was wisest.” So examining this man—for I need not call him by name, but it was one of the public men with regard to whom I had this kind of experience, men of Athens—and conversing with him, this man seemed to me to seem to be wise to many other people and especially to himself, but not to be so; and then I tried to show him that he thought \marginpar{21d} he was wise, but was not. As a result, I became hateful to him and to many of those present; and so, as I went away, I thought to myself, “I am wiser than this man; for neither of us really knows anything fine and good, but this man thinks he knows something when he does not, whereas I, as I do not know anything, do not think I do either. I seem, then, in just this little thing to be wiser than this man at any rate, that what I do not know I do not think I know either.” From him I went to another of those who were reputed \marginpar{21e} to be wiser than he, and these same things seemed to me to be true; and there I became hateful both to him and to many others.

After this then I went on from one to another, perceiving that I was hated, and grieving and fearing, but nevertheless I thought I must consider the god's business of the highest importance. So I had to go, investigating the meaning of the oracle, to all those who were reputed to know anything. And by the Dog, men of Athens \marginpar{22a} —for I must speak the truth to you—this, I do declare, was my experience: those who had the most reputation seemed to me to be almost the most deficient, as I investigated at the god's behest, and others who were of less repute seemed to be superior men in the matter of being sensible. So I must relate to you my wandering as I performed my Herculean labors, so to speak, in order that the oracle might be proved to be irrefutable. For after the public men I went to the poets, those of tragedies, and those of dithyrambs, \marginpar{22b} and the rest, thinking that there I should prove by actual test that I was less learned than they. So, taking up the poems of theirs that seemed to me to have been most carefully elaborated by them, I asked them what they meant, that I might at the same time learn something from them. Now I am ashamed to tell you the truth, gentlemen; but still it must be told. For there was hardly a man present, one might say, who would not speak better than they about the poems they themselves had composed. So again in the case of the poets also I presently recognized this, \marginpar{22c} that what they composed they composed not by wisdom, but by nature and because they were inspired, like the prophets and givers of oracles; for these also say many fine things, but know none of the things they say; it was evident to me that the poets too had experienced something of this same sort. And at the same time I perceived that they, on account of their poetry, thought that they were the wisest of men in other things as well, in which they were not. So I went away from them also thinking that I was superior to them in the same thing in which I excelled the public men.

Finally then I went to the hand-workers. \marginpar{22d} For I was conscious that I knew practically nothing, but I knew I should find that they knew many fine things. And in this I was not deceived; they did know what I did not, and in this way they were wiser than I. But, men of Athens, the good artisans also seemed to me to have the same failing as the poets; because of practicing his art well, each one thought he was very wise in the other most important matters, and this folly of theirs obscured that wisdom, so that I asked myself \marginpar{22e} in behalf of the oracle whether I should prefer to be as I am, neither wise in their wisdom nor foolish in their folly, or to be in both respects as they are. I replied then to myself and to the oracle that it was better for me to be as I am.

Now from this investigation, men of Athens, \marginpar{23a} many enmities have arisen against me, and such as are most harsh and grievous, so that many prejudices have resulted from them and I am called a wise man. For on each occasion those who are present think I am wise in the matters in which I confute someone else; but the fact is, gentlemen, it is likely that the god is really wise and by his oracle means this: “Human wisdom is of little or no value.” And it appears that he does not really say this of Socrates, but merely uses my name, \marginpar{23b} and makes me an example, as if he were to say: “This one of you, O human beings, is wisest, who, like Socrates, recognizes that he is in truth of no account in respect to wisdom.”

Therefore I am still even now going about and searching and investigating at the god's behest anyone, whether citizen or foreigner, who I think is wise; and when he does not seem so to me, I give aid to the god and show that he is not wise. And by reason of this occupation I have no leisure to attend to any of the affairs of the state worth mentioning, or of my own, but am in vast poverty \marginpar{23c} on account of my service to the god.

And in addition to these things, the young men who have the most leisure, the sons of the richest men, accompany me of their own accord, find pleasure in hearing people being examined, and often imitate me themselves, and then they undertake to examine others; and then, I fancy, they find a great plenty of people who think they know something, but know little or nothing. As a result, therefore, those who are examined by them are angry with me, instead of being angry with themselves, and say that “Socrates is a most abominable person \marginpar{23d} and is corrupting the youth.”

And when anyone asks them “by doing or teaching what?” they have nothing to say, but they do not know, and that they may not seem to be at a loss they say these things that are handy to say against all the philosophers, “the things in the air and the things beneath the earth” and “not to believe in the gods” and “to make the weaker argument the stronger.” For they would not, I fancy, care to say the truth, that it is being made very clear that they pretend to know, but know nothing. \marginpar{23e} Since, then, they are jealous of their honor and energetic and numerous and speak concertedly and persuasively about me, they have filled your ears both long ago and now with vehement slanders. From among them Meletus attacked me, and Anytus and Lycon, Meletus angered on account of the poets, and Anytus on account of the artisans and the public men, \marginpar{24a} and Lycon on account of the orators; so that, as I said in the beginning, I should be surprised if I were able to remove this prejudice from you in so short a time when it has grown so great. There you have the truth, men of Athens, and I speak without hiding anything from you, great or small or prevaricating. And yet I know pretty well that I am making myself hated by just that conduct; which is also a proof that I am speaking the truth and that this is the prejudice against me and these are its causes. And whether you investigate \marginpar{24b} this now or hereafter, you will find that it is so.

Now so far as the accusations are concerned which my first accusers made against me, this is a sufficient defence before you; but against Meletus, the good and patriotic, as he says, and the later ones, I will try to defend myself next. So once more, as if these were another set of accusers, let us take up in turn their sworn statement. It is about as follows: it states that Socrates is a wrongdoer because he corrupts the youth and does not believe in the gods the state believes in, but in other \marginpar{24c} new spiritual beings.

Such is the accusation. But let us examine each point of this accusation. He says I am a wrongdoer because I corrupt the youth. But I, men of Athens, say Meletus is a wrongdoer, because he jokes in earnest, lightly involving people in a lawsuit, pretending to be zealous and concerned about things or which he never cared at all. And that this is so I will try to make plain to you also.

Come here, Meletus, tell me: don't you consider it \marginpar{24d} of great importance that the youth be as good as possible? “I do.” Come now, tell these gentlemen who makes them better? For it is evident that you know, since you care about it. For you have found the one who corrupts them, as you say, and you bring me before these gentlemen and accuse me; and now, come, tell who makes them better and inform them who he is. Do you see, Meletus, that you are silent and cannot tell? And yet does it not seem to you disgraceful and a sufficient proof of what I say, that you have never cared about it? But tell, my good man, who \marginpar{24e} makes them better? “The laws.” But that is not what I ask, most excellent one, but what man, who knows in the first place just this very thing, the laws. “These men, Socrates, the judges.” What are you saying, Meletus? Are these gentlemen able to instruct the youth, and do they make them better? “Certainly.” All, or some of them and others not? “All.” Well said, by Hera, and this is a great plenty of helpers you speak of. But how about this?  \marginpar{25a} Do these listeners make them better, or not? “These also.” And how about the senators? “The senators also.” But, Meletus, those in the assembly, the assemblymen, don't corrupt the youth, do they? or do they also all make them better? “They also.” All the Athenians, then, as it seems, make them excellent, except myself, and I alone corrupt them. Is this what you mean? “Very decidedly, that is what I mean.” You have condemned me to great unhappiness! But answer me; does it seem to you to be so in the case of horses, that those who \marginpar{25b} make them better are all mankind, and he who injures them some one person? Or, quite the opposite of this, that he who is able to make them better is some one person, or very few, the horse-trainers, whereas most people, if they have to do with and use horses, injure them? Is it not so, Meletus, both in the case of horses and in that of all other animals? Certainly it is, whether you and Anytus deny it or agree; for it would be a great state of blessedness in the case of the youth if one alone corrupts them, and the others do them good. But, \marginpar{25c} Meletus, you show clearly enough that you never thought about the youth, and you exhibit plainly your own carelessness, that you have not cared at all for the things about which you hale me into court.

But besides, tell us, for heaven's sake, Meletus, is it better to live among good citizens, or bad? My friend, answer; for I am not asking anything hard. Do not the bad do some evil to those who are with them at any time and the good some good? “Certainly.” Is there then anyone who \marginpar{25d} prefers to be injured by his associates rather than benefited? Answer, my good man; for the law orders you to answer. Is there anyone who prefers to be injured? “Of course not.” Come then, do you hale me in here on the ground that I am corrupting the youth and making them worse voluntarily or involuntarily? “Voluntarily I say.” What then, Meletus? Are you at your age so much wiser than I at my age, that you have recognized that the evil always do some evil \marginpar{25e} to those nearest them, and the good some good; whereas I have reached such a depth of ignorance that I do not even know this, that if I make anyone of my associates bad I am in danger of getting some harm from him, so that I do this great evil voluntarily, as you say? I don't believe this, Meletus, nor do I think anyone else in the world does! \marginpar{26a} but either I do not corrupt them, or if I corrupt them, I do it involuntarily, so that you are lying in both events. But if I corrupt them involuntarily, for such involuntary errors the law is not to hale people into court, but to take them and instruct and admonish them in private. For it is clear that if I am told about it, I shall stop doing that which I do involuntarily. But you avoided associating with me and instructing me, and were unwilling to do so, but you hale me in here, where it is the law to hale in those who need punishment, not instruction.

But enough of this, for, men of Athens, this is clear, as I said, that Meletus never \marginpar{26b} cared much or little for these things. But nevertheless, tell us, how do you say, Meletus, that I corrupt the youth? Or is it evident, according to the indictment you brought, that it is by teaching them not to believe in the gods the state believes in, but in other new spiritual beings? Do you not say that it is by teaching this that I corrupt them? “Very decidedly that is what I say.” Then, Meletus, for the sake of \marginpar{26c} these very gods about whom our speech now is, speak still more clearly both to me and to these gentlemen. For I am unable to understand whether you say that I teach that there are some gods, and myself then believe that there are some gods, and am not altogether godless and am not a wrongdoer in that way, that these, however, are not the gods whom the state believes in, but others, and this is what you accuse me for, that I believe in others; or you say that I do not myself believe in gods at all and that I teach this unbelief to other people. “That is what I say, that you do not believe in gods at all.” You amaze me, Meletus! Why do you say this? \marginpar{26d} Do I not even believe that the sun or yet the moon are gods, as the rest of mankind do? “No, by Zeus, judges, since he says that the sun is a stone and the moon earth.” Do you think you are accusing Anaxagoras, my dear Meletus, and do you so despise these gentlemen and think they are so unversed in letters as not to know, that the books of Anaxagoras the Clazomenian are full of such utterances? And forsooth the youth learn these doctrines from me, which they can buy sometimes \marginpar{26e} (if the price is high) for a drachma in the orchestra and laugh at Socrates, if he pretends they are his own, especially when they are so absurd! But for heaven's sake, do you think this of me, that I do not believe there is any god? “No, by Zeus, you don't, not in the least.” You cannot be believed, Meletus, not even, as it seems to me, by yourself. For this man appears to me, men of Athens, to be very violent and unrestrained, and actually to have brought this indictment in a spirit of violence and unrestraint and rashness. For he seems, \marginpar{27a} as it were, by composing a puzzle to be making a test: “Will Socrates, the wise man, recognize that I am joking and contradicting myself, or shall I deceive him and the others who hear me?” For he appears to me to contradict himself in his speech, as if he were to say, “Socrates is a wrongdoer, because he does not believe in gods, but does believe in gods.” And yet this is the conduct of a jester.

Join me, then, gentlemen, in examining how he appears to me to say this; and do you, Meletus, answer; \marginpar{27b} and you, gentlemen, as I asked you in the beginning, please bear in mind not to make a disturbance if I conduct my argument in my accustomed manner.

Is there any human being who believes that there are things pertaining to human beings, but no human beings? Let him answer, gentlemen, and not make a disturbance in one way or another. Is there anyone who does not believe in horses, but does believe in things pertaining to horses? or who does not believe that flute-players exist, but that things pertaining to flute-players do? There is not, best of men; if you do not wish to answer, I say it to you and these others here. But answer at least \marginpar{27c} the next question. Is there anyone who believes spiritual things exist, but does not believe in spirits? “There is not.” Thank you for replying reluctantly when forced by these gentlemen. Then you say that I believe in spiritual beings, whether new or old, and teach that belief; but then I believe in spiritual beings at any rate, according to your statement, and you swore to that in your indictment. But if I believe in spiritual beings, it is quite inevitable that I believe also in spirits; is it not so? It is; for I assume that you agree, since you do not answer. But do we not think the spirits are \marginpar{27d} gods or children of gods? Yes, or no? “Certainly.” Then if I believe in spirits, as you say, if spirits are a kind of gods, that would be the puzzle and joke which I say you are uttering in saying that I, while I do not believe in gods, do believe In gods again, since I believe in spirits; but if, on the other hand, spirits are a kind of bastard children of gods, by nymphs or by any others, whoever their mothers are said to be, what man would believe that there are children of gods, but no gods? It would be just as absurd \marginpar{27e} as if one were to believe that there are children of horses and asses, namely mules, but no horses and asses. But, Meletus, you certainly must have brought this suit either to make a test of us or because you were loss as to what true wrongdoing you could accuse me of; but there is no way for you to persuade any man who has even a little sense that it is possible for the same person to believe in spiritual and divine existences and again for the same person not to believe in spirits or gods or \marginpar{28a} heroes.

Well then, men of Athens, that I am not a wrongdoer according to Meletus's indictment, seems to me not to need much of a defence, but what has been said is enough. But you may be assured that what I said before is true, that great hatred has arisen against me and in the minds of many persons. And this it is which will cause my condemnation, if it is to cause it, not Meletus or Anytus, but the prejudice and dislike of the many. This has condemned many other good men, and I think will do so; \marginpar{28b} and there is no danger that it will stop with me. But perhaps someone might say: “Are you then not ashamed, Socrates, of having followed such a pursuit, that you are now in danger of being put to death as a result?” But I should make to him a just reply: “You do not speak well, Sir, if you think a man in whom there is even a little merit ought to consider danger of life or death, and not rather regard this only, when he does things, whether the things he does are right or wrong and the acts of a good or a bad man. For according to your argument all the demigods \marginpar{28c} would be bad who died at Troy, including the son of Thetis, who so despised danger, in comparison with enduring any disgrace, that when his mother (and she was a goddess) said to him, as he was eager to slay Hector, something like this, I believe, “My son, if you avenge the death of your friend Patroclus and kill Hector, you yourself shall die;
for straightway, after Hector, is death appointed unto you;
”1he, when he heard this, made light of death and danger, \marginpar{28d} and feared much more to live as a coward and not to avenge his friends, and said, “Straightway may I die, after doing vengeance upon the wrongdoer, that I may not stay here, jeered at beside the curved ships, a burden of the earth.”. Do you think he considered death and danger?

For thus it is, men of Athens, in truth; wherever a man stations himself, thinking it is best to be there, or is stationed by his commander, there he must, as it seems to me, remain and run his risks, considering neither death nor any other thing more than disgrace.

So I should have done a terrible thing, \marginpar{28e} if, when the commanders whom you chose to command me stationed me, both at Potidaea and at Amphipolis and at Delium, I remained where they stationed me, like anybody else, and ran the risk of death, but when the god gave me a station, as I believed and understood, with orders to spend my life in philosophy and in examining myself and others, \marginpar{29a} then I were to desert my post through fear of death or anything else whatsoever. It would be a terrible thing, and truly one might then justly hale me into court, on the charge that I do not believe that there are gods, since I disobey the oracle and fear death and think I am wise when I am not. For to fear death, gentlemen, is nothing else than to think one is wise when one is not; for it is thinking one knows what one does not know. For no one knows whether death be not even the greatest of all blessings to man, but they fear it as if they knew that it is the greatest of evils. \marginpar{29b} And is not this the most reprehensible form of ignorance, that of thinking one knows what one does not know? Perhaps, gentlemen, in this matter also I differ from other men in this way, and if I were to say that I am wiser in anything, it would be in this, that not knowing very much about the other world, I do not think I know. But I do know that it is evil and disgraceful to do wrong and to disobey him who is better than I, whether he be god or man. So I shall never fear or avoid those things concerning which I do not know whether they are good or bad rather than those which I know are bad. And therefore, even if \marginpar{29c} you acquit me now and are not convinced by Anytus, who said that either I ought not to have been brought to trial at all, or since was brought to trial, I must certainly be put to death, adding that if I were acquitted your sons would all be utterly ruined by practicing what I teach—if you should say to me in reply to this: “Socrates, this time we will not do as Anytus says, but we will let you go, on this condition, however, that you no longer spend your time in this investigation or in philosophy, and if you are caught doing so again you shall die”; \marginpar{29d} if you should let me go on this condition which I have mentioned, I should say to you, “Men of Athens, I respect and love you, but I shall obey the god rather than you, and while I live and am able to continue, I shall never give up philosophy or stop exhorting you and pointing out the truth to any one of you whom I may meet, saying in my accustomed way: “Most excellent man, are you who are a citizen of Athens, the greatest of cities and the most famous for wisdom and power, not ashamed to care for the acquisition of wealth \marginpar{29e} and for reputation and honor, when you neither care nor take thought for wisdom and truth and the perfection of your soul?” And if any of you argues the point, and says he does care, I shall not let him go at once, nor shall I go away, but I shall question and examine and cross-examine him, and if I find that he does not possess virtue, but says he does, I shall rebuke him for scorning \marginpar{30a} the things that are of most importance and caring more for what is of less worth. This I shall do to whomever I meet, young and old, foreigner and citizen, but most to the citizens, inasmuch as you are more nearly related to me. For know that the god commands me to do this, and I believe that no greater good ever came to pass in the city than my service to the god. For I go about doing nothing else than urging you, young and old, not to care for your persons or your property \marginpar{30b} more than for the perfection of your souls, or even so much; and I tell you that virtue does not come from money, but from virtue comes money and all other good things to man, both to the individual and to the state. If by saying these things I corrupt the youth, these things must be injurious; but if anyone asserts that I say other things than these, he says what is untrue. Therefore I say to you, men of Athens, either do as Anytus tells you, or not, and either acquit me, or not, knowing that I shall not change my conduct even if I am \marginpar{30c} to die many times over.

Do not make a disturbance, men of Athens; continue to do what I asked of you, not to interrupt my speech by disturbances, but to hear me; and I believe you will profit by hearing. Now I am going to say some things to you at which you will perhaps cry out; but do not do so by any means. For know that if you kill me, I being such a man as I say I am, you will not injure me so much as yourselves; for neither Meletus nor Anytus could injure me; \marginpar{30d} that would be impossible, for I believe it is not God's will that a better man be injured by a worse. He might, however, perhaps kill me or banish me or disfranchise me; and perhaps he thinks he would thus inflict great injuries upon me, and others may think so, but I do not; I think he does himself a much greater injury by doing what he is doing now—killing a man unjustly. And so, men of Athens, I am now making my defence not for my own sake, as one might imagine, but far more for yours, that you may not by condemning me err in your treatment of the gift the God gave you. \marginpar{30e} For if you put me to death, you will not easily find another, who, to use a rather absurd figure, attaches himself to the city as a gadfly to a horse, which, though large and well bred, is sluggish on account of his size and needs to be aroused by stinging. I think the god fastened me upon the city in some such capacity, and I go about arousing, \marginpar{31a} and urging and reproaching each one of you, constantly alighting upon you everywhere the whole day long. Such another is not likely to come to you, gentlemen; but if you take my advice, you will spare me. But you, perhaps, might be angry, like people awakened from a nap, and might slap me, as Anytus advises, and easily kill me; then you would pass the rest of your lives in slumber, unless God, in his care for you, should send someone else to sting you. And that I am, as I say, a kind of gift from the god, \marginpar{31b} you might understand from this; for I have neglected all my own affairs and have been enduring the neglect of my concerns all these years, but I am always busy in your interest, coming to each one of you individually like a father or an elder brother and urging you to care for virtue; now that is not like human conduct. If I derived any profit from this and received pay for these exhortations, there would be some sense in it; but now you yourselves see that my accusers, though they accuse me of everything else in such a shameless way, have not been able to work themselves up to such a pitch of shamelessness \marginpar{31c} as to produce a witness to testify that I ever exacted or asked pay of anyone. For I think I have a sufficient witness that I speak the truth, namely, my poverty.

Perhaps it may seem strange that I go about and interfere in other people's affairs to give this advice in private, but do not venture to come before your assembly and advise the state. But the reason for this, as you have heard me say \marginpar{31d} at many times and places, is that something divine and spiritual comes to me, the very thing which Meletus ridiculed in his indictment. I have had this from my childhood; it is a sort of voice that comes to me, and when it comes it always holds me back from what I am thinking of doing, but never urges me forward. This it is which opposes my engaging in politics. And I think this opposition is a very good thing; for you may be quite sure, men of Athens, that if I had undertaken to go into politics, I should have been put to death long ago and should have done \marginpar{31e} no good to you or to myself. And do not be angry with me for speaking the truth; the fact is that no man will save his life who nobly opposes you or any other populace and prevents many unjust and illegal things from happening in the state. \marginpar{32a} A man who really fights for the right, if he is to preserve his life for even a little while, must be a private citizen, not a public man.

I will give you powerful proofs of this not mere words, but what you honor more,—actions. And listen to what happened to me, that you may be convinced that I would never yield to any one, if that was wrong, through fear of death, but would die rather than yield. The tale I am going to tell you is ordinary and commonplace, but true. \marginpar{32b} I, men of Athens, never held any other office in the state, but I was a senator; and it happened that my tribe held the presidency when you wished to judge collectively, not severally, the ten generals who had failed to gather up the slain after the naval battle; this was illegal, as you all agreed afterwards. At that time I was the only one of the prytanes who opposed doing anything contrary to the laws, and although the orators were ready to impeach and arrest me, and though you urged them with shouts to do so, I thought \marginpar{32c} I must run the risk to the end with law and justice on my side, rather than join with you when your wishes were unjust, through fear of imprisonment or death. That was when the democracy still existed; and after the oligarchy was established, the Thirty sent for me with four others to come to the rotunda and ordered us to bring Leon the Salaminian from Salamis to be put to death. They gave many such orders to others also, because they wished to implicate as many in their crimes as they could. Then I, however, \marginpar{32d} showed again, by action, not in word only, that I did not care a whit for death if that be not too rude an expression, but that I did care with all my might not to do anything unjust or unholy. For that government, with all its power, did not frighten me into doing anything unjust, but when we came out of the rotunda, the other four went to Salamis and arrested Leon, but I simply went home; and perhaps I should have been put to death for it, if the government had not \marginpar{32e} quickly been put down. Of these facts you can have many witnesses.

Do you believe that I could have lived so many years if I had been in public life and had acted as a good man should act, lending my aid to what is just and considering that of the highest importance? Far from it, men of Athens; nor could \marginpar{33a} any other man. But you will find that through all my life, both in public, if I engaged in any public activity, and in private, I have always been the same as now, and have never yielded to any one wrongly, whether it were any other person or any of those who are said by my traducers to be my pupils. But I was never any one's teacher. If any one, whether young or old, wishes to hear me speaking and pursuing my mission, I have never objected, \marginpar{33b} nor do I converse only when I am paid and not otherwise, but I offer myself alike to rich and poor; I ask questions, and whoever wishes may answer and hear what I say. And whether any of them turns out well or ill, I should not justly be held responsible, since I never promised or gave any instruction to any of them; but if any man says that he ever learned or heard anything privately from me, which all the others did not, be assured that he is lying.

But why then do some people love \marginpar{33c} to spend much of their time with me? You have heard the reason, men of Athens; for I told you the whole truth; it is because they like to listen when those are examined who think they are wise and are not so; for it is amusing. But, as I believe, I have been commanded to do this by the God through oracles and dreams and in every way in which any man was ever commanded by divine power to do anything whatsoever. This, Athenians, is true and easily tested. For if I am corrupting some of the young men \marginpar{33d} and have corrupted others, surely some of them who have grown older, if they recognize that I ever gave them any bad advice when they were young, ought now to have come forward to accuse me. Or if they did not wish to do it themselves, some of their relatives—fathers or brothers or other kinsfolk—ought now to tell the facts. And there are many of them present, whom I see; first Crito here, \marginpar{33e} who is of my own age and my own deme and father of Critobulus, who is also present; then there is Lysanias the Sphettian, father of Aeschines, who is here; and also Antiphon of Cephisus, father of Epigenes. Then here are others whose brothers joined in my conversations, Nicostratus, son of Theozotides and brother of Theodotus (now Theodotus is dead, so he could not stop him by entreaties), and Paralus, son of Demodocus; Theages was his brother; and \marginpar{34a} Adimantus, son of Aristo, whose brother is Plato here; and Aeantodorus, whose brother Apollodorus is present. And I can mention to you many others, some one of whom Meletus ought certainly to have produced as a witness in his speech; but if he forgot it then, let him do so now; I yield the floor to him, and let him say, if he has any such testimony. But you will find that the exact opposite is the case, gentlemen, and that they are all ready to aid me, the man who corrupts and injures their relatives, as Meletus and Anytus say. \marginpar{34b} Now those who are themselves corrupted might have some motive in aiding me; but what reason could their relatives have, who are not corrupted and are already older men, unless it be the right and true reason, that they know that Meletus is lying and I am speaking the truth?

Well, gentlemen, this, and perhaps more like this, is about all I have to say in my defence. Perhaps some one among you may be offended \marginpar{34c} when he remembers his own conduct, if he, even in a case of less importance than this, begged and besought the judges with many tears, and brought forward his children to arouse compassion, and many other friends and relatives; whereas I will do none of these things, though I am, apparently, in the very greatest danger. Perhaps some one with these thoughts in mind may be harshly disposed toward me and may cast his vote in anger. Now if any one of you is so disposed \marginpar{34d} —I do not believe there is such a person—but if there should be, I think I should be speaking fairly if I said to him, My friend, I too have relatives, for I am, as Homer has it, “not born of an oak or a rock,
”\footnote{Hom. Od. 19.163.}but of human parents, so that I have relatives and, men of Athens, I have three sons, one nearly grown up, and two still children; but nevertheless I shall not bring any of them here and beg you to acquit me. And why shall I not do so? Not because I am stubborn, Athenians, \marginpar{34e} or lack respect for you. Whether I fear death or not is another matter, but for the sake of my good name and yours and that of the whole state, I think it is not right for me to do any of these things in view of my age and my reputation, whether deserved or not; for at any rate the opinion prevails that Socrates \marginpar{35a} is in some way superior to most men. If then those of you who are supposed to be superior either in wisdom or in courage or in any other virtue whatsoever are to behave in such a way, it would be disgraceful. Why, I have often seen men who have some reputation behaving in the strangest manner, when they were on trial, as if they thought they were going to suffer something terrible if they were put to death, just as if they would be immortal if you did not kill them. It seems to me that they are a disgrace to the state and that any stranger might say that those of the Athenians who excel \marginpar{35b} in virtue, men whom they themselves honor with offices and other marks of esteem, are no better than women. Such acts, men of Athens, we who have any reputation at all ought not to commit, and if we commit them you ought not to allow it, but you should make it clear that you will be much more ready to condemn a man who puts before you such pitiable scenes and makes the city ridiculous than one who keeps quiet.

But apart from the question of reputation, gentlemen, I think it is not right \marginpar{35c} to implore the judge or to get acquitted by begging; we ought to inform and convince him. For the judge is not here to grant favors in matters of justice, but to give judgement; and his oath binds him not to do favors according to his pleasure, but to judge according to the laws; therefore, we ought not to get you into the habit of breaking your oaths, nor ought you to fall into that habit; for neither of us would be acting piously. Do not, therefore, men of Athens, demand of me that I act before you in a way which I consider neither honorable nor right nor pious, \marginpar{35d} especially when impiety is the very thing for which Meletus here has brought me to trial. For it is plain that if by persuasion and supplication I forced you to break your oaths I should teach you to disbelieve in the existence of the gods and in making my defence should accuse myself of not believing in them. But that is far from the truth; for I do believe in them, men of Athens, more than any of my accusers, and I entrust my case to you and to God to decide it as shall be best for me and for you. \marginpar{35e} I am not grieved, men of Athens, \marginpar{36a} at this vote of condemnation you have cast against me, and that for many reasons, among them the fact that your decision was not a surprise to me. I am much more surprised by the number of votes for and against it; for I did not expect so small a majority, but a large one. Now, it seems, if only thirty votes had been cast the other way, I should have been acquitted. And so, I think, so far as Meletus is concerned, I have even now been acquitted, and not merely acquitted, but anyone can see that, if Anytus and Lycon had not come forward to accuse me, he would have been fined \marginpar{36b} a thousand drachmas for not receiving a fifth part of the votes.

And so the man proposes the penalty of death. Well, then, what shall I propose as an alternative? Clearly that which I deserve, shall I not? And what do I deserve to suffer or to pay, because in my life I did not keep quiet, but neglecting what most men care for—money-making and property, and military offices, and public speaking, and the various offices and plots and parties that come up in the state—and thinking that I was really too honorable \marginpar{36c} to engage in those activities and live, refrained from those things by which I should have been of no use to you or to myself, and devoted myself to conferring upon each citizen individually what I regard as the greatest benefit? For I tried to persuade each of you to care for himself and his own perfection in goodness and wisdom rather than for any of his belongings, and for the state itself rather than for its interests, and to follow the same method in his care for other things. What, then, does such a man as I deserve? \marginpar{36d} Some good thing, men of Athens, if I must propose something truly in accordance with my deserts; and the good thing should be such as is fitting for me. Now what is fitting for a poor man who is your benefactor, and who needs leisure to exhort you? There is nothing, men of Athens, so fitting as that such a man be given his meals in the prytaneum. That is much more appropriate for me than for any of you who has won a race at the Olympic games with a pair of horses or a four-in-hand. For he makes you seem to be happy, whereas I make you happy in reality; \marginpar{36e} and he is not at all in need of sustenance, but I am needy. So if I must propose a penalty in accordance with my deserts, \marginpar{37a} I propose maintenance in the prytaneum.

Perhaps some of you think that in saying this, as in what I said about lamenting and imploring, I am speaking in a spirit of bravado; but that is not the case. The truth is rather that I am convinced that I never intentionally wronged anyone; but I cannot convince you of this, for we have conversed with each other only a little while. I believe if you had a law, as some other people have, \marginpar{37b} that capital cases should not be decided in one day, but only after several days, you would be convinced; but now it is not easy to rid you of great prejudices in a short time. Since, then, I am convinced that I never wronged any one, I am certainly not going to wrong myself, and to say of myself that I deserve anything bad, and to propose any penalty of that sort for myself. Why should I? Through fear of the penalty that Meletus proposes, about which I say that I do not know whether it is a good thing or an evil? Shall I choose instead of that something which I know to be an evil? What penalty shall I propose? Imprisonment? \marginpar{37c} And why should I live in prison a slave to those who may be in authority? Or shall I propose a fine, with imprisonment until it is paid? But that is the same as what I said just now, for I have no money to pay with. Shall I then propose exile as my penalty? Perhaps you would accept that. I must indeed be possessed by a great love of life if I am so irrational as not to know that if you, who are my fellow citizens, could not \marginpar{37d} endure my conversation and my words, but found them too irksome and disagreeable, so that you are now seeking to be rid of them, others will not be willing to endure them. No, men of Athens, they certainly will not. A fine life I should lead if I went away at my time of life, wandering from city to city and always being driven out! For well I know that wherever I go, the young men will listen to my talk, as they do here; and if I drive them away, they will themselves persuade their elders to drive me out, and if \marginpar{37e} I do not drive them away, their fathers and relatives will drive me out for their sakes.

Perhaps someone might say, “Socrates, can you not go away from us and live quietly, without talking?” Now this is the hardest thing to make some of you believe. For if I say that such conduct would be disobedience to the god and that therefore I cannot keep quiet, you will think I am jesting and will not believe me; \marginpar{38a} and if again I say that to talk every day about virtue and the other things about which you hear me talking and examining myself and others is the greatest good to man, and that the unexamined life is not worth living, you will believe me still less. This is as I say, gentlemen, but it is not easy to convince you. Besides, I am not accustomed to think that I deserve anything bad. If I had money, I would have proposed a fine, \marginpar{38b} as large as I could pay; for that would have done me no harm. But as it is—I have no money, unless you are willing to impose a fine which I could pay. I might perhaps pay a mina of silver. So I propose that penalty; but Plato here, men of Athens, and Crito and Critobulus, and Aristobulus tell me to propose a fine of thirty minas, saying that they are sureties for it. So I propose a fine of that amount, and these men, who are amply sufficient, will be my sureties. \marginpar{38c}

It is no long time, men of Athens, which you gain, and for that those who wish to cast a slur upon the state will give you the name and blame of having killed Socrates, a wise man; for, you know, those who wish to revile you will say I am wise, even though I am not. Now if you had waited a little while, what you desire would have come to you of its own accord; for you see how old I am, how far advanced in life and how near death. I say this not to all of you, \marginpar{38d} but to those who voted for my death. And to them also I have something else to say. Perhaps you think, gentlemen, that I have been convicted through lack of such words as would have moved you to acquit me, if I had thought it right to do and say everything to gain an acquittal. Far from it. And yet it is through a lack that I have been convicted, not however a lack of words, but of impudence and shamelessness, and of willingness to say to you such things as you would have liked best to hear. You would have liked to hear me wailing and lamenting and doing and saying \marginpar{38e} many things which are, as I maintain, unworthy of me—such things as you are accustomed to hear from others. But I did not think at the time that I ought, on account of the danger I was in, to do anything unworthy of a free man, nor do I now repent of having made my defence as I did, but I much prefer to die after such a defence than to live after a defence of the other sort. For neither in the court nor in war ought I \marginpar{39a} or any other man to plan to escape death by every possible means. In battles it is often plain that a man might avoid death by throwing down his arms and begging mercy of his pursuers; and there are many other means of escaping death in dangers of various kinds if one is willing to do and say anything. But, gentlemen, it is not hard to escape death; it is much harder to escape wickedness, for that runs faster than death. \marginpar{39b} And now I, since I am slow and old, am caught by the slower runner, and my accusers, who are clever and quick, by the faster, wickedness. And now I shall go away convicted by you and sentenced to death, and they go convicted by truth of villainy and wrong. And I abide by my penalty, and they by theirs. Perhaps these things had to be so, and I think they are well. \marginpar{39c} And now I wish to prophesy to you, O ye who have condemned me; for I am now at the time when men most do prophesy, the time just before death. And I say to you, ye men who have slain me, that punishment will come upon you straight-way after my death, far more grievous in sooth than the punishment of death which you have meted out to me. For now you have done this to me because you hoped that you would be relieved from rendering an account of your lives, but I say that you will find the result far different. Those who will force you to give an account will be more numerous than heretofore; \marginpar{39d} men whom I restrained, though you knew it not; and they will be harsher, inasmuch as they are younger, and you will be more annoyed. For if you think that by putting men to death you will prevent anyone from reproaching you because you do not act as you should, you are mistaken. That mode of escape is neither possible at all nor honorable, but the easiest and most honorable escape is not by suppressing others, but by making yourselves as good as possible. So with this prophecy to you who condemned me \marginpar{39e} I take my leave.

But with those who voted for my acquittal I should like to converse about this which has happened, while the authorities are busy and before I go to the place where I must die. Wait with me so long, my friends; for nothing prevents our chatting with each other \marginpar{40a} while there is time. I feel that you are my friends, and I wish to show you the meaning of this which has now happened to me. For, judges—and in calling you judges I give you your right name—a wonderful thing has happened to me. For hitherto the customary prophetic monitor always spoke to me very frequently and opposed me even in very small matters, if I was going to do anything I should not; but now, as you yourselves see, this thing which might be thought, and is generally considered, the greatest of evils has come upon me; but the divine sign did not oppose me \marginpar{40b} either when I left my home in the morning, or when I came here to the court, or at any point of my speech, when I was going to say anything; and yet on other occasions it stopped me at many points in the midst of a speech; but now, in this affair, it has not opposed me in anything I was doing or saying. What then do I suppose is the reason? I will tell you. This which has happened to me is doubtless a good thing, and those of us who think death is an evil \marginpar{40c} must be mistaken. A convincing proof of this been given me; for the accustomed sign would surely have opposed me if I had not been going to meet with something good.

Let us consider in another way also how good reason there is to hope that it is a good thing. For the state of death is one of two things: either it is virtually nothingness, so that the dead has no consciousness of anything, or it is, as people say, a change and migration of the soul from this to another place. And if it is unconsciousness, \marginpar{40d} like a sleep in which the sleeper does not even dream, death would be a wonderful gain. For I think if any one were to pick out that night in which he slept a dreamless sleep and, comparing with it the other nights and days of his life, were to say, after due consideration, how many days and nights in his life had passed more pleasantly than that night,—I believe that not only any private person, but even the great King of Persia himself \marginpar{40e} would find that they were few in comparison with the other days and nights. So if such is the nature of death, I count it a gain; for in that case, all time seems to be no longer than one night. But on the other hand, if death is, as it were, a change of habitation from here to some other place, and if what we are told is true, that all the dead are there, what greater blessing could there be, judges? For if a man when he reaches the other world, \marginpar{41a} after leaving behind these who claim to be judges, shall find those who are really judges who are said to sit in judgment there, Minos and Rhadamanthus, and Aeacus and Triptolemus, and all the other demigods who were just men in their lives, would the change of habitation be undesirable? Or again, what would any of you give to meet with Orpheus and Musaeus and Hesiod and Homer? I am willing to die many times over, if these things are true; for I personally should find the life there wonderful, \marginpar{41b} when I met Palamedes or Ajax, the son of Telamon, or any other men of old who lost their lives through an unjust judgement, and compared my experience with theirs. I think that would not be unpleasant. And the greatest pleasure would be to pass my time in examining and investigating the people there, as I do those here, to find out who among them is wise and who thinks he is when he is not. What price would any of you pay, judges, to examine him who led the great army against Troy, \marginpar{41c} or Odysseus, or Sisyphus, or countless others, both men and women, whom I might mention? To converse and associate with them and examine them would be immeasurable happiness. At any rate, the folk there do not kill people for it; since, if what we are told is true, they are immortal for all future time, besides being happier in other respects than men are here.

But you also, judges, must regard death hopefully and must bear in mind this one truth, \marginpar{41d} that no evil can come to a good man either in life or after death, and God does not neglect him. So, too, this which had come to me has not come by chance, but I see plainly that it was better for me to die now and be freed from troubles. That is the reason why the sign never interfered with me, and I am not at all angry with those who condemned me or with my accusers. And yet it was not with that in view that they condemned and accused me, but because \marginpar{41e} they thought to injure me. They deserve blame for that. However, I make this request of them: when my sons grow up, gentlemen, punish them by troubling them as I have troubled you; if they seem to you to care for money or anything else more than for virtue, and if they think they amount to something when they do not, rebuke them as I have rebuked you because they do not care for what they ought, and think they amount to something when they are worth nothing. If you do this, both I and my sons \marginpar{42a} shall have received just treatment from you.

But now the time has come to go away. I go to die, and you to live; but which of us goes to the better lot, is known to none but God.
\stepcounter{chapcount}
\chapter{Part \thechapcount: Life and Times of Socrates}\setcounter{seccount}{1}

With the introduction to philosophy out of the way (as in, the barebones of the method and the quest for answers), we will now move on to the life and times of Socrates, the protagonist and real world person who you should be reading in the assigned reading for this module. This should serve as an ample starting point to understand the raise and fall of Philosophy's most famous proponent.
\section{Part \thechapcount.\theseccount: Birth and Parents}

Socrates was born around spring 469BCE in Athens. At that time, the Persians had just attempted  (and failed) to invade Athens and Athens, on a global scale, was forming an alliance with the other city-states in the region (this alliance, called The Delian League, would grow into the Athenian Empire). This is on the Attic Peninsula and, politically and socially, it was divided into 139 districts, which were in turn, broken up among the 10 tribes which made up Athens. The members of these tribes were automatically Athenian. Socrates was a member of the Antiochis, which was located outside of the city walls, to the south-east. Keeping with the customs of the time, 5 days after Socrates' birth, his father, Sophroniscus, looked the infant Socrates over while walking around the hearth, to make sure that the kid was his, and then admitted him into the family. 5 days after this, Socrates was actually given a name and his father presented him to the local officials for the relevant paperwork (think of this as the equivalent of a birth-certificate and all that). In doing this, Sophroniscus took on the responsibility of ensuring that Socrates got a proper education and was made into a respectable member of society.
\section{Part \thechapcount.\theseccount: Education}\stepcounter{seccount}

In Athens, the ability to read and write was commonplace since around 520BCE and it was unheard of for a young man not to have that skill. When Socrates turned 5, he started the equivalent of elementary school. His education consisted of learning to read and write, gymnastics (it was expected that Athenians be physically fit for military service, think of this like PE), music, and basic mathematics. It is because of this that Socrates, frankly, became a nerd. Socrates, as to be expected, loved philosophy. At that time, philosophy was, and still is, the mother of all other subjects, physics, mathematics, biology, and so on. The term `philosophy' comes from two Greek roots, `philein' meaning love and `sophia' meaning wisdom. For Socrates and other Greeks (as well as any good philosophy/science today), philosophy was done with unaided reason and careful observation of the world. Over time, the careful observation aspect became what we call the sciences. But, to do philosophy, there are certain things which you don't get. You don't get anything in a religious scripture (so I don't want to see you reference a religious text in this class), you don't get what some authority figure says (just because they are an authority), you don't get myths, and you don't get things just because they were always done that way (no tradition/cultural beliefs). For philosophy, all you get is your observations and your own good reason. At that time, philosophers were inventing Geometry, proposing a heliocentric model of the solar system (the Earth revolves around the sun), and also various aspects of the natural sciences which would develop into their own fields (EG biology).

Socrates, by all accounts, was a remarkably ugly person. Many accounts say that his childhood classmates would call him “frogface”. He had bulging eyes, broad, flat nose, and thick lips. He also walked bow-legged and sideways like a crab. If you ever see a bust of Socrates, understand that the carving is an angel compared to him in reality.
\section{Part \thechapcount.\theseccount: “Adult" Socrates}\stepcounter{seccount}

At the age of 17, Socrates graduated from the schools and was sworn in as a full-fledged citizen of Athens (think of it as like turning 18 in the US). Socrates' father took him to the ceremony, but died shortly afterwards. His mother, Phaenarete, remarried and had another son, named Patrocles.  In those days, Athens had one of the first constitutional democracies which enshrined freedom of speech and, because of this, public discussion, voting, and debate about all matters were quite commonplace. Athens had many festivals and gatherings throughout the year, attracting many of the great minds from around Greece. According to some accounts, at 19, Socrates was often found at these festivals discussing things with the philosophers of the day, allowable because of the freedom of speech. Freedom of speech is a definite perk which was not had by many societies in those days, but being an Athenian did come with some responsibilities. Upon becoming a citizen, Socrates was also put on the draft for military service.
\section{Part \thechapcount.\theseccount: Military Career}\stepcounter{seccount}

After being called on for the draft and completing the required 2 years of training, Socrates served in the military for Athens. During his first posting, it was a time of relative peace, so Socrates likely practiced a trade (stonecutting, like his father). But, as the years progressed, Athens was starting to get into a war with Sparta. Socrates served on the front lines for many of the battles in 432BCE,  Socrates was called on to serve in the military for Athens on a few different occasions. After putting down a revolt, Socrates' troop entered into a very heavy conflict near Spartolus, where they suffered heavy casualties. In this fight, Socrates' bravery became legendary. He refused to retreat until he was the last person there and fought off enemy soldiers with another soldier, Alcibiades, on his back, saving his life. This deployment kept him away from home for 3 years. Socrates returned to duty again in 424BCE, where his bravery was again noted by the generals. His commander, Laches, noted Socrates' bravery when writing about the nature of courage, stating that he refused to retreat, even after the order was given, until he was the last person to leave. A year later, Socrates returned to duty and fought in another battle. After this, as far as we can tell, he did not serve in the military any longer. After this point, Athens and Sparta signed a treaty, which gave Athens a few years without the struggles of war. During this time, Socrates married Xanthippe, who, because of Socrates' military exploits, came with a large dowry.
\section{Part \thechapcount.\theseccount: Family Man}\stepcounter{seccount}

Socrates cared little for material possessions. Often wearing the same clothes for many days in a row, including sleeping in them. While in the military, as well as throughout his life, Socrates rarely, if ever, wore footwear, even on ice or snow. Athenians of his day described him as “frugal”. Although it was well within his means to afford various things, like shoes, he practiced “voluntary simplicity”. Xanthippe gave birth to three sons, the third being born while Socrates was in prison awaiting execution, so they did the ceremony in the prison. According to many accounts, Xanthippe had a volatile personality and was very unhappy with her hubby, and with good reason, he had a habit of spending all day and night in the agora (public market space) discussing questions and arguing with people when he could be out there making money (he was a stonecutter, like his father). According to one story, after chewing him out, Xanthippe climbed onto the roof of their house and dumbed a bucket of urine on his head as Socrates went out to debate people.

Because of his willingness and constant engagement in questioning and debate, by his middle age, Socrates became a very recognizable (today we would say famous) person on the streets of Athens. In the Greek comedies of the day, an ugly caricature of Socrates was a re-occurring character. Socrates was made-fun of for his appearance and his love of philosophy in at least 3 prize-winning plays. Socrates took these in stride. At one point a foreigner asked “who’s this loon, Socrates?” in the middle of a play. At which point Socrates stood up and said “ME!”

\section{Part \thechapcount.\theseccount: Trials in Athens}\stepcounter{seccount}

\subsubsection{How Trials Worked in Athens}

In Socrates' time, the procedure for a trial and court cases like this was pretty well established. In that time, also, there was no such thing as what we would call today a `public prosecutor', this is similar to how in London, prior to their police force, they had a hew and cry system. If a citizen suspected another of a crime, then they would report it to the officials to have the case looked over. Any actual citizen of Athens could initiate the procedure. Once such a crime was reported, the trial consisted of three parts. First, there was the Initiation of Criminal Proceedings. Next, they had The Preliminary Hearing (Anakrisis). And third, they had The Trial itself. As evidence for how common and stream-lined this process was in Athens, historically, just prior to his trial, Socrates was engaged in a conversation with a prosecutor for a trial just ending. This conversation was on the nature of goodness and the question related to it is the discussion for this module. However, most cases were settled prior to reaching the actual trial because, as we will see, making the arrangements is quite the undertaking.
\subsubsection{Initiation of Criminal Proceedings}

As I just mentioned, any Athenian could raise charges against another. In the case of Socrates, the proceedings began when Meletus, a poet, arranged a meeting, for a specific date, with the legal magistrate, or in some cases King Archon, in a colonnaded building called the Royal Stoa to answer charges of impiety. Meletus, then delivered an oral summons to Socrates in the presence of witnesses (or callers). Once the magistrate determined – after listening to Socrates and Meletus (and perhaps the other two accusers, Anytus and Lycon) – that the lawsuit was permissible under Athenian law, a date was set for the ``preliminary hearing" (anakrisis) and terms for the hearing were posted as a public notice at the Royal Stoa.
\subsubsection{The Preliminary Hearing (Anakrisis)}

The preliminary hearing before the magistrate at the Royal Stoa began with the reading of the written charge by Socrates' accuser, Meletus. Socrates then formally answered the charge. Both the written charge and denial were then attested to by each, under oath, as being true. The next phase of the preliminary hearing was an interrogation. First, the magistrate questioned both Meletus and Socrates. This is to see whether the issue could be settled out of court and to see whether there was any merit to the charges. Second, both the accuser and defendant were allowed to question each other. This was another chance to change the course of events. The third and final phase, supposing that the magistrate found the case worthy of consideration, the magistrate would draw up formal charges against the accused and set a date for the public trial. For Socrates, these charges (relating to impiety and corruption of youth), the actual paperwork, were preserved as a public document (antomosia), and they survived until at least the second century C.E., but were subsequently lost. 
\subsubsection{The Trial}

The trial of Socrates took place over a nine‐to‐ten hour period in the People's Court, located in the agora, the civic center of Athens. The jury consisted of 500 male citizens over the age of thirty, chosen by lot from among volunteers. People today might think that this is a ridiculously high number for a jury.  But, sometimes, the juries could be as large as 1501 men. This was a protection against bribery. For example, in order to ensure that a case went a certain way, the person would need to bribe at least 251 people, and it's not likely that many could afford to do this.  All jurors were required to swear by the gods of Zeus, Apollo, and Demeter the Heliastic Oath: ``I will cast my vote in consonance with the laws and decrees passed by the Assembly and by the Council, but, if there is no law, in consonance with my sense of what is most just, without favor or enmity. I will vote only on the matters raised in the charge, and I will listen impartially to the accusers and defenders alike."

Most of the jurors were probably farmers, as that was the principal occupation of the day. For their jury service they received payment of three obols. An obol was a currency of the day and it was around a 40TH of an ounce of silver. In general, three obols fed you an paid for a leisurely night. The jurors sat on wooden benches separated from spectators by some sort of barrier or railing. Given Socrates's fame and the notoriousness of the charge against him, the crowd of spectators was most likely large – including, of course, the most famous pupil of Socrates, Plato.

The trial began in the morning with the reading of the formal charges against Socrates by a herald. Few, if any, formal rules of evidence existed. The prosecution presented its case first. Meletus, Anytus, and Lycon had three hours, measured by a waterclock, to make their argument for a finding of guilt. Each accuser spoke from an elevated stage. No record of the prosecution's argument against Socrates survives. Following the prosecution's case, Socrates had three hours to answer the charges. Although many written versions of the defense – or apology – of Socrates at one time circulated, only two have survived: one by Plato and another by Xenophon. After the arguments, the herald of the court called on the jurors to consider their decision. In Athens, jurors did not retire to a juryroom to deliberate – they made their decisions without discussion among themselves, based in large part on their own interpretations of the law. The 500 jurors voted on his guilt or innocence by dropping bronze ballot disks into marked urns. Only a majority vote was necessary for conviction. Four jurors were assigned the task of counting votes. In the case of Socrates, the jury found him guilty on a relatively close vote of 280 to 220. (Interestingly, if less than 100 jurors voted for guilt, the accusers had to pay a fine to cover trial costs, which is similar to something which is found in court cases today.)
\subsubsection{The Final Phase}

If a defendant is convicted, the trial enters a second phase to set punishment. The prosecution and the defendant each propose a punishment and the jury chooses between the two punishment options presented to it. The range of possible punishments included death, imprisonment, loss of civil rights (i.e., the right to vote, the right to serve as a juror, the right to speak in the Assembly), exile, and fines. In the trial of Socrates, the principal accusers proposed the punishment of death. Socrates, if Plato's account is to be believed, proposed first the punishment – or, rather, the nonpunishment – of free meals in the center of the city, then later the extremely modest fine of one mina of silver. Apparently finding Socrates' proposed punishment insultingly light, the jury voted for the prosecution's proposal of death by a larger margin than for conviction, 360 to 140. The execution of Socrates was accomplished through the drinking of a cup of poison hemlock.

\section{Part \thechapcount.\theseccount: Socrates' Defense/Arguments}\stepcounter{seccount}

\subsection{The Charges}

There were three formal charges which Socrates was going to need to reply to in his 3 hour block of time. The first, the real zinger, was corrupting the youth of Athens. This was essentially causing the youth to have rebellious mindsets and be against the state. There was some evidence to this claim as several attempted rebellions and revolutions had taken place with many of the young people which Socrates spoke with often playing central roles, though Socrates himself was more of a bystander than anything in these cases.  The second charge, inventing new gods, was a lesser case. This is more in line with the initial charge of impiety. According to the laws of Athens, blasphemy of this sort was a crime punishable by death.  The third charge, which could have either built up or dismantled the second, was not believing in the gods of Athens. This could have gone two ways, either the accusers could have said that in inventing new gods, Socrates abandoned belief in the Athenian pantheon (which would have built up the case regarding blasphemy) or they could have said that Socrates was an atheist, which directly contradicts the second charge. As it turns out, the accusers, not trained in philosophy, did not see the contradiction and went with the latter option (being an atheist).
\subsection{Socrates' Method}

Philosophy, in this period, was still very much in its infancy, it wasn't as formalized as it is today (throughout this class, you will be learning the philosophic method, which is the great and powerful grandmother to the scientific method which you may be familiar with). For the pre-socratics (which includes Socrates), you can think of philosophy as a game where they are still trying to figure out the rules. If you have ever seen little kids try and play soccer or football for the first time, then you will have the right kind of image in mind.  Socrates, for his part, in the trial, was unfamiliar and inexperienced in how to present publicly, especially given the large number of people in attendance, and was not really the best at being persuasive (rhetoric was an area which he really did not have much respect for). So, Socrates begins his rebuttal by explaining that he will treat this like he would a debate in the Agora. In those cases, Socrates would merely ask questions to people. Rarely, if ever, does Socrates actually state his stance. Rather, Socrates merely restates the stance which the person had stated in reply to his question and then adds further questions to this. If you ever get the chance to see me in a lecture format, I do this methodology for the first few days of lecture, off and on. As a result, Socrates did not address the jury directly, rather just questioned Meletus. Though this method is quite good for teaching and one-on-one debate, it's not the best for a presentation of this sort. 
\section{Part \thechapcount.\theseccount: Socrates' Questions}\stepcounter{seccount}

Since I understand that the translations for this are a bit wordy and the word choice can be a bit unnatural (I have a lot of experience translating Latin and if a person is not careful or doesn't think about the context of the wording, the translation will come out as quite formal), I have taken the liberty of rewording some of the content of The Apology to make the meaning come through a little clearer. It's still important that you read the original, as it's a wonderful work, however, this is here to help. For this, you can read it as a script, with `S' being Socrates and `M' being Meletus.
\subsection{Corrupting the Youth}

This charge, as I have mentioned, is the real big charge, the others, though still punishable by death, don't have as much going for them, as we will see. To prove his point, Socrates asked Meletus, around, 11 questions, which were leading him down a very particular rabbit-hole.
\factoidbox{
    S: Do you think that it is the greatest importance to make the youth as great as possible?

    M: I do.

    S: Who improves them? 

    M:The Laws

    S: You didn’t answer me, Who has knowledge of these laws?

    M: These Jurymen

    S: What do you mean? Do all these people improve the youth?

    M: All of them.}

So, at this point, Meletus starts off by essentially claiming that the laws of Athens are the sort of things which can improve the youth. But this is not correct, laws are passive actors, they aren't the sort of thing which can take an active part in the raising of the youth. That requires a person who knows the laws and has the ability to accurately teach. So, to his credit, Meletus noticed this and backtracked, claiming that the jurymen have the knowledge of the laws and the ability to teach them (think about how it would have sounded if he claimed that the jurors didn't know the laws).
\factoidbox{
    S: By Hera! That’s a lot of people to improve our youth, what great shape we must be in! What about the audience? So they improve them? 

    M: They do.

    S: What about the council?

    M: They improve them too.

    S: Everyone in Athens makes the youth better but me, is that what you are saying?

    M: Yep, everyone makes the youth better but you.}

At this point, Meletus doesn't seem to see where this is going. Socrates has, through his questions, basically forced Meletus to claim that all the people of Athens improve the youth, with Socrates being the sole bad influence. From a rhetorical side, this was an interesting ploy. If Meletus had claimed otherwise, then Socrates could have easily countered by asking why those people weren't on trial with him, why charges hadn't be posed against them. But, since Meletus said that Socrates was the only corrupter, Socrates needs to go a different route, which leads us to the final `build-up' question:
\factoidbox{
    S: Man, you put me in a rough spot. But does this also apply to horses? That all people improve them and only one person corrupts them? Obviously, it is the opposite. The horse breeder (trainer) improves them and all others corrupt them. It would be a wonderful world if our youth only had one corrupting influence in their lives!}

The horse analogy has not aged well, as for contemporary English, calling a person a horse is a bit insulting. However, think about it this way, a good horse was like the nice car of its day. The vast majority of people aren't trained in how to repair and maintain a car, or in this case, care for and train a horse. Rather, there are very few people who really know how to maintain those things, others who use them, rather unintentionally, corrupt them (wear them out, make them forget their training, etc.). In the case of the youth, the vast majority of people aren't trained in the best child-raising techniques, they interact with children in unintentional ways which lead to negative behaviors, and so forth. It would be a great world if the children only had one corrupting influence, as that would get quickly drowned out by all of the other influences around them.

This leads us to the final set of questions concerning this particular accusation.
\factoidbox{
	S: You have made it clear that you care very little for the youth and haven’t put good thought into the reason you’re putting me on trial. Now answer me this: Is it better for a person to live among good fellows or wicked? This isn’t a hard question. Do wicked people harm their neighbors and good people benefit them?

    M: Certainly

    S: Now, is there a person alive who would rather be harmed than benefited?

    M: Certainly not

    S: Also, do you accuse me of corrupting the youth deliberately or unintentionally?

    M: Deliberately

    S: What then, Meletus? Are you so wise that you realize that the wicked do wicked and the good good but I am so stupid that I have not realized this? Have I failed to recognize that if I make anyone of my associates bad, they will harm me? But still you say that I do this great evil voluntarily? I can’t believe you, Meletus, nor do I think anyone else in the world does! But, leaving that aside, either I do not corrupt them, or if I corrupt them, I do it involuntarily. This shows that you are lying either way we go. But if I corrupt them involuntarily, the law is not to haul people into court, but to take them and instruct and admonish them in private. For it is clear that if I am told about it, I shall stop doing that which I do involuntarily. But you avoided associating with me and instructing me, and were unwilling to do so, but you haul me in here, where it is the law to haul in those who need punishment, not instruction.}

This is where we get Socrates' real argument against the charge. It follows a very interesting logical form, called Constructive Dilemma. It is one which Socrates, historically, had just used earlier in the day outside of the court concerning the nature of goodness. In an abstract form, the reasoning goes like this: One of two (or more) options is correct (A or B). If A is correct, then  C and if B is correct, then D. Therefore, either C or D is correct. Here is the argument for the case along with another example used by Socrates in another dialogue:
\noindent
\begin{tabular}{p{2.5in}|p{2.5in}}
Corrupting the Youth& Piety\\

    Either I don't corrupt the youth or I corrupt them unintentionally.&Either something is moral because it's commanded by the gods or it's commanded by the gods because it's moral.\\
    If I don't corrupt the youth, then the charge against me is false (I'm innocent here).&If it's moral because it's commanded by the gods, then the morality of actions is arbitrary (random, not fixed).\\
    If I corrupt the youth unintentionally, then you should have instructed me in private and not taken me to court (I'm innocent here).&If it's commanded by the gods because it's moral, then the morality of actions is not determined by the gods (we don't need to reference them to figure out the morality of actions).\\
    So, either way, I am innocent of this charge.&  So, either morality is arbitrary or it's not determined by the gods.\\
\end{tabular}

The argument regarding piety can also work if you replace `the gods' with `God'. If you are interested in that particular argument, look into the Euthyphro Dilemma.