\part{What is Philosophy?}
\label{ch.modone}
\addtocontents{toc}{\protect\mbox{}\protect\hrulefill\par}
\chapter{Part 1: What Philosophy Is and Isn't}
Philosophy is not just a way of life, it is not just some outlook which a person takes on. It is also not the product of some really deep thinking. Philosophic thought is not philosophy. One does not have a philosophy. That's not how it works. `My philosophy on that is...' is like saying `my science on that is...' What they should say is `my belief is that...' or `I have thought long and hard about it and I think...' Philosophy is critical thinking taken to the ridiculous extreme. Since it is, at its core, thinking, one does not simply study philosophy. Philosophy is an activity, it's something that you do. Doing philosophy involves finding a stance on some issue and then coming up with reasons for that stance. This could be a stance that you hold or it could be the stance that your opposition holds in some debate.\footnote{This would be putting yourself in `their shoes', so to speak, so that you can see where they are coming from and thereby exploit the flaws in their stance.} In order to really truly prove your point, you will want to know why another might believe otherwise and have the skills to show the errors in that reasoning. Engaging in philosophy is how you get the skills to find the errors in the reasoning. 

Since philosophic activities tend to go really abstract really quickly and since the questions which philosophers tend to think about are very difficult to answer, some have claimed that Philosophy is a pointless endeavor. This is, however, not accurate. All fields of study are defined, roughly, in terms of the questions which they are trying to answer or have the ability to answer. The questions in Philosophy tend to be those which other fields can't answer. In those cases, most of the time, the standards are very high, higher than other fields can match. Sometimes, those standards are so high that some claim that they are not possible to answer. Again, this is not accurate. There are answers, even if those answers are difficult to get. Doing philosophy tends to involve the following activities:


\begin{tabular}{p{2in}|p{3.5in}}\hline
Philosophy Involves&Example\\\hline
Resolving Confusion &Noticing that people misunderstand something because of a word-choice.\\\hline
Unmasking Assumptions &Seeing that you/another hold a position for a reason which was not stated.\\\hline
Revealing Presuppositions &Seeing that there are unexamined/unexplained jumps in reasoning or an aspect of the reasoning which is hidden.\\\hline
Distinguishing Importance &Figuring out which things you need to do vs which things you want to do.\\\hline
Testing Positions &Putting yourself in another's shoes to see whether their stance makes sense or to see whether your own stance holds water.\\\hline
Correcting Distortions &Fixing/identifying misrepresented stances or facts, seeing that you are being misinformed/lied to.\\\hline
Looking for Reasons &Asking yourself why a person would think/do what they are thinking/doing\\\hline
Examining World-Views &Seeing whether another's explanation for something or some course of behavior is correct/accurate to the world itself. People can have wrong opinions.\\ \hline
Questioning Conceptual Frameworks &Seeing whether a particular way of carving the world, a way of thinking about it, is better or worse than another way. We may think that the world is one way, but is it?
\end{tabular}

\section{Part 1.1: How Does One Do Philosophy?}
\label{s:p1.1}
As a kind of inquiry, philosophy is aimed at establishing knowledge and understanding. Even when you can't get exact and precise knowledge about a topic, you can often find interesting things to learn (for example, why you can't get that knowledge). So, rational inquiry (critical thinking, doing philosophy) may be interesting and fruitful even when we don't get straight-forward answers. Once we raise a philosophical issue, a question or puzzle, whether about the nature of justice or about the nature of reality, we want to ask what can be said for or against the various possible answers to our question. For example, if my question is `when is it morally permissible to lie?', I look at cases where it seems OK and cases where it just seems wrong. I look at the \textbf{reasons} why they seem that way.

At this point, I am making \glspl{argument}. These are my sets of reasons for a stance. Some arguments give us better reasons or accepting their \glspl{conclusion} (what is being argued for) than others. Once we have an argument, we want to evaluate the reasoning it offers. So, if you want to know what philosophers do, this is a pretty good answer: philosophers formulate and evaluate arguments. We look at the reasons for some stance and figure out whether those reasons actually hold water, so to speak.  Your introduction to philosophy should be as much a training in how to do philosophy as it is a chance to become acquainted with the views of various philosophers. To that end, you should carefully study the sections below on arguments. I personally approach teaching philosophy from the perspective that there's still a lot going on, I show you the relevance of philosophy to questions which we are puzzled with today (like Artificial Intelligence, the Abortion Debate, the Existence of God, and so on). Through this, you will become accuanted with philosophers, both historical and contemporary, but this is more of a byproduct than a goal.

Once we have some philosophical position to think about, we want to ask what arguments can be made for or against it (\textbf{formulate}). We then want to examine the quality of the arguments (\textbf{evaluate}). Evaluating flawed arguments, figuring out why they are wrong, often leads us to other, stronger, arguments and the process of formulating, clarifying, and evaluating arguments continues.

This circular method of question and answer (make an argument, evaluate it, make a stronger one, and repeat) is known as a \gls{dialectic}. A dialectic looks a lot like debate, but a big difference is in the goals of the two activities. The goal of a debate is to win by persuading an audience that your position is right and your opponent’s is wrong.

\newglossaryentry{dialectic}
{
name=dialectic,
description={the art of investigating and determining the truth of some stance or opinion},
plural=dialectics
}


A dialectic, on the other hand, is aimed at inquiry. The goal is to learn something new about the issue under discussion. Unlike debate, in a dialectic your sharpest critic is your best friend. Critical evaluation of your argument brings new evidence and reasoning to light. The person you disagree with on a philosophical issue is often the person you stand to learn the most from (and this doesn’t necessarily depend on which of you is closer to the truth of the matter).

Dialectic is sometimes referred to as the Socratic Method after the famous originator of this systematic style of inquiry. For this module, the reading is one of the more famous of Plato’s dialogues, namely The Apology. This will give you a good sense for how the Socratic Method works. Then watch for how the Socratic Method is deployed throughout the rest of the course.

Doing philosophy boils down to thinking really hard, but in a structured and organized way. The thinking must be critical and comprehensive.

The point of arguments is to get at the \textbf{truth}. But, in the kind of world we live in today, it is worth taking a look at what philosophers mean by `truth', before we get too much into arguments.

\subsection{Part 1.1.1: Truth}
\label{s:p1.1.1}
Since both science and philosophy are mainly concerned with getting at knowledge and understanding about the world (though the kind of knowledge may be argued to be different), it is natural to think that both are after the truth about things. There are some interesting and some confused challenges to the idea that philosophy and science are truth oriented. But for now let’s assume that rational inquiry is truth oriented and address a couple of questions about truth. Let’s focus on just these two:
\begin{earg}
    \item[\ex{truth1}] What is it for a claim to be true?
    \item[\ex{truth2}] How do we determine that a claim is true?
\end{earg}
It’s important to keep these two questions separate. Questions about how we know whether something is true (like the second question) are epistemic questions. Epistemic questions, which we will return to in a later module, are questions about knowledge, beliefs, reasons, or, to a certain extent, faith. But the question of what it is for something to be true (the first question) is not an epistemic issue. The truth of a claim is quite independent of how or whether we know it to be true. Questions about whether or not something is true are, more or less metaphysical questions.

For example, consider these two claims:

\begin{earg}
    \item[] There's extraterrestrial life.
    \item[] There's no extraterrestrial life. 
\end{earg}


I assume we don’t know which of one of these is true, but surely one of them is, for any sensible claim that you make, it's either true or false. Whichever of these claims is true, its being true doesn’t depend in any way on whether or how we know it to be true. So, there is a correct answer to the question ``is there extraterrestrial life?" but as it sits, we don't know what that answer is. There are many truths that will never be known or believed by anyone, and appreciating this is enough to see that the truth of a claim is not relative to belief, knowledge, proof, or any other epistemic notion.\footnote{There are certain facts within mathematics which are not knowable, this could be because of the physical limitations of the computations or because of the nature of the question which the fact is an answer to. For example, there is a first digit to Graham's number, but it is so large that it's physically impossible to determine.} Similarly, unless you hold certain philosophic positions about certain things (we will see this in the Relativism/Meta-Ethics Module), just because you believe something, that does not make it true. How much weight you give a belief also doesn't change how true it is. Unlike somethings which you may encounter, like knowledge, certainty, belief, and so on, truth does not come in degrees (and this is regardless of the stances you take). 

But then what is it for a claim to be true? The ordinary everyday notion of truth would have it that a claim is true if the world is the way the claim says it is. And this is pretty much all we are after for this class. When we make a claim, we represent some part of the world as being a certain way. If how my claim represents the world fits with the way the world is, then my claim is true. Truth, then, is correspondence, or good fit, between what we assert and the way things are. There are other accounts of truth out there, which do better or worse jobs at various things (for example, this account can't make ``Harry Potter is a wizard"\autocite{HarryPotter1}, ``Stepan Arkadyevich Oblonsky's wife is mad at him"\autocite{AnnaKarenina}, or ``Sherlock Holmes is a detective"\autocite{Sherlock} true, because those beings don't exist), but this works for our purposes.  

\subsection{Part 1.1.2: Truth and Meaning}
\label{s:p1.1.2}

A potential confusion about truth comes from confusing a sentence with the meaning behind it. This is where we get into a very basic introduction into Philosophy of Language. This area of philosophy deals with things such as meaning, reference, and so forth (Philosophy of Language is what I mostly work in, it deals with semantics while Linguistics deals with syntax, but the fields overlap quite a bit). Even when we aren't trying very hard, we can use words and sentences in a ton of different ways and there seems to be some vagueness or ambiguity in natural languages. For example, I can easily make examples where sentences might have two or more different interpretations, like double or triple entendres. 
\begin{earg}
\item[\ex{ambi1}] GOP grills the IRS chief over lost emails.
\item[\ex{ambi2}] If your dog poops, you must put it in the trash can.
\item[\ex{ambi3}] A woman gives birth in the UK every 48 seconds.
\item[\ex{ambi4}] People wanted for pickling and canning.
\end{earg}
All four of the above examples have at least 2 different interpretations and, in each case, one seems true while the other seems false. In Example \ref{ambi1}, it could be interpreted to mean that the GOP are having an IRS chief BBQ over a server or it could mean that they are harshly questioning the chief. In Example \ref{ambi2}, one way to understand this is that it's telling you to put your dog in the garbage can, while another is telling you to put the poop therein. In Example \ref{ambi3}, we could say that there's a single woman who is having a baby every 48 seconds (she must be tired) or that some woman, not necessarily the same woman, is having a child in that time. In Example \ref{ambi4}, it could be that we have a cannibal looking to store their meats for the winter using canning and fermenting techniques or it could be that a food fermentation business is looking to hire on some more people in those departments. Given these examples, and how often we misunderstand each other on a daily basis,  it might be tempting, therefore, to think that the truth of a sentence must be relative to its interpretation.  In an even more robust example, imagine the following:

\factoidbox{Suppose that we all collectively switched the sorts of things the words `dog' and `cat' pick-out. So, the word `dog' now is used to point out the meowing critters and the word `cat' refers to the barking ones.  In this case, it would seem that the sentence `dogs are canines' would be false and `dogs are feline' would be true. If we, again, flipped the meanings of the words `feline' and `canine', we would get that `dogs are canines' as true, but for a totally different reason than it was originally.}

But does this make truth open for interpretation? Well, no, but in a sense, yes. When we look at a language, from one perspective, we notice that things like words and sentences are nothing more than characters on a screen/page, random collections of sounds, or certain structured gestures (in the case of sign languages). Those things, on their own, don't have meaning. There must be something extra, beyond the sound and signs, which has the meaning.  Philosophers call the meaning behind a sentence a proposition. A proposition, itself, is not a sentence or a word, rather it's the meaning behind the sentences.


\begin{center}
\begin{tabular}{p{2in}|p{1.5in}|p{2in}}\hline
Sentence &Language &Proposition\\\hline
Snow is white &English &that snow is white\\\hline
Schnee ist weiss &German &that snow is white\\\hline
Nix alba est &Latin &that snow is white\\\hline
La neige est blanche &French &that the snow is white\\\hline
\end{tabular}
\end{center}

All of the above examples are different sentences, made clear because they are in different languages, but they all express the same proposition. The truth of a sentence is relative to the truth of the proposition attached to it, propositions are the things which are true or false, and then the truth of the proposition is determined by its correspondence with reality. Translators, at least the good ones, often take the sentence in one language, figure out the proposition connected to it, and then express the same proposition in the necessary language.\footnote{Sometimes there can be cases where a language has the ability to express a proposition which is not possible to express in another without adding something to that language or giving information beyond what was in the original statement. For example, take Cicero's Epistulae ad Familiares 9.22, the examples Cicero gives in the letter of profanity and why they make sense are  not translatable without explaining some aspect of Latin phonology and semantics.}

So the proposition expressed by a sentence is not itself a linguistic thing. Propositions themselves don't have meaning, but rather they are the meaning behind a sentence. For a bit of language to be open to interpretation is for us to be able to attach different propositions (meanings) to it. But the meanings themselves are not open to further interpretation. And it is the proposition, what is meant by the sentence, which makes the statements, sentences, true or false. So, when I speak of arguments consisting of premises (the supporting evidence and their arrangement in the argument), I am talking about the core meaning behind the sentence, not the sentence itself. If we misinterpret the sentence, then we haven’t yet gotten on to the claim being made and hence probably don’t fully understand the argument. Getting clear on just what an argument says is critical to the dialectical process.

\subsection{Simplified Breakdown}

Even if you are exceptionally bright, you probably found the last couple paragraphs rather challenging. That’s OK. You might work through them again more carefully and come back to it in a day or two if it’s still a struggle. The path to becoming a better critical thinker is more like mountain climbing than a walk in the park, but with this crucial difference: no bones get broken when you fall off an intellectual cliff. So you are always free to try to scale it again. We can sum up the key points of the last few paragraphs as follows:
\begin{earg}
    \item[] We use sentences, bits of language, to express propositions.
    \item[] The proposition, what is meant by the sentence, represents the world as being some way.
    \item[] The proposition is true when it represents the world in a way that corresponds to how the world is.
    \item[] Truth, understood as correspondence between a claim (a proposition) and the way the world is, is not relative to meaning, knowledge, belief, or opinion.
\end{earg}
Hopefully we now have a better grip on what it is for a claim to be true. A claim is true just when it represents things as they are. As is frequently the case in philosophy, the real work here was just getting clear on the issue. Once we clearly appreciate the question at hand, the answer seems pretty obvious. So now we can set aside the issue of what truth is and turn to the rather different issue of how to determine what’s true.

\subsection{The connection}

Arguments are how philosophers, and scientists, and the rest of us get at the truth. We are using them to structure our reasons and prove that the conclusion is true.

\section{Part 1.2: Arguments}
\label{s:p1.2}


The common-sense, everyday, way to tell whether a claim is true or false is to look at the reasons for or against it. Sometimes our observations give us good reasons. For example, I have a good reason for thinking my bicycle has a flat tire when I see the tire sagging on the rim or hear air hissing out of the tube. But often the business of identifying and evaluating reasons is a bit more involved. Logic is the business of identifying and evaluating reasons. You do this all the time. You give yourself reasons to choose one shirt over another for a job interview, you reason your way through making a choice about which classes to take, you believe certain things based on evidence, and you give reasons for why something isn't your fault (when you make excuses). In all of these cases and more, you are making \textbf{arguments}.

This is very different from how you might use the term ‘argument'. In everyday language, we sometimes use the word ‘argument’ to talk about belligerent shouting matches. If you and a friend have an argument in this sense, things are not going well between the two of you. Logic is not concerned with such teeth-gnashing and hair-pulling. They are not arguments, in our sense; they are just disagreements. For a humorous example of how arguments and disagreements differ, take a look at this funny exchange from Monty Python:\autocite{ArgumentClinic}

\factoidbox{\begin{multicols}{2}
Man: Is this the right room for an argument?\\
Other Man:(John Cleese) I've told you once.\\
Man: No you haven't.\\
Other Man: Yes I have.\\
M: When?\\
O: Just now.\\
M: No you didn't!\\
O: Yes I did!\\
M: You didn't!
\ldots\\
O: Oh I'm sorry, is this a five minute argument, or the full half hour?\\
M: Ah! (taking out his wallet and paying) Just the five minutes.\\
O: Just the five minutes. Thank you.\\
O: Anyway, I did.\\
M: You most certainly did not!\\
O: Now let's get one thing quite clear: I most definitely told you!\\
M: Oh no you didn't!\\
O: Oh yes I did!
\ldots\\
M: Oh look, this isn't an argument!\\
(pause)\\
O: Yes it is!\\
M: No it isn't!\\
(pause)\\
M: It's just contradiction!\\
O: No it isn't!\\
M: It IS!\\
O: It is NOT!
\ldots\\
M: (exasperated) Oh, this is futile!!\\
(pause)\\
O: No it isn't!\\
M: Yes it is!\\
(pause)\\
M: I came here for a good argument!\\
O: AH, no you didn't, you came here for an argument!\\
M: An argument isn't just contradiction.\\
O: Well! it CAN be!\\
M: No it can't!\\
M: \emph{An argument is a connected series of statements intended to establish a proposition.}\\
O: No it isn't!\\
M: Yes it is! ‘tisn't just contradiction.\\
O: Look, if I \emph{argue} with you, I must take up a contrary position!\\
M: Yes but it isn't just saying ‘no it isn't'.\\
O: Yes it is!
\ldots\\
M: No it ISN'T! \emph{Argument is an intellectual process.} Contradiction is just the automatic gainsaying of anything the other person says.\\
O: It is NOT!\\
M: It is!
\ldots 
\end{multicols}}

An argument is a reason for taking something to be true. \Glspl{argument} are made out of two or more claims, one of which is a \gls{conclusion}. The conclusion is the claim the argument purports to give a reason for believing. The other claims are the \glspl{premise}. The premises of an argument taken together are offered as a reason for believing its conclusion. In the above exchange, the person looking for an argument claims that ``An argument is a collected series of statements to establish a definite proposition". This is a close approximation to how philosophers use the term, but a better, more exact definition is:


\newglossaryentry{argument}
{
name=argument,
description={A connected series of sentences, divided into \gls{premise}s and \gls{conclusion}},
plural=arguments
}

\newglossaryentry{premise}
{
name=premise,
description={A sentence in an \gls{argument} other than the \gls{conclusion}, often indicated with a premise indicator},
plural=premises
}

\newglossaryentry{conclusion}
{
name=conclusion,
description={The sentence which an argument is intended to support or prove, often indicated with a conclusion indicator.},
plural=conclusions
}

\begin{center}
An argument is a collected series of propositions intended to establish others in the series.
\end{center}
The propositions (statements, in this case) which support others are our premises. The one being supported is the conclusion. That `intent' bit is a heavy hitter as we will soon see.

Some arguments provide better reasons for believing their conclusions than others. In case you have any doubt about that, consider the following examples:


\noindent
\begin{tabular}{p{2.75in}|p{2.75in}}\hline
Example \exarg{linecook1}: &Example \exarg{linecook2}:\\\hline
Sam is a line cook. &Sam is a line cook.\\
Line cooks generally have good kitchen skills. &Line cooks generally aren't paid well.\\
Therefore, Same can probably cook well &Therefore, Sam is probably a billionaire. 
\end{tabular}

\noindent\begin{tabular}{p{2.75in}|p{2.75in}}\hline
Example \exarg{boston3}: &Example \exarg{boston4}:\\\hline
Boston is in Massachusetts. &Boston is in California.\\
Massachusetts is east of the Rockies.&California is west of the Rockies.\\
Therefore, Boston is east of the Rockies.&Therefore, Boston is west of the Rockies.
\end{tabular}

The premises in Example \ref{linecook1} provide pretty good support for thinking Sam can cook well. That is, assuming the premises in the first argument are true, we have a good reason to think that its conclusion is true (at this stage, we are not interesting in whether the premises are actually true, only the structure). Whether or not the argument is any good depends on how well they establish the conclusion, relative to the intent. So, we can say that the reasoning in Example 1 is pretty good (at least). The premises in Example \ref{linecook2} give us no reason to think Sam is a millionaire, let alone a billionaire. So whether or not the premises of an argument support its conclusion is a key issue.

Looking at Examples \ref{boston3} and \ref{boston4}, we see some similarities. The intent is different (in the first two, the goal was to get something with likelihood, in these, the intent is to get something with certainty), but we seem not to like the second (Example \ref{boston4}) and the first seems OK (Example \ref{boston3}). The main problem with Example \ref{boston4} is very different than the problem with \ref{linecook2}. And, in fact, the issue is an entirely different animal. With Example \ref{linecook2}, the issue was with the structure, here, the issue is with the truth. Notice, the structure of the arguments, Examples \ref{boston3} and \ref{boston4}, are exactly the same, there's no difference in how they are being reasoned.  If its premises were true, then we would have a good reason to think the conclusion is true (the best reason, 100\% certainty reasoning). That is, the premises do support the conclusion. But the first premise of the second argument just isn’t true. Boston is not in California. So the latter pair of arguments suggests another key issue for evaluating arguments. Good arguments have premises which support the conclusion, the best arguments have true premises.

That is pretty much it. The best arguments are those which have true premises that, when taken together, support its conclusion. So, evaluating an argument involves just these two essential steps:
\begin{enumerate}
    \item Determine whether or not the premises are true.
    \item Determine whether or not the premises support the conclusion (that is, whether we have grounds to think the conclusion is true if all of the premises are true).
\end{enumerate}
Often, figuring out whether the premises of an argument are true involves looking at further arguments for those premises individually. An argument might be the last link in a long chain of reasoning. In this case, the quality of the argument depends on the whole chain. Really high-quality philosophy papers (I don't expect these in this class) often involve an argument and further arguments for each of the premises. And since arguments can have multiple premises, each of which might be supported by further arguments, evaluating one argument might be more involved yet, since its conclusion is really supported by a rich network of reasoning, not just one link and then another. While the potential for complication should be clear, the basic idea should be pretty familiar. Think of the little kid who would constantly ask their parent “why?” after being an explanation. Even at a young age we understood that the reasons for believing one thing can depend on the reasons for believing a great many other things.

However involved the network of reasons supporting a given conclusion might be, it seems that there must be some starting points. That is, it seems there must be some reasons for believing things that don’t themselves need to be justified in terms of further reasons. There needs to be some sort of bedrock, ground level, foundation at the bottom. Otherwise the network of supporting reasons would go on without end, and the kid would happily ask `why' until the end of time. The problem here is about how do we tell where the ultimate foundations of knowledge and justified belief. This is a big epistemological issue and we will return to it later in the course. For now, let’s consider one potential answer we are already familiar with. In the sciences, our complex chains of reasoning seem to proceed from the evidence of the senses.

We think that evidence provides the foundation for our edifice of scientific knowledge. Sounds great for science, but where does this leave philosophy? Does philosophy entirely lack evidence on which its reasoning can be based? Philosophy does have a kind of evidence to work from and that evidence is provided by philosophical problems. When we encounter a problem in philosophy this often tells us that the principles and assumptions that generate that problem can’t all be correct. This might seem like just a subtle clue that leaves us far from solving the big mysteries. But clues are evidence just the same. Sensory evidence by itself doesn’t tell us as much about the nature of the world as we’d like to suppose. Scientific evidence provides clues, but there remains a good deal of problem solving to do in science as well as in philosophy.

So we can assess the truth or falsity of the premises of an argument by examining evidence or by evaluating further argument in support of the premises. Now we will turn to the other step in evaluating arguments and consider the ways in which premises can support or fail to support their conclusions. The question of support is distinct from the question of whether the premises are true. When we ask whether the premises support the conclusions we are asking whether we’d have grounds for accepting the conclusion assuming the premises are true. In answering this question we will want to apply one of two standards of support: deductive validity or inductive strength.

\subsection{Part 1.2.1: Deductive Validity}
\label{s:p1.2.1}
There are two kinds of arguments which we deal with, one is \glspl{deductive argument} and the other is inductive arguments, here we are going to be concerned with deduction. When you are dealing with this kind of argument, the standard for the `goodness' of the argument is validity. An argument counts as deductive whenever it is aiming at this standard of support. Deductive validity is the strictest standard of support we can uphold. It is also the one which is used in the vast majority of philosophy (with the exception of some, very contemporary, fields).\footnote{This would be Experimental Philosophy, which involves testing philosophic intuitions. This can be useful when you are trying to get a common person, average Joe view.} In a deductively \gls{valid} argument, the truth of the premises guarantees the truth of the conclusion. This standard is not concerned with whether the premises are actually true, that is a different standard. Basically, if you assume that the premises are true, the conclusion must be true. Here are two equivalent definitions of deductive validity:

\newglossaryentry{valid}
{
name=valid,
description={A property of arguments where the truth of the premises guarantee the truth of the conclusion; i.e. it is impossible for the premises to be true and the conclusion false}
}

\begin{tabular}{p{1in}|p{4.5in}}
Deductive&Definition\\\hline
D &A valid argument is an argument where if its premises are true, then its conclusion must be true.\\\hline
D* &A valid argument is an argument where it is not possible for all of its premises to be true and its conclusion false.
\end{tabular}

That (D*) standard is the more formal way, and exact way, of claiming (D). Here are a few examples of deductively valid arguments:


\noindent
\begin{tabular}{p{2in}|p{2in}|p{2in}}
Example \exarg{socrates1} &Example \exarg{mammals2} &Example \exarg{raining3}\\\hline
If Socrates is a man, then Socrates is mortal &All primates are mammals &If it's wet outside, then it's raining\\
Socrates is a man &All humans are primates &It's not raining\\
Therefore, Socrates is mortal &Therefore, all humans are mammals &Therefore, it's not wet outside.
\end{tabular}

If you think about these two examples for a moment, it should be clear that there is no possible way for the premises to all be true and the conclusion false. The truth of the conclusion is guaranteed by the truth of the premises. \ref{raining3}, on the other hand, should have given you pause. Yes, that argument is valid, but it's not \gls{sound}. Soundness involves whether or not the premises are in fact true, validity concerns the structure of the argument. If an argument is sound, then it's valid, but not the other way around. In contrast, the following arguments are not valid:

\newglossaryentry{sound}
{
name=sound,
description={A property of arguments that holds if the argument is valid and has all true premises}
}


\noindent
\begin{tabular}{p{2in}|p{2in}|p{2in}}
Example \exarg{socrates4} &Example \exarg{cookies5} &Example \exarg{raining6}\\\hline
If Socrates the Cat is a man, then Socrates the Cat is mortal &Billy or Sally stole cookies from the jar &If it's wet outside, then it's raining\\\hline
Socrates the Cat is mortal &Billy stole cookies &It's not wet outside\\\hline
Therefore, Socrates the Cat is man &Therefore, Sally didn't steal cookies &Therefore, it's not raining
\end{tabular}

These examples, again, are not valid, they have a flaw in their reasoning, in arguing this way, you will have an error which will lead you astray. To see why, it might require a bit of imagination, but it's a reasonably simple test. Imagine a case where the premises are true and the conclusion is false. If you can't do it, then the argument is valid, if you can, then there's a flaw in the reasoning. Think of this test as like a trial by worst-case scenario. For Example \ref{socrates4}, Socrates the Cat is obviously mortal, and \emph{were} the cat a man, he still would be mortal, but that doesn't mean that a cat is a man. For Example \exarg{cookies5}, it's perfectly possible that both Billy and Sally were naughty, bad, kids and stole some cookies from the jar, just because one did it doesn't mean that the other didn't.\footnote{Some languages, like Latin, have two different words to express the different kinds of disjunctions (or-statements). There are inclusive disjunctions, which allow for both to be true and there are exclusive disjunctions, which only allow for only one of them to be true. English uses both but does not have a built in way to tell the difference between them consistently. Context is your best bet for this, but also, sometimes `either... or...' is used for exclusive and just `... or...' is used for inclusive.} For example \ref{raining6}, this might be easier to imagine for people who have spent time in the desert: In those sorts of regions, there's a phenomenon which I know as sun-showers. This is where the sun is still shining, the ground is perfectly dry, but there's rain coming down from the sky. So, it's perfectly possible for it to be raining and yet not be wet outside.
\factoidbox{
Deductive validity is the gold standard for an argument. Soundness is the platinum standard.}

The deductive arguments we’ve looked at here are pretty intuitive. We only need to think about whether the conclusion could be false even if the premises were true. But most deductive arguments are not so obvious.These example arguments only use one logical rule and only two supporting premises; most of the ones which are really powerful use several logical rules/forms in conjunction with each other and several more lines of supporting evidence.  Logic is the science of deductive validity. For a class on this subject, check out PHIL\&120, which is the barebones, no fluff, mathematics of Philosophy. Philosophy has made some historic advances in logic over the past few centuries, with great advancements happening in the last century.  If you want to see some of the practical side of their good work, what it's actually been used for, just look at your computer. The background coding for that machine, I know because I worked with it for a time and I know the history behind it, is all based on the logical structures which philosophy has used forever.

\subsection{Part 1.2.2: Inductive Strength}
\label{s:p1.2.2}

|n the section, we looked at deductive arguments and their standards are validity and soundness, but those aren't the only kinds of arguments which are used, other fields use a different kind of argumentation, \glspl{inductive argument}. The standard for the goodness here is strength, not validity. Like with deductive arguments, an argument counts as inductive when it's shooting for this kind of support for the conclusion. Inductive strength is a weaker standard of support we can shoot for. That being said, it's also the standard which is found in science.\footnote{As well as in Experimental Philosophy.} An inductively strong argument is one where the premises make the conclusion more likely, give probabilistic support. Strength, again, is not concerned with the truth of the premises, that's a different standard still. Here are some examples of inductive arguments:

\newglossaryentry{deductive argument}
{
name=deductive argument,
description={An argument such that the intent behind it is to guarantee the conclusion. There is no ‘wiggle room'. In general, deductive arguments move from broad general principles and to more particular instances.},
plural=deductive arguments
}

\newglossaryentry{inductive argument}
{
name=inductive argument,
description={An argument such that the intent behind it is \emph{merely} to make the conclusion more likely, assuming the truth of the premises. There is some ‘wiggle room'. In general, inductive arguments move from a collection of particular instances and then use those to support the truth of some general principle},
plural=inductive arguments
}

\noindent
\begin{tabular}{p{2in}|p{2in}|p{2in}}
Example 1: &Example 2: &Example 3:\\\hline
Sam is a line cook &Most things cancerous to mice are cancerous to humans &Oscar was born in North America\\
Line cooks generally can cook well &Various chemicals in tobacco products are cancerous to mice &Oscar was not born in Mexico\\
Therefore, Sam can probably cook well &Therefore, various chemicals in tobacco products are probably cancerous to humans &Therefore, Oscar was probably born in the USA
\end{tabular}

Examples 1 and 2 seem like pretty decent arguments. The premises do support the conclusion (we are looking at inductive strength here). But, none of the above arguments are, in fact, valid. It's perfectly possible for the premises to be true and the conclusion false. Sam could be a brand new cook hired because he’s the manager’s son who has never cooked in his life. The chemicals in tobacco products could be cancerous to mice but not humans.\footnote{In this case, we can only look at the information provided in the premises, same with validity, so if you disagree with this sentence, you would be correct, but that's bringing information not given as support in the argument.} Many arguments give us good reasons for accepting their conclusions even if their premises being true fails to completely guarantee the truth of the conclusion.  The intent behind all of these arguments is different. The point of these arguments is not to guarantee the conclusion, but rather to make them more likely to be true.  We judge how good such an argument is according to how strong it is.  Unlike validity, strength is a bit more wishy-washy, it comes in degrees, which is why I use the terms `probable' and `improbable' in the below definitions:

\begin{tabular}{p{1in}|p{4.5in}}
Inductive&Definition\\\hline
I &An inductively strong argument is an argument in which if its premises are true, its conclusion is probably true.\\
I* &An inductively strong argument is an argument in which it is improbable that its conclusion is false given that its premises are true.
\end{tabular}

As in the case of validity, when we say that an argument is strong, we are only claiming that if the premises are true then the conclusion is likely to be true. Corresponding to the notion of deductive soundness, an inductive argument that is both strong and has true premises is called a cogent inductive argument. Unlike the case with deductively sound arguments, it is possible for an inductively cogent argument to have true premises and a false conclusion. When we are asking about the validity of an argument, we are asking whether it's possible for the premises to be true and the conclusion false. When we are asking about the strength of an argument, we are asking about the probability of the conclusion being false if we assume that the premises are true.  Possibility does not come in degrees, it's either possible or impossible. Probability does come in degrees. In the simplest case, inductive reasoning involves inferring that something is generally the case from a pattern observed in a limited number of cases.\footnote{One way to think about this is that deduction goes from general to particular and induction goes from particular to general.}

    Suppose we conducted a poll of 1000 Seattle voters. The results showed that 600 of them claimed to be Democrats. We could inductively infer that 60\% of the voters in Seattle are Democrats. The results of the poll give a pretty good reason to think that around 60\% of the voters in Seattle are Democrats. But the results of the poll don’t guarantee this conclusion. It is possible that only 50\% of the voters in Seattle are Democrats and Democrats were, just by luck, over represented in the 1000 cases we considered, but it may not be probable.

There are a few factors which tell us how strong an inductive argument is. One is how much evidence we have looked at before inductively generalizing. Our inductive argument above would be stronger is we drew our conclusion from a poll of 100,000 Seattle voters, for instance. And it would be much weaker if we had only polled 100. Similarly, if we were trying to figure out the political stances of all of Washington but solely looked at Enumclaw, we would get a radically different view about our standings than if we had looked at Seattle and Enumclaw.

Also, the strength of an inductive argument depends on how much of the evidence represents the amount in reality. So our inductive argument will be stronger if we randomly select our 1,000 voters from the Seattle phone book than if they are selected from the Ballard phone book (Ballard being a notably liberal neighborhood within Seattle).

So far, we’ve only discussed inductive generalization, where we identify a pattern in a limited number of cases and draw a more general conclusion about a broader class of cases. Inductive argument comes in other varieties as well. In the example we started with about Sam the line cook, we inductively inferred a prediction about Sam based on a known pattern in a broader class of cases. Argument from analogy is another variety of inductive reasoning that can be quite strong. The strength here is in how many commonalities the two cases have, we will see an argument like this in the section concerning the existence of God.

\section{Part 1.3: Common Argument Structures In Philosophy}
\label{s:p1.3}

Since Philosophy relies on arguments to get at the facts about the world, like science, but we go with the gold-standard for arguments, validity, often arguments will come in certain forms, structures. These forms can be just on their own, or in combinations with others. If you structure your arguments in these ways or you make arguments out of combinations of them, you are guaranteed to have a valid argument, soundness is a different story. 

\subsection{Part 1.3.1: Modus Ponens}
\label{s:p1.3.1}

This is another easy and intuitive argument structure. It follows a very similar model to cause-and-effect. If I knock over the glass, it will break; I knocked over the glass; so it broke. Here, I have an if-then statement, which can be phrased in several different ways, I affirm the antecedent (the if-part), and thereby get the consequent (the then-part). But, that doesn't mean it's fool-proof. For example, there's a very common logical fallacy called ``affirming the consequent", this is where you have an if-then statement, you affirm the then-part, and thereby think you get the antecedent (the if-part). This is not valid, it's not a good structure. For example, take this quick argument (which is Modus Ponens done properly):

\factoidbox{If my car won't start, then there's something wrong with the battery terminals. My car won't start. So, there must be something wrong with the battery terminals.}

Using cause-and-effect again, there can be many different possible causes for some event. One of them definitely caused the event, but you can't say with certainty which one caused it just because it happened. For example, if an apple is too heavy, it will fall off a tree. Yes, when an apple is too heavy it will fall from a tree, but loads of other things can cause it to fall, so you can't say that it's weight caused it to fall, it could have been the wind.

\subsection{Part 1.3.2: Modus Tollens}
\label{s:p1.3.2}

Modus Tollens is like Modus Ponens in reverse and negated. In this case, you have an if-then statement, you have that the then-part is false, and so you get that the if-part must be false. This makes sense in the case of cause-and-effect too. You know that one thing would cause another and you know that that other thing didn't happen, so you know that the first thing didn't happen. For example, if I sleep in, I will be late, I wasn't late, so I must not have slept in. 

Again, this is a pretty simple idea, but it's easily misused. The fallacy here is called `denying the antecedent'. This, again, is not good reasoning; you have an if-then statement, you have that the if-part is false, so you think that the then-part must be false. But, just because something wasn't caused by one thing, it doesn't mean that it wasn't caused by another. For example, if God exists, then humans exist; God doesn't exist; therefore, man doesn't exist. Poof, all atheists  just disappeared, right? No, they didn't. This is not a valid argument. 

Modus Tollens, properly, we will mostly see in this class, as it's the easiest for the sort of teaching style which I employ. For example, take the following:

\factoidbox{Ethical Egoism is the stance that the morality of an action is determined by how much it benefits the doer personally. By this theory, an action is moral when it benefits me and immoral when it does not. So, if Ethical Egoism is correct, then donating to charity, which will in no way benefit me, is always morally wrong. But, it seems obvious that the most moral actions we can do are selfless (in this case, donating when it won't benefit me). So, Ethical Egoism is incorrect.}
    
\subsection{Part 1.3.3: Disjunctive Syllogism}
\label{s:p1.3.3}

This is by far the most simple argument structure used in Logic and it's probably the hardest to misuse (think that you have a valid argument, but don't). This does not mean that I have seen bad formulations of this which I will show, but first, here is a story:
\factoidbox{Years ago, while I was living in Arizona, I had some car trouble. Periodically, my car just would not start, or it spontaneously let me start it after a few attempts, seemingly randomly. I eventually called my local mechanic and had them take it to a shop. The mechanic, maybe knowing I was a philosophy professor, said the following (and this is a real quote): ``Either it's your ECM or it's your fuse-box. We tested and it's not your fuse-box. So, the issue must be your ECM."}

Basically, a disjunctive syllogism takes 2 possible options, has that one of them is false, and thereby gets the other. This can be used in various different ways, for example, you could have 3 possible options, show that one of them is false and thereby get that it must be one of the other 2. This is a sort of process of elimination sort of argument structure. This is a very simple, easy, argument structure, but it's still possible to structure it poorly. For example, I have seen people take two options, show that one of them is true and thereby claim that the other is false. This is not a good structure. Disjunctions, in English, without context, don't allow for this sort of move. It is possible for both options to be true. There are some cases where this works, but to play it safe, and stay on the windy side of validity, assume that both can be correct. 

\subsection{Part 1.3.4: Hypothetical Syllogism}
\label{s:p1.3.4}

This is the next, basic, argument structure which I will cover with you here. This argument structure does not require you to have anything other than conditionals, you don't need any proven facts. But, at the end, you don't get any proven facts out either, all you get are conditionals. We often use this sort of reasoning when we are thinking about what will happen next if we go through some possible scenario. For example, take the following:

\factoidbox{If you give a mouse a cookie, he will ask for a glass of milk. If he asks for a glass of milk, then he will ask for a straw. Therefore, if you give a mouse a cookie, then he will ask for a straw.} 

When the antecedent of one conditional is the same as the consequent of another, you can collapse them into one conditional by taking out the middle man. This is a driving force for the slippery slope fallacy, but in that case, the fallacy is not in the reasoning itself, but rather in the truth of the conditionals which it employs. Often in arguments, this is used to shorten the work of Modus Ponens or Modus Tollens (below), but it can be used all on its own to get a point across. 

\subsection{Combining the Structures}

Arguments aren't always merely 3 lines long, sometimes they are far longer and more involved. For example, take the following perfectly valid argument:

\factoidbox{If I eat a ton of food this Thanksgiving, then I will get yelled at by my doctor. If I get yelled at by my doctor, then I will need to work out. I did eat a ton of food this Thanksgiving. Therefore, I will need to work out.}

In this case, I am using several different structures together to get my conclusion. I am using Hypothetical Syllogism and Modus Ponens all at once to get my answer. Similarly, I can have an argument like this:

\factoidbox{If I need to get a math credit, then I will either need to take PHIL\&120 or a college level math course. I need to get a math credit and I really don't want to take a college level math course, so I will need to take PHIL\&120.} 

Here, I am using both Modus Ponens and Disjunctive Syllogism to get the conclusion. And yes, PHIL\&120 is a math credit, but it can be more difficult than other options. All of these examples come from PHIL\&120. 

\section{Part 1.4: Fallacies and Biases}
\label{s:p1.4}

A fallacy is just a mistake in reasoning. Humans are not nearly as rational as we’d like to suppose. In fact we are so prone to certain sorts of mistakes in reasoning that philosophers and logicians refer those mistakes by name. For now I will discuss just one by name but in a little detail. Watch for explanations of other fallacies over the course of the class. For pretty thorough catalogue of logical fallacies, I’ll refer to you The Fallacy Files.\autocite{Curtis1} There is also a section in the appendix of this textbook which lists and explains various informal fallacies. Formal fallacies are cases where the structure of the argument seems fine, but it actually relies on an improper move (outlined when I covered the common structures). Informal fallacies are when improper or incorrect reasons are used in the argument, which are outlined in the module I linked to. Here are some examples of fallacies which are rather egregious and should never be seen in any work, in philosophy or elsewhere: 
\subsection{Ad Hominem Fallacy}

“Ad hominem” is Latin for “against the man.” It is the name for the fallacy of attacking the proponent of a position rather than critically evaluating the reasons offered for the proponent’s position. The reason ad hominem is a fallacy is just that the attack on an individual is simply not relevant to the quality of the reasoning offered by that person. Attacking the person who offers an argument has nothing to do whether or not the premises of the argument are true or support the conclusion. Ad hominem is a particularly rampant and destructive fallacy in our society. What makes it so destructive is that it turns the cooperative social project of inquiry through conversation into polarized verbal combat. This fallacy makes rational communication impossible while it diverts attention from interesting issues that often could be fruitfully investigated.

\factoidbox{Here is a classic example of ad hominem: A car salesman argues for the quality of an automobile and the potential buyer discounts the argument with the thought that the person is just trying to earn a commission. There may be good reason to think the salesman is just trying to earn a commission. But even if there is, this is irrelevant to the evaluation of the reasons the salesman is offering. The reasons should be evaluated on their own merits.}

Notice, it is easy to describe a situation where it is both true that the salesman is just trying to earn a commission and true that he is making good arguments. Consider a salesman who is not too fond of people and cares little for them except that they earn a commission for him. Otherwise he is scrupulously honest and a person of moral integrity. In order to reconcile himself with the duties of a sales job, he carefully researches his product and only accepts a sales position with the business that sells the very best. He then sincerely delivers good arguments for the quality of his product, makes lots of money, and dresses well. This salesman must have been a philosophy major. The customer who rejects his argument on the ad hominem grounds that he is just trying to earn a commission misses an opportunity to buy the best. The moral of the story is just that the salesperson’s motive is logically independent of the quality of his argument.
\subsection{Strawman Fallacy}

The strawman fallacy is one which I encounter occasionally in student works but you really will see it commonly in the political sphere (much to our misfortune, regardless of the side you hold). A strawman is like a scarecrow, a mock façade of a person used as a distraction or a trick. The strawman in the fallacy is not a façade of the person making the argument or claim, per se, but rather it is a façade of the claim or argument. Often, when we are explaining a view or defending our own views, we will need to explain the opposing side and give their arguments for their stance. This is so that we will be able to explain why our own is more reasonable or better. Putting your opponent's stance in your own words is not strawmanning, in fact I encourage you to (because then you will need to put yourself in their shoes and potentially find flaws in your own stance). This becomes strawmanning, however, when the argument or stance which you are attributing to your opponent is not their argument or stance. You have left out key details, misrepresented their findings or claims, or imposed your own stance (which they reject) onto them in order to make them seem absurd. Strawmanning another person's stance doesn't only harm your quest for the right answer to a problem (because you have removed the view of a person (who I am assuming is reasonable) from the discussion) but it also harms others who could take what you say seriously. In the case of politics, if the party or group you support presents the opposing side as having absurd or outlandish views which could never work or have some fundamental flaw which anyone could see, often, rather than thinking that they must have missed something, you will take it as gospel and move on to spread this misinformation. The vast majority of the time, however, the absurd or outlandish view was actually a strawman of the original. The original, if the people behind it are reasonable, would lack the fundamental flaw or have some explanation about how the flaw was handled in connection with other policies. 

The way to avoid this fallacy is to paint all arguments for any claim in the best possible light. If there is a subtle flaw in the argument which could easily be overlooked, make the patch yourself (even and especially if you disagree with the conclusion to it). For example, in this class, when we talk about the first cause argument for the existence of God, I present the argument without a `fatal' flaw which is seen in the ordinary presentation of it because it was easy to see and fix; I did not want to present you with a strawman.  If a stance seems absurd, do some digging into the actual stance and see whether the person presenting it to you missed or intentionally left out something. For example, suppose that a study showed that some program, which would remove a service from the private sector and made it public (have the government provide the service rather than a collection of private companies), would cost the tax payers 2.6 trillion USD. They present this as an absurd amount of money and then show how much your taxes would increase by. What they fail to mention, however, is that the current cost for that program on tax payers is closer 3.6 trillion USD, but rather than this money going to the government, it goes to the private companies. Others may strawman the cost by inflating it by ignoring the basic fact that many middlemen would not exist in the proposed system and the lack of them would reduce the cost. Doing a little deeper digging will save you from this fallacy. 

\section{Part 1.5: Critical Thinking}

This page roughly finishes up our crash course on logic and critical thinking, but we will be seeing more logical forms and critical thinking structures as the course progresses. Most of it was dedicated to logic, and if you are more interested in that, I strongly recommend that you take PHIL\&120. But, for good critical thinking, there are some further standards which are worth noting. These standards follow from the others which we have discussed, but are a wee-bit more practical. These standards are Consistency and Coherency, but there are other aspects which come in when philosophy goes empirical.

\subsection{Consistency}

The first, basic, bare-bones standard for good critical thinking, and thereby good philosophy, is that the thinking must be consistent. This is more basic than saying that it's logical or that it lines-up with reality. For some train of thought to be consistent, it needs to lack two different things. First, it can't have any contradictions. A contradiction is a case where some proposition must be both true and false (at the same time) in order for your reasoning to work. In everyday life, it is (hopefully) very rare for us to encounter a contradiction either in our own or someone else's thinking, but there are cases where they show up, such as in conspiracy theories, and we only notice them well after the stance was explained. Take, for example, a conspiracy theory which has that both the Earth is the center of the universe and that gravity is just the Earth moving upwards at an ever increasing rate. This might not seem too contradictory on the face of it, but once you start digging into what must be true for both of those statements to be true, you will find that some proposition must be both true and false at the same time. There are other more blunt examples, such as:

\begin{tabular}{p{2.5in}|p{2.5in}}
Contradictory &Non-Contradictory\\\hline
The non-existent ghosts stole the painting &The ghosts stole the painting\\\hline
Bobby Joe committed the crime in New York (in person) while in LA. &Bobby Joe made it seem like he was in LA in order to cover up the crime.\\\hline
No one drives in New York because there is too much traffic.\footnote{This is from the Futurama episode ``The Lesser of Two Evils", season 2, episode 11.} &The traffic in New York is so bad that most people walk.\\\hline
All animals are equal, but some are more equal than others.\footnote{This is from George Orwell's Animal Farm.} &Either all animals are equal or they aren't all equal.\\\hline
I make my own choices, with my wife's permission. &What I ultimately choose should be a joint decision between my wife and I.\\\hline
\end{tabular}

The Non-Contradictory column is there to show you how, most of the time, there is an easy way to resolve the contradiction, but there are other cases, which we will see in this class, where the contradiction is more deep seeded and can't be so easily resolved (like the contradictions in Moral Relativism). In order to avoid these, make sure that all of the propositions you are using are consistent, that they are always true or false (depending) throughout your thinking.

The second requirement for consistent thinking is that there aren't any equivocations. This is where you use the same statement or word/phrase in two or more parts of your reasoning and your reasoning relies on them meaning the same thing, but they don't mean the same thing in the two different instances. For example, take a look at these arguments:
\begin{earg}
   \item[] Nothing is better than God.
    \item[]A cheese sandwich is better than nothing.
    \item[]Therefore, a cheese sandwich is better than God.
\end{earg}
In this argument, the term `nothing' is being used in two different ways. In order for us to accept the two lines of this argument, individually, we have different meanings in mind. In the first case, `nothing' means that something along the lines of `it is not the case that there is exists a being such that that being..." or ``no being". So, the proper way to understand this first line is ``it is not the case that there exists a being such that that being is better than God." The second line of the argument uses a different sense of the word `nothing', namely ``not having anything", so the correct way to understand this is that it means ``a cheese sandwich is better than not having anything." Taking these two together, the argument looks like this:
\begin{earg}
    \item[]There isn't anything better than God.
    \item[]Having a cheese sandwich is better than not having anything.
    \item[]Therefore, ...
\end{earg}
As you should be able to see, the flow of the argument isn't there any more, the equivocation was the glue holding it together. Some other examples can be in quick reasoning like:
\begin{earg}
    \item[]André the Giant is so called because of his great size.
    \item[]André René Roussimoff is so called because that's what his mother named him.
    \item[]André the Giant is André René Roussimoff.
    \item[]Therefore, André René Roussimoff is so called because of his great size.
\end{earg}
That last line should seem off to you. This is because there is equivocation in this reasoning. In the first two lines we are talking about the names, but in the third we are talking about what the names pick out in the world. The equivocation is in the shift between talking about the word and the person.

\subsection{Coherency}

The next basic requirement for good critical thinking is that it needs to be coherent. Coherency is not the same as consistency, though there is a fair bit of overlap. Each premise, each proposition used in the reasoning need to relate to each other in a reasonable way. There can't be any strange or unorthodox jumps in the argument. You can think of this like a spider web, each of the strands is connected together and it would not be as strong if some of those strands were removed. For example, take this argument:
\begin{earg}
    \item[]If the moon landing was fake, then the Government did so to deceive us.
    \item[]If the Government did so to deceive us, then it was to make us lose faith in our religion.
    \item[]The moon landing was fake.
    \item[]Therefore, the Government faked it in order to make us lose faith in our religion.
\end{earg}
So, I am not going to make an argument that the moon landing wasn't fake (we really did go to the moon), rather, assuming that it was fake, it does make sense that it would have been faked to deceive us in some way. The real issue with this argument is in the second line. Does it really make sense that the Government would deceive us to make us lose faith in this way? There is a jump in the reasoning. Much more evidence and premises are required to show a connection between the assumption that the Government deceived us with the moon landing and that the purpose of it had something to do with religion. There are many other closer, more relevant potential reasons which need to be discounted first. For example, that the Government did so in order to reaffirm a sense of exceptionalism.

\subsection{Empirical Thinking}

There was a time in which most philosophic thought was empirical, but as the sciences diversified, that became less common (though the trend is swinging back that way in philosophy). Empirical thinking is reasoning which relates to or explains the outside world. In the other sciences, you may have heard a distinction between empirical (applied) and theoretical. Theoretical science is the arm-chair, hypothetical models which are thought about and debated and later tested using empirical methods (like experiments). Empirical science is the science which actually does the tests and uses the materials in question. There are two features to reasonable empirical thinking, which the other sciences should take note of and explain from the beginning. These are that the thinking must be adequate and applicable.

Empirical thinking is adequate when all of the cases you are trying to explain are accounted for. There shouldn't be too many exceptions to the account you are giving and those exceptions should be easily accounted for. For example, although this story is apocryphal, when Galileo presented his findings about there being objects which orbit something other than the Earth, one person claimed that it was because of interference in the telescope. So, they tried it in a different location, and they got the same result. Eventually, the objector said something like ``my hypothesis works so long as you don't look through a telescope". The hypothesis that all objects orbit the Earth wasn't adequate because there were cases which it could not explain and those exceptions could not be easily accounted for. 

Empirical thinking is applicable when there isn't anything in the explanation which doesn't relate back to experience and evidence or data gained from testing in the relevant environment and the explanation/hypothesis is useful (as in, it leads or gives way to further understanding and can be used to make other explanations). Many examples of empirical thoughts failing in this regard can be found when you look at the claims and tests involving pyramid power, magic crystals, feng shui, various vitamins and supplements seen on Dr. Oz, and even copper/magnetic bracelets claimed to treat arthritis. The explanations for each of these contain claims which either can't be demonstrated with experiments or don't relate to experience. For a more scientific example, take this quote from the famous physicist Richard Feynman:
\factoidbox{
    The next question was — what makes planets go around the sun? At the time of Kepler some people answered this problem by saying that there were angels behind them beating their wings and pushing the planets around an orbit.\autocite{physicallaw}}

 The hypothesis that the angels are making the planets move isn't applicable because it doesn't relate back to experience and evidence. If, in some strange world, we could see the angels sweating and pushing the planets really hard, then it would be. However, we can't see the angels and if someone were to claim that they are there, they are just invisible, the hypothesis would be even more inapplicable.

\chapter{Plato's Apology}
I do not know, men of Athens, how my\marginpar{17}
accusers affected you; as for me, I was almost
carried away in spite of myself, so persuasively
did they speak. And yet, hardly anything of
what they said is true. Of the many lies they
told, one in particular surprised me, namely
that you should be careful not to be deceived
by an accomplished speaker like me. That they
were not ashamed to be immediately proved\marginpar{b}
wrong by the facts, when I show myself not to
be an accomplished speaker at all, that I
thought was most shameless on their
part—unless indeed they call an accomplished
speaker the man who speaks the truth. If they
mean that, I would agree that I am an orator,
but not after their manner, for indeed, as I say, practically nothing they said was true. From me you\marginpar{c}
will hear the whole truth, though not, by Zeus, gentlemen, expressed in embroidered and stylized
phrases like theirs, but things spoken at random and expressed in the first words that come to mind,
for I put my trust in the justice of what I say, and let none of you expect anything else. It would
not be fitting at my age, as it might be for a young man, to toy with words when I appear before
you.

One thing I do ask and beg of you, gentlemen: if you hear me making my defence in the same
kind of language as I am accustomed to use in the market place by the bankers' tables, where many
of you have heard me, and elsewhere, do not be surprised or create a disturbance on that account.
The position is this: this is my first appearance in a lawcourt, at the age of seventy; I am therefore\marginpar{d}
simply a stranger to the manner of speaking here. Just as if I were really a stranger, you would
certainly excuse me if I spoke in that dialect and manner in which I had been brought up, so too
my present request seems a just one, for you to pay no attention to my manner of speech—be it\marginpar{18}
better or worse—but to concentrate your attention on whether what I say is just or not, for the
excellence of a judge lies in this, as that of a speaker lies in telling the truth.

It is right for me, gentlemen, to defend myself first against the first lying accusations made
against me and my first accusers, and then against the later accusations and the later accusers.
There have been many who have accused me to you for many years now, and none of their\marginpar{b}
accusations are true. These I fear much more than I fear Anytus and his friends, though they too
are formidable. These earlier ones, however, are more so, gentlemen; they got hold of most of you
from childhood, persuaded you and accused me quite falsely, saying that there is a man called
Socrates, a wise man, a student of all things in the sky and below the earth, who makes the worse
argument the stronger. Those who spread that rumour, gentlemen, are my dangerous accusers, for\marginpar{c}
their hearers believe that those who study these things do not even believe in the gods. Moreover,
these accusers are numerous, and have been at it a long time; also, they spoke to you at an age
when you would most readily believe them, some of you being children and adolescents, and they
won their case by default, as there was no defence.

What is most absurd in all this is that one cannot even know or mention their names unless one
of them is a writer of comedies. Those who maliciously and slanderously persuaded you—who\marginpar{d}
also, when persuaded themselves then persuaded others—all those are most difficult to deal with:
one cannot bring one of them into court or refute him; one must simply fight with shadows, as it
were, in making one's defence, and cross-examine when no one answers. I want you to realize too
that my accusers are of two kinds: those who have accused me recently, and the old ones I
mention; and to think that I must first defend myself against the latter, for you have also heard their
accusations first, and to a much greater extent than the more recent.\marginpar{e}

Very well then. I must surely defend myself and attempt to uproot from your minds in so short
a time the slander that has resided there so long. I wish this may happen, if it is in any way better\marginpar{19}
for you and me, and that my defence may be successful, but I think this is very difficult and I am
fully aware of how difficult it is. Even so, let the matter proceed as the god may wish, but I must
obey the law and make my defence.

Let us then take up the case from its beginning. What is the accusation from which arose the
slander in which Meletus trusted when he wrote out the charge against me? What did they say\marginpar{b}
when they slandered me? I must, as if they were my actual prosecutors, read the affidavit they
would have sworn. It goes something like this: Socrates is guilty of wrongdoing in that he busies
himself studying things in the sky and below the earth; he makes the worse into the stronger
argument, and he teaches these same things to others. You have seen this yourselves in the comedy
of Aristophanes, a Socrates swinging about there, saying he was walking on air and talking a lot\marginpar{c}
of other nonsense about things of which I know nothing at all. I do not speak in contempt of such
knowledge, if someone is wise in these things—lest Meletus bring more cases against me—but,
gentlemen, I have no part in it, and on this point I call upon the majority of you as witnesses. I
think it right that all those of you who have heard me conversing, and many of you have, should
tell each other if anyone of you has ever heard me discussing such subjects to any extent at all.\marginpar{d}
From this you will learn that the other things said about me by the majority are of the same kind.

Not one of them is true. And if you have heard from anyone that I undertake to teach people\marginpar{e}
and charge a fee for it, that is not true either. Yet I think it a fine thing to be able to teach people
as Gorgias of Leontini does, and Prodicus of Ceos, and Hippias of Elis. Each of these men can1
go to any city and persuade the young, who can keep company with anyone of their own fellow-
citizens they want without paying, to leave the company of these, to join with themselves, pay\marginpar{20}
them a fee, and be grateful to them besides. Indeed, I learned that there is another wise man from
Paros who is visiting us, for I met a man who has spent more money on Sophists than everybody
else put together, Callias, the son of Hipponicus. So I asked him—he has two sons—"Callias," I
said, "if your sons were colts or calves, we could find and engage a supervisor for them who would
make them excel in their proper qualities, some horse breeder or farmer. Now since they are men,\marginpar{b}
whom do you have in mind to supervise them? Who is an expert in this kind of excellence, the
human and social kind? I think you must have given thought to this since you have sons. Is there
such a person," I asked, "or is there not?" "Certainly there is," he said. "Who is he?" I asked,
"What is his name, where is he from? and what is his fee?" "His name, Socrates, is Evenus, he
comes from Paras, and his fee is five minas." I thought Evenus a happy man, if he really possesses\marginpar{c}
this art, and teaches for so moderate a fee. Certainly I would pride and preen myself if I had this
knowledge, but I do not have it, gentlemen.

One of you might perhaps interrupt me and say: "But Socrates, what is your occupation? From
where have these slanders come? For surely if you did not busy yourself with something out of the
common, all these rumours and talk would not have arisen unless you did something other than
most people. Tell us what it is, that we may not speak inadvisedly about you." Anyone who says\marginpar{d}
that seems to be right, and I will try to show you what has caused this reputation and slander.
Listen then. Perhaps some of you will think I am jesting, but be sure that all that I shall say is true.
What has caused my reputation is none other than a certain kind of wisdom. What kind of wisdom?
Human wisdom, perhaps. It may be that I really possess this, while those whom I mentioned just
now are wise with a wisdom more than human; else I cannot explain it, for I certainly do not\marginpar{e}
possess it, and whoever says I do is lying and speaks to slander me. Do not create a disturbance,
gentlemen, even if you think I am boasting, for the story I shall tell does not originate with me, but
I will refer you to a trustworthy source. I shall call upon the god at Delphi as witness to the
existence and nature of my wisdom, if it be such. You know Chairephon. He was my friend from
youth, and the friend of most of you, as he shared your exile and your return. You surely know the\marginpar{21}
kind of man he was, how impulsive in any course of action. He went to Delphi at one time and
ventured to ask the oracle—as I say, gentlemen, do not create a disturbance—he asked if any man
was wiser than I, and the Pythian replied that no one was wiser. Chairephon is dead, but his
brother will testify to you about this.

Consider that I tell you this because I would inform you about the origin of the slander. When
I heard of this reply I asked myself: "Whatever does the god mean? What is his riddle? I am very\marginpar{b}
conscious that I am not wise at all; what then does he mean by saying that I am the wisest? For
surely he does not lie; it is not legitimate for him to do so." For a long time I was at a loss as to his
meaning; then I very reluctantly turned to some such investigation as this: I went to one of those
reputed wise, thinking that there, if anywhere, I could refute the oracle and say to it: "This man
is wiser than I, but you said I was." Then, when I examined this man—there is no need for me to
tell you his name, he was one of our public men—my experience was something like this: I\marginpar{c}
thought that he appeared wise to many people and especially to himself, but he was not. I then
tried to show him that he thought himself wise, but that he was not. As a result he came to dislike
me, and so did many of the bystanders. So I withdrew and thought to myself: "I am wiser than this\marginpar{d}
man; it is likely that neither of us knows anything worthwhile, but he thinks he knows something
when he does not, whereas when I do not know, neither do I think I know; so I am likely to be
wiser than he to this small extent, that I do not think I know what I do not know." After this I
approached another man, one of those thought to be wiser than he, and I thought the same thing,
and so I came to be disliked both by him and by many others.\marginpar{e}

After that I proceeded systematically. I realized, to my sorrow and alarm, that I was getting
unpopular, but I thought that I must attach the greatest importance to the god's oracle, so I must
go to all those who had any reputation for knowledge to examine its meaning. And by the dog,
gentlemen of the jury—for I must tell you the truth—I experienced something like this: in my
investigation in the service of the god I found that those who had the highest reputation were\marginpar{22}
nearly the most deficient, while those who were thought to be inferior were more knowledgeable.
I must give you an account of my journeyings as if they were labours I had undertaken to prove
the oracle irrefutable. After the politicians, I went to the poets, the writers of tragedies and
dithyrambs and the others, intending in their case to catch myself being more ignorant then they.
So I took up those poems with which they seemed to have taken most trouble and asked them what
they meant, in order that I might at the same time learn something from them. I am ashamed to tell\marginpar{b}
you the truth, gentlemen, but I must. Almost all the bystanders might have explained the poems
better than their authors could. I soon realized that poets do not compose their poems with
knowledge, but by some inborn talent and by inspiration, like seers and prophets who also say
many fine things without any understanding of what they say. The poets seemed to me to have had\marginpar{c}
a similar experience. At the same time I saw that, because of their poetry, they thought themselves
very wise men in other respects, which they were not. So there again I withdrew, thinking that I
had the same advantage over them as I had over the politicians.

Finally I went to the craftsmen, for I was conscious of knowing practically nothing, and I knew\marginpar{d}
that I would find that they had knowledge of many fine things. In this I was not mistaken; they
knew things I did not know, and to that extent they were wiser than I. But, gentlemen of the jury,
the good craftsmen seemed to me to have the same fault as the poets: each of them, because of his
success at his craft, thought himself very wise in other most important pursuits, and this error of
theirs overshadowed the wisdom they had, so that I asked myself, on behalf of the oracle, whether
I should prefer to be as I am, with neither their wisdom nor their ignorance, or to have both. The\marginpar{e}
answer I gave myself and the oracle was that it was to my advantage to be as I am.
As a result of this investigation, gentlemen of the jury, I acquired much unpopularity, of a kind that
is hard to deal with and is a heavy burden; many slanders came from these people and a reputation\marginpar{23}
for wisdom, for in each case the bystanders thought that I myself possessed the wisdom that I
proved that my interlocutor did not have. What is probable, gentlemen, is that in fact the god is
wise and that his oracular response meant that human wisdom is worth little or nothing, and that
when he says this man, Socrates, he is using my name as an example, as if he said: "This man
among you, mortals, is wisest who, like Socrates, understands that his wisdom is worthless." So\marginpar{b}
even now I continue this investigation as the god bade me—and I go around seeking out anyone,
citizen or stranger, whom I think wise. Then if I do not think he is, I come to the assistance of the
god and show him that he is not wise. Because of this occupation, I do not have the leisure to
engage in public affairs to any extent, nor indeed to look after my own, but I live in great poverty
because of my service to the god.

Furthermore, the young men who follow me around of their own free will, those who have
most leisure, the sons of the very rich, take pleasure in hearing people questioned; they themselves\marginpar{c}
often imitate me and try to question others. I think they find an abundance of men who believe they
have some knowledge but know little or nothing. The result is that those whom they question are
angry, not with themselves but with me. They say: "That man Socrates is a pestilential fellow who
corrupts the young." If one asks them what he does and what he teaches to corrupt them, they are\marginpar{d}
silent, as they do not know, but, so as not to appear at a loss, they mention those accusations that
are available against all philosophers, about "things in the sky and things below the earth," about
"not believing in the gods" and "making the worse the stronger argument;" they would not want
to tell the truth, I'm sure, that they have been proved to lay claim to knowledge when they know
nothing. These people are ambitious, violent and numerous; they are continually and convincingly
talking about me; they have been filling your ears for a long time with vehement slanders against
me. From them Meletus attacked me, and Anytus and Lycon, Meletus being vexed on behalf of\marginpar{e}
the poets, Anytus on behalf of the craftsmen and the politicians, Lycon on behalf of the orators,
so that, as I started out by saying, I should be surprised if I could rid you of so much slander in so
short a time. That, gentlemen of the jury, is the truth for you. I have hidden or disguised nothing.\marginpar{24}
I know well enough that this very conduct makes me unpopular, and this is proof that what I say
is true, that such is the slander against me, and that such are its causes. If you look into this either
now or later, this is what you will find.\marginpar{b}

Let this suffice as a defence against the charges of my earlier accusers. After this I shall try
to defend myself against Meletus, that good and patriotic man, as he says he is, and my later
accusers. As these are a different lot of accusers, let us again take up their sworn deposition. It
goes something like this: Socrates is guilty of corrupting the young and of not believing in the gods
in whom the city believes, but in other new spiritual things. Such is their charge. Let us examine
it point by point.\marginpar{c}

He says that I am guilty of corrupting the young, but I say that Meletus is guilty of dealing
frivolously with serious matters, of irresponsibly bringing people into court, and of professing to
be seriously concerned with things about none of which he has ever cared, and I shall try to prove
that this is so. Come here and tell me, Meletus. Surely you consider it of the greatest importance\marginpar{d}
that our young men be as good as possible? —Indeed I do.

Come then, tell the jury who improves them. You obviously know, in view of your concern.
You say you have discovered the one who corrupts them, namely me, and you bring me here and
accuse me to the jury. Come, inform the jury and tell them who it is. You see, Meletus, that you
are silent and know not what to say. Does this not seem shameful to you and a sufficient proof of
what I say, that you have not been concerned with any of this? Tell me, my good sir, who improves
our young men? —The laws.\marginpar{e}

That is not what I am asking, but what person who has knowledge of the laws to begin
with?—These jurymen, Socrates.

How do you mean, Meletus? Are these able to educate the young and improve
them?—Certainly.

All of them, or some but not others?—All of them.

Very good, by Hera. You mention a great abundance of benefactors. But what about the
audience? Do they improve the young or not?—They do, too.\marginpar{25}

What about the members of Council?—The Councillors, also.
But, Meletus, what about the assembly? Do members of the assembly corrupt the young, or
do they all improve them?—They improve them.

All the Athenians, it seems, make the young into fine good men, except me, and I alone corrupt
them. Is that what you mean?—That is most definitely what I mean.

You condemn me to a great misfortune. Tell me: does this also apply to horses do you think?\marginpar{b}
That all men improve them and one individual corrupts them? Or is quite the contrary true, one
individual is able to improve them, or very few, namely the horse breeders, whereas the majority,
if they have horses and use them, corrupt them? Is that not the case, Meletus, both with horses and
all other animals? Of course it is, whether you and Anytus say so or not. It would be a very happy
state of affairs if only one person corrupted our youth, while the others improved them.

You have made it sufficiently obvious, Meletus, that you have never had any concern for our\marginpar{c}
youth; you show your indifference clearly; that you have given no thought to the subjects about
which you bring me to trial.

And by Zeus, Meletus, tell us also whether it is better for a man to live among good or wicked
fellow-citizens. Answer, my good man, for I am not asking a difficult question. Do not the wicked
do some harm to those who are ever closest to them, whereas good people benefit
them?—Certainly.

And does the man exist who would rather be harmed than benefited by his associates? Answer,\marginpar{d}
my good sir, for the law orders you to answer. Is there any man who wants to be harmed? —Of
course not.

Come now, do you accuse me here of corrupting the young and making them worse
deliberately or unwillingly?—Deliberately.

What follows, Meletus? Are you so much wiser at your age than I am at mine that you
understand that wicked people always do some harm to their closest neighbors while good people\marginpar{e}
do them good, but I have reached such a pitch of ignorance that I do not realize this, namely that
if I make one of my associates wicked I run the risk of being harmed by him so that I do such a
great evil deliberately, as you say? I do not believe you, Meletus, and I do not think anyone else\marginpar{26}
will. Either I do not corrupt the young or, if I do, it is unwillingly, and you are lying in either case.
Now if I corrupt them unwillingly, the law does not require you to bring people to court for such
unwilling wrongdoings, but to get hold of them privately, to instruct them and exhort them; for
clearly, if I learn better, I shall cease to do what I am doing unwillingly. You, however, have
avoided my company and were unwilling to instruct me, but you bring me here, where the law
requires one to bring those who are in need of punishment, not of instruction.

And so, gentlemen of the jury, what I said is clearly true: Meletus has never been at all\marginpar{b}
concerned with these matters. Nonetheless tell us, Meletus, how you say that I corrupt the young;
or is it obvious from your deposition that it is by teaching them not to believe in the gods in whom
the city believes but in other new spiritual things? Is this not what you say I teach and so corrupt
them? —That is most certainly what I do say.

Then by those very gods about whom we are talking, Meletus, make this clearer to me and to
the jury: I cannot be sure whether you mean that I teach the belief that there are some gods—and
therefore I myself believe that there are gods and am not altogether an atheist, nor am I guilty of\marginpar{c}
that—not, however, the gods in whom the city believes, but others, and that this is the charge
against me, that they are others. Or whether you mean that I do not believe in gods at all, and that
this is what I teach to others. —This is what I mean, that you do not believe in gods at all.

You are a strange fellow, Meletus. Why do you say this? Do I not believe, as other men do,
that the sun and the moon are gods?—No, by Zeus, jurymen, for he says that the sun is stone, and\marginpar{d}
the moon earth.

My dear Meletus, do you think you are prosecuting Anaxagoras? Are you so contemptuous
of the jury and think them so ignorant of letters as not to know that the books of Anaxagoras of
Clazomenae are full of those theories, and further, that the young men learn from me what they
can buy from time to time for a drachma, at most, in the bookshops, and ridicule Socrates if he
pretends that these theories are his own, especially as they are so absurd? Is that, by Zeus, what\marginpar{e}
you think of me, Meletus, that I do not believe that there are any gods? —That is what I say, that
you do not believe in the gods at all.

You cannot be believed, Meletus, even, I think, by yourself. The man appears to me,
gentlemen of the jury, highly insolent and uncontrolled. He seems to have made this deposition
out of insolence, violence and youthful zeal. He is like one who composed a riddle and is trying
it out: "Will the wise Socrates realize that I am jesting and contradicting myself, or shall I deceive
him and others?" I think he contradicts himself in the affidavit, as if he said: "Socrates is guilty of\marginpar{27}
not believing in gods but believing in gods," and surely that is the part of a jester!

Examine with me, gentlemen, how he appears to contradict himself, and you, Meletus, answer
us. Remember, gentlemen, what I asked you when I began, not to create a disturbance if I proceed
in my usual manner.\marginpar{b}

Does any man, Meletus, believe in human activities who does not believe in humans? Make
him answer, and not again and again create a disturbance. Does any man who does not believe in
horses believe in horsemen's activities? Or in flute-playing activities but not in flute-players? No,
my good sir, no man could. If you are not willing to answer, I will tell you and the jury. Answer
the next question, however. Does any man believe in spiritual activities who does not believe in
spirits?—No one.\marginpar{c}

Thank you for answering, if reluctantly, when the jury made you. Now you say that I believe
in spiritual things and teach about them, whether new or old, but at any rate spiritual things
according to what you say, and to this you have sworn in your deposition. But if I believe in
spiritual things I must quite inevitably believe in spirits. Is that not so? It is indeed. I shall assume
that you agree, as you do not answer. Do we not believe spirits to be either gods or the children
of gods? Yes or no?—Of course.\marginpar{d}

Then since I do believe in spirits, as you admit, if spirits are gods, this is what I mean when
I say you speak in riddles and in jest, as you state that I do not believe in gods and then again that
I do, since I do believe in spirits. If on the other hand the spirits are children of the gods, bastard
children of the gods by nymphs or some other mothers, as they are said to be, what man would
believe children of the gods to exist, but not gods? That would be just as absurd as to believe the
young of horses and asses, namely mules, to exist, but not to believe in the existence of horses and\marginpar{e}
asses. You must have made this deposition, Meletus, either to test us or because you were at a loss
to find any true wrongdoing of which to accuse me. There is no way in which you could persuade
anyone of even small intelligence that it is possible for one and the same man to believe in spiritual
but not also in divine things, and then again for that same man to believe neither in spirits nor in
gods nor in heroes.\marginpar{28}

I do not think, gentlemen of the jury, that it requires a prolonged defence to prove that I am
not guilty of the charges in Meletus' deposition, but this is sufficient. On the other hand, you know
that what I said earlier is true, that I am very unpopular with many people. This will be my
undoing, if I am undone, not Meletus or Anytus but the slanders and envy of many people. This
has destroyed many other good men and will, I think, continue to do so. There is no danger that\marginpar{b}
it will stop at me.

Someone might say: 'Are you not ashamed, Socrates, to have followed the kind of occupation
that has led to your being now in danger of death?" However, I should be right to reply to him:
"You are wrong, sir, if you think that a man who is any good at all should take into account the risk
of life or death; he should look to this only in his actions, whether what he does is right or wrong,
whether he is acting like a good or a bad man." According to your view, all the heroes who died
at Troy were inferior people, especially the son of Thetis who was so contemptuous of danger\marginpar{c}
compared with disgrace. When he was eager to kill Hector, his goddess mother warned him, as I
believe, in some such words as these: "My child, if you avenge the death of your comrade,
Patroclus, and you kill Hector, you will die yourself, for your death is to follow immediately after
Hector's." Hearing this, he despised death and danger and was much more afraid to live a coward
who did not avenge his friends. "Let me die at once," he said, "when once I have given the\marginpar{d}
wrongdoer his deserts, rather than remain here, a laughing-stock by the curved ships, a burden
upon the earth." Do you think he gave thought to death and danger?

This is the truth of the matter, gentlemen of the jury: wherever a man has taken a position that
he believes to be best, or has been placed by his commander, there he must I think remain and face
danger, without a thought for death or anything else, rather than disgrace. It would have been a
dreadful way to behave, gentlemen of the jury, if, at Potidaea, Amphipolis and Delium, I had, at
the risk of death, like anyone else, remained at my post where those you had elected to command\marginpar{e}
had ordered me, and then, when the god ordered me, as I thought and believed, to live "the life of
a philosopher, to examine myself and others, I had abandoned my post for fear of death or anything
else. That would have been a dreadful thing, and then I might truly have justly been brought here
for not believing that there are gods, disobeying the oracle, fearing death, and thinking I was wise\marginpar{29}
when I was not. To fear death, gentlemen, is no other than to think oneself wise when one is not,
to think one knows what one does not know. No one knows whether death may not be the greatest
of all blessings for a man, yet men fear it as if they knew that it is the greatest of evils. And surely
it is the most blameworthy ignorance to believe that one knows what one does not know. It is
perhaps on this point and in this respect, gentlemen, that I differ from the majority of men, and if\marginpar{b}
I were to claim that I am wiser than anyone in anything, it would be in this that as I have no
adequate knowledge of things in the underworld, so I do not think I have. I do know, however, that
it is wicked and shameful to do wrong, to disobey one's superior, be he god or man. I shall never
fear or avoid things of which I do not know, whether they may not be good rather than things that
I know to be bad. Even if you acquitted me now and did not believe Anytus, who said to you that
either I should not have been brought here in the first place, or that now I am here, you cannot\marginpar{c}
avoid executing me, for if I should be acquitted, your sons would practise the teachings of Socrates
and all be thoroughly corrupted; if you said to me in this regard: "Socrates, we do not believe
Anytus now; we acquit you, but only on condition that you spend no more time on this
investigation and do not practise philosophy, and if you are caught doing so you will die;" if, as
I say, you were to acquit me on those terms, I would say to you: "Gentlemen of the jury, I am\marginpar{d}
grateful and I am your friend, but I will obey the god rather than you, and as long as I draw breath
and am able, I shall not cease to practise philosophy, to exhort you and in my usual way to point
out to anyone of you whom I happen to meet: Good Sir, you are an Athenian, a citizen of the
greatest city with the greatest reputation for both wisdom and power; are you not ashamed of your
eagerness to possess as much wealth, reputation and honours as possible, while you do not care
for nor give thought to wisdom or truth or the best possible state of your soul?" Then, if one of you\marginpar{e}
disputes this and says he does care, I shall not let him go at once or leave him, but I shall question
him, examine him and test him, and if I do not think he has attained the goodness that he says he
has, I shall reproach him because he attaches little importance to the most important things and
greater importance to inferior things. I shall treat in this way anyone I happen to meet, young and
old, citizen and stranger, and more so the citizens because you are more kindred to me. Be sure\marginpar{30}
that this is what the god orders me to do, and I think there is no greater blessing for the city than
my service to the god. For I go around doing nothing but persuading both young and old among
you not to care for your body or your wealth in preference to or as strongly as for the best possible
state of your soul, as I say to you: "Wealth does not bring about excellence, but excellence makes\marginpar{b}
wealth and everything else good for men, both individually and collectively."

Now if by saying this I corrupt the young, this advice must be harmful, but if anyone says that
I give different advice, he is talking nonsense. On this point I would say to you, gentlemen of the
jury: "Whether you believe Anytus or not, whether you acquit me or not, do so on the
understanding that this is my course of action, even if I am to face death many times." Do not\marginpar{c}
create a disturbance, gentlemen, but abide by my request not to cry out at what I say but to listen,
for I think it will be to your advantage to listen, and I am about to say other things at which you
will perhaps cry out. By no means do this. Be sure that if you kill the sort of man I say I am, you
will not harm me more than yourselves. Neither Meletus nor Anytus can harm me in any way; he
could not harm me, for I do not think it is permitted that a better man be harmed by a worse;
certainly he might kill me, or perhaps banish or disfranchise me, which he and maybe others think\marginpar{d}
to be great harm, but I do not think so. I think he is doing himself much greater harm doing what
he is doing now, attempting to have a man executed unjustly. Indeed, gentlemen of the jury, I am
far from making it defence now on my own behalf, as might be thought, but on yours, to prevent
you from wrongdoing by mistreating the god's gift to you by condemning me; for if you kill me\marginpar{e}
you will not easily find another like me. I was attached to this city by the god—though it seems
a ridiculous thing to say—as upon a great and noble horse which was somewhat sluggish because
of its size and needed to be stirred up by a kind of gadfly. It is to fulfill some such function that
I believe the god has placed me in the city. I never cease to rouse each and everyone of you, to
persuade and reproach you all day long and everywhere I find myself in your company.

Another such man will not easily come to be among you, gentlemen, and if you believe me you\marginpar{31}
will spare me. You might easily be annoyed with me as people are when they are aroused from a
doze, and strike out at me; if convinced by Anytus you could easily kill me, and then you could
sleep on for the rest of your days, unless the god, in his care for you, sent you someone else. That
I am the kind of person to be a gift of the god to the city you might realize from the fact that it does
not seem like human nature for me to have neglected all my own affairs and to have tolerated this
neglect now for so many years while I was always concerned with you, approaching each one of\marginpar{b}
you like a father or an elder brother to persuade you to care for virtue (aretë). Now if I profited
from this by charging a fee for my advice, there would be some sense to it, but you can see for
yourselves that, for all their shameless accusations, my accusers have not been able in their
impudence to bring forward a witness to say that I have ever received a fee or ever asked for one.\marginpar{c}
I, on the other hand, have a convincing witness that I speak the truth, my poverty.

It may seem strange that while I go around and give this advice privately and interfere in
private affairs, I do not venture to go to the assembly and there advise the city. You have heard me
give the reason for this in many places. I have a divine or spiritual sign which Meletus has
ridiculed in his deposition. This began when I was a child. It is a voice, and whenever it speaks it
turns me away from something I am about to do, but it never encourages me to do anything. This\marginpar{d}
is what has prevented me from taking part in public affairs, and I think it was quite right to prevent
me. Be sure, gentlemen of the jury, that if I had long ago attempted to take part in politics, I should
have died long ago, and benefited neither you nor myself. Do not be angry with me for speaking
the truth; no man will survive who genuinely opposes you or any other crowd and prevents the
occurrence of many unjust and illegal happenings in the city. A man who really fights for justice\marginpar{e}
must lead a private, not a public, life if he is to survive for even a short time.

I shall give you great proofs of this, not words but what you esteem, deeds. Listen to what\marginpar{32}
happened to me, that you may know that I will not yield to any man contrary to what is right, for
fear of death, even if I should die at once for not yielding. The things I shall tell you are
commonplace and smack of the lawcourts, but they are true. I have never held any other office in
the city, but I served as a member of the Council, and our tribe Antiochis was presiding at the time
when you wanted to try as a body the ten generals who had failed to pick up the survivors of the
naval battle. This was illegal, as you all recognized later. I was the only member of the presiding\marginpar{b}
committee to oppose your doing something contrary to the laws, and I voted against it. The orators
were ready to prosecute me and take me away; and your shouts were egging them on, but I thought
I should run any risk on the side of law and justice rather than join you, for fear of prison or death,
when you were engaged in an unjust course.

This happened when the city was still a democracy. When the oligarchy was established, the
Thirty summoned me to the Hall, along with four others, and ordered us to bring Leon from\marginpar{c}
Salamis, that he might be executed. They gave many such orders to many people, in order to
implicate as many as possible in their guilt. Then I showed again, not in words but in action, that,
if it were not rather vulgar to say so, death is something I couldn't care less about, but that my
whole concern is not to do anything unjust or impious. That government, powerful as it was, did\marginpar{d}
not frighten me into any wrongdoing. When we left the Hall, the other four went to Salamis and
brought in Leon, but I went home. I might have been put to death for this, had not the government
fallen shortly afterwards. There are many who will witness to these events.

Do you think I would have survived all these years if I were engaged in public affairs and,\marginpar{e}
acting as a good man must, came to the help of justice and considered this the most important
thing? Far from it, gentlemen of the jury, nor would any other man. Throughout my life, in any
public activity I may have engaged in, I am the same man as I am in private life. I have never come
to an agreement with anyone to act unjustly, neither with anyone else nor with anyone of those
who they slanderously say are my pupils. I have never been anyone's teacher. If anyone, young or\marginpar{33}
old, desires to listen to me when I am talking and dealing with my own concerns, I have never
begrudged this to anyone, but I do not converse when I receive a fee and not when I do not. I am
equally ready to question the rich and the poor if anyone is willing to answer my questions and\marginpar{b}
listen to what I say. And I cannot justly be held responsible for the good or bad conduct of these
people, as I never promised to teach them anything and have not done so. If anyone says that he
has learned anything from me, or that he heard anything privately that the others did not hear, be
assured that he is not telling the truth.

Why then do some people enjoy spending considerable time in my company? You have heard
why, gentlemen of the jury, I have told you the whole truth. They enjoy hearing those being
questioned who think they are wise, but are not. And this is not unpleasant. To do this has, as I say,\marginpar{c}
been enjoined upon me by the god, by means of oracles and dreams, and in every other way that
a divine manifestation has ever ordered a man to do anything. This is true, gentlemen, and can
easily be established.

If I corrupt some young men and have corrupted others, then surely some of them who have
grown older and realized that I gave them bad advice when they were young should now
themselves come up here to accuse me and avenge themselves. If they were unwilling to do so\marginpar{d}
themselves, then some of their kindred, their fathers or brothers or other relations should recall it
now if their family had been harmed by me. I see many of these present here, first Crito, my
contemporary and fellow demesman, the father of Critoboulos here; next Lysanias of Sphettus, the
father of Aeschines here; also Antiphon the Cephisian, the father of Epigenes; and others whose
brothers spent their time in this way; Nicostratus, the son of Theozotides, brother of Theodotus,
and Theodotus has died so he could not influence him; Paralios here, son of Demodocus, whose\marginpar{e}
brother was Theages; there is Adeimanttls, son of Ariston, brother of Plato here; Acantidorus,
brother of Apollodorus here.

I could mention many others, some one of whom surely Meletus should have brought in as
witness in his own speech. If he forgot to do so, then let him do it now; I will yield time if he has\marginpar{34}
anything of the kind to say. You will find quite the contrary, gentlemen. These men are all ready
to come to the help of the corruptor, the man who has harmed their kindred, as Meletus and Anytus
say. Now those who were corrupted might well have reason to help me, but the uncorrupted, their
kindred who are older men, have no reason to help me except the right and proper one, that they
know that Meletus is lying and that I am telling the truth.

Very well, gentlemen of the jury. This, and maybe other similar things, is what I have to say\marginpar{b}
in my defence. Perhaps one of you might be angry as he recalls that when he himself stood trial
on a less dangerous charge, he begged and pleaded and implored the jury with many tears, that he
brought his children and many of his friends and family into court to arouse as much pity as he
could, but that I do none of these things, even though I may seem to be running the ultimate risk.
Thinking of this, he might feel resentful toward me and, angry about this, cast his vote in anger.\marginpar{c}
If there is such a one among you—I do not deem there is, but if there is—I think it would be right
to say in reply: My good sir, I too have a household and, in Homer's phrase, I am not born "from
oak or rock" but from men, so that I have a family, indeed three sons, gentlemen of the jury, of
whom one is an adolescent while two are children. Nevertheless, I will not beg you to acquit me\marginpar{d}
by bringing them here. Why do I do none of these things? Not through arrogance, gentlemen, nor
through lack of respect for you. Whether I am brave in the face of death is another matter, but with
regard to my reputation and yours and that of the whole city, it does not seem right to me to do
these things, especially at my age and with my reputation. For it is generally believed, whether it
be true or false, that in certain respects Socrates is superior to the majority of men. Now if those\marginpar{e}
of you who are considered superior, be it in wisdom or courage or whatever other virtue makes
them so, are seen behaving like that, it would be a disgrace. Yet I have often seen them do this sort
of thing when standing trial, men who are thought to be somebody, doing amazing things as if they
thought it a terrible thing to die, and as if they were to be immortal if you did not execute them.\marginpar{35}
I think these men bring shame upon the city so that a stranger, too, would assume that those who
are outstanding in virtue among the Athenians, whom they themselves select from themselves to
fill offices of state and receive other honours, are in no way better than women. You should not
act like that, gentlemen of the jury, those of you who have any reputation at all, and if we do, you\marginpar{b}
should not allow it. You should make it very clear that you will more readily convict a man who
performs these pitiful dramatics in court and so makes the city a laughingstock, than a man who
keeps quiet.

Quite apart from the question of reputation, gentlemen, I do not think it right to supplicate the
jury and to be acquitted because of this but to teach and persuade them. It is not the purpose of a
juryman's office to give justice as a favour to whoever seems good to him, but to judge according\marginpar{c}
to law, and this he has sworn to do. We should not accustom you to perjure yourselves, nor should
you make a habit of it. This is irreverent conduct for either of us.

Do not deem it right for me, gentlemen of the jury, that I should act towards you in a way that
I do not consider to be good or just or pious, especially, by Zeus, as I am being prosecuted by
Meletus here for impiety; clearly, if I convinced you by my supplication to do violence to your\marginpar{d}
oath of office, I would be teaching you not to believe that there are gods, and my defence would
convict me of hot believing in them. This is far from being the case, gentlemen, for I do believe
in them as none of my accusers do. I leave it to you and the god to judge me in the way that will
be best for me and for you.

[The jury now gives its verdict of guilty, and Meletus asks for the penalty of death.]

There are many other reasons for my not being angry with you for convicting me, gentlemen
of the jury, and what happened was not unexpected. I am much more surprised at the number of
votes cast on each side, for I did not think the decision would be by so few votes but by a great
many. As it is, a switch of only thirty votes would have acquitted me. I think myself that I have\marginpar{e}
been cleared on Meletus' charges, and not only this, but it is clear to all that, if Anytus and Lycon \marginpar{36}
had not joined him in accusing me, he would have been fined a thousand drachmas for not
receiving a fifth of the votes.

He assesses the penalty at death. So be it. What counter-assessment should I propose to you,\marginpar{b}
gentlemen of the jury? Clearly it should be a penalty I deserve, and what do I deserve to suffer or
to pay because I have deliberately not led a quiet life but have neglected what occupies most
people: wealth, household affairs, the position of general or public orator or the other offices, the
political clubs and factions that exist in the city? I thought myself too honest to survive if I
occupied myself with those things. I did not follow that path that would have made me of no use
either to you or to myself, but I went to each of you privately and conferred upon him what I say
is the greatest benefit, by trying to persuade him not to care for any of his belongings before caring
that he himself should be as good and as wise as possible, not to care for the city's possessions\marginpar{c}
more than for the city itself, and to care for other things in the same way. What do I deserve for
being such a man? Some good, gentlemen of the jury, if I must truly make an assessment according
to my deserts, and something suitable. What is suitable for a poor benefactor who needs leisure
to exhort you? Nothing is more suitable, gentlemen, than for such a man to be fed in the\marginpar{d}
Prytaneum, much more suitable for him than for anyone of you who has won a victory at Olympia5
with a pair or a team of horses. The Olympian victor makes you think yourself happy; I make you
be happy. Besides, he does not need food, but I do. So if I must make a just assessment of what
I deserve, I assess it at this: free meals in the Prytaneum.

When I say this you may think, as when I spoke of appeals to pity and entreaties, that I speak
arrogantly, but that is not the case, gentlemen of the jury; rather it is like this: I am convinced that\marginpar{e}
I never willingly wrong anyone, but I am not convincing you of this, for we have talked together\marginpar{37}
but a short time. If it were the law with us, as it is elsewhere, that a trial for life should not last one
but many days, you would be convinced, but now it is not easy to dispel great slanders in a short
time. Since I am convinced that I wrong no one, I am not likely to wrong myself, to say that I\marginpar{b}
deserve some evil and to make some such assessment against myself. What should I fear? That I
should suffer the penalty Meletus has assessed against me, of which I say I do not know whether
it is good or bad? Am I then to choose in preference to this something that I know very well to be
an evil and assess the penalty at that? Imprisonment? Why should I live in prison, always subjected
to the ruling magistrates the Eleven? A fine, and imprisonment until I pay it? That would be the\marginpar{c}
same thing for me, as I have no money. Exile? for perhaps you might accept that assessment.

I should have to be inordinately fond of life, gentlemen of the jury, to be so unreasonable as
to suppose that other men will easily tolerate my company and conversation when you, my fellow
citizens, have been unable to endure them, but found them a burden and resented them so that you
are now seeking to get rid of them. Far from it, gentlemen. It would be a fine life at my age to be\marginpar{d}
driven out of one city after another, for I know very well that wherever I go the young men will
listen to my talk as they do here. If I drive them away, they will themselves persuade their elders
to drive me out; if I do not drive them away, their fathers and relations will drive me out on their\marginpar{e}
behalf.

Perhaps someone might say: But Socrates, if you leave us will you not be able to live quietly,
without talking? Now this is the most difficult point on which to convince some of you. If I say
that it is impossible for me to keep quiet because that means disobeying the god, you will not
believe me and will think I am being ironical. On the other hand, if I say that it is the greatest good
for a man to discuss virtue every day and those other things about which you hear me conversing
and testing myself and others, for the unexamined life is not worth living for man, you will believe
me even less.\marginpar{38}

What I say is true, gentlemen, but it is not easy to convince you. At the same time, I am not
accustomed to think that I deserve any penalty. If I had money. I would assess the penalty at the
amount I could pay, for that would not hurt me, but I have none, unless you are willing to set the
penalty at the amount I can pay, and perhaps I could pay you one mina of silver. So that is my
assessment.\marginpar{b}

Plato here, gentlemen of the jury, and Crito and Critobulus and Apollodorus bid me put the
penalty at thirty minae, and they will stand surety for the money. Well then, that is my assessment,
and they will be sufficient guarantee of payment.

[The jury now votes again and sentences Socrates to death.]

It is for the sake of a short time, gentlemen of the jury, that you will acquire the reputation and
the guilt, in the eyes of those who want to denigrate the city, of having killed Socrates, a wise man,
for they who want to revile you will say that I am wise even if I am not. If you had waited but a
little while, this would have happened of its own accord. You see my age, that I am already\marginpar{c}
advanced in years and close to death. I am saying this not to all of you but to those who
condemned me to death, and to these same jurors I say: Perhaps you think that I was convicted for
lack of such words as might have convinced you, if I thought I should say or do all I could to avoid
my sentence. Far from it. I was convicted because I lacked not words but boldness and\marginpar{d}
shamelessness and the willingness to say to you what you would most gladly have heard from me,
lamentations and tears and my saying and doing many things that I say are unworthy of me but that
you are accustomed to hear from others. I did not think then that the danger I ran should make me
do anything mean, nor do I now regret the nature of my defence. I would much rather die after this
kind of defence than live after making the other kind. Neither I nor any other man should, on trial\marginpar{e}
or in war, contrive to avoid death at any cost. Indeed it is often obvious in battle that one could
escape death by throwing away one's weapons and by turning to supplicate one's pursuers, and
there are many ways to avoid death in every kind of danger if one will venture to do or say\marginpar{39}
anything to avoid it. It is not difficult to avoid death, gentlemen of the jury, it is much more
difficult to avoid wickedness, for it runs faster than death. Slow and elderly as I am, I have been
caught by the slower pursuer, whereas my accusers, being clever and sharp, have been caught by
the quicker, wickedness. I leave you now, condemned to death by you, but they are condemned\marginpar{b}
by truth to wickedness and injustice. So I maintain my assessment, and they maintain theirs. This
perhaps had to happen, and I think it is as it should be.

Now I want to prophesy to those who convicted me, for I am at the point when men prophesy
most, when they are about to die. I say gentlemen, to those who voted to kill me, that vengeance
will come upon you immediately after my death, a vengeance much harder to bear than that which\marginpar{c}
you took in killing me. You did this in the belief that you would avoid giving an account of your
life, but I maintain that quite the opposite will happen to you. There will be more people to test
you, whom I now held back, but you did not notice it. They will be more difficult to deal with as
they will be younger and you will resent them more. You are wrong if you believe that by killing
people you will prevent anyone from reproaching you for not living in the right way. To escape\marginpar{d}
such tests is neither possible nor good, but it is best and easiest not to discredit others but to
prepare oneself to be as good as possible. With this prophecy to you who convicted me, I part from
you.

I should be glad to discuss what has happened with those who voted for my acquittal during
the time that the officers of the court are busy and I do not yet have to depart to my death. So,
gentlemen, stay with me awhile, for nothing prevents us from talking to each other while it is\marginpar{e}
allowed. To you, as being my friends, I want to show the meaning of what has occurred. A
surprising thing has happened to me, judges—you I would rightly call judges. At all previous times
my familiar prophetic power, my spiritual manifestation frequently opposed me, even in small
matters, when I was about to do something wrong, but now that, as you can see for yourselves, I\marginpar{40}
was faced with what one might think, and what is generally thought to be, the worst of evils, my
divine sign has not opposed me, either when I left home at dawn, or when I came into court, or at
any time that I was about to say something during my speech. Yet in other talks it often held me
back in the middle of my speaking, but now it has opposed no word or deed of mine. What do I
think is the reason for this? I will tell you. What has happened to me may well be a good thing, and\marginpar{b}
those of us who believe death to be an evil are certainly mistaken. I have convincing proof of this,
for it is impossible that my familiar sign did not oppose me if I was not about to do what was right.

Let us reflect in this way, too, that there is good hope that death is a blessing, for it is one of
two things: either the dead are nothing and have no perception of anything, or it is, as we are told,\marginpar{c}
a change and a relocating for the soul from here to another place. If it is complete lack of
perception, like a dreamless sleep, then death would be a great advantage. For I think that if one
had to pick out that night during which a man slept soundly and did not dream, put beside it the
other nights and days of his life, and then see how many days and nights had been better and more\marginpar{d}
pleasant than that night, not only a private person but the great king would find them easy to count
compared with the other days and nights. If death is like this I say it is an advantage, for all eternity
would then seem to be no more than a single night. If, on the other hand, death is a change from
here to another place, and what we are told is true and all who have died are there, what greater\marginpar{e}
blessing could there be, gentlemen of the jury? If anyone arriving in Hades will have escaped from
those who call themselves judges here, and will find those true judges who are said to sit in
judgement there, Minos and Radamanthus and Aeacus and Triptolemus and the other demi-gods\marginpar{41}
who have been upright in their own life, would that be a poor kind of change? Again, what would
one of you give to keep company with Orpheus and Musaeus,Hesiod and Homer? I am willing to
die many times if that is true. It would be a wonderful way for me to spend my time whenever I
met Palamedes and Ajax, the son of Telamon, and any other of the men of old who died through
an unjust conviction, to compare my experience with theirs. I think it would be pleasant. Most
important, I could spend my time testing and examining people there, as I do here, as to who
among them is wise, and who thinks he is, but is not.

What would one not give, gentlemen of the jury, for the opportunity to examine the man who\marginpar{b}
led the great expedition against Troy, or Odysseus, or Sisyphus, and innumerable other men and
women one could mention. It would be an extraordinary happiness to talk with them, to keep
company with them and examine them. In any case, they would certainly not put one to death for\marginpar{c}
doing so. They are happier there than we are here in other respects, and for the rest of time they
are deathless, if indeed what we are told is true.

You too must be of good hope as regards death, gentlemen of the jury, and keep this one truth
in mind, that a good man cannot be harmed either in life or in death, and that his affairs are not
neglected by the gods. What has happened to me now has not happened of itself, but it is clear to
me that it was better for me to die now and to escape from trouble. That is why my divine sign did\marginpar{d}
not oppose me at any point. So I am certainly not angry with those who convicted me, or with my
accusers. Of course that was not their purpose when they accused and convicted me, but they
thought they were hurting me, and for this they deserve blame. This much I ask from them: when
my sons grow up, avenge yourselves by, causing them the same kind of grief that I caused you, if
you think they care for money or anything else more than they care for virtue, or if they think they\marginpar{e}
are somebody when they are nobody. Reproach them as I reproach you, that they do not care for
the right things and think they are worthy when they are not worthy of anything. If you do this, I
shall have been justly treated by you, and my sons also.

Now the hour to part has come. I go to die, you go to live. Which of us goes to the better lot
is known to no one, except the god.\marginpar{42}

\chapter{Part 2: Life and Times of Socrates}

With the introduction to philosophy out of the way (as in, the barebones of the method and the quest for answers), we will now move on to the life and times of Socrates, the protagonist and real world person who you should be reading in the assigned reading for this module. This should serve as an ample starting point to understand the raise and fall of Philosophy's most famous proponent.
\section{Birth and Parents}

Socrates was born around spring 469BCE in Athens. At that time, the Persians had just attempted  (and failed) to invade Athens and Athens, on a global scale, was forming an alliance with the other city-states in the region (this alliance, called The Delian League, would grow into the Athenian Empire). This is on the Attic Peninsula and, politically and socially, it was divided into 139 districts, which were in turn, broken up among the 10 tribes which made up Athens. The members of these tribes were automatically Athenian. Socrates was a member of the Antiochis, which was located outside of the city walls, to the south-east. Keeping with the customs of the time, 5 days after Socrates' birth, his father, Sophroniscus, looked the infant Socrates over while walking around the hearth, to make sure that the kid was his, and then admitted him into the family. 5 days after this, Socrates was actually given a name and his father presented him to the local officials for the relevant paperwork (think of this as the equivalent of a birth-certificate and all that). In doing this, Sophroniscus took on the responsibility of ensuring that Socrates got a proper education and was made into a respectable member of society.
\section{Education}

In Athens, the ability to read and write was commonplace since around 520BCE and it was unheard of for a young man not to have that skill. When Socrates turned 5, he started the equivalent of elementary school. His education consisted of learning to read and write, gymnastics (it was expected that Athenians be physically fit for military service, think of this like PE), music, and basic mathematics. It is because of this that Socrates, frankly, became a nerd. Socrates, as to be expected, loved philosophy. At that time, philosophy was, and still is, the mother of all other subjects, physics, mathematics, biology, and so on. The term `philosophy' comes from two Greek roots, `philein' meaning love and `sophia' meaning wisdom. For Socrates and other Greeks (as well as any good philosophy/science today), philosophy was done with unaided reason and careful observation of the world. Over time, the careful observation aspect became what we call the sciences. But, to do philosophy, there are certain things which you don't get. You don't get anything in a religious scripture (so I don't want to see you reference a religious text in this class), you don't get what some authority figure says (just because they are an authority), you don't get myths, and you don't get things just because they were always done that way (no tradition/cultural beliefs). For philosophy, all you get is your observations and your own good reason. At that time, philosophers were inventing Geometry, proposing a heliocentric model of the solar system (the Earth revolves around the sun), and also various aspects of the natural sciences which would develop into their own fields (EG biology).

Socrates, by all accounts, was a remarkably ugly person. Many accounts say that his childhood classmates would call him “frogface”. He had bulging eyes, broad, flat nose, and thick lips. He also walked bow-legged and sideways like a crab. If you ever see a bust of Socrates, understand that the carving is an angel compared to him in reality.
\section{“Adult" Socrates}

At the age of 17, Socrates graduated from the schools and was sworn in as a full-fledged citizen of Athens (think of it as like turning 18 in the US). Socrates' father took him to the ceremony, but died shortly afterwards. His mother, Phaenarete, remarried and had another son, named Patrocles.  In those days, Athens had one of the first constitutional democracies which enshrined freedom of speech and, because of this, public discussion, voting, and debate about all matters were quite commonplace. Athens had many festivals and gatherings throughout the year, attracting many of the great minds from around Greece. According to some accounts, at 19, Socrates was often found at these festivals discussing things with the philosophers of the day, allowable because of the freedom of speech. Freedom of speech is a definite perk which was not had by many societies in those days, but being an Athenian did come with some responsibilities. Upon becoming a citizen, Socrates was also put on the draft for military service.
\section{Military Career}

After being called on for the draft and completing the required 2 years of training, Socrates served in the military for Athens. During his first posting, it was a time of relative peace, so Socrates likely practiced a trade (stonecutting, like his father). But, as the years progressed, Athens was starting to get into a war with Sparta. Socrates served on the front lines for many of the battles in 432BCE,  Socrates was called on to serve in the military for Athens on a few different occasions. After putting down a revolt, Socrates' troop entered into a very heavy conflict near Spartolus, where they suffered heavy casualties. In this fight, Socrates' bravery became legendary. He refused to retreat until he was the last person there and fought off enemy soldiers with another soldier, Alcibiades, on his back, saving his life. This deployment kept him away from home for 3 years. Socrates returned to duty again in 424BCE, where his bravery was again noted by the generals. His commander, Laches, noted Socrates' bravery when writing about the nature of courage, stating that he refused to retreat, even after the order was given, until he was the last person to leave. A year later, Socrates returned to duty and fought in another battle. After this, as far as we can tell, he did not serve in the military any longer. After this point, Athens and Sparta signed a treaty, which gave Athens a few years without the struggles of war. During this time, Socrates married Xanthippe, who, because of Socrates' military exploits, came with a large dowry.
\section{Family Man}

Socrates cared little for material possessions. Often wearing the same clothes for many days in a row, including sleeping in them. While in the military, as well as throughout his life, Socrates rarely, if ever, wore footwear, even on ice or snow. Athenians of his day described him as “frugal”. Although it was well within his means to afford various things, like shoes, he practiced “voluntary simplicity”. Xanthippe gave birth to three sons, the third being born while Socrates was in prison awaiting execution, so they did the ceremony in the prison. According to many accounts, Xanthippe had a volatile personality and was very unhappy with her hubby, and with good reason, he had a habit of spending all day and night in the agora (public market space) discussing questions and arguing with people when he could be out there making money (he was a stonecutter, like his father). According to one story, after chewing him out, Xanthippe climbed onto the roof of their house and dumbed a bucket of urine on his head as Socrates went out to debate people.

Because of his willingness and constant engagement in questioning and debate, by his middle age, Socrates became a very recognizable (today we would say famous) person on the streets of Athens. In the Greek comedies of the day, an ugly caricature of Socrates was a re-occurring character. Socrates was made-fun of for his appearance and his love of philosophy in at least 3 prize-winning plays. Socrates took these in stride. At one point a foreigner asked “who’s this loon, Socrates?” in the middle of a play. At which point Socrates stood up and said “ME!”

\section{Part 2.1: Trials in Athens}

\section{How Trials Worked in Athens}

In Socrates' time, the procedure for a trial and court cases like this was pretty well established. In that time, also, there was no such thing as what we would call today a `public prosecutor', this is similar to how in London, prior to their police force, they had a hew and cry system. If a citizen suspected another of a crime, then they would report it to the officials to have the case looked over. Any actual citizen of Athens could initiate the procedure. Once such a crime was reported, the trial consisted of three parts. First, there was the Initiation of Criminal Proceedings. Next, they had The Preliminary Hearing (Anakrisis). And third, they had The Trial itself. As evidence for how common and stream-lined this process was in Athens, historically, just prior to his trial, Socrates was engaged in a conversation with a prosecutor for a trial just ending. This conversation was on the nature of goodness and the question related to it is the discussion for this module. However, most cases were settled prior to reaching the actual trial because, as we will see, making the arrangements is quite the undertaking.
\subsection{Initiation of Criminal Proceedings}

As I just mentioned, any Athenian could raise charges against another. In the case of Socrates, the proceedings began when Meletus, a poet, arranged a meeting, for a specific date, with the legal magistrate, or in some cases King Archon, in a colonnaded building called the Royal Stoa to answer charges of impiety. Meletus, then delivered an oral summons to Socrates in the presence of witnesses (or callers). Once the magistrate determined – after listening to Socrates and Meletus (and perhaps the other two accusers, Anytus and Lycon) – that the lawsuit was permissible under Athenian law, a date was set for the ``preliminary hearing" (anakrisis) and terms for the hearing were posted as a public notice at the Royal Stoa.
\subsection{The Preliminary Hearing (Anakrisis)}

The preliminary hearing before the magistrate at the Royal Stoa began with the reading of the written charge by Socrates' accuser, Meletus. Socrates then formally answered the charge. Both the written charge and denial were then attested to by each, under oath, as being true. The next phase of the preliminary hearing was an interrogation. First, the magistrate questioned both Meletus and Socrates. This is to see whether the issue could be settled out of court and to see whether there was any merit to the charges. Second, both the accuser and defendant were allowed to question each other. This was another chance to change the course of events. The third and final phase, supposing that the magistrate found the case worthy of consideration, the magistrate would draw up formal charges against the accused and set a date for the public trial. For Socrates, these charges (relating to impiety and corruption of youth), the actual paperwork, were preserved as a public document (antomosia), and they survived until at least the second century C.E., but were subsequently lost. 
\subsection{The Trial}

The trial of Socrates took place over a nine‐to‐ten hour period in the People's Court, located in the agora, the civic center of Athens. The jury consisted of 500 male citizens over the age of thirty, chosen by lot from among volunteers. People today might think that this is a ridiculously high number for a jury.  But, sometimes, the juries could be as large as 1501 men. This was a protection against bribery. For example, in order to ensure that a case went a certain way, the person would need to bribe at least 251 people, and it's not likely that many could afford to do this.  All jurors were required to swear by the gods of Zeus, Apollo, and Demeter the Heliastic Oath: ``I will cast my vote in consonance with the laws and decrees passed by the Assembly and by the Council, but, if there is no law, in consonance with my sense of what is most just, without favor or enmity. I will vote only on the matters raised in the charge, and I will listen impartially to the accusers and defenders alike."

Most of the jurors were probably farmers, as that was the principal occupation of the day. For their jury service they received payment of three obols. An obol was a currency of the day and it was around a 40TH of an ounce of silver. In general, three obols fed you an paid for a leisurely night. The jurors sat on wooden benches separated from spectators by some sort of barrier or railing. Given Socrates's fame and the notoriousness of the charge against him, the crowd of spectators was most likely large – including, of course, the most famous pupil of Socrates, Plato.

The trial began in the morning with the reading of the formal charges against Socrates by a herald. Few, if any, formal rules of evidence existed. The prosecution presented its case first. Meletus, Anytus, and Lycon had three hours, measured by a waterclock, to make their argument for a finding of guilt. Each accuser spoke from an elevated stage. No record of the prosecution's argument against Socrates survives. Following the prosecution's case, Socrates had three hours to answer the charges. Although many written versions of the defense – or apology – of Socrates at one time circulated, only two have survived: one by Plato and another by Xenophon. After the arguments, the herald of the court called on the jurors to consider their decision. In Athens, jurors did not retire to a juryroom to deliberate – they made their decisions without discussion among themselves, based in large part on their own interpretations of the law. The 500 jurors voted on his guilt or innocence by dropping bronze ballot disks into marked urns. Only a majority vote was necessary for conviction. Four jurors were assigned the task of counting votes. In the case of Socrates, the jury found him guilty on a relatively close vote of 280 to 220. (Interestingly, if less than 100 jurors voted for guilt, the accusers had to pay a fine to cover trial costs, which is similar to something which is found in court cases today.)
\subsection{The Final Phase}

If a defendant is convicted, the trial enters a second phase to set punishment. The prosecution and the defendant each propose a punishment and the jury chooses between the two punishment options presented to it. The range of possible punishments included death, imprisonment, loss of civil rights (i.e., the right to vote, the right to serve as a juror, the right to speak in the Assembly), exile, and fines. In the trial of Socrates, the principal accusers proposed the punishment of death. Socrates, if Plato's account is to be believed, proposed first the punishment – or, rather, the nonpunishment – of free meals in the center of the city, then later the extremely modest fine of one mina of silver. Apparently finding Socrates' proposed punishment insultingly light, the jury voted for the prosecution's proposal of death by a larger margin than for conviction, 360 to 140. The execution of Socrates was accomplished through the drinking of a cup of poison hemlock.

\section{Part 2.2: Socrates' Defense/Arguments}

\section{The Charges}

There were three formal charges which Socrates was going to need to reply to in his 3 hour block of time. The first, the real zinger, was corrupting the youth of Athens. This was essentially causing the youth to have rebellious mindsets and be against the state. There was some evidence to this claim as several attempted rebellions and revolutions had taken place with many of the young people which Socrates spoke with often playing central roles, though Socrates himself was more of a bystander than anything in these cases.  The second charge, inventing new gods, was a lesser case. This is more in line with the initial charge of impiety. According to the laws of Athens, blasphemy of this sort was a crime punishable by death.  The third charge, which could have either built up or dismantled the second, was not believing in the gods of Athens. This could have gone two ways, either the accusers could have said that in inventing new gods, Socrates abandoned belief in the Athenian pantheon (which would have built up the case regarding blasphemy) or they could have said that Socrates was an atheist, which directly contradicts the second charge. As it turns out, the accusers, not trained in philosophy, did not see the contradiction and went with the latter option (being an atheist).
\section{Socrates' Method}

Philosophy, in this period, was still very much in its infancy, it wasn't as formalized as it is today (throughout this class, you will be learning the philosophic method, which is the great and powerful grandmother to the scientific method which you may be familiar with). For the pre-socratics (which includes Socrates), you can think of philosophy as a game where they are still trying to figure out the rules. If you have ever seen little kids try and play soccer or football for the first time, then you will have the right kind of image in mind.  Socrates, for his part, in the trial, was unfamiliar and inexperienced in how to present publicly, especially given the large number of people in attendance, and was not really the best at being persuasive (rhetoric was an area which he really did not have much respect for). So, Socrates begins his rebuttal by explaining that he will treat this like he would a debate in the Agora. In those cases, Socrates would merely ask questions to people. Rarely, if ever, does Socrates actually state his stance. Rather, Socrates merely restates the stance which the person had stated in reply to his question and then adds further questions to this. If you ever get the chance to see me in a lecture format, I do this methodology for the first few days of lecture, off and on. As a result, Socrates did not address the jury directly, rather just questioned Meletus. Though this method is quite good for teaching and one-on-one debate, it's not the best for a presentation of this sort. 
\section{Socrates' Questions}

Since I understand that the translations for this are a bit wordy and the word choice can be a bit unnatural (I have a lot of experience translating Latin and if a person is not careful or doesn't think about the context of the wording, the translation will come out as quite formal), I have taken the liberty of rewording some of the content of The Apology to make the meaning come through a little clearer. It's still important that you read the original, as it's a wonderful work, however, this is here to help. For this, you can read it as a script, with `S' being Socrates and `M' being Meletus.
\subsection{Corrupting the Youth}

This charge, as I have mentioned, is the real big charge, the others, though still punishable by death, don't have as much going for them, as we will see. To prove his point, Socrates asked Meletus, around, 11 questions, which were leading him down a very particular rabbit-hole.
\factoidbox{
    S: Do you think that it is the greatest importance to make the youth as great as possible?

    M: I do.

    S: Who improves them? 

    M:The Laws

    S: You didn’t answer me, Who has knowledge of these laws?

    M: These Jurymen

    S: What do you mean? Do all these people improve the youth?

    M: All of them.}

So, at this point, Meletus starts off by essentially claiming that the laws of Athens are the sort of things which can improve the youth. But this is not correct, laws are passive actors, they aren't the sort of thing which can take an active part in the raising of the youth. That requires a person who knows the laws and has the ability to accurately teach. So, to his credit, Meletus noticed this and backtracked, claiming that the jurymen have the knowledge of the laws and the ability to teach them (think about how it would have sounded if he claimed that the jurors didn't know the laws).
\factoidbox{
    S: By Hera! That’s a lot of people to improve our youth, what great shape we must be in! What about the audience? So they improve them? 

    M: They do.

    S: What about the council?

    M: They improve them too.

    S: Everyone in Athens makes the youth better but me, is that what you are saying?

    M: Yep, everyone makes the youth better but you.}

At this point, Meletus doesn't seem to see where this is going. Socrates has, through his questions, basically forced Meletus to claim that all the people of Athens improve the youth, with Socrates being the sole bad influence. From a rhetorical side, this was an interesting ploy. If Meletus had claimed otherwise, then Socrates could have easily countered by asking why those people weren't on trial with him, why charges hadn't be posed against them. But, since Meletus said that Socrates was the only corrupter, Socrates needs to go a different route, which leads us to the final `build-up' question:
\factoidbox{
    S: Man, you put me in a rough spot. But does this also apply to horses? That all people improve them and only one person corrupts them? Obviously, it is the opposite. The horse breeder (trainer) improves them and all others corrupt them. It would be a wonderful world if our youth only had one corrupting influence in their lives!}

The horse analogy has not aged well, as for contemporary English, calling a person a horse is a bit insulting. However, think about it this way, a good horse was like the nice car of its day. The vast majority of people aren't trained in how to repair and maintain a car, or in this case, care for and train a horse. Rather, there are very few people who really know how to maintain those things, others who use them, rather unintentionally, corrupt them (wear them out, make them forget their training, etc.). In the case of the youth, the vast majority of people aren't trained in the best child-raising techniques, they interact with children in unintentional ways which lead to negative behaviors, and so forth. It would be a great world if the children only had one corrupting influence, as that would get quickly drowned out by all of the other influences around them.

This leads us to the final set of questions concerning this particular accusation.
\factoidbox{
	S: You have made it clear that you care very little for the youth and haven’t put good thought into the reason you’re putting me on trial. Now answer me this: Is it better for a person to live among good fellows or wicked? This isn’t a hard question. Do wicked people harm their neighbors and good people benefit them?

    M: Certainly

    S: Now, is there a person alive who would rather be harmed than benefited?

    M: Certainly not

    S: Also, do you accuse me of corrupting the youth deliberately or unintentionally?

    M: Deliberately

    S: What then, Meletus? Are you so wise that you realize that the wicked do wicked and the good good but I am so stupid that I have not realized this? Have I failed to recognize that if I make anyone of my associates bad, they will harm me? But still you say that I do this great evil voluntarily? I can’t believe you, Meletus, nor do I think anyone else in the world does! But, leaving that aside, either I do not corrupt them, or if I corrupt them, I do it involuntarily. This shows that you are lying either way we go. But if I corrupt them involuntarily, the law is not to haul people into court, but to take them and instruct and admonish them in private. For it is clear that if I am told about it, I shall stop doing that which I do involuntarily. But you avoided associating with me and instructing me, and were unwilling to do so, but you haul me in here, where it is the law to haul in those who need punishment, not instruction.}

This is where we get Socrates' real argument against the charge. It follows a very interesting logical form, called Constructive Dilemma. It is one which Socrates, historically, had just used earlier in the day outside of the court concerning the nature of goodness. In an abstract form, the reasoning goes like this: One of two (or more) options is correct (A or B). If A is correct, then  C and if B is correct, then D. Therefore, either C or D is correct. Here is the argument for the case along with another example used by Socrates in another dialogue:
\noindent
\begin{tabular}{p{2.5in}|p{2.5in}}
Corrupting the Youth& Piety\\

    Either I don't corrupt the youth or I corrupt them unintentionally.&Either something is moral because it's commanded by the gods or it's commanded by the gods because it's moral.\\
    If I don't corrupt the youth, then the charge against me is false (I'm innocent here).&If it's moral because it's commanded by the gods, then the morality of actions is arbitrary (random, not fixed).\\
    If I corrupt the youth unintentionally, then you should have instructed me in private and not taken me to court (I'm innocent here).&If it's commanded by the gods because it's moral, then the morality of actions is not determined by the gods (we don't need to reference them to figure out the morality of actions).\\
    So, either way, I am innocent of this charge.&  So, either morality is arbitrary or it's not determined by the gods.\\
\end{tabular}

The argument regarding piety can also work if you replace `the gods' with `God'. If you are interested in that particular argument, look into the Euthyphro Dilemma.