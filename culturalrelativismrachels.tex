\chapter{The Challenge of Cultural Relativism by James Rachels}\autocite{Rachels1}
\label{challengerelativism}
\newcounter{fz}
\setcounter{fz}{\thefootnote}
\setcounter{footnote}{0}

\factoidbox{``Morality  differs  in  every  society,  and  is  a  convenient  term  for 
socially  approved habits.” \autocite{RuthBenedict1}}

\section{2.1 How Different Cultures Have Different Moral Codes} 
Darius, a king of ancient Persia, was intrigued by the variety of cultures 
he  encountered  in  his  travels.  He  had  found,  for  example,  that  the 
Callatians  (a  tribe  of  Indians)  customarily  ate  the  bodies  of  their  dead 
fathers.  The  Greeks,  of  course,  did  not  do  that—the Greeks  practiced 
cremation and regarded the funeral pyre as the natural and fitting way to 
dispose  of  the  dead.  Darius  thought  that  a  sophisticated  understanding 
of  the  world  must  include  an  appreciation  of  such  differences  between 
cultures. One day, to teach this lesson, he summoned some Greeks who 
happened  to  be  present  at  his  court  and  asked  them  what  they  would 
take  to  eat  the  bodies  of  their  dead  fathers.  They  were  shocked,  as 
Darius knew they would be, and replied that no amount of money could 
persuade them to do such a thing. Then Darius called in some 
Callatians,  and  while  the  Greeks  listened  asked  them  what  they  would 
take to burn their dead fathers' bodies. The Callatians were horrified and 
told Darius not even to mention such a dreadful thing.

This  story,  recounted  by  Herodotus  in  his  History  illustrates  a  recurring 
theme in the literature of social science: Different cultures have different 
moral  codes.  What  is  thought  right  within  one  group  may  be  utterly 
abhorrent  to the members  of another  group, and  vice versa. Should  we 
eat  the  bodies  of  the  dead  or  burn  them?  If  you  were  a  Greek,  one 
answer  would  seem  obviously  correct;  but  if  you  were  a  Callatian,  the 
opposite would seem equally certain. 

It  is  easy  to  give  additional  examples  of  the  same  kind.  Consider  the 
Eskimos.  They  are  a  remote  and  inaccessible  people.  Numbering  only 
about  25,000,  they  live  in  small,  isolated  settlements  scattered  mostly 
along  the  northern  fringes  of  North  America  and  Greenland.  Until  the 
beginning  of  the  20th  century,  the  outside  world knew  little  about  them. 
Then explorers began to bring back strange tales. 

Eskimos customs turned out to be very different from our own. The men 
often  had  more  than  one  wife,  and  they  would  share  their  wives  with 
guests,  lending  them  for  the  night  as  a  sign  of  hospitality.  Moreover, 
within  a  community,  a  dominant  male  might  demand  and  get  regular 
sexual access to other men's wives. The women, however, were free to 
break  these  arrangements  simply  by  leaving  their  husbands  and  taking 
up  with  new  partners—free,  that  is,  so  long  as  their  former  husbands  
chose  not  to  make  trouble.  All  in  all,  the  Eskimo  practice  was  a  volatile 
scheme that bore little resemblance to what we call marriage. 

But  it  was  not only  their  marriage  and sexual  practices  that  were 
different.  The  Eskimos also  seemed  to  have less  regard for  human  life. 
Infanticide,  for  example,  was  common.  Knud  Rasmussen,  one  of  the 
most famous early explorers, reported that be met one woman who bad 
borne  20  children  but had  killed  10  of  them  at  birth.  Female  babies,  he 
found,  were  especially  liable  to  be  destroyed,  and  this  was  permitted 
simply at the parents' discretion, with no social stigma attached to it. Old 
people  also,  when  they  became  too  feeble  to  contribute  to  the  family, 
were left out in the snow .to die. So there seemed to be, in this society, 
remarkably little respect for life. 

To the general public, these were disturbing revelations. Our own way of 
living  seems  so  natural  and  right  that  for  many  of  us  it  is  hard  to 
conceive of  others  living  so  differently.  And  when  we  do  hear  of  such 
things, we tend immediately to categorize those other peoples as 
``backward"  or  ``primitive."  But  to  anthropologists  and  sociologists,  there 
was nothing particularly surprising about the Eskimos. Since the time of 
Herodotus,  enlightened  observers  have  been  accustomed  to  the  idea 
that  conceptions  of  right  and  wrong  differ  from  culture  to  culture.  If  we 
assume that our ideas of right and wrong will be shared by all peoples as 
all times, we are merely naive.

\section{2.2 Cultural Relativism} 
To  many  thinkers,  this  observation—``Different  cultures  have  different 
moral  codes"— has  seemed  to  be  the  key  to  understanding  morality. 
The idea of universal truth in ethics, they say, is a myth. The customs of 
different societies are all that exist. These customs cannot be said to be 
``correct" or ``incorrect," for that implies we have an independent standard 
of  right  and  wrong  by  which  they  may  be  judged.  But  there  is  no  such 
independent standard; every standard is culture-bound. The great 
pioneering  sociologist  William  Graham  Sumner, writing in  1906,  put  the 
point like this: 
\factoidbox{The  ``right"  way  is  the  way  which  the  ancestors  used  and  which 
has  been handed down. The tradition  is its own warrant. It  is not 
held subject to verification by experience. The notion of right is in 
the folkways. It is not outside of them, of independent origin, and 
brought to test them. In the folkways, whatever is, is right. This is 
because they are traditional, and therefore contain in themselves 
the  authority  of  the  ancestral  ghosts.  When  we  come  to  the 
folkways we are at the end of our analysis}

This line of thought has probably persuaded more people to be skeptical 
about  ethics  than  any  other  single  thing.  Cultural  Relativism,  as  it  has 
been called, challenges our ordinary belief in the objectivity and 
universality  of  moral  truth.  It  says,  in  effect,  that  there  is  not  such  thing 
as universal truth in ethics; there are only the various cultural codes, and 
nothing more. Moreover, our own code has no special status; it is merely 
one among many. 

As we shall see, this basic idea is really a compound of several different 
thoughts.  It  is  important  to  separate  the  various  elements  of  the  theory 
because,  on  analysis,  some  parts  turn  out  to  be  correct,  while  others 
seem  to  be mistaken.  As  a  beginning,  we may  distinguish  the  following 
claims, all of which have been made by cultural relativists: 

\begin{enumerate}
\item[1] Different societies have different moral codes.
\item[2] There  is  no  objective  standard  that  can  be  used  to  judge 
one societal code better than another.
\item[3] The moral code of our own society has no special status; it 
is merely one among many. 
\item[4] There  is no ``universal truth"  in ethics; that  is, there are no 
moral truths that hold for all peoples at all times. 
\item[5] The moral code of a society determines what is right within 
that society; that is, if the moral code of a society says that 
a  certain  action  is  right,  then  that  action  is  right,  at  least 
within that society. 
\item[6] It  is  mere  arrogance  for  us  to  try  to  judge  the  conduct  of 
other  peoples.  We  should  adopt  an  attitude  of  tolerance 
toward the practices of other cultures.
\end{enumerate}

Although  it  may  seem  that  these  six  propositions  go  naturally  together, 
they  are  independent  of  one  another,  in  the  sense  that  some  of  them 
might  be  false  even  if  others  are  true.  In  what  follows,  we  will  try  to 
identify what  is  correct in Cultural Relativism, but  we will  also be 
concerned to expose what is mistaken about it.

\section{2.3 The Cultural Differences Argument} 
Cultural Relativism is a theory about the nature of morality. At first blush 
it  seems  quite  plausible.  However,  like  all  such  theories,  it  may  be 
evaluated  by  subjecting  it  to  rational  analysis;  and  when  we  analyze 
Cultural Relativism we find that it is not so plausible as it first appears to 
be. 

The first thing we need to notice is that at the heart of Cultural Relativism 
there  is  a  certain  form  of  argument.  The  strategy  used  by  cultural 
relativists  is  to  argue  from  facts  about  the  differences  between  cultural 
outlooks to a conclusion about the status of morality. Thus we are invited 
to accept this reasoning: 
\begin{enumerate}
\item[1] The Greeks believed it was wrong to eat the dead, 
whereas  the  Callatians  believed  it  was  right  to  eat  the 
dead. 
\item[2] Therefore,  eating  the  dead  is  neither  objectively  fight  nor 
objectively  wrong.  It  is  merely  a  matter  of  opinion,  which 
varies from culture to culture. 
\end{enumerate}
Or, alternatively:
\begin{enumerate} 
\item[1] The  Eskimos  see  nothing  wrong  with  infanticide,  whereas 
Americans believe infanticide is immoral. 
\item[2] Therefore, infanticide is neither objectively right nor 
objectively  wrong.  It  is  merely  a  matter  of  opinion,  which 
varies from culture to culture. 
\end{enumerate}
Clearly,  these  arguments  are  variations  of  one  fundamental  idea  They 
are both special cases of a more general argument, which says: 
\begin{enumerate}
\item[1] Different cultures have different moral codes. 
\item[2] Therefore,  there  is  no  objective  ``truth"  in  morality.  Right 
and  wrong  are  only  matters  of  opinion,  and  opinions  vary 
from culture to culture. 
\end{enumerate}
We may call this  the Cultural Differences  Argument. To many people, it 
is persuasive. But from a logical point of view, is it sound?

It  is  not  sound.  The  trouble  is  that  the  conclusion  does  not  follow  from 
the  premise— that  is,  even  if  the  premise  is  true,  the  conclusion  still 
might  be  false.  The  premise  concerns  what  people  believe.  In  some 
societies,  people  believe  one  thing;  in  other  societies,  people  believe 
differently.  The  conclusion,  however,  concerns  what  really  is  the  case. 
The trouble is that this sort conclusion does not follow logically from this 
sort of premise. 

Consider  again  the  example  of  the  Greeks  and  Callatians.  The  Greeks 
believed  it  was  wrong  to  eat  the  dead;  the  Callatians  believed  it  was 
right. Does it follow, from the mere fact that they disagreed, that there is 
no objective truth in the matter? No, it does not follow; for it could be that 
the practice was objectively right (or wrong) and that one or the other of 
them was simply mistaken. 

To make the point clearer, consider a different matter In some societies, 
people  believe  the  earth  is  flat  In  other  societies,  such  as  our  own,
people believe  the  earth  is  (roughly)  spherical.  Does  it  follow,  from  the 
mere  fact  that  people  disagree,  that  there  is  no  ``objective  truth"  in 
geography?  Of  course  not;  we  would  never  draw  such  a  conclusion 
because we realize that, in their beliefs about the world, the members of 
some societies might simply be wrong. There is no reason to think that if 
the world is round everyone must know it. Similarly, there is no reason to 
think that if there is moral truth everyone must know it. The fundamental 
mistake in the Cultural Differences Argument is that it attempts to derive 
a substantive conclusion about a subject from the mere fact that people 
disagree about it. 

This is a simple point of logic, and it is important not to misunderstand it. 
We are not saying (not yet, anyway) that the conclusion of the argument 
is false. It is still an open question whether the conclusion is true or false. 
The  logical  point  is  just  that  the  conclusion  does  not  follow  from  the 
premise.  This  is  important,  because  in  order  to  determine  whether  the 
conclusion is true, we need arguments in its support. Cultural Relativism 
proposes  this  argument,  but  unfortunately  the  argument  turns  out to  be 
fallacious. So it proves nothing.  

\section{2.4 The Consequences of Taking Cultural Relativism Seriously} 
Even  if  the  Cultural  Differences  Argument  is  invalid,  Cultural  Relativism 
might still be true. What would it be like if it were true? 

In the passage quoted above, William Graham Sumner summarizes the 
essence of Cultural Relativism. He says that there is no measure of right 
and wrong other than the standards of one's society: ``The notion of right 
is  in  the  folkways.  It  is  not  outside  of  them,  of  independent  origin,  and 
brought to test them. In the folkways, whatever is, is right." Suppose we 
took this seriously. What would be some of the consequences? 
\subsection{1. We could no longer say that the customs of other societies are morally 
inferior to our own.} 
This, of course, is one of the main points stressed by 
Cultural  Relativism. We  would  have to  stop condemning  other  societies 
merely  because  they  are  ``different:'  So  long  as  we  concentrate  on 
certain  examples,  such  as  the  funerary  practices  of  the  Greeks  and 
Callatians, this may seem to be a sophisticated, enlightened attitude. 

However,  we  would  also  be  stopped  from  criticizing  other,  less  benign 
practices. Suppose a society waged war on its neighbors for the purpose 
of taking slaves. Or suppose a society was violently anti-Semitic and its 
leaders  set  out  to  destroy the  Jews.  Cultural  Relativism  would  preclude 
us  from  saying that  either  of  these  practices  was  wrong. We  would  not 
even be able to say that a society tolerant of Jews is better than the anti-
Semitic society, for  that would imply some  sort of transcultural standard 
of  comparison.  The  failure  to  condemn  these  practices  does  not  seem 
enlightened;  on  the  contrary,  slavery  and  anti-Semitism  seem  wrong 
wherever they occur. Nevertheless, if we took Cultural Relativism 
seriously,  we  would  have  to  regard  these social  practices  as  also 
immune from criticism. 

\subsection{2. We could decide whether actions are right or wrong just by consulting 
the  standards of  our  society.}
Cultural  Relativism  suggests  a  simple test 
for  determining  what  is  right  and  what  is  wrong:  All  one  need  do  is  ask 
whether  the  action  is  in  accordance  with  the  code  of  one's  society. 
Suppose in 1975, a resident of South Africa was wondering whether his 
country's policy of apartheid—a rigidly racist system—was morally 
correct.  All  he  has  to  do  is  ask  whether  this  policy  conformed  to  his 
society's  moral  code.  If  it  did,  there  would  have  been  nothing  to  worry 
about, at least from a moral point of view. 

This  implication  of  Cultural  Relativism  is  disturbing  because  few  of  us 
think that our society's code is perfect; we can think of ways it might be 
improved. Yet Cultural Relativism would not only forbid us from criticizing 
the  codes  of  other  societies;  it  would  stop  us  from  criticizing  our  own. 
After  all,  if  right  and  wrong  are  relative  to  culture,  this  must  be  true  for 
our own culture just as much as for other cultures. 

\subsection{3. The idea of moral progress is called into doubt.}
Usually, we think that at  least  some  social  changes  are  for  the  better.  (Although,  of  course, 
other  changes  may  be  for  the  worse.)  Throughout  most  of  Western 
history the place of women in society was narrowly circumscribed. They 
could  not  own  property;  they  could  not  vote  or  hold  political  office;  and 
generally they were under the almost absolute control of their husbands. 
Recently  much  of  this  has  changed,  and  most  people  think  of  it  as 
progress.

If  Cultural  Relativism  is  correct,  can  we  legitimately  think  of  this  as 
progress? Progress means replacing a way of doing things with a better 
way.  But  by  what  standard  do  we  judge  the  new  ways as  better?  If  the 
old ways were in accordance with the social standards of their time, then 
Cultural  Relativism  would  say  it  is  a  mistake  to  judge  them  by  the 
standards of a different time. Eighteenth-century society was, in effect, a 
different society from the one we have now. To  say that  we have made 
progress implies  a  judgment  that  present-day society  is better,  and that 
is  just  the  sort  of  transcultural  judgment  that,  according  to  Cultural 
Relativism, is impermissible. 

Our  idea  of  social  reform  will  also  have  to  be  reconsidered.  Reformers 
such as Martin Luther King, Jr., have sought to change their societies for 
the better. Within the constraints imposed by Cultural Relativism, there is 
one way this might be done. If a society is not living up to its own ideals, 
the  reformer  may  be  regarded  as  acting  for  the  best:  The  ideals  of  the 
society  are  the  standard  by  which  we  judge  his  or  her  proposals  as 
worthwhile. But the ``reformer" may not challenge the ideals themselves, 
for those ideals are by definition correct. According to Cultural 
Relativism,  then,  the  idea  of  social  reform  makes  sense  only  in  this 
limited way. 

These three consequences of Cultural Relativism have led many 
thinkers  to reject it as implausible on  its face. It does make  sense, they 
say,  to  condemn  some  practices,  such  as  slavery  and  anti-Semitism, 
wherever  they  occur.  It  makes  sense  to  think  that  our  own  society  has 
made some moral progress, while admitting that it is still imperfect and in 
need  of  reform.  Because  Cultural  Relativism says  that  these  judgments 
make no sense, the argument goes, it cannot be right. 

\section{2.5 Why There Is Less Disagreement Than It Seems} 
The original impetus for Cultural Relativism comes from the observation 
that cultures differ dramatically in their views of right and wrong. But just 
how much do they differ? It is true that there are differences. However, it 
is easy to overestimate  the extent of those differences, Often, when we 
examine  what  seems  to  be  a  dramatic  difference,  we  find  that  the 
cultures do not differ nearly as much as it appears. 

Consider a culture in which people  believe it is wrong to eat cows. This 
may even be a poor culture, in which there is not enough food; still, the 
cows are not to be touched. Such a society would appear to have values 
very  different  from  our  own.  But  does  it?  We  have  not  yet  asked why 
these  people  will not  eat  cows. Suppose  it  is because  they believe  that 
after death the souls of humans inhabit the bodies of animals, especially 
cows, so  that  a  cow  may be  someone's  grandmother.  Now do  we want 
to  say  that  their  values  are  different  from  ours?  No;  the  difference  lies 
elsewhere. The difference is in our belief systems, not in our values. We 
agree that we shouldn't eat Grandma; we simply disagree about whether 
the cow is (or could be) Grandma. 

The point is that many factors work together to produce the customs of a 
society. The  society's values are  only one  of them. Other matters, such 
as the religions and factual beliefs held by its members, and the physical 
circumstances  in  which  they  must  live,  are  also  important.  We  cannot 
conclude, then, merely because customs differ, that there is a 
disagreement about values. The difference in customs may be 
attributable to some other aspects of social life. Thus there may be less 
disagreement about values than there appears to be. 

Consider  again the  Eskimos,  who  often  kill  perfectly  normal  infants, 
especially girls. We do not approve of such things; a parent who killed a 
baby  in  our  society  would  be  locked  up.  Thus  there  appears  to  be  a 
great  difference  in  the  values  of  our  two  cultures.  But  suppose  we  ask 
why  the  Eskimos  do  this.  The  explanation  is  not  that  they  have  less 
affection  for  their  children  or  less  respect  for  human  life.  An  Eskimo 
family will always protect its babies if conditions permit. But they live in a 
harsh environment, where food is in short supply. A fundamental 
postulate  of  Eskimos  thought  is:  ``Life  is  hard,  and  the  margin  of  safety 
small.” A family may want to nourish its babies but be unable to do so. 
As  in many ``primitive" societies,  Eskimo mothers will nurse  their  infants 
over a much longer period of time than mothers in our culture. The child 
will  take  nourishment  from  its  mother's  breast  for  four  years,  perhaps 
even longer. So even in the best of times there are limits to the number 
of  infants  that  one  mother  can  sustain.  Moreover,  the  Eskimos  are  a 
nomadic  people—unable  to  farm,  they  must  move  about  in  search  of 
food.  Infants  must  be carried,  and  a  mother  can  carry only  one  baby in 
her parka as she travels and goes about her outdoor work. Other family 
members help whenever they can. 

Infant girls are more readily disposed of because, first, in this society the 
males are the primary food providers—they are the hunters, according to 
the traditional division of labor—and it is obviously important to maintain 
a  sufficient  number  of  food  providers.  But  there  is  an  important  second 
reason as well. Because the hunters suffer a high casualty rate, the adult 
men who die prematurely far outnumber the women who die early. Thus 
if  male  and  female  infants  survived  in  equal  numbers,  the  female  adult 
population would greatly outnumber the male adult population. 
Examining the available statistics, one writer concluded that ``were it not 
for female infanticide...there  would  be  approximately  one-and-a-half 
times  as  many  females  in  the  average Eskimo  local  group  as  there are 
food-producing males." 

So  among  the  Eskimos,  infanticide  does  not  signal  a  fundamentally 
different  attitude  toward  children.  Instead,  it  is  a  recognition  that  drastic 
measures  are  sometimes  needed  to  ensure  the  family's  survival.  Even 
then, however, killing the baby is not the first option considered. 
Adoption  is  common;  childless  couples  are  especially  happy  to  take  a 
more fertile couple's ``surplus." Killing is only the last resort. I emphasize 
this  in  order  to  show  that  the  raw  data  of  the  anthropologists  can  be 
misleading;  it  can  make  the  differences  in  values  between  cultures 
appear  greater  than  they  are.  The  Eskimos'  values  are  not  all  that 
different from our values. It is only that life forces upon them choices that 
we do not have to make. 

\section{2.6 How All Cultures Have Some Values in Common} 
It  should  not  be  surprising  that,  despite  appearances,  the  Eskimos  are 
protective  of  their  children.  How  could  it  be  otherwise?  How  could  a 
group survive that did not value its young? It is easy to see that, in fact, 
all cultural groups must protect their infants: 
\begin{enumerate}
\item[1] Human  infants  are  helpless  and  cannot  survive  if  they are 
not given extensive care for a period of years. 
\item[2] Therefore,  if  a  group  did  not  care  for  its  young,  the  young 
would  not  survive,  and  the  older  members  of  the  group 
would  not  be  replaced.  After  a  while  the  group  would  die 
out. 
\item[3] Therefore,  any  cultural  group  that  continues  to  exist  must 
care for its young. infants that are not cared for must be the 
exception rather than the rule. 
\end{enumerate}
Similar reasoning shows that other values must be more or less 
universal. Imagine what it would be like for a society to place no value at 
all on truth telling. When one person spoke to another, there would be no 
presumption at all that he was telling the truth for he could just as easily 
be speaking falsely. Within that society, there would be no reason to pay 
attention  to  what  anyone  says.  (I  ask  you  what  time  it  is,  and  you  say 
``Four  o'clock:'  But  there  is  no  presumption  that  you  are  speaking  truly; 
you  could  just  as  easily  have  said  the  first  thing  that  came  into  your 
head. So I have no reason to pay attention to your answer; in fact, there 
was no point  in my asking you in the first place.) Communication would 
then  be  extremely  difficult,  if  not  impossible.  And  because  complex 
societies  cannot  exist  without  communication  among  their  members, 
society would become impossible. It follows that in any complex society 
there  must  be  a  presumption  in  favor  of  truthfulness.  There  may  of 
course be exceptions to  this rule: There may be  situations in which it is 
thought to be permissible to lie. Nevertheless, there will be exceptions to 
a rule that is in force in the society.
 
Here  is  one  further  example  of  the  same  type.  Could  a  society  exist  in 
which  there  was  no  prohibition  on  murder?  What  would  this  be  like? 
Suppose people were free to kill other people at will, and no one thought 
there  was anything  wrong  with  it.  In  such a  ``society,"  no  one  could feel 
secure.  Everyone  would  have  to  be  constantly  on  guard.  People  who 
wanted to survive would have to avoid other people as much as 
possible.  This  would  inevitably  result  in  individuals  trying  to  become  as 
self-sufficient  as  possible— after  all,  associating  with  others  would  be 
dangerous. Society on any large scale would collapse. Of course, people 
might  band  together  in  smaller  groups  with  others  that  they  could  trust 
not  to  harm  them.  But  notice  what  this  means:  They  would  be  forming 
smaller  societies  that  did  acknowledge  a  rule  against  murder:  The 
prohibition of murder, then, is a necessary feature of all societies. 

There  is  a  general  theoretical  point  here,  namely,  that  \emph{there  are  some 
moral  rules  that  all  societies  will  have  in  common,  because  those  rules 
are necessary for society to exist.} The rules against lying and murder are 
two  examples.  And  in  fact,  we  do  find  these  rules  in  force  in  all  viable 
cultures. Cultures may differ in what they regard as legitimate exceptions 
to  the  rules,  but  this  disagreement  exists  against  a  background  of 
agreement on the larger issues. Therefore, it is a mistake to 
overestimate the amount of difference between cultures. Not every moral 
rule can vary from society to society. 
\section{2.7 Judging a Cultural Practice to Be Undesirable} 
In  1996,  a  17-year-old  girl  named  Fauziya Kassindja  arrived  at  Newark 
International  Airport  and  asked  for  asylum.  She  had  fled  her  native 
country  of  Togo,  a  small  west  African  nation,  to  escape  what  people 
there call excision. 

Excision is a permanently disfiguring procedure that is sometimes called 
``female  circumcision," although it bears little resemblance to the Jewish 
ritual.  More  commonly,  at  least  in Western newspapers,  it is  referred  to 
as  ``genital  mutilation."  According  to  the World  Health  Organization,  the 
practice  is  widespread  in  26  African  nations,  and  two  million  girls  each 
year  are  ``excised."  In  some  instances,  excision  is  part  of  an  elaborate 
tribal ritual, performed in small traditional villages, and girls look forward 
to  it  because  it  signals  their  acceptance  into  the  adult  world.  In  other 
instances, the practice is carried out by families living in cities on young 
women who desperately resist. 

Fauziya  Kassindja  was  the  youngest  of  five  daughters  in  a  devoutly 
Muslim  family.  Her  father,  who  owned  a  successful  trucking  business, 
was opposed to excision, and he was able to defy the tradition because 
of his wealth. His first four daughters were married without being 
mutilated.  But  when  Fauziya  was  16,  he  suddenly  died.  Fauziya  then 
came under the authority of his father, who arranged a marriage for her 
and prepared to have her excised. Fauziya was terrified, and her mother 
and oldest sister helped her to escape. Her mother, left without 
resources, eventually had to formally apologize and submit to the 
authority of the patriarch she had offended. 

Meanwhile, in America, Fauziya was imprisoned for  two years while the 
authorities decided what to do with her. She was finally granted asylum, 
but  not  before  she  became  the  center  of  a  controversy  about  how 
foreigners should regard the cultural practices of other peoples. A series 
of articles in the New York Times encouraged the idea that excision is a 
barbaric  practice  that  should  be  condemned.  Other  observers  were 
reluctant  to  be  so  judgmental—live  and  let  live,  they  said;  after  all,  our 
practices probably seem just as strange to them. 

Suppose we are inclined to say that excision is bad. Would we merely be 
applying  the  standards  of  our  own  culture?  If  Cultural  Relativism  is 
correct,  that  is  all  we  can  do,  for  there  is  no  cultural-neutral  moral 
standard to which we may appeal. Is that true? 

\section{Is There a Culture-Neutral Standard of Right and Wrong?} 
There  is,  of  course,  a  lot  that  can  be  said  against  the  practice  of 
excision. Excision is painful and it results in the permanent loss of sexual 
pleasure. Its short-term effects include hemorrhage, tetanus, and 
septicemia. Sometimes the woman dies. Longterm effects include 
chronic infection, scars that hinder walking, and continuing pain. 

Why, then, has it become a widespread social practice? It is not easy to 
say. Excision has no obvious social benefits. Unlike Eskimo infanticide, it 
is  not  necessary  for  the  group's  survival.  Nor  is  it  a  matter  of  religion. 
Excision  is  practiced  by  groups  with  various  religions,  including  Islam 
and Christianity, neither of which commend it. 

Nevertheless, a number of reasons are given in its defense. Women who 
are incapable of sexual pleasure are said to be less likely  to be 
promiscuous; thus there will be fewer unwanted pregnancies in 
unmarried women. Moreover, wives for whom sex is only a duty are less 
likely  to  be  unfaithful  to  their  husbands;  and  because  they  will  not  be 
thinking  about  sex,  they  will  be  more  attentive  to  the  needs  of  their 
husbands  and  children.  Husbands,  for  their  part,  are  said  to  enjoy  sex 
more  with  wives  who  have  been  excised.  (The  women's  own  lack  of 
enjoyment  is  said  to  be  unimportant.)  Men  will  not  want  unexcised 
women,  as  they  are  unclean  and  immature.  And  above  all,  it  has  been 
done since antiquity, and we may not change the ancient ways. 

It would be easy, and perhaps a bit arrogant, to ridicule these 
arguments. But we may notice an important feature of this whole line of 
reasoning:  it  attempts  to  justify  excision  by  showing  that  excision  is 
beneficial— men,  women,  and their  families  are  all  said  to be  better  off 
when women are excised. Thus we might approach this reasoning, and 
excision itself, by asking which is true: Is excision, on the whole, helpful 
or harmful? 

Here,  then,  is  the  standard  that  might  most  reasonably  be  used  in 
thinking  about  excision:  We  may  ask  whether  the  practice  promotes  or 
hinders the welfare of the people whose lives are affected by it. And, as 
a corollary, we may ask if there is an alternative set of social 
arrangements that would do a better job of promoting their welfare. If so, 
we may conclude that the existing practice is deficient. 

But  this  looks  like  just  the  sort  of  independent  moral  standard  that 
Cultural Relativism says cannot exist. It is a single standard that may be 
brought  to  bear  in  judging  the practices  of  any  culture,  at  any  time, 
including our own. Of course, people will not usually see this principle as 
being ``brought in from the outside" to judge them, because, like the rules 
against lying and homicide, the welfare of its members is a value internal 
to all viable cultures. 

\section{Why Thoughtful People  May Nevertheless  Be Reluctant to Criticize 
Other Cultures.}

Although  they  are  personally  horrified  by  excision,  many  thoughtful 
people  are  reluctant  to  say  it  is  wrong,  for  at  least  three  reasons.  First, 
there  is  an  understandable  nervousness  about  ``interfering  in  the  social 
customs of other peoples." Europeans and their cultural descendents in 
America have a shabby history of destroying native cultures in the name 
of Christianity and Enlightenment, not to mention self-interest. Recoiling 
from  this  record,  some  people  refuse  to  make  any  negative  judgments 
about  other  cultures,  especially  cultures  that  resemble  those  that  have 
been  wronged  in  the  past.  We  should  notice,  however,  that  there  is  a 
difference between (a) judging a cultural practice to be morally deficient 
and (b) thinking that we should announce the fact, conduct a campaign, 
apply diplomatic pressure, or send in the army to do something about it. 
The  first  is  just  a  matter  of  trying to  see  the world clearly,  from  a  moral 
point  of  view.  The  second  is  another  matter  altogether.  Sometimes  it 
may be right to ``do something about it," but often it will not be. 

People  also  feel,  rightly  enough,  that  they  should  be  tolerant  of  other 
cultures. Tolerance is, no doubt, a virtue—a tolerant  person is willing to 
live  in  peaceful  cooperation  with  those  who  see  things  differently.  But 
there is nothing in the nature of tolerance that requires you to say that all 
beliefs,  all  religions,  and  all  social  practices  are  equally  admirable.  On 
the contrary, if you did not think that some were better than others, there 
would be nothing for you to tolerate. 

Finally,  people  may  be  reluctant  to  judge  because  they  do  not  want  to 
express  contempt  for  the  society  being  criticized.  But  again,  this  is 
misguided: To condemn a particular practice is not to say that the culture 
is  on  the  whole  contemptible  or  that  it  is  generally  inferior  to  any  other 
culture,  including  one's  own.  It  could  have  many  admirable  features.  In 
fact, we should expect this to be true of most human societies— they are 
mixes of good and bad practices. Excision happens to be one of the bad 
ones.

\setcounter{footnote}{\thefz}
