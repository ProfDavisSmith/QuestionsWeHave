\chapter{Are Dualism and Libertarianism Compatible? by Davis Smith}
\label{libertarianismdualism}
\section{By Davis Smith}
\section{Abstract}
\newcounter{fb}
\setcounter{fb}{\thefootnote}
\setcounter{footnote}{0}

This paper concerns two seemingly unrelated areas of Metaphysics, the Free Will Debate and the Mind-Body Problem. I will be arguing that Libertarianism and Substance Dualism are not compatible. The argument goes like so:  First, if Libertarianism is true then there are some actions such that the actor could have physically done otherwise and which the actor had control over. Second, if Dualism is true, then there are three places where there could be the indeterminacy necessary for Libertarianism. Third, all three of these decrease the actor’s control over their actions.  Therefore, if Dualism is true, then Libertarianism is false.

\section{Introduction}
Metaphysics is a very wide-reaching field and as such one would think that the conclusions one reaches in a far distant area would have little impact on the conclusions one could reasonably hold in a closer one. While this might be true for some areas of Metaphysics, it does not seem to hold for the Free Will Debate and the Mind-Body Problem. In this paper, I will be arguing that Libertarianism and Dualism are not compatible.\footnote{I admit that this is counter-intuitive to some. There have been a few papers in Cognitive Science and Experimental Philosophy which point to a correlation between believing in Dualism and believing in Libertarianism. \cite{Wisniewski1} for one such study. While I have no issue with Dualism, for the purposes of this paper, the issues I have are with Libertarianism.}  In a previous paper\footnote{Davis Smith, “Are Libertarianism and Physicalism Compatible?”, unpublished, presented at the 72nd Northwest Philosophy Conference.  My main intent for this paper is to ultimately merge it with the Physicalism one, showing that Libertarianism is incompatible with the mind, whatever it may be.}, I showed that Libertarianism and Physicalism are contradictory, so here, using a similar methodology, I will show that Libertarianism and Dualism are contradictory. To give a brief overview of the argument:  First, if Libertarianism is true then there are some actions such that the actor could have physically done otherwise and which the actor had control over. Second, if Dualism is true, then there are three places where there could be the indeterminacy necessary for Libertarianism. Third, all three of these decrease the actor’s control over their actions.  Therefore, if Libertarianism is true, then Dualism is false. We will start by being particular about the meanings of the terms central to this proof: Determinism, Dualism, and Libertarianism.
\section{Terms}
\subsection{Determinism}
\begin{center}An event A is deterministic if and only if, given the state of the world and the laws of nature\footnote{“Given the state of the world and the laws of nature” is meant to be neutral in regards to physicalism vs dualism.}  prior to A, A will occur necessarily.\footnote{Put in a more technical and precise way, this means that an event A (at t at w) is deterministic if and only if there is no possible world, w’, such that the conditions of the world prior to t at w’ and the laws of nature at t at w’ are the same as those prior to and at t at w and A does not occur at t at w’.}\end{center} 
Basically, for all possible worlds with the exact same laws of nature and they have the exact same past events leading up to the event, then the event will occur in all the possible worlds, in the same way and at the same time. The opposite of this is to say that an event is non-deterministic, and we can define this like so:
\begin{center}An event A is non-deterministic if and only if, given the state of the world and the laws of nature prior to A, it is possible for A not to occur.\footnote{Put in a more technical way, we have that event A (at t at w) is non-deterministic if and only if there’s possible world, w’, such that the conditions of the world prior to t at w’ and the laws of nature at t at w’ are the same as those prior to and at t at w and A does not occur at t at w’.}\end{center} 
This is not to say that the event will not occur, but rather it is saying that it is possible that the event will not occur. There are two ways in which an event could be non-deterministic. First, the event could be the probabilistic result of a preceding cause or it could be completely uncaused. In the case of the former, the laws of nature would need to be in such a way that for at least some sets of circumstances, the probability of the event occurring is less than 1. The probability of the event occurring could be very close to 1, but if it is less than that, there is still a chance that the event does not occur. A totally uncaused event, on the other hand, would be one in which the preceding events in no way determine or make more likely that it occurs, such an event would be, for all intents and purposes, random.

With these two definitions, we can define ‘Determinism' in the following way: Determinism is the stance that all events in the actual world are deterministic. If Determinism is accurate, then it follows that the laws of nature do not contain randomness and that the preceding events predict with 100\% accuracy future events.\footnote{This way of defining ‘Determinism’ is used in several places, but \cite[most notably in][ ]{Popper1}.}  Some point to the seemingly random events at a quantum level as evidence against determinism, but some recent works\footnote{\cite{Barrett1}}  show that it is impossible to determine whether any notion of randomness can characterize the data in Quantum Mechanics.\footnote{I will often use the phrase ‘deterministic universe’ which means that Determinism is true at the possible world in question. Similarly, ‘non-deterministic universe’ means that Determinism is not true at the possible world in question.}
\subsection{Dualism}
Dualism a stance in the Mind-Body Problem which says that rather than there being one kind of substance in the world, there are two, namely mental and physical. Under many normal interpretations of this stance, the physical substance is the physical body while the mental substance is the immaterial ‘soul’, this houses our feelings, sensations, thoughts, and the ‘what-it’s-like-ness’. Some Dualist theories hold that the mental and physical do not interact, there is no causal relationship between them. It assumes Determinism because it would be very contrary to our experience otherwise. We do not need to spend much time thinking about it for this project. Other Dualist theories, however, do posit that the substances interact. While there is an open question about how this could be, we can assume that they do interact for our purposes. 
\subsection{Libertarianism}
Libertarianism holds that Determinism and responsibility for our actions are not compatible and holds that Determinism is false, so there is responsibility for our actions. From this, it states that there are some non-deterministic events. In general, being responsible for an action implies at least two aspects (according to Libertarianism):
\begin{enumerate}
\item The doer must have physically been able to do otherwise.
\item The events which cause the action which the doer is responsible for must be non-deterministic.\footnote{Mark Balaguer, who we will encounter later, claims that the indeterminacy must increase the control, a concept which we will explore in Part 1 of the argument. \cite{Balaguer2}} 
\end{enumerate}
When it comes to the first feature of Libertarianism, this is the Principle of Alternative Possibilities\footnote{\cite[As presented in][ ]{Frankfurt1}}, but we need to be careful to include the term ‘physically'. Without it, the statement could be one which a Compatibilist would approve of. They would say that, sure, you could do otherwise, but you physically are not able to. Though the Frankfort-style cases found in the literature are convincing to some that this is not an essential feature of responsibility,\footnote{We will see a Frankfurt-style case in Part 1 of the argument.}  I believe that they would be unconvincing to a sensible Libertarian.\footnote{This is because the passive coercion in the Frankfort-style cases is quite like the indirect causal pressures which the past and the laws of nature have on our choices in a deterministic universe. If they reject the possibility of moral responsibility in a deterministic universe, then they would equally need to reject the possibility of moral responsibility in a Frankfort-style case. For an alternative argument \cite[see][ ]{Widerker1}.}  The Libertarians want the ability to physically do otherwise. Similarly, if the events which cause our actions are deterministic, then it would not be possible for us to do otherwise, which means that we would not be responsible for them. Both of these, however, have reference to responsibility. Responsibility, as I will argue later, has at least two features which I believe contradicts these two features. 

Most generally, Libertarians can be divided into two general teams. On one side we have the Agent-Causal Libertarians and on the other we have the Event-Causal Libertarians. Both teams agree that there are at least some actions which we are responsible for, they differ in how they think this functions. Agent-Causal Libertarians hold that the doer has the ability to be the first cause of a chain of events. These Libertarians hold that when we act freely, we are causing something to be without ourselves being caused.\footnote{\cite[687]{Franklin1} Some could see a similarity between this sort of idea and theological considerations. Some hold that God is the first cause of our universe and that God was not caused. Some also hold that God created man in His image. As a result, some could jump to the conclusion that God created us with the ability to be a first cause.}  I hold that Agent-Causal Libertarianism requires Substance Dualism, in other words, it is not possible to have this ability to be the ‘first cause’ without mental substances.\footnote{The primary purpose of this paper is to show that all forms of Libertarianism, as I have constructed it, are incompatible with dualism. Agent-Causal Libertarianism seems a little more brazen about claiming that it requires dualism and the rejection of physicalism. If Agent-Causal Libertarianism does require dualism, then this paper serves as a proof that the stance is contradictory.}  Event-Causal Libertarians take on a more metaphysically modest stance. These theorists generally have the best aspects of the compatibilists’ theories but add the claim that there must be indeterminacy in the action. All Libertarians need to be able to show that the indeterminacy adds something to the control which is lacked in a Compatibilist model.\footnote{I will argue in Part 2 that the indeterminacy detracts from control and thereby responsibility. The more indeterminacy we add to a given model, the less responsibility the agent has.}
  
\section{The Argument}
This argument is broken into three parts, which come together at the end to show that Dualism and Libertarianism are not compatible. 
\subsection{Part 1: If Libertarianism is true, then there are some actions such that the actor could have physically done otherwise and had control over.} 
Proving this line of the argument requires us to show two different things; both of which are derived from a basic understanding of ‘responsibility'. From the very definition of Libertarianism, there must be some responsibility for at least some of our actions. But what exactly does it mean to have moral responsibility for some of our actions?

It seems that there are at least two necessary features for a doer to be responsible for their action, and these are the two features of the above conditional. First, the doer must have been able to do otherwise.\footnote{And this must be a physical ability, not a metaphysical one, according to Libertarianism. So, the doer must have had the ability to physically refrain from doing it or do something other than what they did.}  As an example, take this thought experiment:\footnote{This is a case similar to the ones \cite[found in][ ]{Frankfurt1} \cite[and][ ]{Fischer1}.}
\factoidbox{Suppose that a meteor crashed in a field near your home and psychic alien spores escaped filling the air. Once some spores choose a host, they begin psychically implanting thoughts and take over their mind. Sometimes, the actions which the host would do are the same as those which the spores would cause them to do, however, when they are not, the spores implant thoughts which make the host do as they would want.} 
Since the host could not do otherwise, they are not responsible for their actions (as even if it were their choice, the spores would stop them from making a different one). When we think of cases where a person is forced or coerced to do something, then we are less inclined to hold them responsible for their actions. In the above case, the alien spores are passively forcing the hosts to do actions, so it seems clear that they are at least less responsible for the actions.

The other necessary feature for responsibility is a sense of control or ‘up-to-us-ness.’\footnote{This control requires that the doer had chosen to make the action and directed it to the outcome. Making a choice, it would seem, needs to be both voluntary and with deliberation.} For example, look at this case:
\factoidbox{Suppose that you are at a dinner party having a discussion with various people and sipping wine. During a deep discussion on whether Libertarianism is compatible with Physicalism, you have a muscle spasm which launches the wine all over another’s white cloths.}
Whether the universe in this thought experiment is deterministic is not relevant. You have no control over what happens when you have a muscle spasm like this and the lack of control serves as an excuse to relinquish responsibility for the destruction of another person's clothing.\footnote{For more on this, \cite[see][ ]{Austin1}}  One could also characterize this as an event which was not ultimately up to you. This means that the ability-to-do-otherwise is not enough for responsibility. While it’s possible for you not to have that spasm, but the action must be up to you.
\subsection{Part 2: If Dualism is true, then there are three places where the indeterminacy could appear in an actor’s action; either A) in the physical substance, B) in the mental substance, or C) in the interaction between the mental and the physical substances.}
In the previous part, we looked closely at what it takes to be responsible for our actions; the physical ability to do otherwise and control or ‘up-to-us-ness’. For us to be able to do otherwise, in the Libertarian sense, the action or the events leading to the action\footnote{There could have been an event in the distant past which was not deterministic which lead to the train of event leading to the doer’s action, but such an event seems hardly relevant to the question of whether the doer could have done otherwise. For another example, suppose that one person had the ability to do otherwise while another did not. In their interactions, the free person caused the determined person’s actions and the determined person could have, in a sense, done otherwise, but the determined person’s actions were determined by the free person’s.}  cannot be deterministic, meaning that there must be some indeterminacy in the actual process of making an action. It follows, then, that there are three possible places where the indeterminacy could occur: in the physical substance, in the mental, or in their interactions. Call these, in order, \emph{physical-indeterminacy}, \emph{mental-indeterminacy}, and \emph{interaction-indeterminacy}. Of these, the indeterminacy therein could be totally uncaused or a probabilistic result of previous events. In the following sub-parts, I will show that each of these diminish or eliminate the doer’s control. 
\subsubsection{Part 2A: If there is physical-indeterminacy, then it decreases the control the actor has over their actions.}
In a previous paper, I argued that Physicalism and Libertarianism are incompatible on the grounds that physical-indeterminacy diminishes the doer’s control or the ‘up-to-us-ness’. This holds true here as well, for the same reasons. For example, take the following thought experiment:
\factoidbox{A man is standing in front of a switch, there is a run-away trolley. If the man does nothing, then the trolley will kill 5 people, if they flip the switch, then the trolley will only kill one person. Since this is a non-deterministic universe, there is a 50\% chance that they will flip the switch and a 50\% chance that they will do nothing.}
In this case, the core difference between it and other trolley problems is that there is this element of chance.\footnote{The percent likelihood of the two different possibilities is, I would think, a worst-case scenario for the responsibility of the man in question. Other ratios of possibility are useable.}  In a Dualistic universe, we can suppose that the mental substance had deliberated and, because of the past experiences and events in their life, choose to flip the switch. Since there is physical-indeterminacy, whether to flip the switch was not up to the man.\footnote{It could be generalized that in an extreme case like this, the doer’s reasons and feelings do not play a part in the choice and thereby make it up to him. A similar line of thought, though about artificial intelligence, can be \cite[found in][ ]{Searle1}}.  Generalizing this, if there is any physical-indeterminacy relevant to our actions, then our control is diminished by it. 

In the case of a totally uncaused physical event leading to the action, take this thought experiment:\footnote{This case is similar in form to the ones \cite[seen in][ ]{Elzein1}, but the thought experiments there are used to show that the principle of alternative possibilities is important to moral responsibility and that the alternative possibilities must be relevant to the case.}
\factoidbox{One evening, Jones is sitting back to watch a little television when a quantum particle appears in his brain and then vanishes. This event causes a chain reaction which results in him choosing to begin growing edible mushrooms. Acting on this choice, he becomes very knowledgeable about the subject and eventually he discovers a new form of hypoallergenic penicillin, saving countless lives.}  
For ease of use, we can assume that the appearance of the particle is the only non-deterministic event relevant to the case. Since the initial cause of the choice was completely random and the resulting events which lead to the discovery were the deterministic results of the initial event, we could hardly say that Jones was in control of the action and, thereby, responsible for saving the countless lives.
\subsubsection{Part 2B: If there is mental-indeterminacy, then it decreases the control the actor has over their actions.}
To illustrate is point, we can reuse the thought experiments from the previous part, with some minor alterations. The events in the mental substance could be either totally uncaused or probabilistic. For the probabilistic version, take the trolley problem case from before and think about it as indeterminism in the mental substance. In such a case, the person would not have control over their own thoughts and thereby their actions. They would be, in a sense, undisciplined. One could be faced with a choice where they are unsure what to do and deliberate about it. If the choice contains a hint of randomness, then that diminishes the control they had. Similarly, if the causal mental event is totally uncaused, then it would be very similar to the case involving hypoallergenic penicillin. The thought which spurs the action would be completely random and potentially radically out of character for Jones. This randomness further depreciates the responsibility Jones has. 

\subsubsection{Part 2C: If there is interaction-indeterminacy, then it decreases the control the actor has over their actions.} 
For the third and final place where this indeterminacy could be, take a look at the thought experiments once more. For probabilistic events in the interaction, it would be like a case where the mind chose to pull the leveler and there is a 50-50 shot about whether the body would get the correct message. If the body, magically, got the message, then they would have gotten lucky in regards to doing the action Often, when someone is trying to clam responsibility for an action or event, others will diminish that responsibility by claiming that they got lucky. If the indeterminacy required for the ability to do otherwise was in the interaction between the mind and the body, then it would be lucky that the body did what it was told. For totally uncaused events, this would be like the mental substance not giving an order to the body and the body magically getting an order. This too would be very troubling to how we could say that they are responsible for their actions. 
\subsection{Part 3: So, if Dualism is true, then indeterminacy decreases the control an actor has over their actions.}
This is simple enough to prove. I have shown that, if Dualism is true, then there are three places where the indeterminacy required for responsibility, according to Libertarianism, could appear. Each of these actually diminish or eliminate the control the doer has over their actions. This means that if Dualism is true, then any form of indeterminacy (relevant to how our actions are taken) does not increase the responsibility one may have but rather decreases it.
\section{Conclusion: If Dualism is true, then Libertarianism is false.}
To conclude this paper, I will outline what I have done. First, according to Libertarianism, Determinism is false and there are some actions which we are responsible for. These actions must be ones where we could have physically done otherwise (namely, there is a core aspect of indeterminacy) and those actions were within our control. Dualism allows for three places where the indeterminacy could appear: In the physical, mental, or in their interactions. I then went on to show that indeterminacy in the physical diminishes the control the doer has and it also does so when it is in the mental and also in their interaction. Libertarianism makes a central claim that this indeterminacy increases our responsibility for our actions, not decreases it. The implication from this is that if Dualism is true, then Libertarianism is false because in every place where the indeterminacy could be, it harms our responsibility, not help it.

\setcounter{footnote}{\thefb}

