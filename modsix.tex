\part{What is Knowledge? How Do We Know Things?}
\label{ch.modsix}
\addtocontents{toc}{\protect\mbox{}\protect\hrulefill\par}
\chapter{Meditations on First Philosophy by Ren\'e Descartes}\autocite{Descartes1}

Translated by John Veitch (1901)

%TO THE VERY SAGE AND ILLUSTRIOUS THE DEAN AND DOCTORS OF THE SACRED FACULTY OF THEOLOGY OF PARIS.

%GENTLEMEN,

%1. The motive which impels me to present this Treatise to you is so reasonable, and when you shall learn its design, I am confident that you also will consider that there is ground so valid for your taking it under your protection, that I can in no way better recommend it to you than by briefly stating the end which I proposed to myself in it.

%2. I have always been of the opinion that the two questions respecting God and the Soul were the chief of those that ought to be determined by help of Philosophy rather than of Theology; for although to us, the faithful, it be sufficient to hold as matters of faith, that the human soul does not perish with the body, and that God exists, it yet assuredly seems impossible ever to persuade infidels of the reality of any religion, or almost even any moral virtue, unless, first of all, those two things be proved to them by natural reason. And since in this life there are frequently greater rewards held out to vice than to virtue, few would prefer the right to the useful, if they were restrained neither by the fear of God nor the expectation of another life; and although it is quite true that the existence of God is to be believed since it is taught in the sacred Scriptures, and that, on the other hand, the sacred Scriptures are to be believed because they come from God (for since faith is a gift of God, the same Being who bestows grace to enable us to believe other things, can likewise impart of it to enable us to believe his own existence), nevertheless, this cannot be submitted to infidels, who would consider that the reasoning proceeded in a circle. And, indeed, I have observed that you, with all the other theologians, not only affirmed the sufficiency of natural reason for the proof of the existence of God, but also, that it may be inferred from sacred Scripture, that the knowledge of God is much clearer than of many created things, and that it is really so easy of acquisition as to leave those who do not possess it blameworthy. This is manifest from these words of the Book of Wisdom, chap. xiii., where it is said, Howbeit they are not to be excused; for if their understanding was so great that they could discern the world and the creatures, why did they not rather find out the Lord thereof? And in Romans, chap. i., it is said that they are without excuse; and again, in the same place, by these words,That which may be known of God is manifest in them--we seem to be admonished that all which can be known of God may be made manifest by reasons obtained from no other source than the inspection of our own minds. I have, therefore, thought that it would not be unbecoming in me to inquire how and by what way, without going out of ourselves, God may be more easily and certainly known than the things of the world.

%3. And as regards the Soul, although many have judged that its nature could not be easily discovered, and some have even ventured to say that human reason led to the conclusion that it perished with the body, and that the contrary opinion could be held through faith alone; nevertheless, since the Lateran Council, held under Leo X. (in session viii.), condemns these, and expressly enjoins Christian philosophers to refute their arguments, and establish the truth according to their ability, I have ventured to attempt it in this work.

%4. Moreover, I am aware that most of the irreligious deny the existence of God, and the distinctness of the human soul from the body, for no other reason than because these points, as they allege, have never as yet been demonstrated. Now, although I am by no means of their opinion, but, on the contrary, hold that almost all the proofs which have been adduced on these questions by great men, possess, when rightly understood, the force of demonstrations, and that it is next to impossible to discover new, yet there is, I apprehend, no more useful service to be performed in Philosophy, than if some one were, once for all, carefully to seek out the best of these reasons, and expound them so accurately and clearly that, for the future, it might be manifest to all that they are real demonstrations. And finally, since many persons were greatly desirous of this, who knew that I had cultivated a certain Method of resolving all kinds of difficulties in the sciences, which is not indeed new (there being nothing older than truth), but of which they were aware I had made successful use in other instances, I judged it to be my duty to make trial of it also on the present matter.

%5. Now the sum of what I have been able to accomplish on the subject is contained in this Treatise. Not that I here essayed to collect all the diverse reasons which might be adduced as proofs on this subject, for this does not seem to be necessary, unless on matters where no one proof of adequate certainty is to be had; but I treated the first and chief alone in such a manner that I should venture now to propose them as demonstrations of the highest certainty and evidence. And I will also add that they are such as to lead me to think that there is no way open to the mind of man by which proofs superior to them can ever be discovered for the importance of the subject, and the glory of God, to which all this relates, constrain me to speak here somewhat more freely of myself than I have been accustomed to do. Nevertheless, whatever certitude and evidence I may find in these demonstrations, I cannot therefore persuade myself that they are level to the comprehension of all. But just as in geometry there are many of the demonstrations of Archimedes, Apollonius, Pappus, and others, which, though received by all as evident even and certain (because indeed they manifestly contain nothing which, considered by itself, it is not very easy to understand, and no consequents that are inaccurately related to their antecedents), are nevertheless understood by a very limited number, because they are somewhat long, and demand the whole attention of the reader: so in the same way, although I consider the demonstrations of which I here make use, to be equal or even superior to the geometrical in certitude and evidence, I am afraid, nevertheless, that they will not be adequately understood by many, as well because they also are somewhat long and involved, as chiefly because they require the mind to be entirely free from prejudice, and able with ease to detach itself from the commerce of the senses. And, to speak the truth, the ability for metaphysical studies is less general than for those of geometry. And, besides, there is still this difference that, as in geometry, all are persuaded that nothing is usually advanced of which there is not a certain demonstration, those but partially versed in it err more frequently in assenting to what is false, from a desire of seeming to understand it, than in denying what is true. In philosophy, on the other hand, where it is believed that all is doubtful, few sincerely give themselves to the search after truth, and by far the greater number seek the reputation of bold thinkers by audaciously impugning such truths as are of the greatest moment.

%6. Hence it is that, whatever force my reasonings may possess, yet because they belong to philosophy, I do not expect they will have much effect on the minds of men, unless you extend to them your patronage and approval. But since your Faculty is held in so great esteem by all, and since the name of SORBONNE is of such authority, that not only in matters of faith, but even also in what regards human philosophy, has the judgment of no other society, after the Sacred Councils, received so great deference, it being the universal conviction that it is impossible elsewhere to find greater perspicacity and solidity, or greater wisdom and integrity in giving judgment, I doubt not,if you but condescend to pay so much regard to this Treatise as to be willing, in the first place, to correct it (for mindful not only of my humanity, but chiefly also of my ignorance, I do not affirm that it is free from errors); in the second place, to supply what is wanting in it, to perfect what is incomplete, and to give more ample illustration where it is demanded, or at least to indicate these defects to myself that I may endeavour to remedy them; and, finally, when the reasonings contained in it, by which the existence of God and the distinction of the human soul from the body are established, shall have been brought to such degree of perspicuity as to be esteemed exact demonstrations, of which I am assured they admit, if you condescend to accord them the authority of your approbation, and render a public testimony of their truth and certainty, I doubt not, I say, but that henceforward all the errors which have ever been entertained on these questions will very soon be effaced from the minds of men. For truth itself will readily lead the remainder of the ingenious and the learned to subscribe to your judgment; and your authority will cause the atheists, who are in general sciolists rather than ingenious or learned, to lay aside the spirit of contradiction, and lead them, perhaps, to do battle in their own persons for reasonings which they find considered demonstrations by all men of genius, lest they should seem not to understand them; and, finally, the rest of mankind will readily trust to so many testimonies, and there will no longer be any one who will venture to doubt either the existence of God or the real distinction of mind and body. It is for you, in your singular wisdom, to judge of the importance of the establishment of such beliefs, [who are cognisant of the disorders which doubt of these truths produces].* But it would not here become me to commend at greater length the cause of God and of religion to you, who have always proved the strongest support of the Catholic Church.

%* The square brackets, here and throughout the volume, are used to mark additions to the original of the revised French translation.


%Preface to the Reader


%1. I have already slightly touched upon the questions respecting the existence of God and the nature of the human soul, in the "Discourse on the Method of rightly conducting the Reason, and seeking Truth in the Sciences," published in French in the year 1637; not however, with the design of there treating of them fully, but only, as it were, in passing, that I might learn from the judgment of my readers in what way I should afterward handle them; for these questions appeared to me to be of such moment as to be worthy of being considered more than once, and the path which I follow in discussing them is so little trodden, and so remote from the ordinary route that I thought it would not be expedient to illustrate it at greater length in French, and in a discourse that might be read by all, lest even the more feeble minds should believe that this path might be entered upon by them.

%2. But, as in the " Discourse on Method," I had requested all who might find aught meriting censure in my writings, to do me the favor of pointing it out to me, I may state that no objections worthy of remark have been alleged against what I then said on these questions except two, to which I will here briefly reply, before undertaking their more detailed discussion.

%3. The first objection is that though, while the human mind reflects on itself, it does not perceive that it is any other than a thinking thing, it does not follow that its nature or essence consists only in its being a thing which thinks; so that the word ONLY shall exclude all other things which might also perhaps be said to pertain to the nature of the mind. To this objection I reply, that it was not my intention in that place to exclude these according to the order of truth in the matter (of which I did not then treat),but only according to the order of thought (perception); so that my meaning was, that I clearly apprehended nothing, so far as I was conscious, as belonging to my essence, except that I was a thinking thing, or a thing possessing in itself the faculty of thinking. But I will show hereafter how, from the consciousness that nothing besides thinking belongs to the essence of the mind, it follows that nothing else does in truth belong to it.

%4. The second objection is that it does not follow, from my possessing the idea of a thing more perfect than I am, that the idea itself is more perfect than myself, and much less that what is represented by the idea exists. But I reply that in the term idea there is here something equivocal; for it may be taken either materially for an act of the understanding, and in this sense it cannot be said to be more perfect than I, or objectively, for the thing represented by that act, which, although it be not supposed to exist out of my understanding, may, nevertheless, be more perfect than myself, by reason of its essence. But, in the sequel of this treatise I will show more amply how, from my possessing the idea of a thing more perfect than myself, it follows that this thing really exists.

%5. Besides these two objections, I have seen, indeed, two treatises of sufficient length relating to the present matter. In these, however, my conclusions, much more than my premises, were impugned, and that by arguments borrowed from the common places of the atheists. But, as arguments of this sort can make no impression on the minds of those who shall rightly understand my reasonings, and as the judgments of many are so irrational and weak that they are persuaded rather by the opinions on a subject that are first presented to them, however false and opposed to reason they may be, than by a true and solid, but subsequently received, refutation of them, I am unwilling here to reply to these strictures from a dread of being, in the first instance, obliged to state them. I will only say, in general, that all which the atheists commonly allege in favor of the non-existence of God, arises continually from one or other of these two things, namely, either the ascription of human affections to Deity, or the undue attribution to our minds of so much vigor and wisdom that we may essay to determine and comprehend both what God can and ought to do; hence all that is alleged by them will occasion us no difficulty, provided only we keep in remembrance that our minds must be considered finite, while Deity is incomprehensible and infinite.

%6. Now that I have once, in some measure, made proof of the opinions of men regarding my work, I again undertake to treat of God and the human soul, and at the same time to discuss the principles of the entire First Philosophy, without, however, expecting any commendation from the crowd for my endeavors, or a wide circle of readers. On the contrary, I would advise none to read this work, unless such as are able and willing to meditate with me in earnest, to detach their minds from commerce with the senses, and likewise to deliver themselves from all prejudice; and individuals of this character are, I well know, remarkably rare. But with regard to those who, without caring to comprehend the order and connection of the reasonings, shall study only detached clauses for the purpose of small but noisy criticism, as is the custom with many, I may say that such persons will not profit greatly by the reading of this treatise; and although perhaps they may find opportunity for cavilling in several places, they will yet hardly start any pressing objections, or such as shall be deserving of reply.

%7. But since, indeed, I do not promise to satisfy others on all these subjects at first sight, nor arrogate so much to myself as to believe that I have been able to forsee all that may be the source of difficulty to each ones I shall expound, first of all, in the Meditations, those considerations by which I feel persuaded that I have arrived at a certain and evident knowledge of truth, in order that I may ascertain whether the reasonings which have prevailed with myself will also be effectual in convincing others. I will then reply to the objections of some men, illustrious for their genius and learning, to whom these Meditations were sent for criticism before they were committed to the press; for these objections are so numerous and varied that I venture to anticipate that nothing, at least nothing of any moment, will readily occur to any mind which has not been touched upon in them. Hence it is that I earnestly entreat my readers not to come to any judgment on the questions raised in the Meditations until they have taken care to read the whole of the Objections, with the relative Replies.


%Synopsis


%SYNOPSIS OF THE SIX FOLLOWING MEDITATIONS.

%1. IN THE First Meditation I expound the grounds on which we may doubt in general of all things, and especially of material objects, so long at least, as we have no other foundations for the sciences than those we have hitherto possessed. Now, although the utility of a doubt so general may not be manifest at first sight, it is nevertheless of the greatest, since it delivers us from all prejudice, and affords the easiest pathway by which the mind may withdraw itself from the senses; and finally makes it impossible for us to doubt wherever we afterward discover truth.

%2. In the Second, the mind which, in the exercise of the freedom peculiar to itself, supposes that no object is, of the existence of which it has even the slightest doubt, finds that, meanwhile, it must itself exist. And this point is likewise of the highest moment, for the mind is thus enabled easily to distinguish what pertains to itself, that is, to the intellectual nature, from what is to be referred to the body. But since some, perhaps, will expect, at this stage of our progress, a statement of the reasons which establish the doctrine of the immortality of the soul, I think it proper here to make such aware, that it was my aim to write nothing of which I could not give exact demonstration, and that I therefore felt myself obliged to adopt an order similar to that in use among the geometers, viz., to premise all upon which the proposition in question depends, before coming to any conclusion respecting it. Now, the first and chief prerequisite for the knowledge of the immortality of the soul is our being able to form the clearest possible conception (conceptus—concept) of the soul itself, and such as shall be absolutely distinct from all our notions of body; and how this is to be accomplished is there shown. There is required, besides this, the assurance that all objects which we clearly and distinctly think are true (really exist) in that very mode in which we think them; and this could not be established previously to the Fourth Meditation. Farther, it is necessary, for the same purpose, that we possess a distinct conception of corporeal nature, which is given partly in the Second and partly in the Fifth and Sixth Meditations. And, finally, on these grounds, we are necessitated to conclude, that all those objects which are clearly and distinctly conceived to be diverse substances, as mind and body, are substances really reciprocally distinct; and this inference is made in the Sixth Meditation. The absolute distinction of mind and body is, besides, confirmed in this Second Meditation, by showing that we cannot conceive body unless as divisible; while, on the other hand, mind cannot be conceived unless as indivisible. For we are not able to conceive the half of a mind, as we can of any body, however small, so that the natures of these two substances are to be held, not only as diverse, but even in some measure as contraries. I have not, however, pursued this discussion further in the present treatise, as well for the reason that these considerations are sufficient to show that the destruction of the mind does not follow from the corruption of the body, and thus to afford to men the hope of a future life, as also because the premises from which it is competent for us to infer the immortality of the soul, involve an explication of the whole principles of Physics: in order to establish, in the first place, that generally all substances, that is, all things which can exist only in consequence of having been created by God, are in their own nature incorruptible, and can never cease to be, unless God himself, by refusing his concurrence to them, reduce them to nothing; and, in the second place, that body, taken generally, is a substance, and therefore can never perish, but that the human body, in as far as it differs from other bodies, is constituted only by a certain configuration of members, and by other accidents of this sort, while the human mind is not made up of accidents, but is a pure substance. For although all the accidents of the mind be changed—although, for example, it think certain things, will others, and perceive others, the mind itself does not vary with these changes; while, on the contrary, the human body is no longer the same if a change take place in the form of any of its parts: from which it follows that the body may, indeed, without difficulty perish, but that the mind is in its own nature immortal.

%3. In the Third Meditation, I have unfolded at sufficient length, as appears to me, my chief argument for the existence of God. But yet, since I was there desirous to avoid the use of comparisons taken from material objects, that I might withdraw, as far as possible, the minds of my readers from the senses, numerous obscurities perhaps remain, which, however, will, I trust, be afterward entirely removed in the Replies to the Objections: thus among other things, it may be difficult to understand how the idea of a being absolutely perfect, which is found in our minds, possesses so much objective reality [i. e., participates by representation in so many degrees of being and perfection] that it must be held to arise from a cause absolutely perfect. This is illustrated in the Replies by the comparison of a highly perfect machine, the idea of which exists in the mind of some workman; for as the objective (i.e.., representative) perfection of this idea must have some cause, viz, either the science of the workman, or of some other person from whom he has received the idea, in the same way the idea of God, which is found in us, demands God himself for its cause.

%4. In the Fourth, it is shown that all which we clearly and distinctly perceive (apprehend) is true; and, at the same time, is explained wherein consists the nature of error, points that require to be known as well for confirming the preceding truths, as for the better understanding of those that are to follow. But, meanwhile, it must be observed, that I do not at all there treat of Sin, that is, of error committed in the pursuit of good and evil, but of that sort alone which arises in the determination of the true and the false. Nor do I refer to matters of faith, or to the conduct of life, but only to what regards speculative truths, and such as are known by means of the natural light alone.

%5. In the Fifth, besides the illustration of corporeal nature, taken generically, a new demonstration is given of the existence of God, not free, perhaps, any more than the former, from certain difficulties, but of these the solution will be found in the Replies to the Objections. I further show, in what sense it is true that the certitude of geometrical demonstrations themselves is dependent on the knowledge of God.

%6. Finally, in the Sixth, the act of the understanding (intellectio) is distinguished from that of the imagination (imaginatio); the marks of this distinction are described; the human mind is shown to be really distinct from the body, and, nevertheless, to be so closely conjoined therewith, as together to form, as it were, a unity. The whole of the errors which arise from the senses are brought under review, while the means of avoiding them are pointed out; and, finally, all the grounds are adduced from which the existence of material objects may be inferred; not, however, because I deemed them of great utility in establishing what they prove, viz., that there is in reality a world, that men are possessed of bodies, and the like, the truth of which no one of sound mind ever seriously doubted; but because, from a close consideration of them, it is perceived that they are neither so strong nor clear as the reasonings which conduct us to the knowledge of our mind and of God; so that the latter are, of all which come under human knowledge, the most certain and manifest—a conclusion which it was my single aim in these Meditations to establish; on which account I here omit mention of the various other questions which, in the course of the discussion, I had occasion likewise to consider.

\section{Meditation I OF THE THINGS OF WHICH WE MAY DOUBT.}

1. SEVERAL years have now elapsed since I first became aware that I had accepted, even from my youth, many false opinions for true, and that consequently what I afterward based on such principles was highly doubtful; and from that time I was convinced of the necessity of undertaking once in my life to rid myself of all the opinions I had adopted, and of commencing anew the work of building from the foundation, if I desired to establish a firm and abiding superstructure in the sciences. But as this enterprise appeared to me to be one of great magnitude, I waited until I had attained an age so mature as to leave me no hope that at any stage of life more advanced I should be better able to execute my design. On this account, I have delayed so long that I should henceforth consider I was doing wrong were I still to consume in deliberation any of the time that now remains for action. To-day, then, since I have opportunely freed my mind from all cares [and am happily disturbed by no passions], and since I am in the secure possession of leisure in a peaceable retirement, I will at length apply myself earnestly and freely to the general overthrow of all my former opinions.

2. But, to this end, it will not be necessary for me to show that the whole of these are false—a point, perhaps, which I shall never reach; but as even now my reason convinces me that I ought not the less carefully to withhold belief from what is not entirely certain and indubitable, than from what is manifestly false, it will be sufficient to justify the rejection of the whole if I shall find in each some ground for doubt. Nor for this purpose will it be necessary even to deal with each belief individually, which would be truly an endless labor; but, as the removal from below of the foundation necessarily involves the downfall of the whole edifice, I will at once approach the criticism of the principles on which all my former beliefs rested.

3. All that I have, up to this moment, accepted as possessed of the highest truth and certainty, I received either from or through the senses. I observed, however, that these sometimes misled us; and it is the part of prudence not to place absolute confidence in that by which we have even once been deceived.

4. But it may be said, perhaps, that, although the senses occasionally mislead us respecting minute objects, and such as are so far removed from us as to be beyond the reach of close observation, there are yet many other of their informations (presentations), of the truth of which it is manifestly impossible to doubt; as for example, that I am in this place, seated by the fire, clothed in a winter dressing gown, that I hold in my hands this piece of paper, with other intimations of the same nature. But how could I deny that I possess these hands and this body, and withal escape being classed with persons in a state of insanity, whose brains are so disordered and clouded by dark bilious vapors as to cause them pertinaciously to assert that they are monarchs when they are in the greatest poverty; or clothed [in gold] and purple when destitute of any covering; or that their head is made of clay, their body of glass, or that they are gourds? I should certainly be not less insane than they, were I to regulate my procedure according to examples so extravagant.

5. Though this be true, I must nevertheless here consider that I am a man, and that, consequently, I am in the habit of sleeping, and representing to myself in dreams those same things, or even sometimes others less probable, which the insane think are presented to them in their waking moments. How often have I dreamt that I was in these familiar circumstances, that I was dressed, and occupied this place by the fire, when I was lying undressed in bed? At the present moment, however, I certainly look upon this paper with eyes wide awake; the head which I now move is not asleep; I extend this hand consciously and with express purpose, and I perceive it; the occurrences in sleep are not so distinct as all this. But I cannot forget that, at other times I have been deceived in sleep by similar illusions; and, attentively considering those cases, I perceive so clearly that there exist no certain marks by which the state of waking can ever be distinguished from sleep, that I feel greatly astonished; and in amazement I almost persuade myself that I am now dreaming.

6. Let us suppose, then, that we are dreaming, and that all these particulars—namely, the opening of the eyes, the motion of the head, the forth-putting of the hands—are merely illusions; and even that we really possess neither an entire body nor hands such as we see. Nevertheless it must be admitted at least that the objects which appear to us in sleep are, as it were, painted representations which could not have been formed unless in the likeness of realities; and, therefore, that those general objects, at all events, namely, eyes, a head, hands, and an entire body, are not simply imaginary, but really existent. For, in truth, painters themselves, even when they study to represent sirens and satyrs by forms the most fantastic and extraordinary, cannot bestow upon them natures absolutely new, but can only make a certain medley of the members of different animals; or if they chance to imagine something so novel that nothing at all similar has ever been seen before, and such as is, therefore, purely fictitious and absolutely false, it is at least certain that the colors of which this is composed are real. And on the same principle, although these general objects, viz. [a body], eyes, a head, hands, and the like, be imaginary, we are nevertheless absolutely necessitated to admit the reality at least of some other objects still more simple and universal than these, of which, just as of certain real colors, all those images of things, whether true and real, or false and fantastic, that are found in our consciousness (cogitatio), are formed.

7. To this class of objects seem to belong corporeal nature in general and its extension; the figure of extended things, their quantity or magnitude, and their number, as also the place in, and the time during, which they exist, and other things of the same sort.

8. We will not, therefore, perhaps reason illegitimately if we conclude from this that Physics, Astronomy, Medicine, and all the other sciences that have for their end the consideration of composite objects, are indeed of a doubtful character; but that Arithmetic, Geometry, and the other sciences of the same class, which regard merely the simplest and most general objects, and scarcely inquire whether or not these are really existent, contain somewhat that is certain and indubitable: for whether I am awake or dreaming, it remains true that two and three make five, and that a square has but four sides; nor does it seem possible that truths so apparent can ever fall under a suspicion of falsity [or incertitude].

9. Nevertheless, the belief that there is a God who is all powerful, and who created me, such as I am, has, for a long time, obtained steady possession of my mind. How, then, do I know that he has not arranged that there should be neither earth, nor sky, nor any extended thing, nor figure, nor magnitude, nor place, providing at the same time, however, for [the rise in me of the perceptions of all these objects, and] the persuasion that these do not exist otherwise than as I perceive them ? And further, as I sometimes think that others are in error respecting matters of which they believe themselves to possess a perfect knowledge, how do I know that I am not also deceived each time I add together two and three, or number the sides of a square, or form some judgment still more simple, if more simple indeed can be imagined? But perhaps Deity has not been willing that I should be thus deceived, for he is said to be supremely good. If, however, it were repugnant to the goodness of Deity to have created me subject to constant deception, it would seem likewise to be contrary to his goodness to allow me to be occasionally deceived; and yet it is clear that this is permitted.

10. Some, indeed, might perhaps be found who would be disposed rather to deny the existence of a Being so powerful than to believe that there is nothing certain. But let us for the present refrain from opposing this opinion, and grant that all which is here said of a Deity is fabulous: nevertheless, in whatever way it be supposed that I reach the state in which I exist, whether by fate, or chance, or by an endless series of antecedents and consequents, or by any other means, it is clear (since to be deceived and to err is a certain defect) that the probability of my being so imperfect as to be the constant victim of deception, will be increased exactly in proportion as the power possessed by the cause, to which they assign my origin, is lessened. To these reasonings I have assuredly nothing to reply, but am constrained at last to avow that there is nothing of all that I formerly believed to be true of which it is impossible to doubt, and that not through thoughtlessness or levity, but from cogent and maturely considered reasons; so that henceforward, if I desire to discover anything certain, I ought not the less carefully to refrain from assenting to those same opinions than to what might be shown to be manifestly false.

11. But it is not sufficient to have made these observations; care must be taken likewise to keep them in remembrance. For those old and customary opinions perpetually recur—long and familiar usage giving them the right of occupying my mind, even almost against my will, and subduing my belief; nor will I lose the habit of deferring to them and confiding in them so long as I shall consider them to be what in truth they are, viz, opinions to some extent doubtful, as I have already shown, but still highly probable, and such as it is much more reasonable to believe than deny. It is for this reason I am persuaded that I shall not be doing wrong, if, taking an opposite judgment of deliberate design, I become my own deceiver, by supposing, for a time, that all those opinions are entirely false and imaginary, until at length, having thus balanced my old by my new prejudices, my judgment shall no longer be turned aside by perverted usage from the path that may conduct to the perception of truth. For I am assured that, meanwhile, there will arise neither peril nor error from this course, and that I cannot for the present yield too much to distrust, since the end I now seek is not action but knowledge.

12. I will suppose, then, not that Deity, who is sovereignly good and the fountain of truth, but that some malignant demon, who is at once exceedingly potent and deceitful, has employed all his artifice to deceive me; I will suppose that the sky, the air, the earth, colors, figures, sounds, and all external things, are nothing better than the illusions of dreams, by means of which this being has laid snares for my credulity; I will consider myself as without hands, eyes, flesh, blood, or any of the senses, and as falsely believing that I am possessed of these; I will continue resolutely fixed in this belief, and if indeed by this means it be not in my power to arrive at the knowledge of truth, I shall at least do what is in my power, viz., [suspend my judgment], and guard with settled purpose against giving my assent to what is false, and being imposed upon by this deceiver, whatever be his power and artifice. But this undertaking is arduous, and a certain indolence insensibly leads me back to my ordinary course of life; and just as the captive, who, perchance, was enjoying in his dreams an imaginary liberty, when he begins to suspect that it is but a vision, dreads awakening, and conspires with the agreeable illusions that the deception may be prolonged; so I, of my own accord, fall back into the train of my former beliefs, and fear to arouse myself from my slumber, lest the time of laborious wakefulness that would succeed this quiet rest, in place of bringing any light of day, should prove inadequate to dispel the darkness that will arise from the difficulties that have now been raised.

\section{Meditation II OF THE NATURE OF THE HUMAN MIND; AND THAT IT IS MORE EASILY KNOWN THAN THE BODY.}

1. The Meditation of yesterday has filled my mind with so many doubts, that it is no longer in my power to forget them. Nor do I see, meanwhile, any principle on which they can be resolved; and, just as if I had fallen all of a sudden into very deep water, I am so greatly disconcerted as to be unable either to plant my feet firmly on the bottom or sustain myself by swimming on the surface. I will, nevertheless, make an effort, and try anew the same path on which I had entered yesterday, that is, proceed by casting aside all that admits of the slightest doubt, not less than if I had discovered it to be absolutely false; and I will continue always in this track until I shall find something that is certain, or at least, if I can do nothing more, until I shall know with certainty that there is nothing certain. Archimedes, that he might transport the entire globe from the place it occupied to another, demanded only a point that was firm and immovable; so, also, I shall be entitled to entertain the highest expectations, if I am fortunate enough to discover only one thing that is certain and indubitable.

2. I suppose, accordingly, that all the things which I see are false (fictitious); I believe that none of those objects which my fallacious memory represents ever existed; I suppose that I possess no senses; I believe that body, figure, extension, motion, and place are merely fictions of my mind. What is there, then, that can be esteemed true ? Perhaps this only, that there is absolutely nothing certain.

3. But how do I know that there is not something different altogether from the objects I have now enumerated, of which it is impossible to entertain the slightest doubt? Is there not a God, or some being, by whatever name I may designate him, who causes these thoughts to arise in my mind ? But why suppose such a being, for it may be I myself am capable of producing them? Am I, then, at least not something? But I before denied that I possessed senses or a body; I hesitate, however, for what follows from that? Am I so dependent on the body and the senses that without these I cannot exist? But I had the persuasion that there was absolutely nothing in the world, that there was no sky and no earth, neither minds nor bodies; was I not, therefore, at the same time, persuaded that I did not exist? Far from it; I assuredly existed, since I was persuaded. But there is I know not what being, who is possessed at once of the highest power and the deepest cunning, who is constantly employing all his ingenuity in deceiving me. Doubtless, then, I exist, since I am deceived; and, let him deceive me as he may, he can never bring it about that I am nothing, so long as I shall be conscious that I am something. So that it must, in fine, be maintained, all things being maturely and carefully considered, that this proposition (pronunciatum) I am, I exist, is necessarily true each time it is expressed by me, or conceived in my mind.

4. But I do not yet know with sufficient clearness what I am, though assured that I am; and hence, in the next place, I must take care, lest perchance I inconsiderately substitute some other object in room of what is properly myself, and thus wander from truth, even in that knowledge (cognition) which I hold to be of all others the most certain and evident. For this reason, I will now consider anew what I formerly believed myself to be, before I entered on the present train of thought; and of my previous opinion I will retrench all that can in the least be invalidated by the grounds of doubt I have adduced, in order that there may at length remain nothing but what is certain and indubitable.

5. What then did I formerly think I was ? Undoubtedly I judged that I was a man. But what is a man ? Shall I say a rational animal ? Assuredly not; for it would be necessary forthwith to inquire into what is meant by animal, and what by rational, and thus, from a single question, I should insensibly glide into others, and these more difficult than the first; nor do I now possess enough of leisure to warrant me in wasting my time amid subtleties of this sort. I prefer here to attend to the thoughts that sprung up of themselves in my mind, and were inspired by my own nature alone, when I applied myself to the consideration of what I was. In the first place, then, I thought that I possessed a countenance, hands, arms, and all the fabric of members that appears in a corpse, and which I called by the name of body. It further occurred to me that I was nourished, that I walked, perceived, and thought, and all those actions I referred to the soul; but what the soul itself was I either did not stay to consider, or, if I did, I imagined that it was something extremely rare and subtile, like wind, or flame, or ether, spread through my grosser parts. As regarded the body, I did not even doubt of its nature, but thought I distinctly knew it, and if I had wished to describe it according to the notions I then entertained, I should have explained myself in this manner: By body I understand all that can be terminated by a certain figure; that can be comprised in a certain place, and so fill a certain space as therefrom to exclude every other body; that can be perceived either by touch, sight, hearing, taste, or smell; that can be moved in different ways, not indeed of itself, but by something foreign to it by which it is touched [and from which it receives the impression]; for the power of self-motion, as likewise that of perceiving and thinking, I held as by no means pertaining to the nature of body; on the contrary, I was somewhat astonished to find such faculties existing in some bodies.

6. But [as to myself, what can I now say that I am], since I suppose there exists an extremely powerful, and, if I may so speak, malignant being, whose whole endeavors are directed toward deceiving me ? Can I affirm that I possess any one of all those attributes of which I have lately spoken as belonging to the nature of body ? After attentively considering them in my own mind, I find none of them that can properly be said to belong to myself. To recount them were idle and tedious. Let us pass, then, to the attributes of the soul. The first mentioned were the powers of nutrition and walking; but, if it be true that I have no body, it is true likewise that I am capable neither of walking nor of being nourished. Perception is another attribute of the soul; but perception too is impossible without the body; besides, I have frequently, during sleep, believed that I perceived objects which I afterward observed I did not in reality perceive. Thinking is another attribute of the soul; and here I discover what properly belongs to myself. This alone is inseparable from me. I am—I exist: this is certain; but how often? As often as I think; for perhaps it would even happen, if I should wholly cease to think, that I should at the same time altogether cease to be. I now admit nothing that is not necessarily true. I am therefore, precisely speaking, only a thinking thing, that is, a mind (mens sive animus), understanding, or reason, terms whose signification was before unknown to me. I am, however, a real thing, and really existent; but what thing? The answer was, a thinking thing.

7. The question now arises, am I aught besides ? I will stimulate my imagination with a view to discover whether I am not still something more than a thinking being. Now it is plain I am not the assemblage of members called the human body; I am not a thin and penetrating air diffused through all these members, or wind, or flame, or vapor, or breath, or any of all the things I can imagine; for I supposed that all these were not, and, without changing the supposition, I find that I still feel assured of my existence. But it is true, perhaps, that those very things which I suppose to be non-existent, because they are unknown to me, are not in truth different from myself whom I know. This is a point I cannot determine, and do not now enter into any dispute regarding it. I can only judge of things that are known to me: I am conscious that I exist, and I who know that I exist inquire into what I am. It is, however, perfectly certain that the knowledge of my existence, thus precisely taken, is not dependent on things, the existence of which is as yet unknown to me: and consequently it is not dependent on any of the things I can feign in imagination. Moreover, the phrase itself, I frame an image (effingo), reminds me of my error; for I should in truth frame one if I were to imagine myself to be anything, since to imagine is nothing more than to contemplate the figure or image of a corporeal thing; but I already know that I exist, and that it is possible at the same time that all those images, and in general all that relates to the nature of body, are merely dreams [or chimeras]. From this I discover that it is not more reasonable to say, I will excite my imagination that I may know more distinctly what I am, than to express myself as follows: I am now awake, and perceive something real; but because my perception is not sufficiently clear, I will of express purpose go to sleep that my dreams may represent to me the object of my perception with more truth and clearness. And, therefore, I know that nothing of all that I can embrace in imagination belongs to the knowledge which I have of myself, and that there is need to recall with the utmost care the mind from this mode of thinking, that it may be able to know its own nature with perfect distinctness.

8. But what, then, am I ? A thinking thing, it has been said. But what is a thinking thing? It is a thing that doubts, understands, [conceives], affirms, denies, wills, refuses; that imagines also, and perceives.

9. Assuredly it is not little, if all these properties belong to my nature. But why should they not belong to it ? Am I not that very being who now doubts of almost everything; who, for all that, understands and conceives certain things; who affirms one alone as true, and denies the others; who desires to know more of them, and does not wish to be deceived; who imagines many things, sometimes even despite his will; and is likewise percipient of many, as if through the medium of the senses. Is there nothing of all this as true as that I am, even although I should be always dreaming, and although he who gave me being employed all his ingenuity to deceive me ? Is there also any one of these attributes that can be properly distinguished from my thought, or that can be said to be separate from myself ? For it is of itself so evident that it is I who doubt, I who understand, and I who desire, that it is here unnecessary to add anything by way of rendering it more clear. And I am as certainly the same being who imagines; for although it may be (as I before supposed) that nothing I imagine is true, still the power of imagination does not cease really to exist in me and to form part of my thought. In fine, I am the same being who perceives, that is, who apprehends certain objects as by the organs of sense, since, in truth, I see light, hear a noise, and feel heat. But it will be said that these presentations are false, and that I am dreaming. Let it be so. At all events it is certain that I seem to see light, hear a noise, and feel heat; this cannot be false, and this is what in me is properly called perceiving (sentire), which is nothing else than thinking.

10. From this I begin to know what I am with somewhat greater clearness and distinctness than heretofore. But, nevertheless, it still seems to me, and I cannot help believing, that corporeal things, whose images are formed by thought [which fall under the senses], and are examined by the same, are known with much greater distinctness than that I know not what part of myself which is not imaginable; although, in truth, it may seem strange to say that I know and comprehend with greater distinctness things whose existence appears to me doubtful, that are unknown, and do not belong to me, than others of whose reality I am persuaded, that are known to me, and appertain to my proper nature; in a word, than myself. But I see clearly what is the state of the case. My mind is apt to wander, and will not yet submit to be restrained within the limits of truth. Let us therefore leave the mind to itself once more, and, according to it every kind of liberty [permit it to consider the objects that appear to it from without], in order that, having afterward withdrawn it from these gently and opportunely [and fixed it on the consideration of its being and the properties it finds in itself], it may then be the more easily controlled.

11. Let us now accordingly consider the objects that are commonly thought to be [the most easily, and likewise] the most distinctly known, viz, the bodies we touch and see; not, indeed, bodies in general, for these general notions are usually somewhat more confused, but one body in particular. Take, for example, this piece of wax; it is quite fresh, having been but recently taken from the beehive; it has not yet lost the sweetness of the honey it contained; it still retains somewhat of the odor of the flowers from which it was gathered; its color, figure, size, are apparent (to the sight); it is hard, cold, easily handled; and sounds when struck upon with the finger. In fine, all that contributes to make a body as distinctly known as possible, is found in the one before us. But, while I am speaking, let it be placed near the fire—what remained of the taste exhales, the smell evaporates, the color changes, its figure is destroyed, its size increases, it becomes liquid, it grows hot, it can hardly be handled, and, although struck upon, it emits no sound. Does the same wax still remain after this change ? It must be admitted that it does remain; no one doubts it, or judges otherwise. What, then, was it I knew with so much distinctness in the piece of wax? Assuredly, it could be nothing of all that I observed by means of the senses, since all the things that fell under taste, smell, sight, touch, and hearing are changed, and yet the same wax remains.

12. It was perhaps what I now think, viz, that this wax was neither the sweetness of honey, the pleasant odor of flowers, the whiteness, the figure, nor the sound, but only a body that a little before appeared to me conspicuous under these forms, and which is now perceived under others. But, to speak precisely, what is it that I imagine when I think of it in this way? Let it be attentively considered, and, retrenching all that does not belong to the wax, let us see what remains. There certainly remains nothing, except something extended, flexible, and movable. But what is meant by flexible and movable ? Is it not that I imagine that the piece of wax, being round, is capable of becoming square, or of passing from a square into a triangular figure ? Assuredly such is not the case, because I conceive that it admits of an infinity of similar changes; and I am, moreover, unable to compass this infinity by imagination, and consequently this conception which I have of the wax is not the product of the faculty of imagination. But what now is this extension ? Is it not also unknown ? for it becomes greater when the wax is melted, greater when it is boiled, and greater still when the heat increases; and I should not conceive [clearly and] according to truth, the wax as it is, if I did not suppose that the piece we are considering admitted even of a wider variety of extension than I ever imagined, I must, therefore, admit that I cannot even comprehend by imagination what the piece of wax is, and that it is the mind alone (mens, Lat., entendement, F.) which perceives it. I speak of one piece in particular; for as to wax in general, this is still more evident. But what is the piece of wax that can be perceived only by the [understanding or] mind? It is certainly the same which I see, touch, imagine; and, in fine, it is the same which, from the beginning, I believed it to be. But (and this it is of moment to observe) the perception of it is neither an act of sight, of touch, nor of imagination, and never was either of these, though it might formerly seem so, but is simply an intuition (inspectio) of the mind, which may be imperfect and confused, as it formerly was, or very clear and distinct, as it is at present, according as the attention is more or less directed to the elements which it contains, and of which it is composed.

13. But, meanwhile, I feel greatly astonished when I observe [the weakness of my mind, and] its proneness to error. For although, without at all giving expression to what I think, I consider all this in my own mind, words yet occasionally impede my progress, and I am almost led into error by the terms of ordinary language. We say, for example, that we see the same wax when it is before us, and not that we judge it to be the same from its retaining the same color and figure: whence I should forthwith be disposed to conclude that the wax is known by the act of sight, and not by the intuition of the mind alone, were it not for the analogous instance of human beings passing on in the street below, as observed from a window. In this case I do not fail to say that I see the men themselves, just as I say that I see the wax; and yet what do I see from the window beyond hats and cloaks that might cover artificial machines, whose motions might be determined by springs ? But I judge that there are human beings from these appearances, and thus I comprehend, by the faculty of judgment alone which is in the mind, what I believed I saw with my eyes.

14. The man who makes it his aim to rise to knowledge superior to the common, ought to be ashamed to seek occasions of doubting from the vulgar forms of speech: instead, therefore, of doing this, I shall proceed with the matter in hand, and inquire whether I had a clearer and more perfect perception of the piece of wax when I first saw it, and when I thought I knew it by means of the external sense itself, or, at all events, by the common sense (sensus communis), as it is called, that is, by the imaginative faculty; or whether I rather apprehend it more clearly at present, after having examined with greater care, both what it is, and in what way it can be known. It would certainly be ridiculous to entertain any doubt on this point. For what, in that first perception, was there distinct ? What did I perceive which any animal might not have perceived ? But when I distinguish the wax from its exterior forms, and when, as if I had stripped it of its vestments, I consider it quite naked, it is certain, although some error may still be found in my judgment, that I cannot, nevertheless, thus apprehend it without possessing a human mind.

15. But finally, what shall I say of the mind itself, that is, of myself ? for as yet I do not admit that I am anything but mind. What, then! I who seem to possess so distinct an apprehension of the piece of wax, do I not know myself, both with greater truth and certitude, and also much more distinctly and clearly? For if I judge that the wax exists because I see it, it assuredly follows, much more evidently, that I myself am or exist, for the same reason: for it is possible that what I see may not in truth be wax, and that I do not even possess eyes with which to see anything; but it cannot be that when I see, or, which comes to the same thing, when I think I see, I myself who think am nothing. So likewise, if I judge that the wax exists because I touch it, it will still also follow that I am; and if I determine that my imagination, or any other cause, whatever it be, persuades me of the existence of the wax, I will still draw the same conclusion. And what is here remarked of the piece of wax, is applicable to all the other things that are external to me. And further, if the [notion or] perception of wax appeared to me more precise and distinct, after that not only sight and touch, but many other causes besides, rendered it manifest to my apprehension, with how much greater distinctness must I now know myself, since all the reasons that contribute to the knowledge of the nature of wax, or of any body whatever, manifest still better the nature of my mind ? And there are besides so many other things in the mind itself that contribute to the illustration of its nature, that those dependent on the body, to which I have here referred, scarcely merit to be taken into account.

16. But, in conclusion, I find I have insensibly reverted to the point I desired; for, since it is now manifest to me that bodies themselves are not properly perceived by the senses nor by the faculty of imagination, but by the intellect alone; and since they are not perceived because they are seen and touched, but only because they are understood [or rightly comprehended by thought], I readily discover that there is nothing more easily or clearly apprehended than my own mind. But because it is difficult to rid one's self so promptly of an opinion to which one has been long accustomed, it will be desirable to tarry for some time at this stage, that, by long continued meditation, I may more deeply impress upon my memory this new knowledge.


\section{Meditation III OF GOD: THAT HE EXISTS.}

1. I WILL now close my eyes, I will stop my ears, I will turn away my senses from their objects, I will even efface from my consciousness all the images of corporeal things; or at least, because this can hardly be accomplished, I will consider them as empty and false; and thus, holding converse only with myself, and closely examining my nature, I will endeavor to obtain by degrees a more intimate and familiar knowledge of myself. I am a thinking (conscious) thing, that is, a being who doubts, affirms, denies, knows a few objects, and is ignorant of many,— [who loves, hates], wills, refuses, who imagines likewise, and perceives; for, as I before remarked, although the things which I perceive or imagine are perhaps nothing at all apart from me [and in themselves], I am nevertheless assured that those modes of consciousness which I call perceptions and imaginations, in as far only as they are modes of consciousness, exist in me.

2. And in the little I have said I think I have summed up all that I really know, or at least all that up to this time I was aware I knew. Now, as I am endeavoring to extend my knowledge more widely, I will use circumspection, and consider with care whether I can still discover in myself anything further which I have not yet hitherto observed. I am certain that I am a thinking thing; but do I not therefore likewise know what is required to render me certain of a truth ? In this first knowledge, doubtless, there is nothing that gives me assurance of its truth except the clear and distinct perception of what I affirm, which would not indeed be sufficient to give me the assurance that what I say is true, if it could ever happen that anything I thus clearly and distinctly perceived should prove false; and accordingly it seems to me that I may now take as a general rule, that all that is very clearly and distinctly apprehended (conceived) is true.

3. Nevertheless I before received and admitted many things as wholly certain and manifest, which yet I afterward found to be doubtful. What, then, were those? They were the earth, the sky, the stars, and all the other objects which I was in the habit of perceiving by the senses. But what was it that I clearly [and distinctly] perceived in them ? Nothing more than that the ideas and the thoughts of those objects were presented to my mind. And even now I do not deny that these ideas are found in my mind. But there was yet another thing which I affirmed, and which, from having been accustomed to believe it, I thought I clearly perceived, although, in truth, I did not perceive it at all; I mean the existence of objects external to me, from which those ideas proceeded, and to which they had a perfect resemblance; and it was here I was mistaken, or if I judged correctly, this assuredly was not to be traced to any knowledge I possessed (the force of my perception, Lat.).

4. But when I considered any matter in arithmetic and geometry, that was very simple and easy, as, for example, that two and three added together make five, and things of this sort, did I not view them with at least sufficient clearness to warrant me in affirming their truth? Indeed, if I afterward judged that we ought to doubt of these things, it was for no other reason than because it occurred to me that God might perhaps have given me such a nature as that I should be deceived, even respecting the matters that appeared to me the most evidently true. But as often as this preconceived opinion of the sovereign power of a God presents itself to my mind, I am constrained to admit that it is easy for him, if he wishes it, to cause me to err, even in matters where I think I possess the highest evidence; and, on the other hand, as often as I direct my attention to things which I think I apprehend with great clearness, I am so persuaded of their truth that I naturally break out into expressions such as these: Deceive me who may, no one will yet ever be able to bring it about that I am not, so long as I shall be conscious that I am, or at any future time cause it to be true that I have never been, it being now true that I am, or make two and three more or less than five, in supposing which, and other like absurdities, I discover a manifest contradiction. And in truth, as I have no ground for believing that Deity is deceitful, and as, indeed, I have not even considered the reasons by which the existence of a Deity of any kind is established, the ground of doubt that rests only on this supposition is very slight, and, so to speak, metaphysical. But, that I may be able wholly to remove it, I must inquire whether there is a God, as soon as an opportunity of doing so shall present itself; and if I find that there is a God, I must examine likewise whether he can be a deceiver; for, without the knowledge of these two truths, I do not see that I can ever be certain of anything. And that I may be enabled to examine this without interrupting the order of meditation I have proposed to myself [which is, to pass by degrees from the notions that I shall find first in my mind to those I shall afterward discover in it], it is necessary at this stage to divide all my thoughts into certain classes, and to consider in which of these classes truth and error are, strictly speaking, to be found.

5. Of my thoughts some are, as it were, images of things, and to these alone properly belongs the name IDEA; as when I think [represent to my mind] a man, a chimera, the sky, an angel or God. Others, again, have certain other forms; as when I will, fear, affirm, or deny, I always, indeed, apprehend something as the object of my thought, but I also embrace in thought something more than the representation of the object; and of this class of thoughts some are called volitions or affections, and others judgments.

6. Now, with respect to ideas, if these are considered only in themselves, and are not referred to any object beyond them, they cannot, properly speaking, be false; for, whether I imagine a goat or chimera, it is not less true that I imagine the one than the other. Nor need we fear that falsity may exist in the will or affections; for, although I may desire objects that are wrong, and even that never existed, it is still true that I desire them. There thus only remain our judgments, in which we must take diligent heed that we be not deceived. But the chief and most ordinary error that arises in them consists in judging that the ideas which are in us are like or conformed to the things that are external to us; for assuredly, if we but considered the ideas themselves as certain modes of our thought (consciousness), without referring them to anything beyond, they would hardly afford any occasion of error.

7. But among these ideas, some appear to me to be innate, others adventitious, and others to be made by myself (factitious); for, as I have the power of conceiving what is called a thing, or a truth, or a thought, it seems to me that I hold this power from no other source than my own nature; but if I now hear a noise, if I see the sun, or if I feel heat, I have all along judged that these sensations proceeded from certain objects existing out of myself; and, in fine, it appears to me that sirens, hippogryphs, and the like, are inventions of my own mind. But I may even perhaps come to be of opinion that all my ideas are of the class which I call adventitious, or that they are all innate, or that they are all factitious; for I have not yet clearly discovered their true origin.

8. What I have here principally to do is to consider, with reference to those that appear to come from certain objects without me, what grounds there are for thinking them like these objects. The first of these grounds is that it seems to me I am so taught by nature; and the second that I am conscious that those ideas are not dependent on my will, and therefore not on myself, for they are frequently presented to me against my will, as at present, whether I will or not, I feel heat; and I am thus persuaded that this sensation or idea (sensum vel ideam) of heat is produced in me by something different from myself, viz., by the heat of the fire by which I sit. And it is very reasonable to suppose that this object impresses me with its own likeness rather than any other thing.

9. But I must consider whether these reasons are sufficiently strong and convincing. When I speak of being taught by nature in this matter, I understand by the word nature only a certain spontaneous impetus that impels me to believe in a resemblance between ideas and their objects, and not a natural light that affords a knowledge of its truth. But these two things are widely different; for what the natural light shows to be true can be in no degree doubtful, as, for example, that I am because I doubt, and other truths of the like kind; inasmuch as I possess no other faculty whereby to distinguish truth from error, which can teach me the falsity of what the natural light declares to be true, and which is equally trustworthy; but with respect to [seemingly] natural impulses, I have observed, when the question related to the choice of right or wrong in action, that they frequently led me to take the worse part; nor do I see that I have any better ground for following them in what relates to truth and error.

10. Then, with respect to the other reason, which is that because these ideas do not depend on my will, they must arise from objects existing without me, I do not find it more convincing than the former, for just as those natural impulses, of which I have lately spoken, are found in me, notwithstanding that they are not always in harmony with my will, so likewise it may be that I possess some power not sufficiently known to myself capable of producing ideas without the aid of external objects, and, indeed, it has always hitherto appeared to me that they are formed during sleep, by some power of this nature, without the aid of aught external.

11. And, in fine, although I should grant that they proceeded from those objects, it is not a necessary consequence that they must be like them. On the contrary, I have observed, in a number of instances, that there was a great difference between the object and its idea. Thus, for example, I find in my mind two wholly diverse ideas of the sun; the one, by which it appears to me extremely small draws its origin from the senses, and should be placed in the class of adventitious ideas; the other, by which it seems to be many times larger than the whole earth, is taken up on astronomical grounds, that is, elicited from certain notions born with me, or is framed by myself in some other manner. These two ideas cannot certainly both resemble the same sun; and reason teaches me that the one which seems to have immediately emanated from it is the most unlike.

12. And these things sufficiently prove that hitherto it has not been from a certain and deliberate judgment, but only from a sort of blind impulse, that I believed existence of certain things different from myself, which, by the organs of sense, or by whatever other means it might be, conveyed their ideas or images into my mind [and impressed it with their likenesses].

13. But there is still another way of inquiring whether, of the objects whose ideas are in my mind, there are any that exist out of me. If ideas are taken in so far only as they are certain modes of consciousness, I do not remark any difference or inequality among them, and all seem, in the same manner, to proceed from myself; but, considering them as images, of which one represents one thing and another a different, it is evident that a great diversity obtains among them. For, without doubt, those that represent substances are something more, and contain in themselves, so to speak, more objective reality [that is, participate by representation in higher degrees of being or perfection], than those that represent only modes or accidents; and again, the idea by which I conceive a God [sovereign], eternal, infinite, [immutable], all-knowing, all-powerful, and the creator of all things that are out of himself, this, I say, has certainly in it more objective reality than those ideas by which finite substances are represented.

14. Now, it is manifest by the natural light that there must at least be as much reality in the efficient and total cause as in its effect; for whence can the effect draw its reality if not from its cause ? And how could the cause communicate to it this reality unless it possessed it in itself? And hence it follows, not only that what is cannot be produced by what is not, but likewise that the more perfect, in other words, that which contains in itself more reality, cannot be the effect of the less perfect; and this is not only evidently true of those effects, whose reality is actual or formal, but likewise of ideas, whose reality is only considered as objective. Thus, for example, the stone that is not yet in existence, not only cannot now commence to be, unless it be produced by that which possesses in itself, formally or eminently, all that enters into its composition, [in other words, by that which contains in itself the same properties that are in the stone, or others superior to them]; and heat can only be produced in a subject that was before devoid of it, by a cause that is of an order, [degree or kind], at least as perfect as heat; and so of the others. But further, even the idea of the heat, or of the stone, cannot exist in me unless it be put there by a cause that contains, at least, as much reality as I conceive existent in the heat or in the stone for although that cause may not transmit into my idea anything of its actual or formal reality, we ought not on this account to imagine that it is less real; but we ought to consider that, [as every idea is a work of the mind], its nature is such as of itself to demand no other formal reality than that which it borrows from our consciousness, of which it is but a mode [that is, a manner or way of thinking]. But in order that an idea may contain this objective reality rather than that, it must doubtless derive it from some cause in which is found at least as much formal reality as the idea contains of objective; for, if we suppose that there is found in an idea anything which was not in its cause, it must of course derive this from nothing. But, however imperfect may be the mode of existence by which a thing is objectively [or by representation] in the understanding by its idea, we certainly cannot, for all that, allege that this mode of existence is nothing, nor, consequently, that the idea owes its origin to nothing.

15. Nor must it be imagined that, since the reality which considered in these ideas is only objective, the same reality need not be formally (actually) in the causes of these ideas, but only objectively: for, just as the mode of existing objectively belongs to ideas by their peculiar nature, so likewise the mode of existing formally appertains to the causes of these ideas (at least to the first and principal), by their peculiar nature. And although an idea may give rise to another idea, this regress cannot, nevertheless, be infinite; we must in the end reach a first idea, the cause of which is, as it were, the archetype in which all the reality [or perfection] that is found objectively [or by representation] in these ideas is contained formally [and in act]. I am thus clearly taught by the natural light that ideas exist in me as pictures or images, which may, in truth, readily fall short of the perfection of the objects from which they are taken, but can never contain anything greater or more perfect.

16. And in proportion to the time and care with which I examine all those matters, the conviction of their truth brightens and becomes distinct. But, to sum up, what conclusion shall I draw from it all? It is this: if the objective reality [or perfection] of any one of my ideas be such as clearly to convince me, that this same reality exists in me neither formally nor eminently, and if, as follows from this, I myself cannot be the cause of it, it is a necessary consequence that I am not alone in the world, but that there is besides myself some other being who exists as the cause of that idea; while, on the contrary, if no such idea be found in my mind, I shall have no sufficient ground of assurance of the existence of any other being besides myself, for, after a most careful search, I have, up to this moment, been unable to discover any other ground.

17. But, among these my ideas, besides that which represents myself, respecting which there can be here no difficulty, there is one that represents a God; others that represent corporeal and inanimate things; others angels; others animals; and, finally, there are some that represent men like myself.

18. But with respect to the ideas that represent other men, or animals, or angels, I can easily suppose that they were formed by the mingling and composition of the other ideas which I have of myself, of corporeal things, and of God, although they were, apart from myself, neither men, animals, nor angels.

19. And with regard to the ideas of corporeal objects, I never discovered in them anything so great or excellent which I myself did not appear capable of originating; for, by considering these ideas closely and scrutinizing them individually, in the same way that I yesterday examined the idea of wax, I find that there is but little in them that is clearly and distinctly perceived. As belonging to the class of things that are clearly apprehended, I recognize the following, viz, magnitude or extension in length, breadth, and depth; figure, which results from the termination of extension; situation, which bodies of diverse figures preserve with reference to each other; and motion or the change of situation; to which may be added substance, duration, and number. But with regard to light, colors, sounds, odors, tastes, heat, cold, and the other tactile qualities, they are thought with so much obscurity and confusion, that I cannot determine even whether they are true or false; in other words, whether or not the ideas I have of these qualities are in truth the ideas of real objects. For although I before remarked that it is only in judgments that formal falsity, or falsity properly so called, can be met with, there may nevertheless be found in ideas a certain material falsity, which arises when they represent what is nothing as if it were something. Thus, for example, the ideas I have of cold and heat are so far from being clear and distinct, that I am unable from them to discover whether cold is only the privation of heat, or heat the privation of cold; or whether they are or are not real qualities: and since, ideas being as it were images there can be none that does not seem to us to represent some object, the idea which represents cold as something real and positive will not improperly be called false, if it be correct to say that cold is nothing but a privation of heat; and so in other cases.

20. To ideas of this kind, indeed, it is not necessary that I should assign any author besides myself: for if they are false, that is, represent objects that are unreal, the natural light teaches me that they proceed from nothing; in other words, that they are in me only because something is wanting to the perfection of my nature; but if these ideas are true, yet because they exhibit to me so little reality that I cannot even distinguish the object represented from nonbeing, I do not see why I should not be the author of them.

21. With reference to those ideas of corporeal things that are clear and distinct, there are some which, as appears to me, might have been taken from the idea I have of myself, as those of substance, duration, number, and the like. For when I think that a stone is a substance, or a thing capable of existing of itself, and that I am likewise a substance, although I conceive that I am a thinking and non-extended thing, and that the stone, on the contrary, is extended and unconscious, there being thus the greatest diversity between the two concepts, yet these two ideas seem to have this in common that they both represent substances. In the same way, when I think of myself as now existing, and recollect besides that I existed some time ago, and when I am conscious of various thoughts whose number I know, I then acquire the ideas of duration and number, which I can afterward transfer to as many objects as I please. With respect to the other qualities that go to make up the ideas of corporeal objects, viz, extension, figure, situation, and motion, it is true that they are not formally in me, since I am merely a thinking being; but because they are only certain modes of substance, and because I myself am a substance, it seems possible that they may be contained in me eminently.

22. There only remains, therefore, the idea of God, in which I must consider whether there is anything that cannot be supposed to originate with myself. By the name God, I understand a substance infinite, [eternal, immutable], independent, all-knowing, all-powerful, and by which I myself, and every other thing that exists, if any such there be, were created. But these properties are so great and excellent, that the more attentively I consider them the less I feel persuaded that the idea I have of them owes its origin to myself alone. And thus it is absolutely necessary to conclude, from all that I have before said, that God exists.

23. For though the idea of substance be in my mind owing to this, that I myself am a substance, I should not, however, have the idea of an infinite substance, seeing I am a finite being, unless it were given me by some substance in reality infinite.

24. And I must not imagine that I do not apprehend the infinite by a true idea, but only by the negation of the finite, in the same way that I comprehend repose and darkness by the negation of motion and light: since, on the contrary, I clearly perceive that there is more reality in the infinite substance than in the finite, and therefore that in some way I possess the perception (notion) of the infinite before that of the finite, that is, the perception of God before that of myself, for how could I know that I doubt, desire, or that something is wanting to me, and that I am not wholly perfect, if I possessed no idea of a being more perfect than myself, by comparison of which I knew the deficiencies of my nature?

25. And it cannot be said that this idea of God is perhaps materially false, and consequently that it may have arisen from nothing [in other words, that it may exist in me from my imperfections as I before said of the ideas of heat and cold, and the like: for, on the contrary, as this idea is very clear and distinct, and contains in itself more objective reality than any other, there can be no one of itself more true, or less open to the suspicion of falsity. The idea, I say, of a being supremely perfect, and infinite, is in the highest degree true; for although, perhaps, we may imagine that such a being does not exist, we cannot, nevertheless, suppose that his idea represents nothing real, as I have already said of the idea of cold. It is likewise clear and distinct in the highest degree, since whatever the mind clearly and distinctly conceives as real or true, and as implying any perfection, is contained entire in this idea. And this is true, nevertheless, although I do not comprehend the infinite, and although there may be in God an infinity of things that I cannot comprehend, nor perhaps even compass by thought in any way; for it is of the nature of the infinite that it should not be comprehended by the finite; and it is enough that I rightly understand this, and judge that all which I clearly perceive, and in which I know there is some perfection, and perhaps also an infinity of properties of which I am ignorant, are formally or eminently in God, in order that the idea I have of him may be come the most true, clear, and distinct of all the ideas in my mind.

26. But perhaps I am something more than I suppose myself to be, and it may be that all those perfections which I attribute to God, in some way exist potentially in me, although they do not yet show themselves, and are not reduced to act. Indeed, I am already conscious that my knowledge is being increased [and perfected] by degrees; and I see nothing to prevent it from thus gradually increasing to infinity, nor any reason why, after such increase and perfection, I should not be able thereby to acquire all the other perfections of the Divine nature; nor, in fine, why the power I possess of acquiring those perfections, if it really now exist in me, should not be sufficient to produce the ideas of them.

27. Yet, on looking more closely into the matter, I discover that this cannot be; for, in the first place, although it were true that my knowledge daily acquired new degrees of perfection, and although there were potentially in my nature much that was not as yet actually in it, still all these excellences make not the slightest approach to the idea I have of the Deity, in whom there is no perfection merely potentially [but all actually] existent; for it is even an unmistakable token of imperfection in my knowledge, that it is augmented by degrees. Further, although my knowledge increase more and more, nevertheless I am not, therefore, induced to think that it will ever be actually infinite, since it can never reach that point beyond which it shall be incapable of further increase. But I conceive God as actually infinite, so that nothing can be added to his perfection. And, in fine, I readily perceive that the objective being of an idea cannot be produced by a being that is merely potentially existent, which, properly speaking, is nothing, but only by a being existing formally or actually.

28. And, truly, I see nothing in all that I have now said which it is not easy for any one, who shall carefully consider it, to discern by the natural light; but when I allow my attention in some degree to relax, the vision of my mind being obscured, and, as it were, blinded by the images of sensible objects, I do not readily remember the reason why the idea of a being more perfect than myself, must of necessity have proceeded from a being in reality more perfect. On this account I am here desirous to inquire further, whether I, who possess this idea of God, could exist supposing there were no God.

29. And I ask, from whom could I, in that case, derive my existence ? Perhaps from myself, or from my parents, or from some other causes less perfect than God; for anything more perfect, or even equal to God, cannot be thought or imagined.

30. But if I [were independent of every other existence, and] were myself the author of my being, I should doubt of nothing, I should desire nothing, and, in fine, no perfection would be awanting to me; for I should have bestowed upon myself every perfection of which I possess the idea, and I should thus be God. And it must not be imagined that what is now wanting to me is perhaps of more difficult acquisition than that of which I am already possessed; for, on the contrary, it is quite manifest that it was a matter of much higher difficulty that I, a thinking being, should arise from nothing, than it would be for me to acquire the knowledge of many things of which I am ignorant, and which are merely the accidents of a thinking substance; and certainly, if I possessed of myself the greater perfection of which I have now spoken [in other words, if I were the author of my own existence], I would not at least have denied to myself things that may be more easily obtained [as that infinite variety of knowledge of which I am at present destitute]. I could not, indeed, have denied to myself any property which I perceive is contained in the idea of God, because there is none of these that seems to me to be more difficult to make or acquire; and if there were any that should happen to be more difficult to acquire, they would certainly appear so to me (supposing that I myself were the source of the other things I possess), because I should discover in them a limit to my power.

31. And though I were to suppose that I always was as I now am, I should not, on this ground, escape the force of these reasonings, since it would not follow, even on this supposition, that no author of my existence needed to be sought after. For the whole time of my life may be divided into an infinity of parts, each of which is in no way dependent on any other; and, accordingly, because I was in existence a short time ago, it does not follow that I must now exist, unless in this moment some cause create me anew as it were, that is, conserve me. In truth, it is perfectly clear and evident to all who will attentively consider the nature of duration, that the conservation of a substance, in each moment of its duration, requires the same power and act that would be necessary to create it, supposing it were not yet in existence; so that it is manifestly a dictate of the natural light that conservation and creation differ merely in respect of our mode of thinking [and not in reality].

32. All that is here required, therefore, is that I interrogate myself to discover whether I possess any power by means of which I can bring it about that I, who now am, shall exist a moment afterward: for, since I am merely a thinking thing (or since, at least, the precise question, in the meantime, is only of that part of myself ), if such a power resided in me, I should, without doubt, be conscious of it; but I am conscious of no such power, and thereby I manifestly know that I am dependent upon some being different from myself.

33. But perhaps the being upon whom I am dependent is not God, and I have been produced either by my parents, or by some causes less perfect than Deity. This cannot be: for, as I before said, it is perfectly evident that there must at least be as much reality in the cause as in its effect; and accordingly, since I am a thinking thing and possess in myself an idea of God, whatever in the end be the cause of my existence, it must of necessity be admitted that it is likewise a thinking being, and that it possesses in itself the idea and all the perfections I attribute to Deity. Then it may again be inquired whether this cause owes its origin and existence to itself, or to some other cause. For if it be self-existent, it follows, from what I have before laid down, that this cause is God; for, since it possesses the perfection of self-existence, it must likewise, without doubt, have the power of actually possessing every perfection of which it has the idea—in other words, all the perfections I conceive to belong to God. But if it owe its existence to another cause than itself, we demand again, for a similar reason, whether this second cause exists of itself or through some other, until, from stage to stage, we at length arrive at an ultimate cause, which will be God.

34. And it is quite manifest that in this matter there can be no infinite regress of causes, seeing that the question raised respects not so much the cause which once produced me, as that by which I am at this present moment conserved.

35. Nor can it be supposed that several causes concurred in my production, and that from one I received the idea of one of the perfections I attribute to Deity, and from another the idea of some other, and thus that all those perfections are indeed found somewhere in the universe, but do not all exist together in a single being who is God; for, on the contrary, the unity, the simplicity, or inseparability of all the properties of Deity, is one of the chief perfections I conceive him to possess; and the idea of this unity of all the perfections of Deity could certainly not be put into my mind by any cause from which I did not likewise receive the ideas of all the other perfections; for no power could enable me to embrace them in an inseparable unity, without at the same time giving me the knowledge of what they were [and of their existence in a particular mode].

36. Finally, with regard to my parents [from whom it appears I sprung], although all that I believed respecting them be true, it does not, nevertheless, follow that I am conserved by them, or even that I was produced by them, in so far as I am a thinking being. All that, at the most, they contributed to my origin was the giving of certain dispositions (modifications) to the matter in which I have hitherto judged that I or my mind, which is what alone I now consider to be myself, is inclosed; and thus there can here be no difficulty with respect to them, and it is absolutely necessary to conclude from this alone that I am, and possess the idea of a being absolutely perfect, that is, of God, that his existence is most clearly demonstrated.

37. There remains only the inquiry as to the way in which I received this idea from God; for I have not drawn it from the senses, nor is it even presented to me unexpectedly, as is usual with the ideas of sensible objects, when these are presented or appear to be presented to the external organs of the senses; it is not even a pure production or fiction of my mind, for it is not in my power to take from or add to it; and consequently there but remains the alternative that it is innate, in the same way as is the idea of myself.

38. And, in truth, it is not to be wondered at that God, at my creation, implanted this idea in me, that it might serve, as it were, for the mark of the workman impressed on his work; and it is not also necessary that the mark should be something different from the work itself; but considering only that God is my creator, it is highly probable that he in some way fashioned me after his own image and likeness, and that I perceive this likeness, in which is contained the idea of God, by the same faculty by which I apprehend myself, in other words, when I make myself the object of reflection, I not only find that I am an incomplete, [imperfect] and dependent being, and one who unceasingly aspires after something better and greater than he is; but, at the same time, I am assured likewise that he upon whom I am dependent possesses in himself all the goods after which I aspire [and the ideas of which I find in my mind], and that not merely indefinitely and potentially, but infinitely and actually, and that he is thus God. And the whole force of the argument of which I have here availed myself to establish the existence of God, consists in this, that I perceive I could not possibly be of such a nature as I am, and yet have in my mind the idea of a God, if God did not in reality exist—this same God, I say, whose idea is in my mind—that is, a being who possesses all those lofty perfections, of which the mind may have some slight conception, without, however, being able fully to comprehend them, and who is wholly superior to all defect [and has nothing that marks imperfection]: whence it is sufficiently manifest that he cannot be a deceiver, since it is a dictate of the natural light that all fraud and deception spring from some defect.

39. But before I examine this with more attention, and pass on to the consideration of other truths that may be evolved out of it, I think it proper to remain here for some time in the contemplation of God himself—that I may ponder at leisure his marvelous attributes—and behold, admire, and adore the beauty of this light so unspeakably great, as far, at least, as the strength of my mind, which is to some degree dazzled by the sight, will permit. For just as we learn by faith that the supreme felicity of another life consists in the contemplation of the Divine majesty alone, so even now we learn from experience that a like meditation, though incomparably less perfect, is the source of the highest satisfaction of which we are susceptible in this life.
\chapter{Part 12: Ren\'e Descartes, Life, Times, and Meditations}
Descartes lived during an intellectually vibrant time. The Scholastics had supplemented Catholic doctrine with a tradition of Aristotle scholarship and early scientists like Galileo and Copernicus had challenged the orthodox views of the Scholastics. Surrounded by conflicting yet seemingly authoritative views on many issues, Descartes wants to find a firm foundation on which certain knowledge can be built and doubts can be put to rest. So he proposes to question any belief he has that could possibly turn out to be false and then to methodically reason from the remaining certain foundation of beliefs with the hope of reconstructing a secure structure of knowledge where the truth of each belief is ultimately guaranteed by careful inferences from his foundation of certain beliefs.

When faith and dogma dominate the intellectual scene, “How do we know?” is something of a forbidden question. Descartes dared to ask this question while the influence of Catholic faith was still quite strong. He was apparently a sincere Catholic believer and he thought his reason based philosophy supported the main tenants of Catholicism. Still he roused the suspicion of religious leaders by granting reason authority in the justification of our beliefs.

Descartes is considered by many to be the founder of modern philosophy. He was also an important mathematician and he made significant contributions to the science of optics. You might have heard of Cartesian coordinates. Thank Descartes. Very few contemporary philosophers hold the philosophical views Descartes held. His significance lays in the way he broke with prior tradition and the questions he raised in doing so. Descartes frames some of the big issues philosophers continue to work on today. Notable among these are the foundations of knowledge, the nature of mind and the question of free will. We’ll look briefly at these three areas of influence before taking up a closer examination of Descartes’ philosophy through his Meditations of First Philosophy.\autocite{Descartes1}

To ask “How do we know?” is to ask for reasons that justify our belief in the things we think we know. This is the branch of philosophy called \Gls{epistemology}, which is a fancy word for the study of knowledge (a little meta, I know). The epistemological project of providing systematic justification for the things we take ourselves to know was launched by Descartes and it remains a central endeavor in epistemology to this day. This project carries with it the significant risk of finding that we lack justification for things we think we know. This is the problem of skepticism. \Gls{skepticism} is the view that we can’t know. Skepticism comes in many forms depending on just what we doubt we can know. While Descartes hoped to provide solid justification for many of his beliefs, his project of providing a rational reconstruction of knowledge fails at a key point early on. The unintended result of his epistemological project is known as the problem of Cartesian skepticism. We will explain this problem a bit later in this chapter.


\newglossaryentry{epistemology}
{
name=epistemology,
description={A field of study within Philosophy dedicated to questions concerning knowledge, belief, and justification}
}

\newglossaryentry{skepticism}
{
name=skepticism,
description={The epistemological stance that knowledge is impossible either globally (it is impossible to know anything at all) or locally (knowledge is impossible within certain topics)}
}



Another area where Descartes has been influential is in the philosophy of mind (Module 3). Descartes defends a metaphysical view known as dualism that remains popular among many religious believers. According to this view, the world is made up of two fundamentally different kinds of substance, matter and spirit (or mind). Material stuff occupies space and time and is subject to strictly deterministic laws of nature. But spiritual things, minds, are immaterial, exist eternally and have free will. If dualism reminds you of Plato’s theory of the Forms, this would not be accidental. Descartes thinks his rationalist philosophy validates Catholic doctrine and this in turn was highly influenced by Plato through St. Augustine.

The intractable problem for Descartes’ dualism is that if mind and matter are so different in nature, then it is hard to see how they could interact at all. And yet when I look out the window, an image of trees and sky affects my mind. When I will to go for a walk, my material body does so under the influence of my mind. This problem of mind body interaction was famously and forcefully raised by one of the all too rare female philosophers of the time, princess Elizabeth of Bohemia.

Previously, we explored the philosophy of mind which was  launched in the wake of problems for substance dualism and because of the advancements there, Descartes' fruitful failure lead to neuroscience, cognitive psychology and information science. We also see how undeserved philosophy’s reputation for failing to answer its questions is. While many distinctively philosophical issues concerning the mind remain, the credit for progress will go largely to the newly minted science of mind. The history of philosophy nicely illustrates how parenthood can be such worthwhile but thankless work. As soon as you produce something of real value, it takes credit for 

The final big issue that Descartes brought enduring attention to is the problem of free will (Module 4). We all have the subjective sense that when we choose something we have acted freely or autonomously. We think that we made a choice and we could have made a different choice. The matter was entirely up to us and independent of outside considerations. Advertisers count on us taking complete credit and responsibility for our choices even as they very effectively go about influencing our choices. Is this freedom we have a subjective sense of genuine or illusory? How could we live in a world of causes and effects and yet will and act independent of these? And what are the ramifications for personal responsibility? This is difficult nettle of problems that continues to interest contemporary philosophers.

Descartes’ is also a scientific revolution figure. He flourished after Galileo and Copernicus and just a generation before Newton. The idea of the physical world operating like a clockwork mechanism according to strict physical laws is coming into vogue. Determinism is the view that all physical events are fully determined by prior causal factors in accordance with strict mechanistic natural laws. Part of Descartes’ motivation for taking mind and matter to be fundamentally different substances is to grant the pervasive presence of causation in the material realm while preserving a place for free will in the realm of mind or spirit. This compromise ultimately doesn’t work out so well. If every event in the material realm is causally determined by prior events and the laws of nature, this would include the motions of our physical bodies. But if these are causally determined, then there doesn’t appear to be any entering wedge for our mental free will to have any influence over out bodily movements.

\chapter{Part 13: An Overview of Descartes' Meditations on First Philosophy}

\Gls{epistemology} (from the Greek word $\eta\pi\iota\sigma\tau\eta\mu\eta$ (epist\-em\-e) meaning ‘knowledge’) is a branch of philosophy which deals with knowledge and belief. Some of the questions which are found there are:
\begin{enumerate}
    \item What does it take to know something?
    \item What is belief?
    \item What if faith? (This is a more recent question)
    \item What are the different kinds of knowledge? Are some better than others?
    \item What is wisdom?
    \item What separates knowledge from opinion?
    \item And there are many, many others.
\end{enumerate}
Despite what others may tell you, and I have debated this on many occasions, Descartes' Meditations concern epistemology, not metaphysics. So, Descartes' project in his meditations is to carry out a rational reconstruction of knowledge. Descartes is living during an intellectually vibrant time and he is troubled by the lack of certainty. With the Protestant Reformation challenging the doctrines of the Catholic Church and scientific thinkers like Galileo and Copernicus applying the empirical methods Aristotle recommends to the end of challenging the scientific views handed down from Aristotle, the credibility of authority has being challenged on multiple fronts. So Descartes sets out to determine what can be known with certainty without relying on any authority and then to see what knowledge can be securely justified based on that foundation.

The meditations are broken up into 6 parts and each serves a different purpose. I encourage you to read all 6, but the first 2 are enough for this module.
\section{The First Meditation}
Below is the first small section of this meditation, written in more plan language. The point of these meditations is to break down everything, doubt everything, until we can find something which can’t be doubted and use that as the basis for building up a 100\% guaranteed true set of beliefs/knowledge. The first meditation concerns what can be doubted.

\factoidbox{Several years ago, I came to the opinion that many of the things which I believed were true, were, in fact, false; and when that happened, I thought that I should do the job of trying to rid myself of these false opinions in order to get an indubitable structure for knowledge (science). But I waited until I was older and wiser so that I could have very little doubt in my abilities. On this account, I have delayed so long that it would be wrong to keep thinking on it and I should just put pen to paper. Today, then, since I don’t have anything which I need to do because I am retired, I will at length apply myself earnestly and freely to the general doubting of all my former opinions.}

For the second section of this meditation, Descartes starts us off by explaining his method for getting at that which cannot be doubted. Rather that dealing with each belief individually, he chooses to deal with classes, or categories, of beliefs (such as ones which all came from the senses, or were told to him by some person). If he finds reason to doubt some of the beliefs in a class, then he can doubt the entire class. The type of reason to doubt is also important. The reason must be one which is at the foundation of those beliefs. If you take out the foundation, the rest collapses. Here is an example of this general idea in practice:

\factoidbox{Take the beliefs that you got because someone told them to you. But, as we all know, people do sometimes lie or are mistaken about what they are telling others. This means that there is reason to doubt all of the things which you were taught in school, or by your folks, or by other people. Since there is reason to doubt those things, they can’t be the foundation of indubitable knowledge.}

Given the times Descartes was living in, with the rie of empirical science (the idea that knowledge is primarily gained from the senses, through experience and experiments), Descartes wanted to show that the fundamental basis for this knowledge was actually grounded in knowledge you can gain without appealing to your senses, called \emph{a priori} knowledge. So, he turns to whether one can doubt their senses. To get this off the ground, Descartes needs a way of getting rid of knowledge through your senses, as in, knowledge via experience. This means he needs a systematic way of making a worst case scenario (think Murphy's Law, if it can go wrong it will), making it so that the foundation of his knowledge is internal, not external.  According to Descartes, you can doubt your senses, meaning that they can't be at the foundation of an indubitable knowledge structure. He gives us two ways, with the second stronger than the first.
\begin{enumerate}
    \item What if he is having a really realistic dream? (this is in sections 6-8)
    \item What if he is under the influence of some evil demon? (this is in sections 9-12)
\end{enumerate}
The third section of this meditation gives us a basis for making the claim that our senses cannot be the foundational basis for knowledge. Put in more plain language, he says:

\factoidbox{Everything that I have held with certainty, I have gotten through my senses. But, I know that my senses have sometime misled me; as a general rule, you should not trust something (100\%) if it has lied to you.}

Every generation or so has a movie or show which draws inspiration from Descartes' Meditations or Plato's Cave (they both use similar thought experiments, but draw different, though related, conclusions. One great example is the first Matrix movie. So much inspiration was drawn that we have this little quote from Morpheus to Neo:

\factoidbox{Morpheus: Have you ever had a dream, Neo, that you were so sure was real? What if you were unable to wake from that dream? How would you know the difference between the dream world and the real world?}

The point here is that we could be deceiving ourselves, what if we are having a really, super, realistic dream? It would seem that, then, we couldn't trust our senses, rather what we experience would be an illusion.

\subsection{The dream case (this is section 6)}
At this point, Descartes gives the first thought experiment, or possible scenario, where his senses could be misleading, allowing him to discard them as the foundation for knowledge. Put more plainly, this section goes like this:

\factoidbox{Hey, you know, I am a man and, you know, I need to sleep. When I sleep, I tend to dream and be in those dreams. Sometimes those dreams are really crazy, but other times they make a lot of sense. Sometimes, I cannot tell the difference between my dreams and reality. How can I be sure that I am not, right now, dreaming something really realistic?}

This case, however, is not strong enough for Descartes' purposes. If he were in a very realistic dream, then Descartes could stil gain some knowledge from his senses, but this knowledge would be fairly abstract. Here is the plain language version of his reasoning:
\factoidbox{So, let's suppose that I am having this really realistic dream. This would mean that everything I am seeing now are nothing more than illusions from the unconscious mind. But we have to admit that the things in my dreams are based on things in the real world, no matter how mangled and jumbled they may be. Because of this, we can say that although the objects in dreams may be doubted, we must claim that there are things which they are based on. So, you can still not doubt your experienced whole-hog, like I'm trying to do.}

As you can see, if he were in a very realistic dream, then he would still have some knowledge about the external world. This is because dreams are based on experiences in the waking world. One can still have knowledge of abstract things like shape, color, motion, and the like. In order to completely discount knowledge gained from the senses, he needs a stronger possible case.
\subsection{God and Deception}

He believes that there is an all good God, but how could this fit with being mistaken in this way? If allowing a person to be mistaken did not jive with God, He would not allow it. But He does, we are mistaken all the time, so allowing a person to be mistaken must be just fine by God.

\subsection{The Evil Demon Case (this is section 12)}
Here we have the Evil Demon, which is the thought experiment strong enough to discredit all knowledge gained from the senses. This thought experiment is the basis for many books, movies, and shows.\footnote{Some examples include: The Matrix (1999), The Truman Show (1998), The Lego Movie (2014), Sword Art Online (2012-present), Inception (2010), Pan’s Labyrinth (2006), The Village (2004), Dark City (1998), The Signal (2014), Shutter Island (2010), and The Island (2005).} In plain language, his introduction of this case goes like this:
 
\factoidbox{I will suppose that some Evil Demon has employed all of their power to deceive me. All of my senses and the things which I learned from them are nothing more than either illusions or are the product of the Demon’s deceptions. I will consider myself as without  a body or anything external to the mind, as they are, too, illusions by the Demon. But this this is a really hard job, and difficult to maintain; so I tend to fall back into thinking that the external world is indubitable when I leave my armchair. This tendency does not mean that they are, in fact, indubitable, but rather, that we have a built in bias.}
\section{Descartes' Second Meditation}


\factoidbox{The last meditation left me with no knowledge at all, everything could be doubted, and I have a hard time forgetting this. Also, I don’t see a way out of the situation, it is like I have been plunged in deep water and can’t find my way up or down. Despite this, I will forge on, treating everything with the slightest doubt of its truth as if it were false. I will continue to do this until I have found something which I know for certain or, at the very least, that I know for certain that nothing else is know for certain. For Archimedes to move the world from one place to another only demanded a  firm and immovable point. In the same way, if I can find one thing which is indubitable, then I can have great hopes for this project.}


\factoidbox{Right now, I suppose that everything I see are illusions or are not actual. I can’t trust my memory, so I must assume that the things I remember never existed. My senses are also dubitable, so I must assume that I don’t have them. Because I learned about things like body, figure, extension and motion from my senses, I must assume that they are not actual as well. Since I can’t have any of those things, what can I have which is true? Maybe only that nothing is certain.}



\factoidbox{How do I know that there isn’t something different than those things which I can’t doubt? Didn’t I suppose that there was some kind of Demon who could cause these thoughts to come to my mind? But why do I even need to bring up this Demon, when I would deceive myself. Even if there is some demon, I am still something, I still exist. I have supposed that I don’t have a body, but then I must ask what comes from that? Am I so dependent on the body and the senses that without them, I don’t exist? But I had been persuaded that there was nothing in the world. When  I was persuaded, was I also persuaded that I don’t exist? Not at all, I certainly exist, since I was persuaded. At the same time, I supposed that there was some great demon deceiving me. This further shows that I exist, since I am being deceived. No matter how much this demon deceives me, he can’t make it that I don’t exist, so long as I am conscious, I am something. There is at least one thing which I cannot doubt, I am, I exist.}


\factoidbox{Even though I know that I exist, I don’t know what I am. The next step is one I need to take care in making, because if I say that I am the wrong sort of thing, then I will have strayed from the truth. This means that I need to think about what I thought I was and cast those aside. I will apply the same methods as before in order to get at what is undoubtable.}


\factoidbox{But what can I say I am if I suppose that there is this demon? Can I say that I still have all of the things which belong to the body? After looking into it, I can say that done really belong to me. Let’s move on, then, to the attributes of the soul. The first mentioned were the powers of nutrition and walking; but, if I don’t  have a body, then I can’t be able to either walk or be nourished.

Second, we had perception. In the same way as before, perception requires that I have a body, besides, I have often thought my self having perceptions in dreams.

The third and last attribute is thinking; and this is what really belongs to me. This alone is inseparable from me. I am--I exist: this is certain; but, when do I exist? I exist whenever I am thinking. If I stop thinking, it would seem that I cease to exist. I am therefore only a thinking thing, that is, a mind. I am, however, a real thing, and really existing; but what kind of thing? The answer was a thinking thing.}

\subsection{Commentary:}

Here, Descartes has his Evil Demon case, think of it like the Matrix without the mad glitches. This puts him in a state where he can doubt almost everything. Descartes ends up finding one thing which he can't doubt.

Even an evil deceiver could not deceive Descartes about his belief that he thinks. At least this belief is completely immune from doubt, because Descartes would have to be thinking in order for the evil deceiver to deceive him. In fact there is a larger class of beliefs about the content of one’s own mind that can be defended as indubitable even in the face of the evil deceiver hypothesis. When I look at the grey wall behind my desk I form a belief about the external world; that I am facing a grey wall. I might be wrong about this. I might be dreaming or deceived by an evil deceiver. But I also form another belief about the content of my experience. I form the belief that I am having a visual experience of greyness. This belief about the content of my sense experience may yet be indubitable. For how could the evil deceiver trick me into thinking that I am having such an experience without in fact giving me that experience? So perhaps we can identify a broader class of beliefs that are genuinely indubitable. These are our beliefs about the contents of our own mind. We couldn’t be wrong about these because we have immediate access to them and not even an evil deceiver could misdirect us.

The problem Descartes faces at this point is how to justify his beliefs about the external world based on the very narrow foundation of his indubitable beliefs about the contents of his own mind. And this brings us to one of the more famous arguments in philosophy: Descartes’ “Cogito Ergo Sum” or “I think, therefore I exist.” Descartes argues that if he knows with certainty that he thinks, then he can know with certainty that he exists as a thinking being. Many philosophers since then have worried about the validity of this inference. Perhaps all we are entitled to infer is that there is thinking going on and we move beyond our indubitable foundation when we attribute that thinking to an existing subject (the “I” in “I exist”). There are issues to explore here too. 

\section{Descartes' Third Meditation}
At this point, Descartes only has two justified beliefs. One is about his thoughts (the content of his mind) and the other is about his own existence (I think therefore I am). At the moment, he doesn't have any knowledge, or at least justified beliefs about anything outside of his mind. For example, he can't say ``I know I have hands" or ``I know there's a piece of paper in front of me."

Also, Descartes doesn't have any information about truths we know through reason alone, like 2+2=4 or other things like that. Those just don't have the justification yet. To get some basis to justify that stuff, Descartes sets out to prove that God exists and is not like his evil demon. Once the evil deceiver hypothesis is taken out because of God, facts we get from reason and perhaps those from our senses could get the justification to be knowable. However, not just any argument for the existence of God and that God's all-good will do the trick (we will be looking into a lot of different arguments for the existence of God in Module 5). The trick for Descartes’ project of a rational reconstruction of knowledge is to prove the existence of a good God by reasoning using only those beliefs that he has identified as indubitable and foundational.

Descartes argument for the existence of a good God goes roughly as follows:
\begin{enumerate}
    \item I find in my mind the idea of a perfect being.
    \item The cause of my idea of a perfect being must have at least as much perfection and reality as I find in the idea.
    \item I am not that perfect.
    \item Nothing other than a good and perfect God could be the cause of my idea of a perfect being.
    \item So, a good and perfect God must exist.
\end{enumerate}
This argument is a really simple version of the argument which Descartes gives (it's deep and complex) but the core features are there and can show the flaw in his reasoning (all arguments for the existence of God have at least one flaw (most of them have the same flaw, as we will see). Let's grant that this argument is valid and move over to look at it's soundness. Keep in mind that for Descartes to do the job he is trying to, all of his premises in this argument need to be indubitable and foundational. The first premise is a belief about his own thoughts, so we will give it to him. Though it is not as clear, premise three might arguably count as a foundational belief about the contents of Descartes' own mind. An evil deceiver, being evil, would lack perfection found in Descartes idea of a perfect being. So as powerful as such a being could be, the cause of Descartes idea of a perfect being must be more perfect than any evil deceiver. Perhaps any being so perfect would have to be a good God.

But the issue for Descartes is in the second premise. What reason do we have for thinking that the cause of something must have at least as much perfection as its effect?

The idea that there are degrees of perfection and the notion that the less perfect can only be explained in terms of the more perfect traces it's way back to Plato and his theory of Forms. Like many ideas of this type, it will strike many of us as implausible or even incomprehensible. What, exactly, is perfection supposed to mean here?  And even once we’ve spelled this out, why think causes must be more perfect?

It doesn't seem all that rare for less prefect things to make more perfect things (if you are a parent, it might seem clear that your kid is more perfect (better) than you are, or the ugly depressed artist making a beautiful painting).

In any case, whether the second premise can be explained and defended at all, the fatal flaw for Descartes’ project is that it is not foundational. It is not an indubitable belief about the contents of Descartes’ own mind, but rather a substantive belief about how things are beyond the bounds of Descartes’ own mind. So Descartes’ attempt to provide a rational justification for a substantive body of knowledge leaves us with an enduring skeptical problem. All we have immediate intellectual access to is the contents of our own minds. How can we ever have knowledge of anything beyond the contents of our own mind based on this? This is the problem of Cartesian skepticism.

Having diagnosed the fatal flaw in Descartes’ project, we should briefly consider how his rational reconstruction of knowledge was to go from there. Given knowledge of God’s existence and good nature, we would appeal to this to assure the reliability of knowledge had through reason and later also through the senses. God being the most perfect and good being would rule out the possibility of interference by an evil deceiver. We might still make mistakes in reasoning or be misinformed by the senses. But this would be due to our failure to use these faculties correctly. A good God, however, would not equip us with faculties that could not be trusted to justify our beliefs if used properly. This is a very cursory summary of the later stages of Descartes’ attempted rational reconstruction of knowledge in his Meditations. But it will suffice for our purposes.

\chapter{Part 14: Knowledge and Justification}

In the previous section of this module, we were discussing Ren\'e Descartes and his quest for certainty and knowledge. He did this by employing a form of methedological skepticism, doubt everything which can be doubted. In philosophy, \gls{knowledge}, as we are going to use it, is justified true belief. This allows us to make a nice little equation:\footnote{This equation will appear later.}

\begin{center}

K=TJB

Knowledge is True Justified Belief
\end{center}

Different stances in Epistemology regarding knowledge can be reduced to different ways of adding variables to this equation or even removing them. Some might claim that one can know something without believing it, others might claim that knowledge just is belief, and so on. The different stances in Epistemology can also be characterized by what they take each of these terms to mean (with the exception of knowledge). 

Justification for knowledge is different than other forms for justification which you might encounter (and even then there is some debate about the nature of that justification). In ordinary life, you are entitled to your medical records, which means that you are justified in asking for them. This kind of justification is related to, but not the same as, the kind of justification which we have for knowledge. Epistemic justification is the kind found when it comes to knowledge. This is background reasoning which supports the belief. A belief is epistemically justified if we have sufficiently good reason to think that the belief is true. 
\section{Why justification is important}

Some might try to answer the question ``what is knowledge?'' with a relatively simple answer ``you know it if you are correct.'' Using the equation from before, this stance could be characterized as K=T\footnote{if it's true then you know it}, K=B\footnote{knowledge is belief, there is no difference.}, or  K=TB\footnote{if it's true and you believe it, then you know it.} All three of these have a problem with getting lucky, where you guess and it turn out you were right. As an example, take this case:

\factoidbox{Recently, my little brother and I were binging the show Dragon Ball Z. My little brother came to me with a theory about the time travel based plots in the show.  As an expert on theories of time travel, I thought this was great. He came up with the idea that there had to be four different timelines, given what he had seen thus far. Explaining that there had to be two different versions of the same character and so forth. As it turns out, he was correct, there did need to be two different versions of that character and there needed to be 4 timelines, but for completely different reasons than the ones he laid out. He later said that it didn't feel like he was correct, something was missing.}

I quickly recognized what was missing: adequate \gls{justification}. The structure of his reasoning, though it lead to the correct answer, wasn't accurate. As a result, merely being correct is not enough for us to claim that we knew it. At the same time, merely having reasons to think that the belief is true is not enough to say that you knew it. Here is a, relatively, classic example of justification going astray. There have been various thought experiments which follow this line of thought, but I will go with the one which is earliest, as far as I am aware, from Alexius Meinong: 

\thoughtex{The Park Violinist}{Imagine that you live near a park and most days at noon, a violinist plays. You can hear the beautiful music from your dinning room and, it seems right to say that if you hear the music, then you know that the violinist is playing. However, one day, because you, unknowingly, drank some bad tea, at noon, you go temporarily deaf and have an auditory hallucination of a violin playing. While it is true that the violinist is out there and you hear the music, does it follow that you know the violinist is out there?}{parkviolinist.jpg}

For cases like these, most people would say that, though you were right, you don't know that the violinist is playing. The reason, as before, is that there was something wrong with the justification. 

It should be noted that 100\% certainty, no chance of being wrong, is not necessarily required for justification. For example, in a real world case, you can know that your sink isn't leaking without being in the room with the sink. There is a chance you are wrong, given your evidence. Justification can come in degrees and the more justified a belief is, then the more likely it is to be accurate/true. In the violinist example, experience and prior evidence certainly gives ample justification for knowledge in most cases, but you would have had the same evidence if the violinist had been sick and unable to play. This is why good critical thinkers question, doubt, or even reject beliefs if they lack adequate justification.

We see that the that the real area of debate, what really separates knowledge from a lucky belief, is the justification. As a result, much of the debate in Epistemology centers around the standards for justification. If you believe something and that thing happens to be true, what else is needed for you to know it? Examples like this can be found all over the place, where you were correct, but for totally unexpected reasons. This is why good critical thinkers look deeply at the structure of their justification, how their beliefs relate to each other, which also calls back to one of Socrates' core questions; what is your evidence?



\newglossaryentry{knowledge}
{
name=knowledge,
description={Justified True Belief; You have good reason to believe it, you do believe it, and your belief is, in fact, true}
}


\newglossaryentry{justification}
{
name=justification,
description={The reasons one has to believe something, why they think it is true}
}

Justification for knowledge is always how your beliefs relate to each other. You can think of this as a building or a puzzle (these ways of picturing the structure lead to two different theories of justification). After doing just a little introspection, you will find that your beliefs relate to each other in various ways, and some make others more likely to be accurate. For example, take this case:

    \factoidbox{My belief that the weather-forecast in Washington is unreliable is based on my belief that they have claimed many times that it would be sunny and on my belief that the weather on many of those occasions was, in fact, rainy. My belief about their claims is based on my belief that I remember their statements correctly. My belief about the weather on those occasions is based on my belief that my memory is correct.}

For this, we see that my beliefs form a pyramid, or tower, like structure, which leads us right into the first theory of justification, Foundationalism, which is what Descartes used in the Meditations.

\section{Foundationalism}
Much of the debate in Epistemology, as I mentioned, centers around justification. There are three dominate theories of justification which we will cover. Doing a little introspection should make it clear that some beliefs are built upon others. One belief is justified by another belief. For a particularly strange example, take this case: 

\thoughtex{Cube Earth}{I believe that the Earth is actually a cube. It is not spherical but it is also not flat. I believe this because I think that the middle of two extremes is the most  likely to be accurate. I also believe that there are two extreme stances in the debate over the shape of the Earth. One side claims that the Earth is spherical while the other side claims that the Earth is a flat disk. The two beliefs about the extreme stances with the general belief about the accuracy of a middle ground leads me to the conclusion that the Earth must be a cube.}{CubeEarth.jpg} 

You might notice that this reasoning is much like an argument which we have seen previously. I have given reasons and some general principles to connect them. Laid out this way, we can diagnose what went wrong with the justification. It is correct that there are two extremes in the debate over the shape of the Earth and their descriptions are, roughly, accurate. The issue lies in the principle which links them together. The middle of two extremes is not always the most accurate. This causes us to come to the wrong conclusion. The debate about justification is how the beliefs either need to relate to each other or the sort of cement used to link them together.

The most basic and common stance in Epistemology is \Gls{foundationalism}. This is the stance that our beliefs, in order to be justified, need to be ultimately justified on the basis of certain, core, foundational beliefs. Here is an example of a system of reasoning using Foundationalism: 

\factoidbox{I have the belief that the weather-forecast in Washington state is unreliable. I believe this because I also beleive that that the forecasters have claimed many times that it would be sunny and on my belief that the weather on many of those occasions was, in fact, rainy. My beliefs about their claims is based on my belief that I remember their statements correctly. My beliefs about the weather on those occasions is based on my belief that my memory and senses are correct.} 

Notice that the reasoning and the beliefs used to justify my stance that the weather forecasts in Washington state are unreliable is structured like a tower or maybe a pyramid. At the bottom of this structure are the foundational beliefs, which support all of the others. In this case, the foundational beliefs are that my memory and senses are accurate. 

Foundational beliefs are your rock bottom, these aren't supported or justified by anything else. If your beliefs are structured in this way and you want to get at knowledge, certainty, then you need to look very closely at the foundational beliefs. These beliefs need to be self-evident, they are not  supported by other beliefs, they need to be strong enough to carry all the weight, all by themselves.  If you have any experience with construction, or Minecraft, this analogy will work nicely for you. The foundationalist thinks of the intellectual structure of knowledge, how the beliefs relate to each other, as a tall building. Buildings can take many different shapes, from tall towers to pyramids, to strange Brutalist architecture, with a small foundation and an expanding roof. But, at the end of the day, the foundation, the concrete slab it's built on, needs to be tough enough to take the weight. 

Finding your foundational beliefs might require a little help, or a highly questioning mind. Little kids will often continually ask `why'. It could start with something fairly simple like `mommy, why is the sky blue?' and then you will answer with something accurate to the best of your knowledge, and then they will ask `why', you will reply, and they will ask `why' again, and again, and again. Eventually, you will get to a point where you just need to throw your hands up and say `that's just the way it is.' For example, last time I had this happen, I just stopped at `because some things exist and other things don't.' This final belief is your foundation. If you keep on digging, there's nothing left. Is this strong enough, on its own, to hold up the weight? In my weather forcast example, the foundations could be that my senses and memory are accurate (for those cases). Is that strong enough, all on its own? 

Foundational beliefs tend to be very basic logical or experiential truths. For example, something like `my senses are accurate' or `if P implies Q and P is true, then Q is true', or that there can't be contradictions. These all can be used to build up a structure of beliefs, which, though possible to be inaccurate, still count as justification. Additionally, Foundationalism holds that the beliefs you have, if they are justified, cannot directly or indirectly justify themselves. This is to say that there cannot be circular justification. For example, take this strange structure of justification:

\factoidbox{I believe that Mt. Rainer is an active volcano. I believe this because my middle school teacher told me so and I believe that she was accurate in her claim. I believe she was accurate in claiming this because I believe that Mt. Rainer is an active volcano.}

Notice that the first and final beliefs in this structure of reasoning are the same. Ultimately, it does not reach out to anything more than my belief that Mt. Rainser is an active volcano. If someone explained their stance in this sort of way, most people would be, correctly, very skeptical of their reasoning. Some might claim, however, that circular reasoning like this is not always bad. This case, in isolation, is but if we add more beliefs and increase the network of beliefs, showing more connections, then the circles will improve rather than deminish the justification. 

\newglossaryentry{foundationalism}
{
name=foundationalism,
description={The stance in Epistemology concerning justification which states that in order to know something the justification necessary must be structured in such a way as they fundamentally rely on basic, foundational beliefs which are self-evident and do not need further justification. The structure is often compared to a pyramid or a building}
}


\section{Infnite Regress Theory}

The \glspl{infinite regress theory} of justification points the finger at the foundational beliefs. For a belief to count as knowledge, according to the foundationalist, need to be built up from basic, foundations, which are self-evident. Those beliefs don't require any further justification. This theory claims that all beliefs need to be justified. It agrees with foundationalism that beliefs can't directly or indirectly support themselves, but it claims that foundational beliefs can't lead to knowledge. For a belief to count as knowledge, according to this theory, it must be justified by some other beliefs, which in turn need to be further justified by others, and those need to be justified by still others, and so on. For you to know something, there would need to be an infinite string of justifying beliefs, never making a circle, which support it. Such an account of justification makes knowledge, frankly, impossible.

\newglossaryentry{infinite regress theory}
{
name=infinite regress theory,
description={The stance in Epistemology concerning justificaton that all beliefs must be justified, there can be no `foundational' beliefs but also beliefs cannot indirectly justify themselves. The structure is like an infinitely long chain of branching beliefs},
plural=Infinite Regress Theory
}


Some of you might have heard the phrase `it's turtles all the way down', which has a humorous story behind it.  This story has changed and mutated quite a bit with time, but it gets the general idea:

    \thoughtex{Turtles All The Way Down}{Long ago, a philosopher was giving a lecture to the public. In it, they mention the structure of the solar system and the planets. Afterwards, an older lady comes up to the philosopher and says ``your theory that the sun is at the center and the Earth is a ball is very convincing, but I have a better one." Amused, the philosopher asks about it and she replies ``we live on the crust of the Earth which rests on a turtle's back." Instead of demolishing her stance with evidence, the philosopher chose to follow Socrates and ask follow-up questions, ``if that is correct, ma'am, then what does the turtle stand on?" Equally amused, the lady replied ``the first turtle stands on the back of a second turtle." ``But," replied the philosopher, ``what does the second turtle stand on?" She then said ``why, the back of yet another, bigger turtle, it's no use, Mister, it's turtles all the way down."}{TurtlesDown.jpg}

Infinite Regress Theory follows a very similar line of thought. Rather than their being foundational beliefs, in order to have knowledge, there must be beliefs all the way down.  However, having an infinite chain of knowledge doesn't, on its own, give us knowledge. For example, suppose that there's at least 0 soda cans on my desk. Here is a chain of beliefs; if there's at least 2 soda cans on my desk, then there's at least 1; if there are 3 soda cans on my desk then there are at least 2; if there are 4... Continuing for all numbers. None of these beliefs are really justified,  the chain will never reach back to reality and justify my claim that there's no soda cans on my desk. This also means that, if this theory is correct, knowledge is impossible and Global Skepticism is the way to go. 

\section{Coherentism}

Building off of the previous two, Foundationalism claims that there are bedrock beliefs which don't require further justification, knowledge is justified on the basis of other beliefs (aside from the foundational ones), and beliefs can't directly or indirectly justify themselves to count as knowledge. Infinite Regress Theory claims that there are no foundational beliefs, knowledge must be justified on the basis of further more basic beliefs, and beliefs can't directly or indirectly justify themselves.  \Gls{coherentism} is like Infinite Regress Theory in that it claims that there are no foundational beliefs in the structure of knowledge, but it rejects that beliefs can't indirectly justify themselves. Beliefs are justified in how they cohere or fit together. Several philosophers have called the structure involved in this justification a ``web of belief''\autocite{Quine1} and this is an apt picture to keep in mind when thinking about this structure of justification. In a spider web, every strand is connected, any where you start on the strand, you can follow the strands to any other one. But, it is not enough that the beliefs are all interconnected to be justfied. One could come up with radically strange reasons to move from ``green is a color'' to ``the Earth is flat'' but it doesn't follow that the person is justified in claiming that the Earth is flat.  

First, all of the beliefs in the web need to be consistent with each other. This should seem basic at this point, but in ordered to be properly justified, the beliefs you have can't be in conflict with each other. One cannot, for example, believe that Spiderman is a fictional character, that fictional characters do not exist, and that Spiderman is waiting for them in their apartment.\footnote{Unless they mean something different by `Spiderman' in these beliefs. In such a case, the beliefs may be consistent but they would lack a point.} 

Second, the beliefs must all be connected (as we saw before with the `web of belief') but weak connections between beliefs weaken, rather than strengthen, the degree of justification one has for any belief. At the same time, strong connections between the beliefs strengthen one's justification. One way to think about this is to consider likelihood. How likely is it that the inference you have made connecting two beliefs is accurate? The more likely the connection is, the stronger the connection. 

And third, one's justification is increased by the number of connections one makes or can reasonably make between their beliefs. This is to say that the more links one has connecting beliefs to others and the more integrated the beliefs are together, the more justified they are in their beliefs.\footnote{This is not to say that the more beliefs you have, the more justified you are in any of them. Rather, the more densely packed the connections are, the more they link together, the more justified you are.}   

It is worth noting that without an infinite number of beliefs, which no person can have, and without foundational beliefs, it's not possible to have a system of beliefs fit the coherentist model without there being some kind of circle, at least one. Coherentists embrace this circularity and claim that while small obvious loops are clearly a sign of bad reasoning, large, interconnected, multidirectional loops, are a sign of good justification. Because of this circularity, Coherentism allows for a sort of back and forth process; with your beliefs being shaped by your experiences and your experiences being shaped by your beliefs. Justification can come in two different directions. 

For example, consider a, probably young, person, Skie, who is unsure about the shape of the Earth. She goes out and looks at photos taken from orbit, listens to first hand accounts from astronauts, and even goes up in a space shuttle to see the shape for herself. In this case, the many individual observations, testamonies, and facts build up evidence for a more grand generalized claim that the Earth is round. The particular observations give us a basis for the principles. Here, the justification is going `bottom-up', moving from particulars to generalizations. 

At the same time, sometimes, we have cases where some evidence, an observation, just can't fit with a generalization we have made; sometimes, the evidence disproves our generalization. For example, suppose that after years of bird watching in states aside from Washington and seeing thousands of ravens, I generalize to the belief that all ravens are black. But, one day, while bird-watching in Washington, I spot two white ravens (yes, there is a beach in Washington with a family of white ravens). This will clearly conflict with my core belief that all ravens are black, but that's OK. After confirming that those are in fact ravens and doing my due diligence, I amend my belief to something like ``most ravens are black." But that is still fine with the Foundationalist, they would say that the process is still bottom-up.  

Coherentism allows for our beliefs to shape our experiences, going `top down'. In these cases, the generalized beliefs justify our particular evidence. The generalized claims stick around by categorizing or making sense of our observations.This can be because of the biases which we have covered in this class or it could be because of other reasons. Coherentism claims that our observations can be `theory laden', or influenced by our general theories. For example, a person, Mary, strongly believes in miracles. When she sees, for example, a statue of Jesus with water dripping from the cheeks, Mary will instantly color her experience of the `weeping statue' as a miracle. Similarly, Norm, who strongly holds that miracles don't happen, will see such an event and reject it almost instantly as his eyes playing tricks on him or will jump to some explanation in order to explain the event. This is the kind of back and forth which Coherentism entails. People who strongly believe in Sasquatch are likely to see Sasquatch because their beliefs color their experience and their experience shapes their beliefs. In Mary's case, her belief that the weeping statue is a miracle justifies her stance that miracles happen and her stance that miracles happen justifies her stance that it was a miracle. In Norm's case, his belief that miracles don't happen justifies his belief that the weeping statue was not a miracle. 


\newglossaryentry{coherentism}
{
name=coherentism,
description={The stance in Epistemology concerning justificaton that all beliefs must be justified, there can be no `foundational' beliefs. For this theory, it is necessary that some beliefs indirectly justify themselves. For this theory, beliefs are justified in how they cohere or fit together. The structure is like a spider web or loosely woven fabric}
}


\section{Native American Epistemologies}
By ``Native American Epistemologies'', I am referring to the systems of justification and knowledge used by the peoples indigenous to what is today the continental United States. The peoples, cultures, and epistemological systems native to the Americas (North and South) are very diverse, with each having subtle differences from their neighbors and those differences become greater the greater the distances between them. It pays, therefore, to be mindful that this is a cursory overview of the common features had by some of the peoples in a relatively small section of the Americas as a whole.\footnote{There are also the Epistemological systems used by the peoples native to Mesoamerica and various regions across South America. These systems may or may not have all or some of these common features.}  

All cultures, the world over, can have their requirements for knowledge be characterized, roughly, as `True Justified Belief'. In fact, we can make this into a nice little equation, like so:

\begin{center}
K = TJB
\end{center}

Thanks to this, we can see that the Epistemologies employed by different peoples and cultures can be reduced to disagreements about the answer to one or more of these questions:
\begin{enumerate}
\item What is `truth' as it relates to knowledge? Does this differ from `truth' as in a fact about the world?
\item What is a `belief'? Can I believe something without being able to use it?
\item What does it take to be justified in claiming something? What structure does the reasoning take?
\end{enumerate}
The `old world' Epistemologies, those found in Europe, North Africa, the Middle East, and Asia, all tend to agree, on an abstract level, about the answers to these three questions. Starting with the last, those systems all tend to be Foundationalist. There are basic, foundational beliefs (which don't need justification) and those build up to more complex beliefs in a tower like structure. These old world Epistemologies also all tend to hold that the nature of beliefs (as they relate to knowledge) is propositional, you know \emph{that} the sky is blue, \emph{that} carbon has 6 protons, \emph{that} a person needs to sleep, and so on. And, finally, they all agree that the proposition must correspond with reality, be a fact about the world, independent of you, in order for you to know it. This is to say that there is no difference betweem truth as a condition for knowledge and truth as in a fact about the world. 

Native American Epistemologies disagree with the `old world' ones on all three points.\autocite[18]{Burkhart1} Truth for knowledge is different than truth as in a fact about the world, there's something more to it. Beliefs are not just propositional, they are far more `global', all encompassing, they are not isolated. And, finally, the justification for knowledge is not Foundationalist, but rather Coherentist.   
\subsection{Truth}
As mentioned previously, the sense of truth used as a requirement for knowledge in the Epistemologies of Europe, Asia, the Middle East, and Africa\footnote{At the very least North Africa.} is the same as the sense of truth used in statements like ``it's true that there is no largest prime number'' or ``it's true that the Earth is spherical''.\autocite[57]{NortonSmith1} This makes truth a feature or property of propositions. It's true when it accurately corresponds or represents the way the world is, independent of the mind. This kind of knowledge or truth is sometimes called `propositional knowledge' or `propositional truth'. In English, we say things of the form `I know that...' the `...' there is the proposition, in isolation, which is claimed to be true because it corresponds with the what the world \emph{really} is. In these Epistemological systems, this `knowledge-that' is treated as fundamental, the basic form which knowledge takes and other forms of knowledge are explainable in terms of a collection of known propositions. 

Native American Epistemologies do not treat propositional knowledge, knowledge-that, as fundamental. Rather, the fundamental kind of knowledge is \emph{procedural} knowledge, or `knowledge-how'.\autocite[19]{Burkhart1} Procedural knowledge is measured by practicality in addressing some practical concern or reaching some goal. The kind of truth related to knowledge in Native American Epistemologies is procedural as well. Rather than being a property or feature of statements, it is a property of a performance or action.\autocite[63]{NortonSmith1} 

This points out the first of a few consequences of taking on this notion of truth: Truth comes in degrees. Some methods are hands-down more successful in reaching a goal than others, making them more true, in this sense. Compare, for example, two different farming practices. First, we have the Native American method of the three sisters. Corn (maize), squash, and peas are all planted together and grown together in the same spot (each corn stalk has peas climbing it and squash around the base). This method produces a large quantity of food and you can plant in the same mounds for generations without loss of soil integrity, plant vigor, or other concerns. Second, we have the methods used by many farmers today. This is large scale monoculture,\footnote{Only one kind of plant grown in the field} using synthetic fertilizers, pesticides, and the like. These methods are commonplace because they produce far more food than the three sisters or other similar methods can. Even without the synthetic fertilizers and pesticides, such methods produce more foods. So, if our goal was simply to produce as much food (for humans) as possible, the monoculture method would be more true than the three sisters method. But, the use of the monocultural methods harms soil quality and the various additives and pesticides harm the rivers, fish, and the rest of the food chain from there.\autocite[21-22]{Burkhart1} 

This points out that there must be an overarching goal, a greater goal, such that the lesser goals are judged by how well they achieve or progress towards meeting it. This goal is to ``walk the right path'', be a good person and make good actions. This is fundamentally a moral component:

\factoidbox{The idea is simply that the universe is moral. Facts, truth, meaning, even our own existence are normative. In this way, there is no difference between that is true and what is right. On this account, then, all investigation is moral investigation. The guiding question for the entire philosophical enterprise is, then: what is the right road for humans to walk?\autocite[17]{Burkhart1}}
At least from this perspective, the monocultural practices are not on the path which a human should walk. It follows, then, that the monocultural methods, though more successful at achieving the lesser goal of producing the most food, are less successful at achieving the greater goal of being morally right, because it causes harm to our non-human counterparts. This, however, does not imply that what is the case about the world is right or as it should be. Remember that `true' or `truth' in this sense is different from `truth' in the sense of being a fact of the world, independent of us. Everything we do, say, and even think has a moral component to it. From this perspective, the `truth' required for knowledge is that it is good, moral, on the road you ought to walk. 

Following Thomas M. Norton-Smith, a Shawnee philosopher of clan Turkey, we can characterize this notion of truth as a biconditional:
\factoidbox{For subject(s), S, her, his, or their purpose or goal g, and action or performance p, p is true for S for g if and only if p is \emph{respectfully successful} in achieving g.\autocite[64]{NortonSmith1}}
This \emph{respectful success} of an action just means that the action is the morally right one to do, especially with respect towards the consequences, not just for humans, but non-humans as well.\footnote{This hints at a Consequentialist notion of ethics, which we will explore in Module \ref{ch.modeight}, or at least a form of Dao Consequentialism. Dao Consequentialism or `Way Consequentialism' states that an action is right if and only if it is a part of a way or pattern of life which as a whole causes more good and less bad. Different forms of Consequentialism propose different goods and bads which ought to be promoted or mitigated respectfully.}  

\subsection{Belief}
As the Choctaw Native American philosopher Lee Hester puts it, western science and philosophy is based on \emph{belief}.\autocite[321]{Hester1} Similar to the nature of truth in those Epistemologies, beliefs are taken to be isolated propositions which one claims to be the case; true beliefs would be those which one claims to be the case and accurately correspond with reality, independent of the person in question. The beliefs, taken together, form a map or a depiction of the world (territory). This map is taken to be the true depiction of the territory, there can only be one. 

In Native American Epistemologies, both the map and the territory are real, but the map is not mistaken for the territory.\autocite[321]{Hester1} The map is taken to be one of many ways to see the world around us, different ways of seeing may be more true than others according to how well they guide us and help us walk the path we ought to.  
\factoidbox{An old chief of the Crow tribe from Montana was asked to describe the difference between his tribe and the whites who lived nearby. Pausing slightly and drawing his conclusions, he remarked that the white man has ideas [beliefs], the Indian has visions\autocite[15]{Deloria1}}
\emph{Visions}, as the Crow chief calls them, or \emph{recognitions}, as Vine Deloria Jr. later calls them,\autocite[362]{Deloria1} are far more robust than beliefs. They are \emph{integrating experiences} which shape and inform how one sees the world in all matters, ``culturally, spiritually, psychologically, politically, and in matters of subsistence and use of technology''.\autocite[321]{Hester1} Unlike beliefs, visions are not isolated, they are fundamentally interconnected with all aspects of the person. One could, digging a little deeper, think that these Native American Epistemologies have propositional beliefs just as much as the old world ones, maybe even more of them. This could be correct but they are not the basis for knowledge, rather the fundamental unit is the \emph{vision}, composed of a litany of propositional beliefs. 

\subsection{Justification}

Thus far, we have seen how Native American Epistemologies differ from the old world Epistemologies in two different ways. First, `truth' as a requirement for knowledge is not the same as `truth' as in an accurate description of the world. `Truth' in Native American Epistemologies is assessed more pragmatically, something is true for the person if it that thing is \emph{respectfully successful} at achieving the person's goal. This respectfulness relates to the overarching goal of walking the right path, of being a good person and acting in a moral manner. Second, the \emph{vision} is far more important than the belief. Recall that this is an integrating experience which links all of your beliefs into a cohesive network. This points out the fundamental system of justification in Native American Epistemologies, Coherentism. As Norton-Smith puts it: 
\factoidbox{All beings and their actions in the American Indian world are related and interconnected, so knowing about the world involves actively seeking out newly emerging connections between experiences.\autocite[58]{NortonSmith1}}
We ought to note that \emph{experiences} are not just the waking ones which are undiluted and un-muddied (so to speak); all experiences are to be treated equally and integrated. In the western epistemic tradition, at least, experiences like dreams, prophetic visions, and the like are seen as \emph{deceptive};\autocite[71]{NortonSmith1} they are not the sort of experiences which give an accurate account of the territory and ought not to be included in the map of the world. This sharp distinction between appearance and reality is absent from Native American Epistemologies, all experiences, even those which would be rejected by western sciences, are worthy of integration into the larger network. More over, in the western tradition at least, first-hand experiences are taken to be privileged. This is to say that the person who had the experience is in a better position to use that experience in their justifications than a person who merely heard about it second hand. This notion of privilege is also absent from Native American Epistemologies, as Burkhart says, ``What place do I have to tell you that your experiences are invalid because I do not share them?''\autocite[26]{Burkhart1} This allows for \emph{tribal knowledge} or communal knowledge, knowledge which comes from the merging of many minds and experiences, handed down by tribal elders to the young, often through stories, advice, and council.\footnote{This is, in part, why the stories and histories of a tribe are so important. They serve as an integral part of the tribal knowledge which has been building for generations upon generations.}

Following Norton-Smith once again, we can give the following biconditional for what it takes to be justified in Native American Epistemologies, noting that this is closely tied in with the notions of truth and visions which we have seen previously:
\factoidbox{For subject(s), S, her, his, or their purpose or goal g, and action or performance p, p is justified... for S for g if and only if either (1) S directly experiences the respectful success of p in achieving g, or (2) the respectful succes of p in achieving g is endorsed by S's trbal tradition.\autocite[73]{NortonSmith1}}
Using this and the other aspects of knowledge which we have seen in this section, we can construct an account of what it takes to know how to do something according to Native American Epistemologies and we can even make an account of what it takes to know something (have knowledge-that) within those systems. Starting with the procedural knowledge, again following Norton-Smith, we have:
\factoidbox{For subject(s), S, her, his, or their purpose or goal g, and action or performance p, S knows how to p to achieve g if and only if (1) p is true for S for g and (2) p is justified ... for S for g.\autocite[74]{NortonSmith1}}
Although Norton-Smith primarily focuses on the procedural knowledge, the know-how, which seems primary in Native American Epistemologies, it is possible for us to faithfully generalize the various aspects to give an account of propositional knowledge, knowledge-that, by amending the standards, only when necessary, like so:
\factoidbox{For a person, S, the proposition p is true if and only if p is \emph{respectful}, meaning that p is in accordance with morality and would assist in living the kind of life one ought.

For a person S, p is justified for S if and only if S had a vision concerning or related to p (remember that a vision is an integrating experience) AND either (1) S directly experienced p, (2) S is told, in a reliable fashion, that another experienced, or (3) p is a part of S's tribal knowledge. 

For a person S, S knows that p if and only if p is true for S and p is justified for S.
}

Note that the final biconditional does not make reference to beliefs directly. This is because, as mentioned before, the \emph{vision} is what matters in Native American Epistemologies and this appears as a fundamental component of the conditions for justification. 
   